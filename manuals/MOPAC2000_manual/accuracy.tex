\chapter{Accuracy of Methods in MOPAC} 
\index{MOPAC!accuracy}
\index{Accuracy of methods}
\index{MNDO!accuracy of}
\index{MINDO/3!accuracy of}
\index{AM1!accuracy of}
The following tables illustrate the relative accuracy of three semiempirical
methods: MNDO, AM1, and PM3. PM3 is a re-parameterization of the MNDO method,
in which the AM1 form of the core-core interaction is used. In cases where two
sets of MNDO parameters have been published, the more recent set will be used. 

\section{Protocols used in Determining Accuracy}
The selection of species to be used to demonstrate the accuracy of the various
methods presented some difficulty.   As the designer of the AM1 method, I have
an understandable bias to show that AM1 is better than MNDO.  As the sole
author of PM3, I also have an understandable bias to show that PM3 is better
than AM1.  However, in order to be credible from a reader's point of view,
protocols need to be set up which will convince the reader that these biases 
have not affected the discussions or conclusions in this Chapter.

After some thought, I decided to adopt the following protocols:
\begin{enumerate}
\item All species for which reliable experimental data were available would be
used in generating the raw data for the statistical analyses.
\item Experimental data of low or unknown accuracy was used as reference data
if, and only if, a calculation using that data was done.
\item All results of all data would be reported in un-corrected and un-modified
form.
\item Only data for which references were available would be used.
\item Experimental data included some high-level {\em ab initio} results. 
Where {\em ab initio} results were used, the {\em ab initio} method (basis set
plus post-Hartree Fock corrections, if necessary) was of much higher
reliability than the semiempirical method.  In other words, any differences
between {\em ab initio} and semiempirical results would be due to errors in the
semiempirical method.
\end{enumerate}

The second of these protocols needs a word of explanation.  Consider the case
of a compound whose $\Delta H_f$ is of highly questionable accuracy, say 
10$\pm$30 kcal/mol.  Should this system be used in discussions of accuracy? If
the calculated PM3 $\Delta H_f$ was 11 kcal/mol and the calculated AM1 $\Delta
H_f$ was 40 kcal/mol, then PM3 would obviously look better than AM1.  In that
case, my bias would be to include the compound.  Conversely, if the results
were reversed, then my bias would suggest that clearly the reference datum was
very inaccurate,  and therefore should not be included.  This is obviously
faulty reasoning.  Therefore, in order to prevent bias affecting the selection
of compounds to be used in these discussions, the decision of whether to
include any specific species was made before the calculation was done.

By adopting these protocols, and by extensive use of automatic table
generation, and by an absolute minimum  of human intervention, I hope the
reader will be convinced that bias has been kept to an absolute minimum.

In the Tables, the accuracy of the reference data is not given.  This is a
known fault. However, addition of the accuracy data would be a lot of work,
and  has not yet been done.  

\section{Structure of the Tables}
Because of the size of these Tables, finding any specific system can be quite
time-consuming.  To help locate species, the order of occurrence of species
within a Table follows very specific rules.  These are:

\begin{itemize}
\item The order of occurrence is dictated by the empirical formula only.
\item Elements in the empirical formula are arranged as in 
J. O. Cox, G. Pilcher, ``Thermochemistry of Organic and Organometallic
Compounds'', Academic Press, New York, N.Y., 1970, e.g.\
H $<$ C $<$ O $<$ N $<$ S $<$ F $<$ Cl $<$ Br $<$ I $<$ Li $<$ Na $<$ K $<$ 
Rb $<$ Cs $<$ Be $<$ Mg $<$ Ca $<$ Sr $<$ Ba $<$ B $<$ Al $<$ Ga $<$ In $<$ 
Tl $<$ Si $<$ Ge $<$ Sn $<$ Pb $<$ P $<$ As $<$ Zn $<$ Cd $<$ Hg $<$ Sb $<$
Se $<$ Te $<$ Bi.
\end{itemize}
 
\index{$\Delta H_f$!average errors in}
\index{Heat of Formation!average errors in}
Average $\Delta H_f$ errors for different types of compound are given  in
Table~\ref{avehof}, and summarized by element in Table~\ref{elehof}. Because of
the importance of $\Delta H_f$s, the raw data are presented in
Table~\ref{hoftab}.  Users often want to know where methods do not work well,
so the worst predicted $\Delta H_f$ are presented for each method.  These are
shown in Tables~\ref{wemndo}, \ref{weam1} and \ref{wepm3}.

Summarizing geometry errors proved difficult because of the sheer volume of
data. Because bond-lengths are very important, and because of the relatively
small amount of data on angles and dihedrals, only average bond-length errors
are presented. For bond-types, average errors are given in
Table~\ref{avegeos}.  This data is further summarized by element in
Table~\ref{elegeos}.  All the raw data are given in Table~\ref{geotab}.

Although Ionization Potential (I.P.) and dipole moment data is important,  the
importance is much less than $\Delta H_f$ and geometric data, therefore only
summaries are given here, see Tables~\ref{weimndo}, \ref{weiam1} and
\ref{weipm3} for the worst predicted values of I.P.s,  and Table~\ref{eleips}
for a summary by element.  For dipole moments, the corresponding Tables are
\ref{wedmndo}, \ref{wedam1}, \ref{wedpm3}, and \ref{eledip}.

\section{Limitations of MNDO}\index{MNDO!accuracy}
MNDO is the oldest of the three methods surveyed here, and as a direct result,
is the least accurate. MNDO has many advantages over earlier semiempirical
methods, but as MNDO is such a large improvement over those methods,
enumeration of MNDO's good points would be invidious to the other methods.
Instead, only the limitations likely to be encountered by users will be
mentioned. The principal drawbacks to MNDO are:
\begin{enumerate}
\item Sterically crowded molecules are too unstable.   Example: neopentane.
\item Four-membered rings are too stable.   Example: cubane.
\item The hydrogen bond is virtually non-existent.   Example: water dimer.
\item Hypervalent compounds are too unstable.   Example: sulfuric acid.
\item Activation barriers are generally too high.
\item Non-classical structures are predicted to be unstable relative to the 
classical structure.   Example: ethyl radical.
\item   Oxygenated substituents on aromatic rings are out-of-plane.   Example: 
nitrobenzene.
\item The peroxide bond is systematically too short by about 0.17\AA.
\item The C--O--C angle in ethers is too large by about 9$^\circ$. 
\end{enumerate}

\section{Accuracy of AM1}\index{AM1!accuracy}
AM1 is a distinct improvement over MNDO, in that the overall accuracy is
considerably improved. Specific improvements are:
\begin{enumerate}
\item The strength of the hydrogen bond in the water dimer is 5.5 kcal/mol, in
accordance with experiment.
\item Activation barriers for reaction are markedly better than those of MNDO.
\item Hypervalent phosphorus compounds are considerably improved relative to
MNDO.
\item In general, errors in $\Delta H_f$ obtained using AM1 are about 40\%
less than those given by MNDO.
\end{enumerate}

Unfortunately, with this improvement a few deficiencies were introduced. The
most important of these are:
\begin{enumerate}
\item\index{Phosphorus!AM1 fault} AM1 phosphorus has a spurious and very sharp
potential barrier at 3.0 \AA ngstroms. The effect of this is to distort
otherwise symmetric geometries and to introduce spurious activation barriers. A
vivid example is given by P$_4$O$_6$, in which the notionally equal P--P bonds
are predicted by AM1 to differ by 0.4 \AA . This is by far the most severe
limitation of AM1.
\item Alkyl groups have a systematic error due to the heat of formation of the
CH$_2$ fragment being too negative by about 2 kcal/mol.
\item Nitro compounds, although considerably improved relative to MNDO, are
still systematically too positive in energy.
\item The lowest-energy conformer for ethanol is predicted to have a gauche 
C--C--O--H angle.  High level {\em ab initio} calculations, and experimental 
evidence, suggests that the correct geometry is trans.  This error was 
reported by Andreas Jabs, at Martin-Luther-Universitaet Halle-Wittenberg,
Halle/Saale, Germany.
\item The peroxide bond is still systematically too short by about 0.17\AA.
\end{enumerate}

\section{Accuracy of PM3}\index{PM3!accuracy}
PM3 is a distinct improvement over AM1. Evidence of this is:
\begin{enumerate}
\item Hypervalent compounds are predicted with considerably improved accuracy.
\item Overall errors in $\Delta H_f$ are reduced by about 40\% relative to AM1.
\end{enumerate}

\section{Faults and Errors in PM3}
Although PM3 is more accurate in predicting the $\Delta H_f$ of many systems
than either AM1 or MNDO, it does have some limitations. The more important of
these are:

\begin{enumerate}
\item The barrier to rotation in formamide is practically non-existent. In
part, this can be corrected by the use of the \comp{MMOK} option, see
Section~\ref{mmok}. Fault discovered by Prof Tim Clark.
\item Minima are predicted for which no experimental systems are known.  By
inference, these minima are spurious~\cite{klebe}.
\item Proton affinities are inaccurate.  The distribution of proton affinities
is essentially ``shotgun.''  In other words, there is a high, non-systematic
scatter from the experimental values compared with all other popular
methods~\cite{schmiedekamp} (including AM1).
\item CBr$_4$ is predicted to have a D$_{2d}$ rather than T$_d$ geometry (W.\
Thiel, private comm.).
\item The lowest-energy conformer for ethanol is predicted to have a gauche
C--C--O--H angle.  High level {\em ab initio} calculations, and experimental
evidence, suggests that the correct geometry is trans.  This error was reported
by Andreas Jabs, at Martin-Luther-Universitaet Halle-Wittenberg, Halle/Saale,
Germany.
\item Some polycyclic rings are predicted to have butterfly shapes rather than
being flat.  
\item Almost all $sp^3$ nitrogen systems are predicted to be pyramidal, in
variance with experiment and other methods.
\item The charge on nitrogen atoms are most likely of incorrect sign  and of
unrealistic magnitude. This was due to faults in the parameterization.
\item H-non-bonded H contacts are too short, at about 1.7\AA . They should be
more than 2\AA .  One consequence is that hydrogen bonds are too short by about
0.1\AA .
\item The LUMO of methane is of symmetry $a_1$ instead of the correct symmetry 
of $t_2$. 
\end{enumerate}

The tables given here have been carefully checked for accuracy. In many
instances the calculated results differ from those of earlier publications. In
most cases, the difference has been investigated and found to be due to errors
in the earlier work. This does not, of course, mean that the tables are fully
accurate, but only that care has been taken to minimize error.

% \section{Comparison of Accuracy of MNDO, AM1, and PM3}
\begin{table}
\caption{\label{wemndo} Worst $\Delta H_f$ Errors in MNDO}
\compresstable
\begin{center}
\begin{tabular}{llrrr}
 Formula & Chemical Name & Exp. $\Delta H_f$ & Calc. $\Delta H_f$ & Diff.\\
 \hline
 O$_7$Cl$_2$          & Cl2O7                                  &   65.00   &   678.55   &  613.5\\
 S$_2$F$_1$$_0$       & S2F10                                  & -493.40   &    50.79   &  544.2\\
 F$_7$I               & Iodine heptafluoride                   & -229.70   &   105.16   &  334.9\\
 F$_5$I               & Iodine pentafluoride                   & -200.84   &   130.85   &  331.7\\
 O$_3$FCl             & Perchloryl fluoride                    &   -5.12   &   323.22   &  328.3\\
 SF$_6$               & Sulfur hexafluoride                    & -291.40   &    29.18   &  320.6\\
 SOF$_4$              & SOF4                                   & -235.50   &    33.51   &  269.0\\
 P$_4$O$_1$$_0$       & Phosphorus pentoxide                   & -694.09   &  -431.88   &  262.2\\
 F$_5$Cl              & Chlorine pentafluoride                 &  -54.00   &   204.73   &  258.7\\
 SF$_5$               & Sulfur pentafluoride                   & -217.10   &    -8.99   &  208.1\\
 F$_5$Br              & Bromine pentafluoride                  & -102.50   &   104.84   &  207.3\\
 SO$_2$F$_2$          & Sulfuryl fluoride                      & -181.30   &    21.90   &  203.2\\
 SOF$_3$              & SOF3                                   & -185.10   &    -5.00   &  180.1\\
 H$_2$SO$_4$          & Sulfuric acid                          & -175.70   &   -13.13   &  162.6\\
 N                    & Nitrogen (++)                          & 1133.90   &   973.15   & -160.8\\
 SF$_5$               & Sulfur pentafluoride (-)               & -291.00   &  -131.66   &  159.3\\
 C$_2$H$_6$SO$_4$     & Dimethyl sulfate                       & -164.10   &    -5.58   &  158.5\\
 SO$_3$               & Sulfur trioxide                        &  -94.60   &    58.46   &  153.1\\
 PF$_6$               & Phosphorus hexafluoride (-)            & -522.00   &  -369.69   &  152.3\\
 C$_3$H$_8$SO$_2$     & Methylethyl sulfone                    &  -97.60   &    47.01   &  144.6\\
 C$_4$H$_1$$_0$SO$_2$ & Diethyl sulfone                        & -102.50   &    40.49   &  143.0\\
 C$_2$H$_6$SO$_2$     & Dimethyl sulfone                       &  -89.10   &    53.64   &  142.7\\
 SOCl$_4$             & SOCl4                                  &  -55.70   &    86.91   &  142.6\\
 SCl$_6$              & Sulfur hexachloride                    &  -19.80   &   118.29   &  138.1\\
 C$_4$H$_6$SO$_2$     & Divinyl sulfone                        &  -36.00   &   101.33   &  137.3\\
 SF$_4$               & Sulfur tetrafluoride                   & -182.40   &   -46.51   &  135.9\\
 PF$_5$               & Phosphorus pentafluoride               & -381.10   &  -248.82   &  132.3\\
 SO$_2$Cl$_2$         & Sulfuryl chloride                      &  -86.20   &    44.58   &  130.8\\
 SO$_2$Br$_2$         & Sulfuryl bromide                       &  -59.50   &    68.33   &  127.8\\
 SO$_2$F              & SO2F                                   & -113.20   &    13.99   &  127.2\\
 Al$_2$I$_6$          & Al2I6                                  & -117.00   &     7.41   &  124.4\\
 Si$_2$Br$_6$         & Hexabromodisilane                      & -182.80   &   -65.96   &  116.8\\
 F$_3$Cl              & Chlorine trifluoride                   &  -38.00   &    78.69   &  116.7\\
 O$_2$Cl              & Chlorine dioxide                       &   25.00   &   135.99   &  111.0\\
 SBr$_6$              & Sulfur hexabromide                     &   58.80   &   166.34   &  107.5\\
 S$_2$F$_2$           & SSF2                                   &  -95.94   &     7.94   &  103.9\\
 C$_5$H$_8$N$_4$O$_1$ & Pentaerythritol tetranitrate           &  -92.50   &     1.23   &   93.7\\
 SO$_2$Cl             & SO2Cl                                  &  -66.40   &    25.80   &   92.2\\
 SF$_3$               & Sulfur trifluoride                     & -130.00   &   -38.53   &   91.5\\
 Al$_2$Br$_6$         & Al2Br6                                 & -224.00   &  -132.67   &   91.3\\
 POF$_3$              & Phosphorus oxyfluoride                 & -289.50   &  -199.54   &   90.0\\
 SiOI$_2$             & SiOI2                                  &  -99.40   &    -9.74   &   89.7\\
 SO$_2$Br             & SO2Br                                  &  -52.80   &    36.64   &   89.4\\
 CH$_5$PO$_3$         & Methylphosphonic acid                  & -240.50   &  -152.09   &   88.4\\
 CH$_3$POF$_2$        & Methylphosphonodifluoride              & -233.23   &  -147.22   &   86.0\\
 SiOBr$_2$            & SiOBr2                                 & -137.40   &   -51.46   &   85.9\\
 PSBr$_3$             & Phosphorus thiobromide                 &  -67.20   &    18.43   &   85.6\\
\hline
\end{tabular}
\end{center}
\end{table}

\begin{table}
\caption{\label{weam1} Worst $\Delta H_f$ Errors in AM1}
\compresstable
\begin{center}
\begin{tabular}{llrrr}
 Formula & Chemical Name & Exp. $\Delta H_f$ & Calc. $\Delta H_f$ & Diff.\\
 \hline
 O$_7$Cl$_2$          & Cl2O7                                  &   65.00   &   554.56   &  489.6\\
 F$_5$I               & Iodine pentafluoride                   & -200.84   &    94.43   &  295.3\\
 F$_7$I               & Iodine heptafluoride                   & -229.70   &    44.36   &  274.1\\
 O$_3$FCl             & Perchloryl fluoride                    &   -5.12   &   246.46   &  251.6\\
 P$_4$O$_6$           & Phosphorus trioxide                    & -529.23   &  -321.67   &  207.6\\
 Be$_3$O$_3$          & Be3O3                                  & -251.90   &   -45.39   &  206.5\\
 F$_5$Br              & Bromine pentafluoride                  & -102.50   &    81.11   &  183.6\\
 N                    & Nitrogen (++)                          & 1133.90   &   972.93   & -161.0\\
 F$_5$Cl              & Chlorine pentafluoride                 &  -54.00   &    90.39   &  144.4\\
 Be$_2$OF$_2$         & Be(OF)2                                & -287.90   &  -162.01   &  125.9\\
 PSBr$_3$             & Phosphorus thiobromide                 &  -67.20   &    44.74   &  111.9\\
 SI$_4$               & Sulfur tetraiodide                     &  120.20   &    17.48   & -102.7\\
 SBr$_4$              & Sulfur tetrabromide                    &   53.00   &   -35.51   &  -88.5\\
 BeH$_2$O$_2$         & Beryllium di-hydroxide                 & -161.80   &   -76.01   &   85.8\\
 SI$_3$               & Sulfur triiodide                       &  100.30   &    16.33   &  -84.0\\
 F$_3$Br              & Bromine trifluoride                    &  -61.10   &    21.47   &   82.6\\
 SCl$_4$              & Sulfur tetrachloride                   &   -0.70   &   -82.73   &  -82.0\\
 O$_2$Cl              & Chlorine dioxide                       &   25.00   &   105.87   &   80.9\\
 B$_2$H$_4$           & B2H4 D2d                               &   45.70   &   126.57   &   80.9\\
 SiOI$_2$             & SiOI2                                  &  -99.40   &   -18.69   &   80.7\\
 Be$_2$O              & Beryllium oxide                        &  -15.00   &    65.34   &   80.3\\
 SiOI                 & SiOI                                   &  -53.30   &    23.43   &   76.7\\
 SBr$_3$              & Sulfur tribromide                      &   50.20   &   -26.16   &  -76.4\\
 POI$_3$              & Phosphorus oxyiodide                   &  -39.70   &    36.16   &   75.9\\
 AlOF$_2$             & AlF2O                                  & -265.00   &  -191.91   &   73.1\\
 SCl$_3$              & Sulfur trichloride                     &    8.80   &   -60.10   &  -68.9\\
 SI$_5$               & Sulfur pentaiodide                     &  130.90   &    62.25   &  -68.7\\
 POBr$_3$             & Phosphorus oxybromide                  &  -97.00   &   -29.05   &   68.0\\
 F                    & Fluoride (-)                           &  -61.00   &     3.44   &   64.4\\
 Zn                   & Zinc (++)                              &  665.10   &   729.44   &   64.3\\
 PBr$_5$              & Phosphorus pentabromide                &  -11.00   &    52.99   &   64.0\\
 SF$_5$               & Sulfur pentafluoride                   & -217.10   &  -280.38   &  -63.3\\
 P                    & Phosphorus (++)                        &  775.10   &   712.09   &  -63.0\\
 SiOBr$_2$            & SiOBr2                                 & -137.40   &   -77.39   &   60.0\\
 COI                  & COI                                    &   63.50   &     4.72   &  -58.8\\
 PSCl$_3$             & Phosphorus thiochloride                &  -91.00   &   -32.31   &   58.7\\
 F$_3$Cl              & Chlorine trifluoride                   &  -38.00   &    20.19   &   58.2\\
 SI$_2$               & Sulfur diiodide                        &   81.90   &    25.68   &  -56.2\\
 BeF$_2$              & Beryllium difluoride                   & -189.70   &  -134.91   &   54.8\\
 SCl$_6$              & Sulfur hexachloride                    &  -19.80   &    34.76   &   54.6\\
 AlOF$_2$             & AlF2O (-)                              & -311.56   &  -257.16   &   54.4\\
 CI$_4$               & Carbon tetraiodide                     &  108.20   &    54.18   &  -54.0\\
 SOCl$_4$             & SOCl4                                  &  -55.70   &    -2.82   &   52.9\\
 CI$_3$               & Triiodomethyl                          &  117.30   &    64.59   &  -52.7\\
 CGe                  & Germanium carbide                      &  151.00   &   203.55   &   52.6\\
 C$_3$H$_6$N$_6$      & Melamine                               &   12.40   &    64.25   &   51.9\\
 BeBr$_2$             & Beryllium dibromide                    &  -54.80   &  -106.08   &  -51.3\\
 H                    & Hydrogen (+)                           &  365.72   &   314.91   &  -50.8\\
 SiOCl$_2$            & SiOCl2                                 & -167.70   &  -118.35   &   49.3\\
\hline
\end{tabular}
\end{center}
\end{table}

\begin{table}
\caption{\label{wepm3} Worst $\Delta H_f$ Errors in PM3}
\compresstable
\begin{center}
\begin{tabular}{llrrr}
 Formula & Chemical Name & Exp. $\Delta H_f$ & Calc. $\Delta H_f$ & Diff.\\
 \hline
 N                    & Nitrogen (++)                          & 1133.90   &   963.89   & -170.0\\
 P                    & Phosphorus (++)                        &  775.10   &   633.00   & -142.1\\
 O$_7$Cl$_2$          & Cl2O7                                  &   65.00   &   -26.27   &  -91.3\\
 S$_2$F$_1$$_0$       & S2F10                                  & -493.40   &  -410.71   &   82.7\\
 COI                  & COI                                    &   63.50   &    -2.93   &  -66.4\\
 Al$_2$O$_2$          & Al2O2  (+)                             &  135.60   &    72.50   &  -63.1\\
 Al                   & Al (+)                                 &  218.10   &   279.75   &   61.7\\
 PSBr$_3$             & Phosphorus thiobromide                 &  -67.20   &    -7.84   &   59.4\\
 C$_2$                & Carbon, dimer                          &  200.20   &   258.21   &   58.0\\
 AlOF$_2$             & AlF2O                                  & -265.00   &  -208.55   &   56.4\\
 SI$_4$               & Sulfur tetraiodide                     &  120.20   &    69.22   &  -51.0\\
 HAlO                 & AlOH  (-)                              &  -55.00   &  -105.68   &  -50.7\\
 AlN                  & Aluminum nitride                       &  125.00   &    75.45   &  -49.5\\
 SiOI$_2$             & SiOI2                                  &  -99.40   &   -50.23   &   49.2\\
 C$_4$                & Carbon, tetramer                       &  232.00   &   280.29   &   48.3\\
 SiOCl$_2$            & SiOCl2                                 & -167.70   &  -121.23   &   46.5\\
 SI$_3$               & Sulfur triiodide                       &  100.30   &    55.26   &  -45.0\\
 SiOBr$_2$            & SiOBr2                                 & -137.40   &   -94.09   &   43.3\\
 C$_3$InH$_9$         & Trimethyl indium                       &   40.80   &    -2.15   &  -42.9\\
 HPO$_3$              & HPO3                                   & -135.00   &  -177.27   &  -42.3\\
 OCl$_2$              & Chlorine monoxide                      &   25.00   &   -16.26   &  -41.3\\
 S$_2$F$_2$           & SSF2                                   &  -95.94   &   -56.11   &   39.8\\
 InO                  & Indium oxide                           &   92.50   &    52.97   &  -39.5\\
 SbN                  & Antimony nitride                       &   63.70   &   102.52   &   38.8\\
 Al$_2$               & Al2                                    &  116.40   &    80.69   &  -35.7\\
 Ga$_2$S              & Ga2S                                   &    5.00   &    40.20   &   35.2\\
 C$_8$H$_8$           & Cubane                                 &  148.70   &   113.66   &  -35.0\\
 MgN                  & Magnesium nitride                      &   69.00   &   103.99   &   35.0\\
 AlO                  & AlO   (+)                              &  237.30   &   202.36   &  -34.9\\
 C$_3$GaH$_9$         & Trimethyl gallium                      &  -10.80   &    23.82   &   34.6\\
 SBr$_4$              & Sulfur tetrabromide                    &   53.00   &    18.73   &  -34.3\\
 Pb$_2$               & Lead, dimer                            &   79.50   &    45.33   &  -34.2\\
 SBr$_3$              & Sulfur tribromide                      &   50.20   &    16.38   &  -33.8\\
 C$_6$AsH$_1$$_5$     & Triethylarsine                         &   13.40   &   -20.44   &  -33.8\\
 Al                   & Aluminum, atom                         &   79.49   &    46.03   &  -33.5\\
 C$_6$SbH$_1$$_5$     & Triethylstibine                        &   10.40   &   -23.08   &  -33.5\\
 MgHO                 & Magnesium hydroxide                    &  -22.00   &   -55.05   &  -33.0\\
 AlI                  & AlI                                    &   16.24   &    49.32   &   33.1\\
 C$_6$H$_1$$_5$PO$_4$ & Triethyl phosphate                     & -284.50   &  -251.56   &   32.9\\
 GaO                  & Gallium oxide                          &   66.80   &    35.33   &  -31.5\\
 CH$_5$PO$_3$         & Methylphosphonic acid                  & -240.50   &  -209.04   &   31.5\\
 POI$_3$              & Phosphorus oxyiodide                   &  -39.70   &    -8.28   &   31.4\\
 HCl$_2$              & Hydrogen dichloride (-)                & -142.00   &  -111.02   &   31.0\\
 COBr                 & COBr                                   &   20.50   &   -10.25   &  -30.8\\
 PSCl$_3$             & Phosphorus thiochloride                &  -91.00   &   -60.23   &   30.8\\
 CS                   & Carbon sulfide                         &   67.00   &    97.32   &   30.3\\
 SI$_2$               & Sulfur diiodide                        &   81.90   &    51.57   &  -30.3\\
 SCl$_6$              & Sulfur hexachloride                    &  -19.80   &    10.27   &   30.1\\
 AlCl$_2$             & AlCl2 (-)                              & -115.00   &  -144.99   &  -30.0\\
\hline
\end{tabular}
\end{center}
\end{table}

\clearpage
\begin{table}
\caption{\label{avehof}Average Signed and Unsigned Errors in $\Delta H_f$ by
 Compound Type}
\compresstable
\begin{center}
\begin{tabular}{lrrrrrrrrr}
\multicolumn{1}{c}{Elements} & \multicolumn{3}{c}{PM3} & \multicolumn{3}{c}{MNDO} &
\multicolumn{3}{c}{AM1} \\
\multicolumn{1}{c}{in compounds} &  No. & Unsigned & Signed & No. & Unsigned & Signed & No. & Unsigned & Signed \\
\hline\\
      H       &      2 &   12.765 &  -12.765 &  &   19.885 &  -19.165 &  &   27.995 &  -27.995 \\
      CH      &    118 &    5.375 &    1.220 & 116 &    7.368 &    0.522 & 116 &    5.772 &    1.008 \\
      C       &      3 &   38.977 &   38.977 &  &   37.407 &   37.407 &  &   26.770 &   26.770 \\
      HO      &      5 &   11.018 &    5.078 &  &   10.018 &    0.950 &  &    7.288 &    2.168 \\
      CO      &      3 &    5.767 &    4.660 &  &   13.527 &   12.760 &  &   14.223 &   14.223 \\
      CHO     &     61 &    4.539 &   -2.286 &  &    5.401 &   -2.576 &  &    6.024 &   -2.226 \\
      O       &      3 &    8.263 &    3.050 &  &   13.203 &   -3.617 &  &   17.327 &  -14.933 \\
      N       &      3 &   64.833 &  -48.507 &  &   57.473 &  -49.693 &  &   60.163 &  -47.150 \\
      HN      &      6 &    4.742 &    0.122 &  &    6.387 &   -0.720 &  &    5.728 &   -2.652 \\
      CN      &      3 &    9.707 &    9.707 &  &   15.893 &    0.953 &  &    7.683 &   -0.770 \\
      CHN     &     35 &    6.182 &   -0.825 &  &    7.258 &   -2.579 &  &    6.456 &    2.429 \\
      NO      &     10 &   10.365 &   -5.821 &  &   16.068 &    6.700 &  &   13.226 &   -2.438 \\
      CNO     &      2 &    9.005 &   -9.005 &  &   38.660 &   37.640 &  &   17.575 &   17.575 \\
      CHNO    &     45 &    5.742 &    0.582 &  &   36.746 &   35.758 &  &   14.945 &   13.195 \\
      HNO     &      2 &    4.925 &   -1.035 &  &   18.195 &   -3.695 &  &   13.040 &  -13.040 \\
      S       &      4 &    6.643 &    2.658 &  &   11.922 &   11.543 &  &    6.835 &   -0.750 \\
      HS      &      7 &    5.140 &   -0.837 &  &   15.104 &   14.930 &  &    5.380 &    1.326 \\
      CS      &      2 &   19.620 &   19.620 &  &   23.180 &   23.180 &  &   19.180 &    8.660 \\
      CHS     &     31 &    4.459 &    1.871 &  &    7.217 &   -6.547 &  &    4.784 &   -1.152 \\
      SO      &      3 &   15.077 &   -1.597 &  &   77.143 &   77.143 &  &   13.140 &    2.833 \\
      CSO     &      1 &   10.090 &   10.090 &  &   10.930 &   10.930 &  &    4.870 &    4.870 \\
      CHSO    &      9 &   10.690 &    5.068 &  &   95.592 &   95.592 &  &   12.962 &    4.033 \\
      HSO     &      1 &   12.430 &  -12.430 &  &  162.570 &  162.570 &  &   10.670 &  -10.670 \\
      CHNS    &      4 &   14.465 &    9.760 &  &   14.625 &    7.365 &  &    8.655 &   -0.460 \\
      CNS     &      1 &    3.990 &   -3.990 &  &   11.850 &  -11.850 &  &   18.210 &  -18.210 \\
      F       &      2 &   25.735 &    4.035 &  &   25.595 &   25.595 &  &   43.465 &   20.975 \\
      HF      &      1 &    2.390 &    2.390 &  &    5.400 &    5.400 &  &    9.140 &   -9.140 \\
      CF      &     13 &    9.562 &   -3.618 &  &   22.062 &    0.381 &  &   17.625 &   -6.375 \\
      CHF     &     27 &    6.261 &    3.570 &  &    6.564 &   -1.221 &  &    9.381 &   -6.841 \\
      OF      &      3 &    8.493 &   -1.913 &  &   12.593 &    9.673 &  &    4.670 &    2.337 \\
      HOF     &      1 &    8.350 &   -8.350 &  &    2.260 &    2.260 &  &    1.710 &   -1.710 \\
      COF     &      6 &   11.832 &   -8.112 &  &   12.318 &    9.748 &  &    9.737 &   -5.893 \\
      CHOF    &      4 &    7.295 &    7.295 &  &    8.962 &    8.962 &  &    7.258 &    5.777 \\
      CNF     &      2 &    2.710 &    0.590 &  &    8.060 &   -2.910 &  &    7.075 &   -7.075 \\
      NOF     &      4 &    8.412 &    4.092 &  &   30.710 &   26.160 &  &   12.732 &    7.323 \\
      SF      &     10 &   20.535 &    5.281 &  &  162.532 &  162.532 &  &   28.202 &  -22.926 \\
      SOF     &      6 &    7.083 &   -0.657 &  &  149.787 &  149.787 &  &   16.598 &  -15.825 \\
      NF      &      6 &    6.113 &    5.607 &  &   14.287 &  -14.287 &  &   10.005 &    1.678 \\
      Cl      &      3 &    5.417 &   -2.303 &  &    3.963 &   -3.157 &  &   10.803 &    1.357 \\
      HCl     &      2 &   16.305 &   16.305 &  &   27.020 &   27.020 &  &   25.050 &   22.540 \\
      CCl     &      7 &    5.297 &   -5.174 &  &    6.571 &   -2.677 &  &    8.793 &   -8.604 \\
      CHCl    &     17 &    3.782 &    3.203 &  &    2.952 &   -1.878 &  &    2.945 &   -1.555 \\
      HOCl    &      1 &   16.500 &  -16.500 &  &    2.120 &    2.120 &  &    3.980 &   -3.980 \\
      COCl    &      2 &    2.305 &    1.195 &  &    0.330 &   -0.330 &  &    2.840 &    2.400 \\
\hline
\end{tabular}
\end{center}
\end{table}

\begin{table}
\caption{\label{avehofb}Average Signed and Unsigned Errors in $\Delta H_f$ by
 Compound Type (contd.)}
\compresstable
\begin{center}
\begin{tabular}{lrrrrrrrrr}
\multicolumn{1}{c}{Elements} & \multicolumn{3}{c}{PM3} & \multicolumn{3}{c}{MNDO} &
\multicolumn{3}{c}{AM1} \\
\multicolumn{1}{c}{in compounds} &  No. & Unsigned & Signed & No. & Unsigned & Signed & No. & Unsigned & Signed \\
\hline\\
      CHOCl   &      1 &    7.630 &    7.630 &  &    2.600 &    2.600 &  &   10.270 &   10.270 \\
      OCl     &      3 &   52.027 &  -52.027 &  &  243.607 &  243.607 &  &  191.990 &  188.297 \\
      CNCl    &      1 &    0.030 &   -0.030 &  &    1.640 &    1.640 &  &    6.970 &   -6.970 \\
      NOCl    &      2 &   11.900 &  -11.900 &  &   15.480 &   -1.110 &  &    9.690 &    1.980 \\
      SCl     &      7 &   14.544 &   -5.641 &  &   45.359 &    6.561 &  &   42.119 &  -26.530 \\
      SOCl    &      6 &    9.723 &    5.157 &  &   74.258 &   74.258 &  &   21.852 &    3.178 \\
      FCl     &      3 &    8.510 &    2.090 &  &  131.903 &  131.903 &  &   68.047 &   68.047 \\
      HFCl    &      1 &    5.310 &    5.310 &  &   15.820 &   15.820 &  &   23.610 &   23.610 \\
      CHFCl   &      3 &    5.530 &    5.530 &  &    2.693 &   -1.920 &  &    2.177 &    0.343 \\
      COFCl   &      1 &    8.370 &    8.370 &  &    9.730 &    9.730 &  &    9.980 &    9.980 \\
      OFCl    &      1 &   19.630 &   19.630 &  &  328.340 &  328.340 &  &  251.580 &  251.580 \\
      CFCl    &      3 &    1.047 &    0.993 &  &    6.867 &    6.867 &  &    8.340 &    8.340 \\
      Br      &      4 &    3.978 &    0.768 &  &    9.103 &    4.562 &  &   13.075 &    2.870 \\
      HBr     &      2 &   13.815 &   13.815 &  &   14.560 &   14.560 &  &    7.065 &   -7.065 \\
      CBr     &      4 &    8.950 &    7.605 &  &   16.637 &  -11.942 &  &   15.900 &  -12.610 \\
      CHBr    &      9 &    6.326 &    6.326 &  &    1.870 &   -1.250 &  &    2.129 &    1.956 \\
      OBr     &      1 &    9.280 &   -9.280 &  &    5.270 &    5.270 &  &    5.600 &    5.600 \\
      HOBr    &      1 &   13.950 &  -13.950 &  &    2.740 &   -2.740 &  &    4.730 &   -4.730 \\
      COBr    &      2 &   18.005 &  -18.005 &  &   20.170 &  -20.170 &  &   13.915 &  -11.695 \\
      CHOBr   &      2 &    2.525 &    2.525 &  &    1.455 &    1.455 &  &   11.660 &   11.660 \\
      CNBr    &      1 &   10.320 &   10.320 &  &    3.310 &   -3.310 &  &   10.850 &  -10.850 \\
      NOBr    &      1 &   13.040 &  -13.040 &  &   17.790 &  -17.790 &  &    1.640 &    1.640 \\
      SBr     &      7 &   21.330 &   -3.764 &  &   37.234 &    0.994 &  &   38.696 &  -29.707 \\
      SOBr    &      6 &   10.348 &   -2.612 &  &   50.580 &   50.580 &  &   19.877 &   -2.667 \\
      FBr     &      3 &   15.953 &   11.113 &  &   99.827 &   99.827 &  &   90.987 &   90.987 \\
      CFBr    &      2 &    4.065 &   -4.065 &  &   12.585 &   12.585 &  &   17.075 &   17.075 \\
      ClBr    &      2 &    9.530 &   -9.530 &  &    9.395 &   -3.595 &  &   13.870 &  -13.870 \\
      CClBr   &      1 &    4.790 &   -4.790 &  &    5.830 &   -5.830 &  &    5.410 &   -5.410 \\
      I       &      2 &   11.970 &   -6.150 &  &   23.190 &   23.190 &  &   24.590 &   24.590 \\
      HI      &      1 &   22.500 &   22.500 &  &    9.440 &    9.440 &  &    1.630 &    1.630 \\
      CI      &      4 &    4.285 &   -3.375 &  &   38.013 &  -34.458 &  &   28.520 &  -24.845 \\
      CHI     &     23 &    6.925 &    6.925 &  &    6.618 &   -4.881 &  &    3.547 &   -1.143 \\
      OI      &      1 &   10.900 &  -10.900 &  &    4.820 &    4.820 &  &    4.880 &   -4.880 \\
      COI     &      2 &   38.755 &  -38.755 &  &   38.810 &  -38.810 &  &   30.925 &  -30.925 \\
      CHOI    &      3 &    3.577 &    3.577 &  &    2.440 &    1.353 &  &    7.020 &    7.020 \\
      CNI     &      1 &    9.820 &    9.820 &  &   14.140 &  -14.140 &  &   11.140 &  -11.140 \\
      NOI     &      1 &    8.640 &   -8.640 &  &    5.900 &   -5.900 &  &    5.420 &    5.420 \\
      SI      &      7 &   23.399 &  -21.084 &  &   37.767 &  -23.861 &  &   56.807 &  -56.807 \\
      SOI     &      6 &   13.080 &  -10.530 &  &   28.090 &   22.373 &  &   17.052 &  -13.598 \\
      FI      &      3 &    6.237 &    4.837 &  &  226.633 &  226.633 &  &  194.290 &  194.290 \\
      CFI     &      1 &    2.520 &    2.520 &  &   12.270 &   12.270 &  &    7.820 &    7.820 \\
      ClI     &      1 &    6.210 &    6.210 &  &   11.460 &  -11.460 &  &    9.180 &   -9.180 \\
      BrI     &      1 &    5.880 &    5.880 &  &    2.540 &   -2.540 &  &    3.770 &   -3.770 \\
      Be      &      2 &    7.795 &   -7.795 &  &   13.445 &  -13.445 &  &   13.445 &  -13.445 \\
      BeH     &      3 &   16.983 &    1.323 &  &   25.870 &  -25.870 &  &   21.647 &  -21.647 \\
      BeO     &      3 &    9.467 &    4.160 &  &   39.423 &   39.423 &  &  108.250 &  108.250 \\
      BeHO    &      2 &   10.680 &   10.680 &  &   13.320 &   13.320 &  &   61.625 &   61.625 \\
      BeF     &      2 &    5.320 &   -2.950 &  &    6.570 &   -6.570 &  &   36.475 &   36.475 \\
      BeCl    &      2 &    4.600 &   -4.600 &  &    7.280 &   -7.280 &  &   13.815 &  -13.815 \\
\hline
\end{tabular}
\end{center}
\end{table}

\begin{table}
\caption{\label{avehofc}Average Signed and Unsigned Errors in $\Delta H_f$ by
 Compound Type (contd.)}
\compresstable
\begin{center}
\begin{tabular}{lrrrrrrrrr}
\multicolumn{1}{c}{Elements} & \multicolumn{3}{c}{PM3} & \multicolumn{3}{c}{MNDO} &
\multicolumn{3}{c}{AM1} \\
\multicolumn{1}{c}{in compounds} &  No. & Unsigned & Signed & No. & Unsigned & Signed & No. & Unsigned & Signed \\
\hline\\
      BeBr    &      2 &    3.755 &   -2.425 &  &    5.960 &    5.810 &  &   39.810 &  -39.810 \\
      BeI     &      2 &    5.110 &   -4.230 &  &    4.965 &   -3.655 &  &    8.545 &   -8.545 \\
      BeOF    &      1 &    9.430 &    9.430 &  &    1.800 &    1.800 &  &  125.890 &  125.890 \\
  Mg        &   4 &    3.050 &   -0.055 &     &          &          &     &          &          \\
  MgH       &   1 &    1.310 &    1.310 &     &          &          &     &          &          \\
  CMgH      &   1 &    1.130 &   -1.130 &     &          &          &     &          &          \\
  MgO       &   1 &    1.950 &   -1.950 &     &          &          &     &          &          \\
  MgHO      &   2 &   20.645 &  -12.405 &     &          &          &     &          &          \\
  MgN       &   1 &   34.990 &   34.990 &     &          &          &     &          &          \\
  MgS       &   1 &   19.640 &  -19.640 &     &          &          &     &          &          \\
  MgF       &   3 &    9.757 &   -0.030 &     &          &          &     &          &          \\
  MgCl      &   3 &   12.677 &   -5.077 &     &          &          &     &          &          \\
  MgBr      &   3 &    6.733 &   -2.167 &     &          &          &     &          &          \\
  MgI       &   1 &    3.480 &   -3.480 &     &          &          &     &          &          \\
      Al      &      3 &   43.607 &   -2.507 &  &   13.057 &   -3.097 &  &    2.357 &   -0.823 \\
      HAl     &      1 &    8.060 &    8.060 &  &   16.030 &  -16.030 &  &    9.110 &   -9.110 \\
      CHAl    &      2 &    9.985 &    9.985 &  &   17.960 &  -17.960 &  &    6.850 &   -6.850 \\
      AlO     &      9 &   19.863 &  -11.206 &  &   25.150 &  -16.192 &  &   10.878 &    4.160 \\
      HAlO    &      5 &   16.366 &   -3.906 &  &   14.070 &   -2.030 &  &   17.100 &   11.544 \\
      AlN     &      1 &   49.550 &  -49.550 &  &   15.910 &   15.910 &  &    7.090 &   -7.090 \\
      AlS     &      1 &   25.700 &  -25.700 &  &    4.520 &   -4.520 &  &    8.260 &   -8.260 \\
      AlF     &      8 &    6.566 &    1.056 &  &   16.358 &   -5.665 &  &   16.690 &   -2.295 \\
      AlOF    &      3 &   32.110 &   32.110 &  &   47.183 &   47.183 &  &   52.113 &   52.113 \\
      AlCl    &      7 &   10.541 &   -0.890 &  &   12.701 &   -7.470 &  &    9.584 &   -9.584 \\
      AlOCl   &      1 &   10.780 &   10.780 &  &   14.510 &   14.510 &  &   16.760 &   16.760 \\
      AlFCl   &      4 &    6.285 &    6.285 &  &    3.872 &   -3.872 &  &    3.425 &   -0.970 \\
      AlBr    &      2 &    6.585 &    5.665 &  &   64.565 &   64.565 &  &   13.305 &  -11.665 \\
      AlI     &      3 &   13.247 &   13.007 &  &   65.703 &   65.703 &  &   12.740 &   12.740 \\
  Ga        &   6 &    9.708 &   -3.565 &     &          &          &     &          &          \\
  GaH       &   1 &   14.840 &   14.840 &     &          &          &     &          &          \\
  CGaH      &   3 &   24.837 &   -1.757 &     &          &          &     &          &          \\
  GaO       &   1 &   31.470 &  -31.470 &     &          &          &     &          &          \\
  GaHO      &   1 &    4.600 &   -4.600 &     &          &          &     &          &          \\
  GaF       &   1 &   14.040 &   14.040 &     &          &          &     &          &          \\
  GaCl      &   3 &   16.087 &    2.147 &     &          &          &     &          &          \\
  GaHNCl    &   1 &   21.960 &   21.960 &     &          &          &     &          &          \\
  GaBr      &   1 &   19.860 &  -19.860 &     &          &          &     &          &          \\
  GaI       &   2 &    9.890 &   -9.890 &     &          &          &     &          &          \\
  GaS       &   1 &   35.200 &   35.200 &     &          &          &     &          &          \\
  In        &   5 &   11.656 &    5.184 &     &          &          &     &          &          \\
  InH       &   1 &    0.920 &    0.920 &     &          &          &     &          &          \\
  CInH      &   1 &   42.950 &  -42.950 &     &          &          &     &          &          \\
  InO       &   1 &   39.530 &  -39.530 &     &          &          &     &          &          \\
  InHO      &   1 &    4.800 &   -4.800 &     &          &          &     &          &          \\
  InS       &   2 &    3.835 &    1.175 &     &          &          &     &          &          \\
  InF       &   1 &   13.740 &   13.740 &     &          &          &     &          &          \\
  InCl      &   4 &   10.530 &    0.115 &     &          &          &     &          &          \\
\hline
\end{tabular}
\end{center}
\end{table}

\begin{table}
\caption{\label{avehofd}Average Signed and Unsigned Errors in $\Delta H_f$ by
 Compound Type (contd.)}
\compresstable
\begin{center}
\begin{tabular}{lrrrrrrrrr}
\multicolumn{1}{c}{Elements} & \multicolumn{3}{c}{PM3} & \multicolumn{3}{c}{MNDO} &
\multicolumn{3}{c}{AM1} \\
\multicolumn{1}{c}{in compounds} &  No. & Unsigned & Signed & No. & Unsigned & Signed & No. & Unsigned & Signed \\
\hline\\
  InBr      &   2 &    5.700 &    1.790 &     &          &          &     &          &          \\
  InI       &   2 &    9.785 &    0.785 &     &          &          &     &          &          \\
  Tl        &   4 &    5.697 &    4.932 &     &          &          &     &          &          \\
  TlCl      &   3 &    2.043 &    2.043 &     &          &          &     &          &          \\
  TlBr      &   2 &   10.810 &    0.670 &     &          &          &     &          &          \\
  TlI       &   1 &   14.370 &  -14.370 &     &          &          &     &          &          \\
      Si      &      3 &    1.910 &   -1.537 &  &   31.283 &   31.283 &  &   11.360 &   10.427 \\
      HSi     &      9 &    8.008 &   -2.457 &  &   12.311 &   -1.613 &  &   10.784 &   -8.509 \\
      CHSi    &     14 &    7.901 &   -1.279 &  &   10.875 &   -7.991 &  &   10.564 &   -4.304 \\
      SiO     &      2 &    9.030 &   -9.030 &  &   25.660 &   25.660 &  &   13.665 &   13.665 \\
      CHSiO   &      2 &    2.630 &    2.630 &  &    5.675 &   -5.675 &  &   13.440 &   13.440 \\
      SiF     &      5 &   11.780 &  -10.688 &  &   18.246 &   -4.974 &  &   14.322 &  -11.290 \\
      HSiF    &      3 &    9.243 &    9.243 &  &    5.803 &   -4.530 &  &    4.973 &    4.100 \\
      CHSiF   &      1 &   25.020 &   25.020 &  &   33.750 &   33.750 &  &   24.070 &   24.070 \\
      SiOF    &      2 &   15.090 &   15.090 &  &   49.210 &   49.210 &  &   20.190 &   20.190 \\
      SiCl    &      6 &   12.740 &   -7.500 &  &   15.953 &   -2.107 &  &   18.155 &  -17.528 \\
      HSiCl   &      3 &    5.713 &    5.713 &  &    7.050 &   -6.103 &  &    5.533 &   -5.533 \\
      CHSiCl  &      5 &    3.872 &    3.200 &  &    1.956 &   -1.732 &  &    4.112 &   -4.112 \\
      SiOCl   &      2 &   37.070 &   37.070 &  &   58.860 &   58.860 &  &   38.960 &   38.960 \\
      SiFCl   &      1 &    0.080 &    0.080 &  &   17.460 &   17.460 &  &    7.960 &    7.960 \\
      SiBr    &      5 &   11.658 &   -4.418 &  &   47.098 &   47.098 &  &   11.994 &    6.946 \\
      HSiBr   &      3 &    3.923 &   -3.923 &  &   14.653 &   12.820 &  &    3.573 &   -2.060 \\
      CHSiBr  &      1 &    1.220 &    1.220 &  &    7.600 &    7.600 &  &    3.510 &    3.510 \\
      SiOBr   &      2 &   33.775 &   33.775 &  &   64.355 &   64.355 &  &   45.535 &   45.535 \\
      SiI     &      4 &    7.035 &    2.325 &  &   47.308 &   47.308 &  &   19.867 &   19.867 \\
      HSiI    &      3 &    5.223 &    5.223 &  &   31.810 &   31.810 &  &   13.717 &   13.717 \\
      SiOI    &      2 &   34.380 &   14.790 &  &   72.915 &   72.915 &  &   78.720 &   78.720 \\
      CHSiN   &      1 &    7.020 &   -7.020 &  &    9.800 &   -9.800 &  &    4.170 &   -4.170 \\
      Ge      &      2 &   22.760 &  -22.760 &  &   38.790 &   38.790 &  &   15.170 &  -15.170 \\
      GeH     &      3 &    7.113 &    7.113 &  &   12.460 &    6.960 &  &    6.707 &   -1.613 \\
      CGe     &      1 &   27.610 &   27.610 &  &   73.430 &   73.430 &  &   52.550 &   52.550 \\
      CGeH    &      8 &   10.894 &   -3.679 &  &   15.488 &   -1.718 &  &   15.783 &   -2.757 \\
      GeO     &      1 &    5.120 &    5.120 &  &   11.850 &   11.850 &  &   20.240 &   20.240 \\
      CGeHO   &      3 &    7.850 &   -4.903 &  &    4.210 &    1.657 &  &   13.433 &   -3.007 \\
      CGeHN   &      1 &   12.420 &  -12.420 &  &    3.990 &   -3.990 &  &    1.130 &    1.130 \\
      GeF     &      3 &    7.977 &    4.077 &  &    7.130 &   -0.370 &  &   11.073 &    3.247 \\
      GeCl    &      4 &   13.197 &   -8.098 &  &   17.237 &    3.772 &  &   10.980 &   -8.845 \\
      CGeHCl  &      1 &    3.650 &    3.650 &  &    1.700 &    1.700 &  &    5.110 &    5.110 \\
      GeBr    &      3 &    9.987 &    0.640 &  &   22.470 &   13.550 &  &    6.930 &   -1.777 \\
      CGeHBr  &      1 &    5.270 &    5.270 &  &   14.820 &   14.820 &  &   12.030 &   12.030 \\
      GeI     &      2 &   11.845 &   11.845 &  &   59.935 &   59.935 &  &   19.435 &   19.435 \\
  Sn        &   1 &    0.000 &    0.000 &     &    0.000 &    0.000 &     &          &          \\
  SnH       &   2 &    8.835 &    5.415 &     &    4.555 &   -3.845 &     &          &          \\
  CSnH      &  10 &    9.040 &   -6.168 &     &   12.578 &   -9.052 &     &          &          \\
  SnO       &   1 &    0.920 &    0.920 &     &    9.610 &    9.610 &     &          &          \\
  CSnHN     &   1 &    4.900 &   -4.900 &     &    7.030 &    7.030 &     &          &          \\
  SnS       &   1 &    9.230 &    9.230 &     &    1.420 &   -1.420 &     &          &          \\
\hline
\end{tabular}
\end{center}
\end{table}

\begin{table}
\caption{\label{avehofe}Average Signed and Unsigned Errors in $\Delta H_f$ by
 Compound Type (contd.)}
\compresstable
\begin{center}
\begin{tabular}{lrrrrrrrrr}
\multicolumn{1}{c}{Elements} & \multicolumn{3}{c}{PM3} & \multicolumn{3}{c}{MNDO} &
\multicolumn{3}{c}{AM1} \\
\multicolumn{1}{c}{in compounds} &  No. & Unsigned & Signed & No. & Unsigned & Signed & No. & Unsigned & Signed \\
\hline\\
  SnF       &   2 &    6.520 &   -1.960 &     &    6.395 &   -5.035 &     &          &          \\
  SnCl      &   3 &    6.927 &    1.307 &     &   19.990 &  -11.543 &     &          &          \\
  CSnHCl    &   5 &   12.102 &   10.402 &     &    7.510 &    3.398 &     &          &          \\
  SnBr      &   3 &   13.710 &   -3.757 &     &   20.573 &    9.207 &     &          &          \\
  CSnHBr    &   2 &    6.270 &   -6.270 &     &    4.645 &    4.645 &     &          &          \\
  SnI       &   2 &   11.900 &  -11.900 &     &    1.915 &    1.915 &     &          &          \\
  CSnHI     &   1 &    0.830 &   -0.830 &     &   10.860 &   10.860 &     &          &          \\
  CGeSnH    &   1 &    7.480 &    7.480 &     &   10.050 &   10.050 &     &          &          \\
  Pb        &   3 &   11.753 &  -11.753 &     &    2.530 &   -2.530 &     &          &          \\
  PbH       &   2 &    1.605 &   -1.605 &     &    9.580 &   -6.190 &     &          &          \\
  CPbH      &   6 &   10.560 &    3.813 &     &   11.782 &    0.872 &     &          &          \\
  PbO       &   1 &    4.510 &    4.510 &     &   12.310 &   12.310 &     &          &          \\
  PbS       &   1 &    8.400 &    8.400 &     &    0.460 &    0.460 &     &          &          \\
  PbF       &   2 &    8.065 &    6.285 &     &    6.615 &    3.215 &     &          &          \\
  PbCl      &   5 &    6.108 &   -2.436 &     &   19.620 &  -11.944 &     &          &          \\
  PbBr      &   2 &    7.875 &   -7.875 &     &    9.925 &   -9.925 &     &          &          \\
  PbI       &   2 &   14.375 &  -14.375 &     &    4.920 &    4.920 &     &          &          \\
      P       &      3 &   54.893 &  -47.047 &  &   23.133 &  -19.727 &  &   33.507 &  -20.653 \\
      HP      &      4 &    5.857 &    0.557 &  &    9.925 &    5.985 &  &    7.393 &    7.393 \\
      CP      &      1 &   12.090 &   12.090 &  &   22.120 &   22.120 &  &   23.580 &   23.580 \\
      CHP     &      7 &    7.393 &   -3.799 &  &   19.817 &  -17.989 &  &    7.740 &    2.511 \\
      PO      &      5 &   13.260 &   -6.000 &  &   72.368 &   65.104 &  &   50.902 &   37.394 \\
      CHPO    &      6 &   23.477 &   14.377 &  &   60.502 &   40.432 &  &    9.535 &    1.468 \\
      HPO     &      1 &   42.270 &  -42.270 &  &   29.840 &   29.840 &  &   33.620 &  -33.620 \\
      NP      &      1 &    7.860 &    7.860 &  &    8.870 &    8.870 &  &    7.460 &    7.460 \\
      PF      &      7 &   16.131 &  -12.180 &  &   55.154 &   49.634 &  &    9.651 &    5.583 \\
      POF     &      3 &   15.017 &    1.430 &  &   51.350 &   49.737 &  &   14.397 &   12.190 \\
      CHPOF   &      9 &    5.136 &   -2.749 &  &   61.624 &   61.624 &  &   15.476 &  -14.242 \\
      PCl     &      5 &   17.888 &  -16.560 &  &   26.640 &   -4.672 &  &   18.896 &  -13.188 \\
      POCl    &      3 &   11.387 &   -1.353 &  &   32.397 &   25.657 &  &   13.340 &   13.067 \\
      CHPOCl  &      1 &    4.790 &   -4.790 &  &   48.660 &   48.660 &  &    9.270 &    9.270 \\
      PSCl    &      1 &   30.770 &   30.770 &  &   62.610 &   62.610 &  &   58.690 &   58.690 \\
      PBr     &      5 &   10.678 &   -7.986 &  &   19.434 &   -2.338 &  &   22.112 &    8.096 \\
      POBr    &      3 &   13.797 &    7.330 &  &   35.370 &   32.903 &  &   39.183 &   39.183 \\
      PSBr    &      1 &   59.360 &   59.360 &  &   85.630 &   85.630 &  &  111.940 &  111.940 \\
      PI      &      5 &    6.446 &   -3.242 &  &   11.790 &  -11.318 &  &   17.218 &  -16.930 \\
      POI     &      3 &   15.517 &    8.697 &  &   33.197 &   33.197 &  &   42.190 &   42.190 \\
  As        &   5 &    5.012 &   -2.380 &     &          &          &     &          &          \\
  AsH       &   1 &    3.210 &   -3.210 &     &          &          &     &          &          \\
  CAsH      &   3 &   20.550 &  -13.810 &     &          &          &     &          &          \\
  CAsHO     &   3 &    4.097 &   -1.397 &     &          &          &     &          &          \\
  AsN       &   1 &   23.230 &   23.230 &     &          &          &     &          &          \\
  AsS       &   1 &    2.100 &   -2.100 &     &          &          &     &          &          \\
  AsF       &   2 &    5.085 &    2.685 &     &          &          &     &          &          \\
  AsCl      &   2 &    6.445 &   -1.265 &     &          &          &     &          &          \\
  AsBr      &   1 &   11.300 &   11.300 &     &          &          &     &          &          \\
  AsI       &   1 &   20.380 &   20.380 &     &          &          &     &          &          \\
\hline
\end{tabular}
\end{center}
\end{table}

\begin{table}
\caption{\label{avehoff}Average Signed and Unsigned Errors in $\Delta H_f$ by
 Compound Type (contd.)}
\compresstable
\begin{center}
\begin{tabular}{lrrrrrrrrr}
\multicolumn{1}{c}{Elements} & \multicolumn{3}{c}{PM3} & \multicolumn{3}{c}{MNDO} &
\multicolumn{3}{c}{AM1} \\
\multicolumn{1}{c}{in compounds} &  No. & Unsigned & Signed & No. & Unsigned & Signed & No. & Unsigned & Signed \\
\hline\\
  AsO       &     1 &    1.110 &    1.110 &     &          &          &     &          &          \\
      Zn      &      3 &    5.487 &   -5.487 &  &   21.647 &   15.093 &  &   23.183 &   19.710 \\
      ZnH     &      1 &    2.480 &    2.480 &  &   11.240 &  -11.240 &  &    3.450 &   -3.450 \\
      CZnH    &      6 &    6.442 &    2.768 &  &   11.097 &   11.097 &  &    6.045 &    6.045 \\
      ZnS     &      1 &    3.180 &    3.180 &  &   30.710 &   30.710 &  &   30.040 &   30.040 \\
      ZnCl    &      2 &    5.430 &    5.430 &  &   21.735 &   21.735 &  &   10.370 &   10.370 \\
  Cd        &   1 &    0.000 &    0.000 &     &          &          &     &          &          \\
  CCdH      &   3 &    3.400 &    0.027 &     &          &          &     &          &          \\
      Hg      &      5 &    6.276 &   -0.396 &  &   15.554 &  -13.866 &  &   12.496 &  -11.680 \\
      HgH     &      1 &    9.140 &   -9.140 &  &   19.420 &  -19.420 &  &    1.270 &   -1.270 \\
      CHgH    &      9 &    4.232 &    1.359 &  &    8.867 &    0.782 &  &    4.314 &    2.794 \\
      CHgN    &      1 &    2.080 &    2.080 &  &   21.940 &  -21.940 &  &    6.770 &   -6.770 \\
      HgF     &      2 &   13.695 &    5.455 &  &   14.400 &  -11.070 &  &   13.845 &    2.165 \\
      HgCl    &      2 &    8.760 &   -6.460 &  &   14.185 &  -14.185 &  &   16.320 &  -16.320 \\
      CHgHCl  &      2 &    7.125 &    7.125 &  &    3.635 &   -3.635 &  &    0.720 &   -0.720 \\
      HgBr    &      2 &   10.365 &  -10.365 &  &   16.805 &    6.685 &  &   25.975 &  -25.975 \\
      CHgHBr  &      2 &    1.945 &    1.375 &  &    9.155 &    9.155 &  &    7.175 &   -7.175 \\
      CHgHI   &      2 &    2.370 &    1.910 &  &    9.690 &    9.690 &  &   14.870 &   14.870 \\
      HgI     &      1 &    0.130 &   -0.130 &  &   25.210 &   25.210 &  &   23.200 &   23.200 \\
  Sb        &   5 &   10.850 &   -0.814 &     &          &          &     &          &          \\
  SbH       &   1 &   23.370 &   23.370 &     &          &          &     &          &          \\
  CSbH      &   3 &   15.770 &  -15.770 &     &          &          &     &          &          \\
  SbO       &   1 &   11.770 &  -11.770 &     &          &          &     &          &          \\
  SbN       &   1 &   38.820 &   38.820 &     &          &          &     &          &          \\
  SbF       &   1 &   10.620 &   10.620 &     &          &          &     &          &          \\
  SbCl      &   4 &   10.110 &    2.200 &     &          &          &     &          &          \\
  SbOCl     &   1 &   18.390 &   18.390 &     &          &          &     &          &          \\
  SbBr      &   1 &    9.990 &    9.990 &     &          &          &     &          &          \\
  InSb      &   2 &    9.325 &   -9.325 &     &          &          &     &          &          \\
  Se        &   1 &    0.000 &    0.000 &     &          &          &     &          &          \\
  SeH       &   1 &   15.620 &   15.620 &     &          &          &     &          &          \\
  CSeH      &   1 &   23.690 &  -23.690 &     &          &          &     &          &          \\
  SeO       &   2 &   14.880 &   14.880 &     &          &          &     &          &          \\
  SeOF      &   1 &   12.650 &  -12.650 &     &          &          &     &          &          \\
  SeF       &   2 &    7.860 &   -7.630 &     &          &          &     &          &          \\
  CSe       &   1 &    3.360 &    3.360 &     &          &          &     &          &          \\
  AsTe      &   1 &    6.390 &    6.390 &     &          &          &     &          &          \\
  ZnTe      &   1 &   12.740 &  -12.740 &     &          &          &     &          &          \\
  Te        &   1 &   13.310 &   13.310 &     &          &          &     &          &          \\
  TeO       &   1 &   18.470 &  -18.470 &     &          &          &     &          &          \\
  TeF       &   1 &   10.580 &   10.580 &     &          &          &     &          &          \\
  Bi        &   4 &    9.690 &    1.055 &     &          &          &     &          &          \\
  CBiH      &   3 &   10.690 &  -10.690 &     &          &          &     &          &          \\
  BiF       &   1 &   14.100 &   14.100 &     &          &          &     &          &          \\
  BiCl      &   2 &   16.315 &   16.315 &     &          &          &     &          &          \\
  BiSe      &   1 &   12.550 &   12.550 &     &          &          &     &          &          \\
  CLiH      &     &          &          &  2   &   24.470 &  -24.470 &     &          &          \\
\hline
\end{tabular}
\end{center}
\end{table}

\begin{table}
\caption{\label{avehofg}Average Signed and Unsigned Errors in $\Delta H_f$ by
 Compound Type (contd.)}
\compresstable
\begin{center}
\begin{tabular}{lrrrrrrrrr}
\multicolumn{1}{c}{Elements} & \multicolumn{3}{c}{PM3} & \multicolumn{3}{c}{MNDO} &
\multicolumn{3}{c}{AM1} \\
\multicolumn{1}{c}{in compounds} &  No. & Unsigned & Signed & No. & Unsigned & Signed & No. & Unsigned & Signed \\
\hline\\
  BH        &     &          &          &  7  &   28.837 &    3.649 &     &   33.640 &   12.311 \\
  CBH       &     &          &          &  4  &    9.922 &   -9.922 &     &    5.205 &   -2.965 \\
  BO        &     &          &          &  5  &   14.142 &  -11.806 &     &   12.074 &  -11.682 \\
  BHO       &     &          &          &  9 &   15.498 &   13.020 &     &    9.312 &    0.948 \\
  CBHO      &     &          &          &  5 &   11.010 &    1.794 &     &    7.962 &    6.702 \\
  BN        &     &          &          &   1 &   25.290 &   25.290 &     &   18.010 &   18.010 \\
  CBHN      &     &          &          &   5 &   20.188 &   20.188 &     &   20.858 &   20.674 \\
  BS        &     &          &          &   3 &   21.453 &  -12.140 &     &   12.273 &    0.713 \\
  CBHS      &     &          &          &   2 &    4.620 &   -3.830 &     &    4.950 &    4.950 \\
  BF        &     &          &          &   3 &   16.427 &   -6.187 &     &   12.660 &  -12.660 \\
  BOF       &     &          &          &   3 &   19.023 &   19.023 &     &    5.063 &    2.230 \\
  BHOF      &     &          &          &   4 &   18.845 &   18.845 &     &    5.213 &    0.653 \\
  CBHF      &     &          &          &   2 &    5.965 &    5.965 &     &    6.430 &    6.430 \\
  CBHOF     &     &          &          &   2 &   18.405 &   18.405 &     &    2.325 &   -2.165 \\
  CBHNF     &     &          &          &   1 &   40.180 &   40.180 &     &   32.120 &   32.120 \\
  BCl       &     &          &          &   3 &   14.110 &   -3.543 &     &    8.630 &   -8.630 \\
  BOCl      &     &          &          &   2 &   27.325 &   10.285 &     &   24.580 &    0.290 \\
  CBHOCl    &     &          &          &   1 &   12.290 &   12.290 &     &   12.750 &   12.750 \\
  BFCl      &     &          &          &   2 &   10.530 &   10.530 &     &    4.635 &    4.635 \\
  BBr       &     &          &          &   2 &   19.350 &   -3.870 &     &   10.325 &  -10.325 \\
  BFBr      &     &          &          &   1 &   12.820 &   12.820 &     &    5.910 &    5.910 \\
  BI        &     &          &          &   1 &    8.390 &   -8.390 &     &    2.410 &    2.410 \\
  B         &     &          &          &   1 &    5.230 &   -5.230 &     &    8.750 &    8.750 \\
  BHN       &     &          &          &   1 &    8.920 &   -8.920 &     &   19.970 &  -19.970 \\
  BHOCl     &     &          &          &   2 &   13.655 &   13.655 &     &    0.670 &   -0.520 \\
  BHNCl     &     &          &          &   1 &   12.660 &   12.660 &     &   20.750 &  -20.750 \\
  CBHP      &     &          &          &   1 &   20.220 &  -20.220 &     &    4.150 &   -4.150 \\
  BHPF      &     &          &          &   1 &   13.160 &  -13.160 &     &   30.500 &  -30.500 \\
  CBHPF     &     &          &          &   1 &    2.340 &    2.340 &     &   30.850 &   30.850 \\
\hline
\end{tabular}
\end{center}
\end{table}

\clearpage
\begin{table}
\caption{\label{elehof} Average Errors in $\Delta H_f$ by Element} 
\compresstable
\begin{center}
\begin{tabular}{lrrrrrrrrr}
Element & \multicolumn{3}{c}{PM3} & \multicolumn{3}{c}{MNDO} & \multicolumn{3}{c}{AM1} \\
    & No. & Signed &    RMS    & No. & Signed &    RMS    & No. & Signed &    RMS   \\ \hline
  H & 613 &    6.951 &    9.711 & 630 &   13.854 &   24.420 & 598 &    8.498 &   12.746 \\
 Li &     &          &          &   2 &   24.470 &   24.495 &     &          &          \\
 Be &  19 &    8.595 &   11.450 &  19 &   15.829 &   23.996 &  19 &   45.422 &   66.926 \\
  B &     &          &          &  76 &   16.544 &   21.183 &  76 &   12.833 &   18.902 \\
  C & 595 &    6.962 &   10.153 & 597 &   14.016 &   24.367 & 569 &    8.578 &   12.354 \\
  N & 140 &    9.032 &   18.044 & 145 &   21.136 &   30.283 & 144 &   12.396 &   19.821 \\
  O & 305 &   10.424 &   15.457 & 321 &   32.881 &   62.721 & 319 &   17.858 &   40.932 \\
  F & 177 &    9.673 &   13.600 & 184 &   39.491 &   83.527 & 180 &   20.997 &   45.279 \\
 Mg &  21 &    9.690 &   13.949 &     &          &          &     &          &          \\
 Al &  50 &   16.124 &   23.016 &  50 &   22.516 &   32.833 &  50 &   13.617 &   19.861 \\
 Si &  79 &   10.116 &   14.262 &  79 &   22.510 &   33.132 &  79 &   14.730 &   21.824 \\
  P &  74 &   14.939 &   23.657 &  77 &   38.352 &   55.538 &  77 &   21.186 &   35.850 \\
  S & 129 &   11.851 &   17.296 & 129 &   48.854 &   88.441 & 127 &   19.129 &   29.126 \\
 Cl & 156 &   10.030 &   14.834 & 144 &   26.118 &   69.410 & 131 &   19.757 &   53.791 \\
 Zn &  14 &    6.026 &    8.035 &  13 &   16.688 &   22.445 &  13 &   12.312 &   20.774 \\
 Ga &  21 &   16.322 &   19.623 &     &          &          &     &          &          \\
 Ge &  34 &   10.867 &   12.944 &  34 &   18.969 &   28.055 &  33 &   13.481 &   17.554 \\
 As &  22 &    8.626 &   11.964 &     &          &          &     &          &          \\
 Se &  10 &   11.335 &   14.110 &     &          &          &     &          &          \\
 Br &  98 &   10.799 &   14.273 &  91 &   24.829 &   41.901 &  84 &   20.173 &   34.835 \\
 Cd &   4 &    2.550 &    3.482 &     &          &          &     &          &          \\
 In &  22 &   11.801 &   16.725 &     &          &          &     &          &          \\
 Sn &  35 &    8.664 &   10.137 &  35 &   10.257 &   13.429 &     &          &          \\
 Sb &  20 &   13.681 &   17.333 &     &          &          &     &          &          \\
 Te &   5 &   12.298 &   12.910 &     &          &          &     &          &          \\
  I &  96 &   10.592 &   15.518 &  90 &   28.916 &   59.613 &  85 &   24.643 &   52.386 \\
 Hg &  29 &    5.839 &    7.653 &  29 &   12.410 &   15.481 &  29 &   10.012 &   13.547 \\
 Tl &  10 &    6.491 &    9.108 &     &          &          &     &          &          \\
 Pb &  24 &    8.580 &   11.488 &  24 &   10.468 &   13.194 &     &          &          \\
 Bi &  11 &   11.828 &   14.421 &     &          &          &     &          &          \\ \hline
 Totals &  1116 &  9.5  & 15.0  &   1050  & 22.4  & 48.9   &   989  & 14.8    & 32.2 \\
 \hline
\end{tabular}
\end{center}
\end{table}

\clearpage
\begin{table}
\caption{\label{avegeos} Average Errors in Bond-Length by Bond-Type} 
\compresstable
\begin{center}
\begin{tabular}{llrrrrrrrrr}
\multicolumn{2}{c}{Elements} & \multicolumn{3}{c}{PM3} &
\multicolumn{3}{c}{MNDO} & \multicolumn{3}{c}{AM1} \\
 & &  No. & Unsigned & Signed & No. & Unsigned & Signed & No. & Unsigned & Signed  \\ \hline
  H &   H &   1 &    0.042 &   -0.042 &   1 &    0.078 &   -0.078 &   1 &    0.064 &   -0.064 \\
  H &  Be &   2 &    0.038 &   -0.038 &   2 &    0.059 &   -0.059 &   2 &    0.039 &   -0.039 \\
  H &   B &     &          &          &   3 &    0.042 &   -0.042 &   3 &    0.016 &   -0.006 \\
  H &   C &  68 &    0.010 &    0.002 &  67 &    0.011 &    0.006 &  66 &    0.015 &    0.013 \\
  H &   N &   9 &    0.012 &   -0.010 &   8 &    0.063 &    0.056 &   8 &    0.046 &    0.029 \\
  H &   O &   8 &    0.014 &   -0.013 &   8 &    0.014 &   -0.011 &   8 &    0.013 &    0.008 \\
  H &   F &   3 &    0.055 &   -0.029 &   3 &    0.425 &    0.425 &   3 &    0.205 &    0.082 \\
  H &  Mg &   1 &    0.043 &   -0.043 &     &          &          &     &          &          \\
  H &  Al &   1 &    0.015 &    0.015 &   1 &    0.222 &   -0.222 &   1 &    0.186 &   -0.186 \\
  H &  Si &  12 &    0.016 &    0.011 &  10 &    0.112 &   -0.112 &  10 &    0.029 &   -0.029 \\
  H &   P &   3 &    0.114 &   -0.114 &   3 &    0.097 &   -0.009 &   3 &    0.091 &    0.014 \\
  H &   S &   3 &    0.026 &   -0.026 &   3 &    0.026 &   -0.026 &   3 &    0.005 &   -0.005 \\
  H &  Cl &   1 &    0.007 &   -0.007 &   1 &    0.073 &    0.073 &   1 &    0.009 &    0.009 \\
  H &  Ga &   2 &    0.027 &    0.012 &     &          &          &     &          &          \\
  H &  Ge &  17 &    0.020 &   -0.016 &  17 &    0.041 &   -0.041 &  17 &    0.019 &    0.019 \\
  H &  As &   1 &    0.007 &    0.007 &     &          &          &     &          &          \\
  H &  Se &   3 &    0.013 &    0.011 &     &          &          &     &          &          \\
  H &  Br &   1 &    0.056 &    0.056 &   1 &    0.025 &    0.025 &   1 &    0.006 &    0.006 \\
  H &  In &   1 &    0.104 &   -0.104 &     &          &          &     &          &          \\
  H &  Sn &   5 &    0.022 &   -0.014 &   5 &    0.125 &   -0.125 &     &          &          \\
  H &  Sb &   1 &    0.005 &   -0.005 &     &          &          &     &          &          \\
  H &  Te &   1 &    0.017 &    0.017 &     &          &          &     &          &          \\
  H &   I &   1 &    0.068 &    0.068 &   1 &    0.042 &   -0.042 &   1 &    0.022 &   -0.022 \\
  H &  Hg &   1 &    0.044 &   -0.044 &   1 &    0.190 &   -0.190 &   1 &    0.069 &   -0.069 \\
  H &  Pb &   1 &    0.110 &   -0.110 &   1 &    0.181 &   -0.181 &     &          &          \\
 Be &   O &   1 &    0.027 &   -0.027 &   1 &    0.004 &    0.004 &   1 &    0.071 &    0.071 \\
 Be &   F &   2 &    0.007 &    0.007 &   2 &    0.079 &    0.079 &   2 &    0.106 &    0.106 \\
 Be &   S &   1 &    0.090 &   -0.090 &   1 &    0.118 &   -0.118 &   1 &    0.057 &   -0.057 \\
 Be &  Cl &   2 &    0.047 &   -0.047 &   2 &    0.130 &    0.130 &   2 &    0.042 &    0.042 \\
 Be &  Br &   1 &    0.096 &   -0.096 &   1 &    0.102 &    0.102 &   1 &    0.013 &   -0.013 \\
 Be &   I &   2 &    0.071 &    0.019 &   2 &    0.012 &    0.012 &   2 &    0.012 &   -0.007 \\
  B &   B &     &          &          &   5 &    0.035 &    0.035 &   5 &    0.025 &   -0.015 \\
  B &   C &     &          &          &   5 &    0.033 &    0.008 &   5 &    0.016 &    0.001 \\
  B &   O &     &          &          &   2 &    0.021 &   -0.021 &   2 &    0.028 &   -0.028 \\
  B &   F &     &          &          &   1 &    0.007 &   -0.007 &   1 &    0.020 &   -0.020 \\
  B &   P &     &          &          &   1 &    2.386 &    2.386 &   1 &    0.321 &    0.321 \\
  B &   S &     &          &          &   1 &    0.118 &   -0.118 &   1 &    0.131 &   -0.131 \\
  C &   C &  91 &    0.017 &   -0.009 &  90 &    0.018 &    0.009 &  90 &    0.018 &   -0.006 \\
  C &   N &  30 &    0.020 &    0.015 &  28 &    0.023 &    0.006 &  28 &    0.027 &   -0.003 \\
  C &   O &  31 &    0.015 &    0.007 &  31 &    0.023 &    0.017 &  31 &    0.029 &    0.027 \\
  C &   F &  25 &    0.017 &    0.009 &  24 &    0.019 &    0.007 &  24 &    0.030 &    0.026 \\
  C &  Si &   4 &    0.012 &    0.011 &   4 &    0.054 &   -0.054 &   4 &    0.052 &   -0.052 \\
  C &   P &   6 &    0.076 &   -0.064 &   6 &    0.101 &   -0.101 &   6 &    0.108 &   -0.108 \\
  C &   S &  13 &    0.044 &   -0.040 &  12 &    0.068 &   -0.068 &  12 &    0.076 &   -0.076 \\
  C &  Cl &  12 &    0.040 &   -0.015 &  12 &    0.034 &    0.034 &  12 &    0.033 &   -0.008 \\
  C &  Zn &   5 &    0.030 &    0.030 &   5 &    0.037 &   -0.037 &   5 &    0.035 &   -0.004 \\
  C &  Ge &  14 &    0.024 &    0.011 &  14 &    0.019 &   -0.007 &  14 &    0.048 &    0.042 \\
\hline
\end{tabular}
\end{center}
\end{table}

\begin{table}
\caption{\label{avegeosb} Average Errors in Bond-Length by Bond-Type (contd.)} 
\compresstable
\begin{center}
\begin{tabular}{llrrrrrrrrr}
\multicolumn{2}{c}{Elements} & \multicolumn{3}{c}{PM3} &
\multicolumn{3}{c}{MNDO} & \multicolumn{3}{c}{AM1} \\
 & &  No. & Unsigned & Signed & No. & Unsigned & Signed & No. & Unsigned & Signed \\ \hline
  C &  As &   2 &    0.029 &   -0.002 &     &          &          &     &          &          \\
  C &  Se &  11 &    0.035 &   -0.020 &     &          &          &     &          &          \\
  C &  Br &  10 &    0.016 &    0.001 &  10 &    0.055 &   -0.049 &  10 &    0.040 &   -0.013 \\
  C &  Cd &   1 &    0.077 &   -0.077 &     &          &          &     &          &          \\
  C &  Sn &   6 &    0.027 &    0.008 &   6 &    0.078 &   -0.078 &     &          &          \\
  C &  Sb &   1 &    0.007 &    0.007 &     &          &          &     &          &          \\
  C &  Te &   1 &    0.027 &   -0.027 &     &          &          &     &          &          \\
  C &   I &   3 &    0.089 &   -0.089 &   3 &    0.076 &   -0.076 &   3 &    0.068 &   -0.038 \\
  C &  Hg &   6 &    0.035 &    0.016 &   6 &    0.102 &   -0.048 &   6 &    0.036 &    0.001 \\
  C &  Tl &   1 &    0.007 &    0.007 &     &          &          &     &          &          \\
  C &  Pb &   2 &    0.052 &   -0.052 &   2 &    0.071 &   -0.071 &     &          &          \\
  C &  Bi &   1 &    0.004 &   -0.004 &     &          &          &     &          &          \\
  N &   N &   8 &    0.056 &   -0.052 &   8 &    0.101 &   -0.098 &   8 &    0.093 &   -0.073 \\
  N &   O &  19 &    0.028 &   -0.009 &  19 &    0.053 &   -0.007 &  19 &    0.053 &   -0.017 \\
  N &   F &   2 &    0.085 &   -0.085 &   2 &    0.136 &   -0.136 &   2 &    0.082 &   -0.082 \\
  N &  Si &   1 &    0.108 &   -0.108 &   1 &    0.092 &   -0.092 &   1 &    0.087 &   -0.087 \\
  N &   P &   1 &    0.077 &   -0.077 &   1 &    0.093 &   -0.093 &   1 &    0.109 &   -0.109 \\
  N &   S &   2 &    0.065 &    0.020 &   2 &    0.058 &    0.003 &   2 &    0.053 &   -0.053 \\
  N &  Cl &   2 &    0.099 &   -0.099 &   2 &    0.094 &   -0.094 &   2 &    0.139 &   -0.139 \\
  N &  Ga &   2 &    0.294 &    0.064 &     &          &          &     &          &          \\
  N &  Ge &   3 &    0.020 &   -0.003 &   3 &    0.013 &    0.012 &   3 &    0.016 &   -0.016 \\
  N &  Se &   1 &    0.053 &    0.053 &     &          &          &     &          &          \\
  N &  Br &   1 &    0.252 &   -0.252 &   1 &    0.271 &   -0.271 &   1 &    0.218 &   -0.218 \\
  N &  Cd &   1 &    0.022 &   -0.022 &     &          &          &     &          &          \\
  N &  Te &   1 &    0.045 &   -0.045 &     &          &          &     &          &          \\
  O &   O &   3 &    0.026 &   -0.005 &   3 &    0.090 &   -0.090 &   3 &    0.115 &   -0.115 \\
  O &   F &   1 &    0.034 &   -0.034 &   1 &    0.131 &   -0.131 &   1 &    0.058 &   -0.058 \\
  O &  Mg &   1 &    0.031 &    0.031 &     &          &          &     &          &          \\
  O &  Al &   2 &    0.028 &   -0.025 &   2 &    0.133 &   -0.133 &   2 &    0.067 &   -0.067 \\
  O &   P &   5 &    0.043 &    0.017 &   5 &    0.032 &   -0.007 &   5 &    0.018 &    0.002 \\
  O &   S &  11 &    0.039 &    0.023 &  11 &    0.070 &    0.070 &  11 &    0.042 &   -0.006 \\
  O &  Cl &   3 &    0.003 &   -0.002 &   3 &    0.120 &    0.108 &   3 &    0.150 &    0.150 \\
  O &  Ga &   1 &    0.028 &   -0.028 &     &          &          &     &          &          \\
  O &  Ge &   2 &    0.014 &    0.014 &   2 &    0.035 &   -0.035 &   2 &    0.040 &    0.038 \\
  O &  Se &   7 &    0.050 &   -0.017 &     &          &          &     &          &          \\
  O &  In &   1 &    0.021 &   -0.021 &     &          &          &     &          &          \\
  O &  Sn &   1 &    0.006 &    0.006 &   1 &    0.084 &   -0.084 &     &          &          \\
  O &  Te &   2 &    0.093 &   -0.093 &     &          &          &     &          &          \\
  O &  Hg &   1 &    0.001 &    0.001 &   1 &    0.042 &    0.042 &   1 &    0.222 &    0.222 \\
  O &  Pb &   1 &    0.016 &    0.016 &   1 &    0.038 &   -0.038 &     &          &          \\
  F &   F &   1 &    0.062 &   -0.062 &   1 &    0.146 &   -0.146 &   1 &    0.015 &    0.015 \\
  F &  Mg &   2 &    0.006 &   -0.002 &     &          &          &     &          &          \\
  F &  Al &   3 &    0.006 &    0.004 &   3 &    0.059 &   -0.059 &   3 &    0.073 &   -0.073 \\
  F &  Si &   5 &    0.017 &    0.011 &   5 &    0.021 &    0.014 &   5 &    0.038 &    0.038 \\
  F &   P &   9 &    0.018 &   -0.014 &   8 &    0.031 &    0.000 &   8 &    0.027 &   -0.026 \\
  F &   S &  13 &    0.027 &   -0.010 &  13 &    0.235 &    0.222 &  13 &    0.037 &   -0.033 \\
  F &  Cl &   4 &    0.051 &    0.015 &   4 &    0.057 &    0.057 &   4 &    0.043 &    0.035 \\
  F &  Zn &   1 &    0.002 &   -0.002 &   1 &    0.062 &   -0.062 &   1 &    0.006 &    0.006 \\
  F &  Ga &   2 &    0.010 &   -0.002 &     &          &          &     &          &          \\
  F &  Ge &   7 &    0.016 &   -0.011 &   7 &    0.020 &    0.002 &   7 &    0.035 &   -0.031 \\
\hline
\end{tabular}
\end{center}
\end{table}

\begin{table}
\caption{\label{avegeosc} Average Errors in Bond-Length by Bond-Type (contd.)} 
\compresstable
\begin{center}
\begin{tabular}{llrrrrrrrrr}
\multicolumn{2}{c}{Elements} & \multicolumn{3}{c}{PM3} &
\multicolumn{3}{c}{MNDO} & \multicolumn{3}{c}{AM1} \\
 & &  No. & Unsigned & Signed & No. & Unsigned & Signed & No. & Unsigned & Signed \\ \hline
  F &  As &   3 &    0.010 &   -0.007 &     &          &          &     &          &          \\
  F &  Se &   8 &    0.020 &    0.010 &     &          &          &     &          &          \\
  F &  Br &   5 &    0.033 &    0.025 &   5 &    0.037 &    0.006 &   5 &    0.055 &    0.055 \\
  F &  Cd &   1 &    0.004 &   -0.004 &     &          &          &     &          &          \\
  F &  In &   1 &    0.001 &    0.001 &     &          &          &     &          &          \\
  F &  Sb &   2 &    0.402 &   -0.402 &     &          &          &     &          &          \\
  F &  Te &   4 &    0.007 &    0.001 &     &          &          &     &          &          \\
  F &   I &   5 &    0.226 &    0.219 &   5 &    0.236 &    0.234 &   5 &    0.224 &    0.214 \\
  F &  Hg &   2 &    0.021 &   -0.011 &   2 &    0.050 &   -0.050 &   2 &    0.028 &   -0.024 \\
  F &  Tl &   2 &    0.027 &    0.022 &     &          &          &     &          &          \\
  F &  Pb &   2 &    0.018 &   -0.018 &   2 &    0.050 &   -0.050 &     &          &          \\
 Mg &  Mg &   1 &    3.110 &    3.110 &     &          &          &     &          &          \\
 Mg &   S &   1 &    0.215 &    0.215 &     &          &          &     &          &          \\
 Mg &  Cl &   2 &    0.311 &   -0.311 &     &          &          &     &          &          \\
 Mg &  Br &   2 &    0.010 &    0.004 &     &          &          &     &          &          \\
 Mg &   I &   1 &    0.100 &   -0.100 &     &          &          &     &          &          \\
 Al &  Al &   1 &    0.073 &    0.073 &   1 &    0.175 &   -0.175 &   1 &    0.060 &   -0.060 \\
 Al &  Cl &   2 &    0.138 &   -0.138 &   2 &    0.030 &   -0.025 &   2 &    0.240 &   -0.240 \\
 Al &  Br &   2 &    0.199 &   -0.199 &   2 &    0.094 &   -0.094 &   2 &    0.028 &   -0.028 \\
 Al &   I &   1 &    0.012 &   -0.012 &   1 &    0.174 &   -0.174 &   1 &    0.111 &   -0.111 \\
 Si &  Si &   3 &    0.051 &    0.051 &   3 &    0.202 &   -0.202 &   3 &    0.196 &   -0.109 \\
 Si &   S &   1 &    0.105 &    0.105 &   1 &    0.070 &   -0.070 &   1 &    0.176 &    0.176 \\
 Si &  Cl &   3 &    0.060 &   -0.018 &   3 &    0.046 &    0.046 &   3 &    0.045 &   -0.007 \\
 Si &  Ge &   1 &    0.047 &    0.047 &   1 &    0.017 &   -0.017 &   1 &    0.002 &    0.002 \\
 Si &  As &   1 &    0.018 &    0.018 &     &          &          &     &          &          \\
 Si &  Se &   2 &    0.103 &    0.020 &     &          &          &     &          &          \\
 Si &  Br &   2 &    0.318 &   -0.318 &   2 &    0.051 &    0.051 &   2 &    0.116 &    0.116 \\
 Si &  Sb &   1 &    0.022 &   -0.022 &     &          &          &     &          &          \\
 Si &   I &   2 &    0.231 &   -0.194 &   2 &    0.074 &   -0.074 &   2 &    0.004 &   -0.004 \\
  P &   P &   3 &    0.084 &   -0.084 &   3 &    0.189 &   -0.189 &   3 &    0.243 &   -0.243 \\
  P &   S &   5 &    0.049 &    0.022 &   3 &    0.104 &   -0.033 &   3 &    0.008 &   -0.005 \\
  P &  Cl &   3 &    0.045 &   -0.020 &   3 &    0.045 &   -0.045 &   3 &    0.101 &   -0.101 \\
  P &  Ge &   2 &    0.090 &   -0.090 &   2 &    0.150 &   -0.150 &   2 &    0.185 &   -0.185 \\
  P &  Se &   1 &    0.062 &    0.062 &     &          &          &     &          &          \\
  P &  Br &   3 &    0.077 &   -0.077 &   3 &    0.085 &   -0.085 &   3 &    0.058 &   -0.058 \\
  S &   S &   9 &    0.115 &    0.084 &   9 &    0.454 &    0.325 &   9 &    0.264 &    0.234 \\
  S &  Cl &   5 &    0.042 &    0.036 &   5 &    0.064 &   -0.002 &   5 &    0.075 &   -0.015 \\
  S &  Ge &   2 &    0.035 &   -0.004 &   2 &    0.079 &   -0.079 &   2 &    0.039 &    0.039 \\
  S &  Br &   4 &    0.166 &    0.166 &   4 &    0.129 &    0.009 &   4 &    0.166 &    0.118 \\
  S &  Cd &   1 &    0.001 &   -0.001 &     &          &          &     &          &          \\
  S &  Sn &   1 &    0.071 &   -0.071 &   1 &    0.208 &   -0.208 &     &          &          \\
  S &  Te &   1 &    0.234 &    0.234 &     &          &          &     &          &          \\
  S &  Pb &   1 &    0.138 &   -0.138 &   1 &    0.179 &   -0.179 &     &          &          \\
 Cl &  Cl &   1 &    0.049 &    0.049 &   1 &    0.010 &    0.010 &   1 &    0.068 &   -0.068 \\
 Cl &  Zn &   1 &    0.002 &    0.002 &   1 &    0.053 &    0.053 &   1 &    0.005 &    0.005 \\
 Cl &  Ga &   6 &    0.119 &    0.049 &     &          &          &     &          &          \\
 Cl &  Ge &   8 &    0.061 &    0.012 &   8 &    0.102 &    0.102 &   8 &    0.035 &    0.001 \\
 Cl &  As &   1 &    0.002 &    0.002 &     &          &          &     &          &          \\
 Cl &  Se &   2 &    0.006 &    0.001 &     &          &          &     &          &          \\
 Cl &  Br &   1 &    0.040 &    0.040 &   1 &    0.056 &   -0.056 &   1 &    0.072 &   -0.072 \\
\hline
\end{tabular}
\end{center}
\end{table}

\begin{table}
\caption{\label{avegeosd} Average Errors in Bond-Length by Bond-Type (contd.)} 
\compresstable
\begin{center}
\begin{tabular}{llrrrrrrrrr}
\multicolumn{2}{c}{Elements} & \multicolumn{3}{c}{PM3} &
\multicolumn{3}{c}{MNDO} & \multicolumn{3}{c}{AM1} \\
 & &  No. & Unsigned & Signed & No. & Unsigned & Signed & No. & Unsigned & Signed \\ \hline
 Cl &  Cd &   1 &    0.015 &    0.015 &     &          &          &     &          &          \\
 Cl &  In &   1 &    0.002 &    0.002 &     &          &          &     &          &          \\
 Cl &  Sn &   4 &    0.047 &    0.044 &   4 &    0.029 &   -0.027 &     &          &          \\
 Cl &  Sb &   3 &    0.037 &    0.035 &     &          &          &     &          &          \\
 Cl &  Te &   1 &    0.072 &    0.072 &     &          &          &     &          &          \\
 Cl &   I &   1 &    0.135 &   -0.135 &   1 &    0.065 &   -0.065 &   1 &    0.109 &   -0.109 \\
 Cl &  Hg &   3 &    0.025 &   -0.025 &   3 &    0.024 &    0.024 &   3 &    0.018 &   -0.011 \\
 Cl &  Tl &   1 &    0.004 &    0.004 &     &          &          &     &          &          \\
 Cl &  Pb &   3 &    0.124 &    0.021 &   3 &    0.104 &    0.028 &     &          &          \\
 Cl &  Bi &   1 &    0.006 &   -0.006 &     &          &          &     &          &          \\
 Zn &  Br &   1 &    0.107 &   -0.107 &   1 &    0.031 &    0.031 &   1 &    0.093 &   -0.093 \\
 Zn &   I &   1 &    0.003 &    0.003 &   1 &    0.019 &   -0.019 &   1 &    0.055 &   -0.055 \\
 Ga &  Ga &   1 &    0.583 &    0.583 &     &          &          &     &          &          \\
 Ga &  Br &   4 &    0.031 &    0.014 &     &          &          &     &          &          \\
 Ga &   I &   2 &    0.077 &    0.041 &     &          &          &     &          &          \\
 Ge &  Ge &   1 &    0.010 &   -0.010 &   1 &    0.121 &    0.121 &   1 &    0.037 &   -0.037 \\
 Ge &  Se &   1 &    0.208 &   -0.208 &     &          &          &     &          &          \\
 Ge &  Br &   5 &    0.037 &    0.031 &   5 &    0.056 &    0.049 &   5 &    0.053 &   -0.053 \\
 Ge &  Te &   1 &    0.338 &   -0.338 &     &          &          &     &          &          \\
 Ge &   I &   2 &    0.034 &   -0.034 &   2 &    0.013 &    0.010 &   2 &    0.083 &   -0.083 \\
 As &  Br &   1 &    0.008 &   -0.008 &     &          &          &     &          &          \\
 As &   I &   1 &    0.041 &   -0.041 &     &          &          &     &          &          \\
 Se &  Se &   1 &    0.040 &    0.040 &     &          &          &     &          &          \\
 Se &  In &   1 &    0.070 &   -0.070 &     &          &          &     &          &          \\
 Se &  Sn &   1 &    0.002 &   -0.002 &     &          &          &     &          &          \\
 Se &  Pb &   1 &    0.041 &   -0.041 &     &          &          &     &          &          \\
 Br &  Br &   1 &    0.160 &    0.160 &   1 &    0.115 &   -0.115 &   1 &    0.099 &   -0.099 \\
 Br &  Cd &   1 &    0.034 &   -0.034 &     &          &          &     &          &          \\
 Br &  In &   1 &    0.253 &   -0.253 &     &          &          &     &          &          \\
 Br &  Sn &   4 &    0.041 &   -0.041 &   4 &    0.087 &   -0.087 &     &          &          \\
 Br &  Sb &   1 &    0.019 &   -0.019 &     &          &          &     &          &          \\
 Br &  Te &   2 &    0.075 &   -0.075 &     &          &          &     &          &          \\
 Br &   I &   1 &    0.076 &    0.076 &   1 &    0.135 &   -0.135 &   1 &    0.131 &   -0.131 \\
 Br &  Hg &   3 &    0.148 &   -0.148 &   3 &    0.045 &   -0.020 &   3 &    0.146 &   -0.146 \\
 Br &  Tl &   1 &    0.059 &   -0.059 &     &          &          &     &          &          \\
 Br &  Pb &   2 &    0.025 &   -0.007 &   2 &    0.103 &   -0.103 &     &          &          \\
 Br &  Bi &   1 &    0.026 &   -0.026 &     &          &          &     &          &          \\
 Cd &   I &   1 &    0.038 &    0.038 &     &          &          &     &          &          \\
 In &  Te &   1 &    0.004 &   -0.004 &     &          &          &     &          &          \\
 In &   I &   2 &    0.012 &   -0.008 &     &          &          &     &          &          \\
 Sn &  Te &   1 &    0.049 &    0.049 &     &          &          &     &          &          \\
 Sn &   I &   3 &    0.055 &   -0.055 &   3 &    0.172 &   -0.172 &     &          &          \\
 Sb &  Sb &   1 &    0.289 &   -0.289 &     &          &          &     &          &          \\
 Te &  Te &   1 &    0.145 &    0.145 &     &          &          &     &          &          \\
 Te &  Pb &   1 &    0.141 &    0.141 &     &          &          &     &          &          \\
  I &   I &   1 &    0.002 &    0.002 &   1 &    0.151 &   -0.151 &   1 &    0.128 &   -0.128 \\
  I &  Hg &   3 &    3.441 &    3.330 &   3 &    0.071 &   -0.071 &   3 &    0.063 &   -0.039 \\
  I &  Tl &   1 &    0.090 &   -0.090 &     &          &          &     &          &          \\
  I &  Pb &   2 &    0.031 &    0.006 &   2 &    0.181 &   -0.181 &     &          &          \\
 Pb &  Pb &   1 &    0.012 &   -0.012 &   1 &    0.110 &   -0.110 &     &          &          \\
\hline
\end{tabular}
\end{center}
\end{table}

\clearpage
\begin{table}
\caption{\label{elegeos} Average Errors  in Bond-Lengths for each Element}
\begin{center}
\begin{tabular}{lrrrrrrrrr}
Element    &    No.  &  PM3     &      No. & MNDO & No.  & AM1  \\ \hline
Hydrogen   &     147 &    0.019 &     133 &    0.048 &     126 &    0.027 &  \\
Beryllium  &      11 &    0.049 &      11 &    0.071 &      11 &    0.049 &  \\
Carbon     &     344 &    0.021 &     320 &    0.027 &     311 &    0.028 &  \\
Nitrogen   &      83 &    0.041 &      75 &    0.054 &      75 &    0.052 &  \\
Oxygen     &     101 &    0.026 &      89 &    0.045 &      87 &    0.045 &  \\
Fluorine   &     115 &    0.038 &      88 &    0.091 &      86 &    0.055 &  \\
Magnesium  &      11 &    0.378 &         &          &         &          &   \\
Aluminum   &      12 &    0.071 &      12 &    0.105 &      12 &    0.104 &  \\
Silicon    &      38 &    0.059 &      32 &    0.082 &      32 &    0.060 &  \\
Phosphorus &      41 &    0.057 &      37 &    0.080 &      37 &    0.079 &  \\
Sulfur     &      74 &    0.063 &      68 &    0.159 &      66 &    0.087 &  \\
Chlorine   &      78 &    0.060 &      59 &    0.062 &      52 &    0.063 &  \\
Zinc       &       9 &    0.029 &       9 &    0.039 &       9 &    0.037 &  \\
Gallium    &      20 &    0.113 &         &          &         &          &   \\
Germanium  &      66 &    0.037 &      64 &    0.046 &      64 &    0.040 &  \\
Arsenic    &      10 &    0.017 &         &          &         &          &   \\
Selenium   &      40 &    0.041 &         &          &         &          &   \\
Bromine    &      61 &    0.075 &      47 &    0.074 &      41 &    0.077 &  \\
Cadmium    &       7 &    0.027 &         &          &         &          &   \\
Indium     &       9 &    0.053 &         &          &         &          &   \\
Tin        &      26 &    0.035 &      24 &    0.099 &         &          &   \\
Antimony   &      10 &    0.126 &         &          &         &          &   \\
Tellurium  &      18 &    0.080 &         &          &         &          &   \\
Iodine     &      36 &    0.371 &      28 &    0.117 &      23 &    0.099 &  \\
Mercury    &      19 &    0.586 &      19 &    0.072 &      19 &    0.065 &  \\
Thallium   &       6 &    0.036 &         &          &         &          &   \\
Lead       &      17 &    0.064 &      15 &    0.109 &         &          &   \\
Bismuth    &       3 &    0.012 &         &          &         &          &   \\ \hline
 TOTALS:   &     771 &    0.058 &     627 &    0.062 &     587 &    0.048 &  \\ \hline
\end{tabular}
\end{center}
\end{table}

\clearpage
\begin{table}
\caption{\label{weimndo} Worst Ionization Potential  Errors in MNDO}
\compresstable
\begin{center}
\begin{tabular}{llrrr}
 Formula & Chemical Name & Exp.\ I.P. & Calc.\   I.P.       & Diff.\\
\hline
 H$_2$O               & Water, fourth I.P.                     &   32.20   &    40.03   &    7.8\\
 CCl$_3$              & Trichloromethyl                        &    8.28   &    13.09   &    4.8\\
 SnF$_2$              & Tin difluoride                         &    8.00   &    12.60   &    4.6\\
 NS                   & Sulfur nitride                         &    8.87   &    13.26   &    4.4\\
 CF$_3$Cl             & Trifluorochloromethane                 &   10.30   &    14.13   &    3.8\\
 SOCl$_3$             & SOCl3                                  &    9.63   &    13.40   &    3.8\\
 SnH$_4$              & Tin tetrahydride (stannane)            &    9.20   &    12.74   &    3.5\\
 PF$_3$               & Phosphorus trifluoride                 &    9.71   &    13.13   &    3.4\\
 PS                   & Phosphorus sulfide                     &    9.00   &    12.05   &    3.1\\
 C$_3$H$_7$NO$_2$     & Alanine                                &    8.10   &    10.82   &    2.7\\
 C$_5$H$_1$$_0$       & 2-Methyl-1-butene                      &    7.40   &     9.81   &    2.4\\
 C$_5$H$_8$O$_2$      & Acetylacetone                          &    8.38   &    10.78   &    2.4\\
 PbO$_2$              & Lead dioxide                           &    8.87   &    11.06   &    2.2\\
 SO$_3$               & Sulfur trioxide                        &   11.00   &    13.16   &    2.2\\
 C$_5$H$_1$$_0$       & 1-Pentene                              &    7.90   &     9.97   &    2.1\\
 C$_4$GeH$_1$$_2$     & Tetramethylgermanium                   &    9.29   &    11.36   &    2.1\\
 CF$_3$I              & Trifluoroiodomethane                   &   10.45   &    12.48   &    2.0\\
 SiCl$_4$             & Silicon tetrachloride                  &   11.79   &    13.82   &    2.0\\
 C$_4$SnH$_1$$_2$     & Tetramethyltin                         &    8.90   &    10.91   &    2.0\\
 GeO                  & Germanium oxide                        &   11.25   &     9.26   &   -2.0\\
 P$_4$                & Phosphorus tetramer                    &    9.54   &    11.50   &    2.0\\
 C$_8$H$_1$$_4$       & Bicyclo(2.2.2)-octane                  &    9.45   &    11.40   &    2.0\\
 GeCl$_4$             & Germanium tetrachloride                &   11.88   &    13.78   &    1.9\\
 C$_2$Cl$_6$          & Hexachloroethane                       &   11.20   &    13.08   &    1.9\\
 C$_6$H$_1$$_1$I      & Iodocyclohexane                        &    8.91   &    10.77   &    1.9\\
 C$_5$H$_1$$_2$       & n-Pentane                              &   10.30   &    12.15   &    1.8\\
 GeI$_4$              & Germanium tetraiodide                  &    9.42   &    11.26   &    1.8\\
 CHgH$_3$Br           & Methylmercuric bromide                 &    9.25   &    11.09   &    1.8\\
 C$_8$PbH$_2$$_0$     & Tetraethyllead                         &    8.13   &     9.87   &    1.7\\
 C$_6$Pb$_2$H$_1$$_8$ & Hexamethyldiplumbane                   &    7.41   &     9.15   &    1.7\\
 CBr$_4$              & Carbon tetrabromide                    &   10.30   &    12.03   &    1.7\\
 C$_5$H$_1$$_2$       & 2-Methylbutane                         &   10.30   &    12.00   &    1.7\\
 C$_6$GeSnH$_1$$_8$   & GeMe3-SnMe3                            &    8.20   &     9.88   &    1.7\\
 C$_1$$_0$H$_1$$_6$   & Adamantane                             &    9.60   &    11.27   &    1.7\\
 C$_2$F$_4$Br$_2$     & 1,2-Dibromotetrafluoroethane           &   14.44   &    12.77   &   -1.7\\
 C$_8$GeH$_2$$_0$     & Tetraethylgermanium                    &    9.30   &    10.97   &    1.7\\
 PCl$_3$              & Phosphorus trichloride                 &   10.50   &    12.17   &    1.7\\
 S$_2$Cl$_2$          & ClSSCl                                 &    9.40   &    11.06   &    1.7\\
 I$_2$                & Iodine, second I.P.                    &   11.03   &    12.69   &    1.7\\
 C$_7$GeH$_1$$_8$     & t-Butyltrimethylgermane                &    8.98   &    10.64   &    1.7\\
 SnCl$_2$             & Tin dichloride                         &   10.10   &    11.72   &    1.6\\
 NO$_2$Cl             & Nitryl chloride                        &   11.40   &    13.01   &    1.6\\
 C$_4$H$_9$I          & 1-Butyl iodide                         &    9.20   &    10.80   &    1.6\\
 C$_7$PbH$_1$$_8$     & t-Butyltrimethyllead                   &    7.99   &     9.60   &    1.6\\
 Be                   & Beryllium, atom                        &    9.20   &     7.60   &   -1.6\\
 C$_3$H$_9$PO$_3$     & Trimethyl phosphite                    &    9.00   &    10.60   &    1.6\\
 C$_4$ZnH$_1$$_0$     & Diethylzinc                            &    8.60   &    10.20   &    1.6\\
 C$_8$SnH$_2$$_0$     & Tetraethyltin                          &    8.90   &    10.47   &    1.6\\
 C$_5$H$_1$$_0$       & Cyclopentane                           &   10.50   &    12.06   &    1.6\\
 PbCl$_2$             & Lead dichloride                        &   10.34   &    11.90   &    1.6\\
\hline
\end{tabular}
\end{center}
\end{table}

\clearpage
\begin{table}
\caption{\label{weiam1} Worst Ionization Potential  Errors in AM1}
\compresstable
\begin{center}
\begin{tabular}{llrrr}
 Formula & Chemical Name & Exp.\   I.P.       & Calc.\   I.P.       & Diff.\\
 \hline
 H$_2$O               & Water, fourth I.P.                     &   32.20   &    36.42   &    4.2\\
 CCl$_3$              & Trichloromethyl                        &    8.28   &    12.24   &    4.0\\
 SOCl$_3$             & SOCl3                                  &    9.63   &    13.36   &    3.7\\
 SO$_3$               & Sulfur trioxide                        &   11.00   &    14.26   &    3.3\\
 NS                   & Sulfur nitride                         &    8.87   &    11.96   &    3.1\\
 PS                   & Phosphorus sulfide                     &    9.00   &    11.95   &    2.9\\
 CF$_3$Cl             & Trifluorochloromethane                 &   10.30   &    13.24   &    2.9\\
 C$_2$F$_4$Br$_2$     & 1,2-Dibromotetrafluoroethane           &   14.44   &    11.50   &   -2.9\\
 C$_5$H$_8$O$_2$      & Acetylacetone                          &    8.38   &    10.77   &    2.4\\
 C$_5$H$_1$$_0$       & 2-Methyl-1-butene                      &    7.40   &     9.70   &    2.3\\
 C$_3$H$_7$NO$_2$     & Alanine                                &    8.10   &    10.37   &    2.3\\
 P$_4$                & Phosphorus tetramer                    &    9.54   &    11.68   &    2.1\\
 SF$_4$               & Sulfur tetrafluoride                   &   12.05   &     9.92   &   -2.1\\
 C$_5$H$_1$$_0$       & 1-Pentene                              &    7.90   &    10.00   &    2.1\\
 HF                   & Hydrogen fluoride                      &   16.06   &    14.09   &   -2.0\\
 GeI$_4$              & Germanium tetraiodide                  &    9.42   &    11.33   &    1.9\\
 PCl$_5$              & Phosphorus pentachloride               &   10.80   &    12.70   &    1.9\\
 GeO                  & Germanium oxide                        &   11.25   &     9.40   &   -1.8\\
 SO$_2$               & Sulfur dioxide                         &   12.30   &    10.49   &   -1.8\\
 SOF$_2$              & Thionyl fluoride                       &   12.58   &    10.80   &   -1.8\\
 C$_6$Ge$_2$H$_1$$_8$ & GeMe3-GeMe3                            &    8.18   &     9.92   &    1.7\\
 C$_4$GeH$_1$$_2$     & Tetramethylgermanium                   &    9.29   &    10.96   &    1.7\\
 Be                   & Beryllium, atom                        &    9.20   &     7.60   &   -1.6\\
 NO$_2$Cl             & Nitryl chloride                        &   11.40   &    12.94   &    1.5\\
 CF$_3$I              & Trifluoroiodomethane                   &   10.45   &    11.97   &    1.5\\
 CHF$_3$              & Trifluoromethane                       &   14.80   &    13.32   &   -1.5\\
 C$_6$H$_1$$_1$I      & Iodocyclohexane                        &    8.91   &    10.38   &    1.5\\
 C$_2$I$_2$           & Diiodoacetylene                        &    9.03   &    10.50   &    1.5\\
 GeF$_4$              & Germanium tetrafluoride                &   16.06   &    14.58   &   -1.5\\
 I$_2$                & Iodine, second I.P.                    &   11.03   &    12.48   &    1.5\\
 SO$_2$F$_2$          & Sulfuryl fluoride                      &   13.04   &    14.41   &    1.4\\
 C$_4$H$_4$N$_2$      & Pyridazine                             &    9.30   &    10.67   &    1.4\\
 C$_2$F$_6$           & Hexafluoroethane                       &   14.60   &    13.23   &   -1.4\\
 N$_2$O$_5$           & Dinitrogen pentoxide                   &   12.30   &    13.64   &    1.3\\
 I$_2$                & Iodine                                 &    9.34   &    10.67   &    1.3\\
 CHgH$_3$Br           & Methylmercuric bromide                 &    9.25   &    10.59   &    1.3\\
 N$_2$                & Nitrogen                               &   15.60   &    14.32   &   -1.3\\
 POCl$_3$             & Phosphorus oxychloride                 &   11.85   &    13.12   &    1.3\\
 H$_2$SiF$_2$         & Difluorosilane                         &   12.85   &    11.61   &   -1.2\\
 C$_4$H$_9$I          & 1-Butyl iodide                         &    9.20   &    10.43   &    1.2\\
 CH$_3$F              & Fluoromethane                          &   13.31   &    12.10   &   -1.2\\
 C$_2$H$_4$I$_2$      & 1,2-Diiodoethane                       &    9.50   &    10.71   &    1.2\\
 SF$_2$               & Sulfur difluoride                      &   10.20   &     8.99   &   -1.2\\
 SiCl$_4$             & Silicon tetrachloride                  &   11.79   &    12.98   &    1.2\\
 C$_4$H$_2$O$_3$      & Malaic anhydride                       &   10.84   &    12.02   &    1.2\\
 C$_3$H$_7$I          & 1-Iodopropane                          &    9.27   &    10.43   &    1.2\\
 CH$_2$F$_2$          & Difluoromethane                        &   13.17   &    12.02   &   -1.2\\
 N$_2$F$_2$           & trans-Difluorodiazene                  &   13.40   &    12.25   &   -1.2\\
 PI$_3$               & Phosphorus triiodide                   &    9.15   &    10.30   &    1.2\\
 B$_2$F$_4$           & B2F4                                   &   13.26   &    12.12   &   -1.1\\
\hline
\end{tabular}
\end{center}
\end{table}

\clearpage
\begin{table}
\caption{\label{weipm3} Worst Ionization Potential Errors in PM3}
\compresstable
\begin{center}
\begin{tabular}{llrrr} Formula & Chemical Name & Exp.\ I.P.         & Calc.\ I.P.         & Diff.\\
 \hline
 H$_2$O               & Water, fourth I.P.                     &   32.20   &    36.83   &    4.6\\
 GaBr$_3$             & Gallium tribromide                     &   10.94   &    15.02   &    4.1\\
 InBr$_3$             & Indium tribromide                      &   10.30   &    14.20   &    3.9\\
 SnH$_4$              & Tin tetrahydride (stannane)            &    9.20   &    12.87   &    3.7\\
 C$_2$F$_4$Br$_2$     & 1,2-Dibromotetrafluoroethane           &   14.44   &    11.03   &   -3.4\\
 NS                   & Sulfur nitride                         &    8.87   &    12.26   &    3.4\\
 C$_1$$_0$MgH$_1$$_0$ & Dicyclopentadienyl magnesium           &   11.00   &     8.36   &   -2.6\\
 C$_5$H$_8$O$_2$      & Acetylacetone                          &    8.38   &    10.89   &    2.5\\
 C$_5$H$_1$$_0$       & 2-Methyl-1-butene                      &    7.40   &     9.85   &    2.4\\
 SOCl$_3$             & SOCl3                                  &    9.63   &    12.06   &    2.4\\
 HgBr$_2$             & Mercury dibromide                      &   10.62   &    12.91   &    2.3\\
 C$_5$H$_1$$_0$       & 1-Pentene                              &    7.90   &    10.15   &    2.2\\
 SnF$_2$              & Tin difluoride                         &    8.00   &    10.26   &    2.3\\
 CCl$_3$              & Trichloromethyl                        &    8.28   &    10.36   &    2.1\\
 H$_2$SiF$_2$         & Difluorosilane                         &   12.85   &    10.82   &   -2.0\\
 SnBr$_4$             & Tin tetrabromide                       &   11.00   &    13.02   &    2.0\\
 SO$_3$               & Sulfur trioxide                        &   11.00   &    12.91   &    1.9\\
 PS                   & Phosphorus sulfide                     &    9.00   &    10.86   &    1.9\\
 N$_2$                & Nitrogen                               &   15.60   &    13.80   &   -1.8\\
 C$_4$SnH$_1$$_2$     & Tetramethyltin                         &    8.90   &    10.70   &    1.8\\
 C$_3$H$_7$NO$_2$     & Alanine                                &    8.10   &     9.88   &    1.8\\
 SO$_2$               & Sulfur dioxide                         &   12.30   &    10.55   &   -1.8\\
 C$_3$SnH$_9$Br       & Trimethyltin bromide                   &    9.60   &    11.35   &    1.8\\
 ZnBr$_2$             & Zinc dibromide                         &   10.89   &    12.62   &    1.7\\
 P$_2$                & Phosphorus dimer                       &   10.62   &     8.91   &   -1.7\\
 C$_1$$_8$BiH$_1$$_5$ & Triphenylbismuth                       &    7.45   &     9.16   &    1.7\\
 HCl                  & Hydrogen chloride                      &   12.75   &    11.06   &   -1.7\\
 PbO$_2$              & Lead dioxide                           &    8.87   &    10.55   &    1.7\\
 SF$_4$               & Sulfur tetrafluoride                   &   12.05   &    10.41   &   -1.6\\
 CF$_3$Cl             & Trifluorochloromethane                 &   10.30   &    11.88   &    1.6\\
 C$_6$Pb$_2$H$_1$$_8$ & Hexamethyldiplumbane                   &    7.41   &     9.00   &    1.6\\
 GeO                  & Germanium oxide                        &   11.25   &     9.68   &   -1.6\\
 BeFCl                & Beryllium chloride fluoride            &   13.00   &    11.44   &   -1.6\\
 PbF$_2$              & Lead difluoride                        &   11.84   &    10.28   &   -1.6\\
 C$_1$$_8$SbH$_1$$_5$ & Triphenylstibine                       &    8.08   &     9.64   &    1.6\\
 SOF$_2$              & Thionyl fluoride                       &   12.58   &    11.03   &   -1.6\\
 C$_4$GeH$_1$$_2$     & Tetramethylgermanium                   &    9.29   &    10.83   &    1.5\\
 C$_8$H$_1$$_4$       & Bicyclo(2.2.2)-octane                  &    9.45   &    10.94   &    1.5\\
 N$_2$F$_2$           & trans-Difluorodiazene                  &   13.40   &    11.91   &   -1.5\\
 NF$_3$               & Nitrogen trifluoride                   &   13.73   &    12.24   &   -1.5\\
 CHgH$_3$Br           & Methylmercuric bromide                 &    9.25   &    10.70   &    1.4\\
 H$_2$SiCl$_2$        & Dichlorosilane                         &   11.70   &    10.27   &   -1.4\\
 C$_3$SbH$_9$         & Trimethylstibine                       &    8.48   &     9.90   &    1.4\\
 C$_6$Ge$_2$H$_1$$_8$ & GeMe3-GeMe3                            &    8.18   &     9.59   &    1.4\\
 NOF                  & Nitrosyl fluoride                      &   12.94   &    11.54   &   -1.4\\
 C$_8$PbH$_2$$_0$     & Tetraethyllead                         &    8.13   &     9.54   &    1.4\\
 PbTe                 & Lead telluride                         &    8.04   &     9.44   &    1.4\\
 SnBr$_2$             & Tin dibromide                          &    9.80   &    11.19   &    1.4\\
 C$_4$PbH$_1$$_2$     & Tetramethyllead                        &    8.90   &    10.27   &    1.4\\
 C$_6$GeSnH$_1$$_8$   & GeMe3-SnMe3                            &    8.20   &     9.55   &    1.4\\
\hline
\end{tabular}
\end{center}
\end{table}

\clearpage
\begin{table}
\caption{\label{eleips} Average Errors (eV) in Ionization Potential by Element}
\begin{center}
\begin{tabular}{lrrrrrrrrr}
Element & \multicolumn{3}{c}{PM3} & \multicolumn{3}{c}{MNDO} & \multicolumn{3}{c}{AM1} \\
 & No. & Signed &    RMS    & No. & Signed &    RMS    & No. & Signed &    RMS   \\ 
 \hline
  H & 244 &    0.623 &    0.848 & 232 &    0.818 &    1.097 & 221 &    0.550 &    0.744 \\
 Li &     &          &          &   1 &    0.120 &    0.120 &     &          &          \\
 Be &   8 &    0.844 &    0.927 &   8 &    0.565 &    0.788 &   8 &    0.586 &    0.767 \\
  B &     &          &          &   6 &    0.765 &    0.882 &   6 &    0.793 &    0.823 \\
  C & 253 &    0.587 &    0.769 & 241 &    0.819 &    1.041 & 231 &    0.575 &    0.784 \\
  N &  45 &    0.616 &    0.879 &  46 &    0.705 &    1.026 &  46 &    0.608 &    0.843 \\
  O &  79 &    0.706 &    0.983 &  79 &    0.876 &    1.339 &  77 &    0.770 &    1.094 \\
  F &  67 &    0.692 &    0.935 &  64 &    0.695 &    1.104 &  62 &    0.754 &    0.992 \\
 Mg &   5 &    0.932 &    1.271 &     &          &          &     &          &          \\
 Al &   2 &    0.485 &    0.501 &   2 &    0.995 &    0.998 &   2 &    0.520 &    0.522 \\
 Si &  11 &    0.694 &    0.868 &  11 &    0.871 &    1.050 &  11 &    0.680 &    0.822 \\
  P &  17 &    0.774 &    0.922 &  17 &    1.426 &    1.621 &  17 &    0.902 &    1.192 \\
  S &  48 &    0.573 &    0.900 &  48 &    0.833 &    1.204 &  47 &    0.828 &    1.203 \\
 Cl &  51 &    0.691 &    0.863 &  47 &    1.321 &    1.578 &  43 &    0.752 &    1.158 \\
 Zn &   5 &    0.986 &    1.063 &   5 &    1.086 &    1.120 &   5 &    0.492 &    0.517 \\
 Ga &   3 &    1.813 &    2.457 &     &          &          &     &          &          \\
 Ge &  13 &    0.914 &    1.014 &  13 &    1.437 &    1.517 &  12 &    1.103 &    1.258 \\
 As &   5 &    0.616 &    0.677 &     &          &          &     &          &          \\
 Se &   6 &    0.517 &    0.562 &     &          &          &     &          &          \\
 Br &  27 &    1.191 &    1.620 &  24 &    1.136 &    1.198 &  20 &    0.639 &    0.878 \\
 Cd &   5 &    0.386 &    0.614 &     &          &          &     &          &          \\
 In &   2 &    2.065 &    2.763 &     &          &          &     &          &          \\
 Sn &  12 &    1.378 &    1.691 &  12 &    1.837 &    2.127 &     &          &          \\
 Sb &   5 &    1.188 &    1.250 &     &          &          &     &          &          \\
 Te &   2 &    0.890 &    1.026 &     &          &          &     &          &          \\
  I &  30 &    0.443 &    0.537 &  26 &    1.292 &    1.350 &  25 &    1.083 &    1.134 \\
 Hg &  12 &    0.727 &    0.946 &  12 &    1.116 &    1.157 &  12 &    0.522 &    0.673 \\
 Tl &   2 &    1.220 &    1.220 &     &          &          &     &          &          \\
 Pb &  12 &    1.068 &    1.192 &  12 &    1.343 &    1.433 &     &          &          \\
 Bi &   4 &    0.632 &    0.914 &     &          &          &     &          &          \\
\hline
Totals: & 375  & 0.7 &      1.0 & 343 &     0.9  &      1.2 & 319 &      0.7 &  0.9     \\ 
\hline
\end{tabular}
\end{center}
\end{table}

\clearpage
\begin{table}
\caption{\label{wedmndo}Worst Dipole Moment Errors in MNDO}
\compresstable
\begin{center}
\begin{tabular}{llrrr}
 Formula & Chemical Name & Exp.\ Dipole & Calc.\ Dipole & Diff.\\
 \hline
 GeS                  & Germanium sulfide                      &    2.00   &     5.48   &    3.5\\
 H$_2$SiCl$_2$        & Dichlorosilane                         &    1.18   &     3.49   &    2.3\\
 SO$_2$               & Sulfur dioxide                         &    1.58   &     3.73   &    2.1\\
 GeF$_2$              & Germanium difluoride                   &    2.61   &     4.58   &    2.0\\
 GeH$_3$Br            & Bromogermane                           &    1.70   &     3.65   &    2.0\\
 H$_2$SiBr$_2$        & Dibromosilane                          &    1.43   &     3.33   &    1.9\\
 HSiCl$_3$            & Trichlorosilane                        &    0.86   &     2.73   &    1.9\\
 CGeH$_3$N            & Cyanogermane                           &    3.99   &     2.19   &   -1.8\\
 C$_2$H$_3$N          & Methyl isocyanide                      &    3.85   &     2.17   &   -1.7\\
 C$_2$GeH$_3$Br$_3$   & Tribromovinylgermane                   &    2.47   &     4.13   &    1.7\\
 CHgH$_3$I            & Methylmercuric iodide                  &    1.30   &     2.88   &    1.6\\
 HSiF$_3$             & Trifluorosilane                        &    1.27   &     2.80   &    1.5\\
 GeH$_3$Cl            & Chlorogermane                          &    2.10   &     3.59   &    1.5\\
 CH$_6$N$_2$          & Methylhydrazine                        &    1.66   &     0.23   &   -1.4\\
 C$_4$B$_2$H$_6$      & C4B2H6                                 &    2.26   &     0.83   &   -1.4\\
 PF$_3$               & Phosphorus trifluoride                 &    1.03   &     2.44   &    1.4\\
 SnO                  & Tin oxide                              &    4.30   &     2.93   &   -1.4\\
 GeH$_3$I             & Iodogermane                            &    1.81   &     3.17   &    1.4\\
 NOF                  & Nitrosyl fluoride                      &    1.81   &     0.51   &   -1.3\\
 C$_2$H$_3$N          & Acetonitrile                           &    3.92   &     2.63   &   -1.3\\
 CNF                  & Cyanogen fluoride                      &    2.17   &     0.89   &   -1.3\\
 HgCl$_2$             & Mercury dichloride                     &    1.23   &     0.00   &   -1.2\\
 AlF                  & Aluminum fluoride                      &    1.53   &     0.31   &   -1.2\\
 C$_4$GeH$_1$$_2$O$_3$ & Trimethoxymethylgermane                &    1.91   &     0.70   &   -1.2\\
 C$_2$HgH$_3$N        & Methylmercuric cyanide                 &    4.69   &     3.53   &   -1.2\\
 CBH$_3$O             & BH3CO                                  &    1.80   &     2.96   &    1.2\\
 CF$_3$I              & Trifluoroiodomethane                   &    1.00   &     2.13   &    1.1\\
 CH$_6$SiS            & CH3-S-SiH3                             &    1.38   &     2.47   &    1.1\\
 C$_3$BH$_1$$_2$N     & Trimethyamine-Borane adduct            &    4.84   &     5.87   &    1.0\\
 C$_2$BH$_6$Cl        & dimethylchloroborane                   &    0.86   &     1.89   &    1.0\\
 PbO                  & Lead oxide                             &    4.64   &     3.62   &   -1.0\\
 C$_3$SnH$_9$I        & Trimethyltin iodide                    &    3.37   &     2.37   &   -1.0\\
 HgBr$_2$             & Mercury dibromide                      &    0.95   &     0.00   &   -1.0\\
 C$_2$H$_3$NS         & Methyl isothiocyanate                  &    4.03   &     3.10   &   -0.9\\
 H$_4$P$_2$           & P2H4                                   &    0.92   &     0.01   &   -0.9\\
 C$_3$H$_3$N          & Acrylonitrile                          &    3.87   &     2.97   &   -0.9\\
 C$_2$NF$_3$          & Trifluoroacetonitrile                  &    1.26   &     0.36   &   -0.9\\
 C$_2$HF              & Fluoroacetylene                        &    0.70   &     1.58   &    0.9\\
 C$_2$GeH$_4$         & Germylacetylene                        &    0.14   &     0.94   &    0.8\\
 C$_2$GeH$_6$Cl$_2$   & Dimethylgermanium dichloride           &    3.14   &     3.94   &    0.8\\
 BH                   & BH                                     &    1.27   &     0.49   &   -0.8\\
 C$_6$H$_4$F$_2$      & o-Difluorobenzene                      &    2.59   &     3.36   &    0.8\\
 CGeH$_3$Cl$_3$       & Trichloromethylgermane                 &    2.70   &     3.48   &    0.8\\
 H$_3$P               & Phosphine                              &    0.58   &     1.35   &    0.8\\
 GeH$_3$N$_3$         & Germylazide                            &    2.58   &     1.84   &   -0.7\\
 C$_3$H$_9$P          & Trimethylphosphine                     &    1.19   &     1.93   &    0.7\\
 CH$_2$PF             & CH2=P-F                                &    1.35   &     2.09   &    0.7\\
 C$_2$H$_8$Si         & Ethylsilane                            &    0.81   &     0.10   &   -0.7\\
 GeH$_3$F             & Fluorogermane                          &    2.30   &     1.58   &   -0.7\\
 C$_2$SnH$_6$Cl$_2$   & Dimethyltin dichloride                 &    4.41   &     3.70   &   -0.7\\
\end{tabular}
\end{center}
\end{table}

\clearpage
\begin{table}
\caption{\label{wedam1}Worst Dipole Moment Errors in AM1}
\compresstable
\begin{center}
\begin{tabular}{llrrr}
 Formula & Chemical Name & Exp.\ Dipole       & Calc.\ Dipole    & Diff.\\
 \hline
 SO$_2$               & Sulfur dioxide                         &    1.58   &     4.29   &    2.7\\
 GeS                  & Germanium sulfide                      &    2.00   &     4.25   &    2.2\\
 C$_4$B$_2$H$_6$      & C4B2H6                                 &    2.26   &     0.11   &   -2.1\\
 CHgH$_3$I            & Methylmercuric iodide                  &    1.30   &     3.10   &    1.8\\
 H$_3$P               & Phosphine                              &    0.58   &     2.29   &    1.7\\
 CS                   & Carbon sulfide                         &    0.68   &     2.24   &    1.6\\
 NOF                  & Nitrosyl fluoride                      &    1.81   &     0.38   &   -1.4\\
 C$_5$BeH$_6$         & Cyclopentadienylberyllium hydride      &    2.08   &     0.72   &   -1.4\\
 C$_2$H$_3$NS         & Methyl isothiocyanate                  &    4.03   &     2.73   &   -1.3\\
 C$_2$NF$_3$          & Trifluoroacetonitrile                  &    1.26   &     0.03   &   -1.2\\
 C$_2$B$_4$H$_6$      & C2B4H6                                 &    1.50   &     0.27   &   -1.2\\
 HgCl$_2$             & Mercury dichloride                     &    1.23   &     0.00   &   -1.2\\
 C$_2$H$_3$N          & Acetonitrile                           &    3.92   &     2.89   &   -1.0\\
 C$_2$H$_3$N          & Methyl isocyanide                      &    3.85   &     2.83   &   -1.0\\
 C$_2$BH$_6$Cl        & dimethylchloroborane                   &    0.86   &     1.84   &    1.0\\
 C$_4$GeH$_1$$_2$O$_3$ & Trimethoxymethylgermane                &    1.91   &     0.93   &   -1.0\\
 CNF                  & Cyanogen fluoride                      &    2.17   &     1.21   &   -1.0\\
 HgBr$_2$             & Mercury dibromide                      &    0.95   &     0.00   &   -1.0\\
 H$_4$P$_2$           & P2H4                                   &    0.92   &     0.00   &   -0.9\\
 CH$_5$P              & Methylphosphine                        &    1.10   &     2.02   &    0.9\\
 CH$_6$N$_2$          & Methylhydrazine                        &    1.66   &     0.77   &   -0.9\\
 H$_2$S               & Hydrogen sulfide                       &    0.97   &     1.86   &    0.9\\
 POF$_3$              & Phosphorus oxyfluoride                 &    1.76   &     2.64   &    0.9\\
 C$_3$H$_3$N          & Acrylonitrile                          &    3.87   &     3.00   &   -0.9\\
 CGeH$_3$N            & Cyanogermane                           &    3.99   &     3.14   &   -0.9\\
 HI                   & Hydrogen iodide                        &    0.44   &     1.27   &    0.8\\
 CGeH$_3$F$_3$        & Methyltrifluorogermane                 &    3.80   &     2.98   &   -0.8\\
 OBr                  & BrO                                    &    1.61   &     2.36   &    0.7\\
 GeH$_3$N$_3$         & Germylazide                            &    2.58   &     1.89   &   -0.7\\
 C$_3$HN              & CH.C.CN                                &    3.72   &     3.04   &   -0.7\\
 CF$_3$I              & Trifluoroiodomethane                   &    1.00   &     1.67   &    0.7\\
 O$_3$                & Ozone                                  &    0.53   &     1.20   &    0.7\\
 GeO                  & Germanium oxide                        &    3.28   &     2.62   &   -0.7\\
 C$_2$HgH$_5$Cl       & Ethylmercuric chloride                 &    2.99   &     3.65   &    0.7\\
 CH$_6$SiS            & CH3-S-SiH3                             &    1.38   &     2.03   &    0.6\\
 OCl                  & Cl-O                                   &    1.24   &     1.88   &    0.6\\
 PF$_3$               & Phosphorus trifluoride                 &    1.03   &     1.66   &    0.6\\
 C$_2$HgH$_3$N        & Methylmercuric cyanide                 &    4.69   &     4.06   &   -0.6\\
 CHN                  & Hydrogen cyanide                       &    2.98   &     2.36   &   -0.6\\
 HOF                  & Hypofluorous acid                      &    2.23   &     1.60   &   -0.6\\
 C$_3$GeH$_1$$_0$     & Trimethylgermane                       &    0.67   &     0.05   &   -0.6\\
 GeH$_3$Br            & Bromogermane                           &    1.70   &     2.31   &    0.6\\
 C$_3$H$_4$O          & Acrolein                               &    3.12   &     2.53   &   -0.6\\
 H$_2$SiCl$_2$        & Dichlorosilane                         &    1.18   &     1.76   &    0.6\\
 C$_2$HBr             & Bromoacetylene                         &    0.00   &     0.57   &    0.6\\
 CBH$_3$OF$_2$        & Methoxydifluoroborane                  &    2.62   &     2.05   &   -0.6\\
 C$_2$H$_7$P          & Dimethylphosphine                      &    1.23   &     1.79   &    0.6\\
 CHOF                 & HCOF                                   &    2.02   &     2.57   &    0.6\\
 HBr                  & Hydrogen bromide                       &    0.83   &     1.38   &    0.6\\
 HSiCl$_3$            & Trichlorosilane                        &    0.86   &     1.42   &    0.6\\
\end{tabular}
\end{center}
\end{table}

\clearpage
\begin{table}
\caption{\label{wedpm3}Worst Dipole Moment Errors in PM3}
\compresstable
\begin{center}
\begin{tabular}{llrrr}
 Formula & Chemical Name & Exp.\ Dipole       & Calc.\ Dipole       & Diff.\\
 \hline
 OCl                  & Cl-O                                   &    1.24   &     3.67   &    2.4\\
 SO$_2$               & Sulfur dioxide                         &    1.58   &     3.63   &    2.0\\
 AlF                  & Aluminum fluoride                      &    1.53   &     3.29   &    1.8\\
 OBr                  & BrO                                    &    1.61   &     3.32   &    1.7\\
 C$_3$SnH$_9$Br       & Trimethyltin bromide                   &    3.45   &     5.14   &    1.7\\
 H$_2$SiBr$_2$        & Dibromosilane                          &    1.43   &     3.09   &    1.7\\
 NOF                  & Nitrosyl fluoride                      &    1.81   &     0.26   &   -1.6\\
 GeS                  & Germanium sulfide                      &    2.00   &     0.48   &   -1.5\\
 H$_2$SiCl$_2$        & Dichlorosilane                         &    1.18   &     2.69   &    1.5\\
 HSiCl$_3$            & Trichlorosilane                        &    0.86   &     2.38   &    1.5\\
 CHgH$_3$I            & Methylmercuric iodide                  &    1.30   &     2.75   &    1.4\\
 SnO                  & Tin oxide                              &    4.30   &     2.92   &   -1.4\\
 GaF                  & Gallium fluoride                       &    2.50   &     3.85   &    1.4\\
 CH$_6$N$_2$          & Methylhydrazine                        &    1.66   &     0.32   &   -1.3\\
 InF                  & Indium fluoride                        &    3.40   &     2.08   &   -1.3\\
 C$_3$SnH$_9$I        & Trimethyltin iodide                    &    3.37   &     4.65   &    1.3\\
 H$_4$P$_2$           & P2H4                                   &    0.92   &     2.19   &    1.3\\
 CSeS                 & Thiocarbonyl selenide                  &    0.03   &     1.29   &    1.3\\
 HgCl$_2$             & Mercury dichloride                     &    1.23   &     0.00   &   -1.2\\
 PF$_3$               & Phosphorus trifluoride                 &    1.03   &     2.25   &    1.2\\
 GeH$_3$Br            & Bromogermane                           &    1.70   &     2.92   &    1.2\\
 O$_3$                & Ozone                                  &    0.53   &     1.71   &    1.2\\
 SeH$_2$              & Hydrogen selenide                      &    0.24   &     1.35   &    1.1\\
 C$_2$GeH$_3$Br$_3$   & Tribromovinylgermane                   &    2.47   &     3.53   &    1.1\\
 C$_3$GeH$_9$Br       & Trimethylbromogermane                  &    2.84   &     3.81   &    1.0\\
 C$_4$SeH$_4$         & Selenophene                            &    0.37   &     1.34   &    1.0\\
 HgBr$_2$             & Mercury dibromide                      &    0.95   &     0.00   &   -1.0\\
 CGeH$_3$F$_3$        & Methyltrifluorogermane                 &    3.80   &     2.85   &   -0.9\\
 C$_2$NF$_3$          & Trifluoroacetonitrile                  &    1.26   &     0.32   &   -0.9\\
 C$_6$H$_5$I          & Iodobenzene                            &    1.70   &     0.79   &   -0.9\\
 SeO$_2$              & Selenium dioxide                       &    2.70   &     3.61   &    0.9\\
 GeF$_2$              & Germanium difluoride                   &    2.61   &     1.71   &   -0.9\\
 PbO                  & Lead oxide                             &    4.64   &     3.77   &   -0.9\\
 SeOF$_2$             & Seleninyl difluoride                   &    2.84   &     3.70   &    0.9\\
 FBr                  & Bromine fluoride                       &    1.42   &     2.25   &    0.8\\
 H$_2$S               & Hydrogen sulfide                       &    0.97   &     1.78   &    0.8\\
 GeH$_3$I             & Iodogermane                            &    1.81   &     2.59   &    0.8\\
 C$_3$H$_4$O          & Acrolein                               &    3.12   &     2.36   &   -0.8\\
 C$_3$H$_7$NO         & Dimethylformamide                      &    3.82   &     3.06   &   -0.8\\
 CS                   & Carbon sulfide                         &    0.68   &     1.42   &    0.7\\
 C$_2$H$_3$N          & Acetonitrile                           &    3.92   &     3.21   &   -0.7\\
 PbS                  & Lead sulfide                           &    3.59   &     2.94   &   -0.6\\
 C$_3$H$_3$N          & Acrylonitrile                          &    3.87   &     3.25   &   -0.6\\
 C$_2$H$_3$NS         & Methyl isothiocyanate                  &    4.03   &     3.42   &   -0.6\\
 C$_2$HgH$_3$N        & Methylmercuric cyanide                 &    4.69   &     4.08   &   -0.6\\
 CH$_6$SiS            & CH3-S-SiH3                             &    1.38   &     1.99   &    0.6\\
 H$_3$P               & Phosphine                              &    0.58   &     1.18   &    0.6\\
 C$_2$H$_6$S$_2$      & 2,3-Dithiabutane                       &    1.98   &     2.57   &    0.6\\
 C$_2$HgH$_5$Br       & Ethylmercuric bromide                  &    2.80   &     3.39   &    0.6\\
 CF$_3$I              & Trifluoroiodomethane                   &    1.00   &     1.55   &    0.6\\
\end{tabular}
\end{center}
\end{table}

\begin{table}
\caption{\label{eledip}  \  Average Errors (Debye) in Dipole Moment by Element}
\begin{center}
\begin{tabular}{lrrrrrrrrr}
Element & \multicolumn{3}{c}{PM3} & \multicolumn{3}{c}{MNDO} & \multicolumn{3}{c}{AM1} \\
 & No. & Signed &   RMS     & No. & Signed &   RMS     & No. & Signed &   RMS    \\
\hline
  H & 160 &    0.371 &    0.508 & 159 &    0.495 &    0.678 & 151 &    0.368 &    0.514 \\
 Be &   1 &    0.250 &    0.250 &   1 &    0.070 &    0.070 &   1 &    1.360 &    1.360 \\
  B &     &          &          &  11 &    0.686 &    0.798 &  11 &    0.615 &    0.855 \\
  C & 144 &    0.338 &    0.455 & 145 &    0.434 &    0.585 & 137 &    0.359 &    0.517 \\
  N &  34 &    0.401 &    0.529 &  35 &    0.596 &    0.778 &  35 &    0.520 &    0.640 \\
  O &  51 &    0.460 &    0.703 &  51 &    0.410 &    0.584 &  49 &    0.355 &    0.576 \\
  F &  42 &    0.480 &    0.650 &  41 &    0.566 &    0.730 &  41 &    0.371 &    0.498 \\
 Al &   1 &    1.760 &    1.760 &   1 &    1.220 &    1.220 &   1 &    0.530 &    0.530 \\
 Si &  11 &    0.638 &    0.867 &  11 &    1.025 &    1.259 &  11 &    0.335 &    0.389 \\
  P &   8 &    0.456 &    0.667 &   8 &    0.778 &    0.823 &   8 &    0.750 &    0.881 \\
  S &  18 &    0.681 &    0.825 &  17 &    0.637 &    1.090 &  15 &    0.726 &    1.100 \\
 Cl &  23 &    0.505 &    0.774 &  21 &    0.693 &    0.911 &  18 &    0.391 &    0.504 \\
 Zn &   4 &    0.153 &    0.210 &   4 &    0.155 &    0.212 &   4 &    0.150 &    0.209 \\
 Ga &   1 &    1.350 &    1.350 &     &          &          &     &          &          \\
 Ge &  22 &    0.490 &    0.649 &  22 &    1.060 &    1.294 &  22 &    0.544 &    0.700 \\
 As &   7 &    0.311 &    0.342 &     &          &          &     &          &          \\
 Se &  10 &    0.695 &    0.778 &     &          &          &     &          &          \\
 Br &  26 &    0.592 &    0.778 &  24 &    0.505 &    0.747 &  23 &    0.323 &    0.403 \\
 In &   2 &    0.720 &    0.937 &     &          &          &     &          &          \\
 Sn &   9 &    0.699 &    0.891 &   9 &    0.467 &    0.636 &     &          &          \\
 Sb &   1 &    0.420 &    0.420 &     &          &          &     &          &          \\
 Te &   1 &    0.040 &    0.040 &     &          &          &     &          &          \\
  I &  14 &    0.521 &    0.668 &  12 &    0.654 &    0.812 &  11 &    0.519 &    0.692 \\
 Hg &   9 &    0.631 &    0.777 &   9 &    0.674 &    0.857 &   9 &    0.676 &    0.866 \\
 Tl &   3 &    0.347 &    0.350 &     &          &          &     &          &          \\
 Pb &   4 &    0.398 &    0.544 &   3 &    0.470 &    0.621 &     &          &          \\
 Totals: & 212  & 0.4  & 0.6   &  197   &0.5   &0.8   &  185   &0.4  & 0.6 \\ 
\hline
\end{tabular}
\end{center}
\end{table}

\clearpage

\section{Comparison of Calc'd and Observed $\Delta H_f$ for MNDO, AM1, and PM3}
The following Table lists experimental $\Delta H_f$, in kcal/mol, for  1200
species.  In order to save space, only the difference between the calculated
and observed $\Delta H_f$ is reported.  The calculated $\Delta H_f$ is,
however, easily obtained by adding the difference to the experimental value. 
Thus, the $\Delta H_f$ of  methane, CH$_4$,  calculated using MNDO, is
5.95+(--17.89) = --11.96 kcal/mol.

All the results reported in this and the following Table can be easily
reproduced using MOPAC.  In some cases, care must be taken to ensure that the
correct spin state and symmetry are specified.  For example, triplet oxygen
must be specified using \comp{OPEN(2,2) TRIPLET}, in order for the correct
$\Delta H_f$ to be obtained.   Most open-shell systems will not be indicated as
open-shell.

Results for the iodine atom are missing.  Although MNDO, AM1, and PM3 all give
a zero error, the PM3 result is incorrect.  The correct PM3 result for the
$\Delta H_f$ of iodine atom is -46.87 kcal/mol.  This can be easily checked by
running an iodine atom using keywords \comp{PM3 OPEN(7,4)}.

Many reference data are of highly questionable accuracy, for example the sulfur
halides, and some other reference data refer to highly exotic species, e.g.\
N$^{2+}$. Results for calculations on all these systems are reported, as
required by the protocols for determining accuracy of methods.  If only those
systems for which accurate data were used, and the choice of system limited to
conventional molecules, then the average errors reported in the earlier Tables
would decrease markedly.

\subsection*{Notes on the Table}
\begin{itemize}
\item The $\Delta H_f$ of COI is probably incorrect.  Given that the
experimental $\Delta H_f$s of CO and I are -26.4 and +25.5 kcal/mol,
respectively, and  the $\Delta H_f$ of COI is 63.6 kcal/mol, then the CO-I bond
has a `strength' of -64.5 kcal/mol.  This is unlikely.
\item $\Delta H_f$ for the elements are included. Isolated atoms are used in
the NDDO methods as the standard, and the defined $\Delta H_f$ should be
exactly zero.  For PM3 several elements have negative $\Delta H_f$.  This is a
result of  PM3 incorrectly predicting the atomic configuration.
\end{itemize}

\begin{table}
\caption{\label{hoftab} Comparison of Calculated and Observed $\Delta H_f$ 
for MNDO, AM1, and PM3}
\begin{center}
\compresstable
\begin{tabular}{llrrrrr}
Empirical & Chemical Name & $\Delta H_f$ & \multicolumn{3}{c}{Difference} & \\
Formula   &               & Exp. & PM3 &  MNDO  &  AM1 &     Ref.\\
\hline

 H           & Hydrogen (+)                   &   365.72    &   -12.14  &   -39.05  &   -50.81  &      a\\
 H$_2$          & Hydrogen                       &     0.00    &   -13.39  &     0.72  &    -5.18  &      b\\
 CH          & Methylidyne                    &   142.40    &     4.43  &     1.18  &     2.63  &      c\\
 CH$_2$         & Methylene, triplet             &    92.30    &   -16.66  &   -15.01  &   -11.50  &      d\\
 CH$_2$         & Methylene, singlet             &    99.80    &    13.42  &     7.56  &    11.05  &      d\\
 CH$_3$         & Methyl (+)                     &   261.00    &    -4.46  &   -17.13  &    -8.66  &      e\\
 CH$_3$         & Methyl radical                 &    34.80    &    -5.04  &    -9.01  &    -3.56  &      d\\
 CH$_4$         & Methane                        &   -17.89    &     4.86  &     5.93  &     9.10  &      f\\
 C$_2$          & Carbon, dimer                  &   200.20    &    58.01  &    35.21  &    16.03  &      g\\
 C$_2$H$_2$        & Acetylene                      &    54.34    &    -3.65  &     3.53  &     0.44  &      f\\
 C$_2$H$_3$        & Vinyl (+)                      &   266.00    &    -2.09  &    -0.34  &    -4.57  &      h\\
 C$_2$H$_3$        & Vinyl                          &    59.60    &     3.66  &     4.24  &     5.15  &      h\\
 C$_2$H$_4$        & Ethylene                       &    12.45    &     4.16  &     2.93  &     4.00  &      f\\
 C$_2$H$_4$        & Ethylene (+)                   &   257.00    &    -8.31  &   -18.71  &   -13.17  &      i\\
 C$_2$H$_4$        & Methylmethylene                &    90.30    &    -1.69  &    -1.95  &    -2.63  &      j\\
 C$_2$H$_5$        & Ethyl (+)  (classical)         &   216.00    &     6.44  &     3.64  &     0.77  &      k\\
 C$_2$H$_5$        & Ethyl radical                  &    25.00    &    -7.69  &   -12.20  &    -6.89  &      l\\
 C$_2$H$_6$        & Ethane                         &   -20.24    &     2.08  &     0.49  &     2.80  &      f\\
 C$_3$          & Carbon, trimer                     196.00    &    10.63  &    24.30  &    16.43  &      d\\
 C$_3$H$_3$        & Propynyl (+)                   &   281.00    &    -5.70  &   -15.60  &    -7.36  &      i\\
 C$_3$H$_3$        & Cyclopropenyl (+)              &   257.00    &    12.81  &    15.52  &    19.37  &      i\\
 C$_3$H$_4$        & Allene                         &    45.63    &     1.40  &    -1.74  &     0.48  &      f\\
 C$_3$H$_4$        & Cyclopropene                   &    66.20    &     1.94  &     2.07  &     8.58  &      f\\
 C$_3$H$_4$        & Propyne                        &    44.39    &    -4.20  &    -3.03  &    -1.02  &      f\\
 C$_3$H$_5$        & Allyl (+)                      &   226.00    &     6.66  &    -4.65  &     0.18  &      i\\
 C$_3$H$_5$        & Propenyl (+)                   &   237.00    &     1.20  &     3.06  &    -3.36  &      i\\
 C$_3$H$_5$        & Allyl                          &    40.00    &    -0.41  &    -4.67  &    -1.45  &      h\\
 C$_3$H$_5$        & Cyclopropyl (+)                &   235.00    &    26.78  &    23.15  &    25.57  &      i\\
 C$_3$H$_6$        & Cyclopropane                   &    12.73    &     3.50  &    -1.55  &     5.01  &      f\\
 C$_3$H$_6$        & Propene                        &     4.88    &     1.49  &     0.07  &     1.66  &      f\\
 C$_3$H$_7$        & Propyl (+)                     &   208.00    &     6.33  &     4.32  &    -0.28  &      i\\
 C$_3$H$_7$        & 2-Propyl (+)                   &   187.00    &    10.22  &    13.65  &     4.83  &      k\\
 C$_3$H$_7$        & i-Propyl radical               &    16.80    &   -11.57  &   -15.32  &   -10.20  &      l\\
 C$_3$H$_8$        & Propane                        &   -24.83    &     1.17  &    -0.15  &     0.53  &      f\\
 C$_4$          & Carbon, tetramer                   232.00    &    48.29  &    52.71  &    47.85  &      d\\
 C$_4$H$_2$        & Diacetylene                    &   113.00    &   -10.58  &    -9.85  &    -6.96  &      m\\
 C$_4$H$_4$        & Vinylacetylene                 &    72.80    &    -6.47  &    -7.26  &    -4.91  &      m\\
 C$_4$H$_6$        & 1-Methylcycloprop-1-ene        &    58.20    &    -0.87  &    -4.55  &     6.41  &      f\\
 C$_4$H$_6$        & Bicyclobutane                  &    51.90    &    17.29  &    12.11  &    26.15  &      f\\
 C$_4$H$_6$        & Methylenecyclopropane          &    47.92    &    -3.45  &   -10.07  &    -0.31  &      f\\
\hline
\end{tabular}
\end{center}
\end{table}
\clearpage

\begin{table}
\caption{Comparison of Calculated and Observed $\Delta H_f$ 
for MNDO, AM1, and PM3 (contd.)}
\begin{center}
\compresstable
\begin{tabular}{llrrrrr}
Empirical & Chemical Name & $\Delta H_f$ & \multicolumn{3}{c}{Difference} & \\
Formula   &               & Exp. & PM3 &  MNDO  &  AM1 &     Ref.\\
\hline
 C$_4$H$_6$        & 1,2-Butadiene                  &    38.80    &    -0.85  &    -5.30  &    -1.73  &      f\\
 C$_4$H$_6$        & 1-Butyne                       &    39.50    &    -3.86  &    -3.40  &    -2.05  &      f\\
 C$_4$H$_6$        & 2-Butyne                       &    34.70    &    -4.97  &    -9.85  &    -2.77  &      f\\
 C$_4$H$_6$        & Cyclobutene                    &    37.45    &     0.17  &    -6.48  &     8.26  &      n\\
 C$_4$H$_6$        & 1,3-Butadiene                  &    26.01    &     4.98  &     2.90  &     3.86  &      f\\
 C$_4$H$_7$        & 2-Butenyl (+)                  &   200.00    &    12.51  &     6.91  &     6.23  &      i\\
 C$_4$H$_7$        & Cyclobutyl (+)                 &   213.00    &    12.55  &     8.35  &    13.11  &      i\\
 C$_4$H$_8$        & 1-Butene                       &    -0.20    &     1.93  &     0.53  &     0.56  &      f\\
 C$_4$H$_8$        & cis-2-Butene                   &    -1.86    &    -0.65  &    -2.52  &    -0.32  &      f\\
 C$_4$H$_8$        & Cyclobutane                    &     6.78    &   -10.62  &   -18.73  &    -7.82  &      f\\
 C$_4$H$_8$        & Isobutene                      &    -4.30    &     0.93  &     2.25  &     3.10  &      f\\
 C$_4$H$_8$        & trans-2-Butene                 &    -3.00    &    -0.82  &    -2.14  &    -0.40  &      f\\
 C$_4$H$_9$        & n-Butyl (+)                    &   201.00    &     6.95  &     5.83  &    -1.10  &      i\\
 C$_4$H$_9$        & 1-Methyl propyl (+)            &   183.00    &     7.72  &    11.29  &     0.91  &      i\\
 C$_4$H$_9$        & Isobutyl                       &     4.50    &   -10.45  &   -11.33  &    -7.21  &      l\\
 C$_4$H$_9$        & Isobutyl (+)                   &   176.00    &     2.70  &    11.92  &    -1.26  &      i\\
 C$_4$H$_1$$_0$       & n-Butane                       &   -30.36    &     1.25  &     0.60  &    -0.82  &      f\\
 C$_4$H$_1$$_0$       & Isobutane                      &   -32.41    &     2.83  &     5.58  &     2.99  &      f\\
 C$_5$H$_5$        & Cyclopentadienyl (--)            &    21.30    &    -5.45 &    -2.43  &     3.81  &      o\\
 C$_5$H$_6$        & Cyclopentadiene                 &    32.10    &    -0.40 &    -0.08  &     4.90  &      n\\
 C$_5$H$_8$        & 1,2-Dimethylcyclopropene        &    46.40    &     0.26  &    -7.24  &     8.13  &      p\\
 C$_5$H$_8$        & Methylene cyclobutane           &    29.10    &    -9.46  &   -18.27  &    -4.03  &      q\\
 C$_5$H$_8$        & 1,cis-3-Pentadiene              &    19.10    &     1.91  &     0.90  &     1.78  &      f\\
 C$_5$H$_8$        & Cyclopentene                    &     8.30    &    -5.34  &    -8.70  &    -5.39  &      n\\
 C$_5$H$_8$        & Bicyclo(2.1.0)-pentane          &    37.30    &     0.44  &    -7.17  &     8.69  &      p\\
 C$_5$H$_8$        & 1,4-Pentadiene                  &    25.30    &     1.22  &    -0.80  &    -0.57  &      f\\
 C$_5$H$_8$        & Spiropentane                    &    44.30    &    -1.25  &   -10.68  &     6.09  &      f\\
 C$_5$H$_8$        & 1,trans-3-Pentadiene            &    18.10    &     3.11  &     1.22  &     2.42  &      f\\
 C$_5$H$_9$        & Cyclopentyl (+)                 &   188.00    &     5.42  &     6.14  &    -2.14  &      r\\
 C$_5$H$_1$$_0$       & 1-Pentene                       &    -5.30    &     1.23  &     0.20  &    -1.44  &      f\\
 C$_5$H$_1$$_0$       & 2-Methyl-1-butene               &    -8.60    &     0.67  &     2.00  &     1.71  &      f\\
 C$_5$H$_1$$_0$       & 2-Methyl-2-butene               &   -10.10    &    -2.15  &    -0.45  &     0.09  &      f\\
 C$_5$H$_1$$_0$       & 3-Methyl-1-butene               &    -6.60    &     2.60  &     4.25  &     2.16  &      f\\
 C$_5$H$_1$$_0$       & cis-2-Pentene                   &    -7.00    &    -0.78  &    -1.86  &    -2.01  &      f\\
 C$_5$H$_1$$_0$       & cis-Dimethylcyclopropane        &     1.30    &     0.05  &    -3.47  &     3.40  &      p\\
 C$_5$H$_1$$_0$       & Cyclopentane                    &   -18.30    &    -5.66  &   -12.24  &   -10.56  &      n\\
 C$_5$H$_1$$_0$       & trans-2-Pentene                 &    -7.90    &    -0.89  &    -2.46  &    -1.88  &      f\\
 C$_5$H$_1$$_1$       & 1-Pentyl (+)                    &   194.00    &     8.31  &     7.79  &    -1.25  &      i\\
 C$_5$H$_1$$_1$       & 2-Pentyl (+)                    &   173.00    &    11.59  &    15.93  &     3.33  &      i\\
 C$_5$H$_1$$_1$       & 2-Ethylisopropyl (+)            &   156.00    &    15.71  &    26.87  &    10.41  &      i\\
 C$_5$H$_1$$_1$       & Neopentyl (+)                   &   188.00    &    20.50  &    12.14  &    17.71  &      s\\
 C$_5$H$_1$$_2$       & 2-Methylbutane                  &   -36.80    &     2.49  &     6.71  &     1.34  &      f\\
 C$_5$H$_1$$_2$       & Neopentane                      &   -40.30    &     4.40  &    15.63  &     7.46  &      f\\
 C$_5$H$_1$$_2$       & n-Pentane                       &   -35.10    &     0.55  &     0.61  &    -2.94  &      f\\
 C$_6$H$_6$        & Benzene                         &    19.81    &     3.58  &     1.44  &     2.14  &      f\\
 C$_6$H$_6$        & Fulvene                         &    47.50    &     8.64  &     6.11  &    15.08  &      f\\
 C$_6$H$_8$        & 1,3-Cyclohexadiene              &    25.40    &    -5.06  &   -10.98  &    -7.93  &      f\\
 C$_6$H$_1$$_0$       & 2,3-Dimethyl-1,3-butadiene      &    10.80    &     3.16  &     4.28  &     6.47  &      f\\
\hline
\end{tabular}
\end{center}
\end{table}
\clearpage

\begin{table}
\caption{Comparison of Calculated and Observed $\Delta H_f$ 
for MNDO, AM1, and PM3 (contd.)}
\begin{center}
\compresstable
\begin{tabular}{llrrrrr}
Empirical & Chemical Name & $\Delta H_f$ & \multicolumn{3}{c}{Difference} & \\
Formula   &               & Exp. & PM3 &  MNDO  &  AM1 &     Ref.\\
\hline
 C$_6$H$_1$$_0$       & Cyclohexene                     &    -1.08    &    -3.87  &    -8.91  &    -9.05  &      f\\
 C$_6$H$_1$$_0$       & 1,5-Hexadiene                   &    20.10    &     0.90  &    -0.54  &    -2.37  &      f\\
 C$_6$H$_1$$_0$       & 1,2-Dimethylcyclobutene         &    19.80    &    -3.71  &   -13.46  &     7.07  &      p\\
 C$_6$H$_1$$_0$       & Bicyclopropyl                   &    30.90    &     5.10  &    -2.33  &     8.58  &      f\\
 C$_6$H$_1$$_1$       & 1-Methylcyclopentyl (+)         &   165.00    &     9.42  &    13.55  &     2.36  &      i\\
 C$_6$H$_1$$_1$       & Cyclohexyl (+)                  &   177.00    &     9.00  &     9.80  &    -2.85  &      r\\
 C$_6$H$_1$$_2$       & Cyclohexane                     &   -29.49    &    -1.62  &    -5.35  &    -9.13  &      n\\
 C$_6$H$_1$$_4$       & n-Hexane                        &   -39.92    &    -0.06  &     0.71  &    -4.98  &      f\\
 C$_7$H$_7$        & Benzyl (+)                      &   212.00    &    15.34  &     5.96  &    10.04  &      t\\
 C$_7$H$_7$        & Tropylium (+)                   &   209.00    &    11.92  &    -1.41  &     1.37  &      u\\
 C$_7$H$_8$        & Cycloheptatriene                &    43.20    &    -0.77  &    -9.47  &    -4.98  &      n\\
 C$_7$H$_8$        & Norbornadiene                   &    59.70    &    -1.01  &     3.10  &     7.91  &      p\\
 C$_7$H$_8$        & Toluene                         &    11.99    &     2.04  &     1.52  &     2.35  &      f\\
 C$_7$H$_1$$_1$       & 2-Norbornyl (+)                 &   182.00    &    26.37  &    30.99  &    20.88  &      i\\
 C$_7$H$_1$$_2$       & Norbornane                      &   -12.40    &    -1.34  &     1.98  &    -2.09  &      v\\
 C$_7$H$_1$$_6$       & n-Heptane                       &   -44.85    &    -0.56  &     0.91  &    -6.91  &      f\\
 C$_8$H$_8$        & Cubane                          &   148.70    &   -35.04  &   -49.76  &     2.41  &      f\\
 C$_8$H$_8$        & Styrene                         &    35.30    &     3.81  &     3.62  &     3.33  &      f\\
 C$_8$H$_1$$_0$       & Ethylbenzene                    &     7.15    &     2.21  &     1.50  &     1.42  &      f\\
 C$_8$H$_1$$_0$       & m-Xylene                        &     4.14    &     0.54  &     1.68  &     2.61  &      f\\
 C$_8$H$_1$$_0$       & o-Xylene                        &     4.56    &     0.88  &     3.67  &     2.78  &      f\\
 C$_8$H$_1$$_0$       & p-Xylene                        &     4.31    &     0.34  &     1.34  &     2.39  &      f\\
 C$_8$H$_1$$_4$       & Bicyclo(2.2.2)-octane           &   -24.10    &    -3.79  &    -2.36  &   -12.05  &      v\\
 C$_8$H$_1$$_8$       & 2,2,3,3-Tetramethylbutane       &   -53.83    &     9.38  &    44.92  &    17.41  &      f\\
 C$_8$H$_1$$_8$       & n-Octane                        &   -49.86    &    -0.99  &     1.20  &    -8.76  &      f\\
 C$_9$H$_2$$_0$       & 3,3-Diethylpentane              &   -55.41    &     3.90  &    28.35  &     4.03  &      f\\
 C$_9$H$_2$$_0$       & n-Nonane                        &   -54.66    &    -1.62  &     1.28  &   -10.82  &      f\\
 C$_1$$_0$H$_8$       & Azulene                         &    73.50    &     7.73  &    -1.50  &    10.79  &      f\\
 C$_1$$_0$H$_8$       & Naphthalene                     &    36.05    &     4.51  &     2.15  &     4.42  &      f\\
 C$_1$$_0$H$_1$$_6$      & Adamantane                      &   -31.90    &    -2.88  &     5.37  &   -11.45  &      w\\
 C$_1$$_2$H$_1$$_0$      & Biphenyl                        &    43.53    &     4.39  &     2.39  &     3.93  &      f\\
 C$_1$$_2$H$_1$$_8$      & Hexamethylbenzene               &   -20.75    &    -6.73  &    20.80  &     7.36  &      f\\
 C$_1$$_2$H$_2$$_6$      & n-Duodecane                     &   -69.24    &    -3.35  &     1.68  &   -16.82  &      f\\
 C$_1$$_4$H$_1$$_0$      & Anthracene                      &    55.20    &     6.30  &     3.47  &     7.56  &      f\\
 C$_1$$_4$H$_1$$_0$      & Phenanthrene                    &    49.50    &     5.36  &     6.00  &     7.78  &      f\\
 HO          & Hydroxide (--)                   &   -33.20    &    15.70  &    27.42  &    19.08  &      o\\
 HO          & Hydroxyl radical                &     9.50    &    -6.55  &    -9.00  &    -8.59  &      x\\
 H$_2$O         & Water                           &   -57.80    &     4.37  &    -3.15  &    -1.45  &      d\\
 H$_3$O         & Hydronium (+)                   &   138.90    &    20.17  &    -4.75  &     4.56  &      d\\
 CO          & Carbon monoxide                 &   -26.42    &     6.66  &    20.49  &    20.72  &      d\\
 CHO         & HCO (+)                         &   199.00    &   -22.11  &   -14.14  &   -11.52  &      x\\
 CHO         & HCO                             &    10.40    &   -19.68  &   -10.77  &   -11.38  &      d\\
 CH$_2$O        & Formaldehyde                    &   -25.95    &    -8.15  &    -6.95  &    -5.56  &      f\\
 CH$_2$O        & Hydroxymethylene (trans)        &    53.20    &    -2.75  &    -6.79  &    -5.05  &      j\\
 CH$_2$O        & Hydroxymethylene (cis)          &    58.50    &    -8.36  &   -12.19  &   -11.26  &      j\\
 CH$_3$O        & CH2OH (+)                       &   168.00    &    -1.73  &   -12.46  &    -6.71  &      i\\
 CH$_3$O        & Methoxy (--)                     &   -36.00    &    -1.96  &    -3.77  &    -2.52  &      o\\
 CH$_3$O        & Methoxy                         &    -0.50    &    -6.33  &     0.32  &    -3.22  &      l\\
\hline
\end{tabular}
\end{center}
\end{table}
\clearpage

\begin{table}
\caption{Comparison of Calculated and Observed $\Delta H_f$ 
for MNDO, AM1, and PM3 (contd.)}
\begin{center}
\compresstable
\begin{tabular}{llrrrrr}
Empirical & Chemical Name & $\Delta H_f$ & \multicolumn{3}{c}{Difference} & \\
Formula   &               & Exp. & PM3 &  MNDO  &  AM1 &     Ref.\\
\hline
 CH$_4$O        & Methanol                        &   -48.10    &    -3.80  &    -9.28  &    -8.95  &      f\\
 C$_2$H$_2$O       & Ketene                          &   -11.40    &     2.17  &     4.57  &     5.71  &      f\\
 C$_2$H$_2$O       & HCCOH                           &    36.00    &   -12.81  &   -16.72  &   -11.49  &      j\\
 C$_2$H$_4$O       & Acetaldehyde                    &   -39.73    &    -4.50  &    -2.59  &    -1.87  &      f\\
 C$_2$H$_4$O       & Ethylene oxide                  &   -12.58    &     4.41  &    -2.99  &     3.59  &      f\\
 C$_2$H$_5$O       & Ethoxy (--)                      &   -47.50    &     2.71  &     2.16  &     1.97  &      o\\
 C$_2$H$_6$O       & Ethanol                         &   -56.20    &    -0.69  &    -6.83  &    -6.50  &      f\\
 C$_2$H$_6$O       & Dimethyl ether                  &   -43.99    &    -4.36  &    -7.27  &    -9.22  &      f\\
 C$_3$H$_6$O       & Acetone                         &   -51.99    &    -1.37  &     2.54  &     2.75  &      f\\
 C$_3$H$_6$O       & Propanal                        &   -45.50    &    -3.86  &    -2.58  &    -2.85  &      f\\
 C$_3$H$_6$O       & Trimethylene oxide              &   -19.25    &    -7.51  &   -17.96  &    -6.35  &      f\\
 C$_3$H$_8$O       & Isopropanol                     &   -65.10    &    -0.71  &    -0.05  &    -4.49  &      f\\
 C$_3$H$_8$O       & Propanol                        &   -61.20    &    -2.47  &    -6.38  &    -9.45  &      f\\
 C$_4$H$_4$O       & Furan                           &    -8.30    &     4.21  &    -0.37  &    11.19  &      f\\
 C$_4$H$_6$O       & 2-Butenal                       &   -24.00    &    -3.17  &    -2.90  &    -1.70  &      f\\
 C$_4$H$_6$O       & Divinyl ether                   &    -3.27    &     2.48  &     1.22  &     4.53  &      f\\
 C$_4$H$_8$O       & 2-Butanone                      &   -57.02    &    -0.48  &     2.91  &     1.80  &      y\\
 C$_4$H$_8$O       & Butanal                         &   -48.90    &    -5.86  &    -3.96  &    -6.27  &      f\\
 C$_4$H$_8$O       & Tetrahydrofuran                 &   -44.02    &    -7.37  &   -15.33  &   -14.45  &      f\\
 C$_4$H$_1$$_0$O      & Diethyl ether                   &   -60.30    &     0.61  &     0.06  &    -4.76  &      f\\
 C$_4$H$_1$$_0$O      & t-Butanol                       &   -74.70    &     3.36  &    10.36  &     3.04  &      f\\
 C$_5$H$_8$O       & Cyclopentanone                  &   -46.03    &    -9.21  &   -11.07  &    -9.37  &      f\\
 C$_5$H$_1$$_0$O      & Tetrahydropyran                 &   -53.40    &    -4.09  &    -8.72  &   -13.74  &      f\\
 C$_5$H$_1$$_2$O      & 3-Pentanol                      &   -75.20    &     1.31  &     1.66  &    -5.59  &      f\\
 C$_6$H$_5$O       & Phenoxy (--)                     &   -40.50    &    -3.69  &    -1.81  &    -0.55  &      o\\
 C$_6$H$_6$O       & Phenol                          &   -23.04    &     1.29  &    -3.72  &     0.71  &      f\\
 C$_6$H$_1$$_0$O      & Cyclohexanone                   &   -54.04    &    -6.20  &    -6.17  &    -9.38  &      f\\
 C$_7$H$_6$O       & Benzaldehyde                    &    -8.80    &    -1.94  &    -0.86  &    -0.21  &      f\\
 C$_7$H$_8$O       & Anisole                         &   -17.30    &     2.65  &    -0.47  &     1.35  &      f\\
 C$_8$H$_1$$_0$O      & Phenetole                       &   -26.30    &     5.77  &     3.63  &     4.85  &      f\\
 C$_1$$_0$H$_8$O      & 1-Naphthol                      &    -5.10    &     0.85  &    -1.88  &     2.80  &      f\\
 C$_1$$_0$H$_8$O      & 2-Naphthol                      &   -10.10    &     5.41  &     0.30  &     6.16  &      f\\
 O$_2$          & Oxygen (Singlet)                &    22.00    &    -3.63  &    -9.87  &   -21.31  &      z\\
 O$_2$          & Oxygen (Triplet)                &     0.00    &    -4.19  &   -15.36  &   -27.08  &      b\\
 H$_2$O$_2$        & Hydrogen peroxide               &   -32.50    &    -8.30  &    -5.77  &    -2.76  &      d\\
 CO$_2$         & Carbon dioxide                  &   -94.05    &     8.98  &    18.94  &    14.19  &      d\\
 CHO$_2$        & HCOO (--)                        &  -106.60    &    -4.34  &     4.95  &    -2.87  &      o\\
 CH$_2$O$_2$       & Formic acid                     &   -90.50    &    -3.94  &    -2.11  &    -6.92  &      f\\
 C$_2$H$_2$O$_2$      & trans Glyoxal                   &   -50.70    &   -13.66  &   -10.73  &    -8.05  &      f\\
 C$_2$H$_3$O$_2$      & CH3COO (--)                      &  -122.50    &     2.82  &    12.44  &     7.05  &      o\\
 C$_2$H$_4$O$_2$      & Acetic acid                     &  -103.30    &     1.26  &     2.14  &     0.27  &      f\\
 C$_2$H$_4$O$_2$      & Methyl formate                  &   -83.60    &    -3.48  &    -1.97  &    -7.50  &      m\\
 C$_2$H$_6$O$_2$      & Dimethyl peroxide               &   -30.10    &    -4.05  &     1.75  &     4.32  &      f\\
 C$_2$H$_6$O$_2$      & Ethylene glycol                 &   -93.90    &    -4.70  &   -12.70  &   -17.60  &      f\\
 C$_3$O$_2$        & Carbon suboxide                 &   -22.40    &    -1.66  &    -1.15  &     7.76  &      f\\
 C$_3$H$_4$O$_2$      & beta-Propiolactone              &   -67.60    &    -3.08  &    -3.32  &    -3.46  &      f\\
\hline
\end{tabular}
\end{center}
\end{table}
\clearpage

\begin{table}
\caption{Comparison of Calculated and Observed $\Delta H_f$ 
for MNDO, AM1, and PM3 (contd.)}
\begin{center}
\compresstable
\begin{tabular}{llrrrrr}
Empirical & Chemical Name & $\Delta H_f$ & \multicolumn{3}{c}{Difference} & \\
Formula   &               & Exp. & PM3 &  MNDO  &  AM1 &     Ref.\\
\hline
 C$_3$H$_6$O$_2$      & Propionic acid                  &  -108.40    &     1.97  &     2.04  &    -0.75  &      f\\
 C$_3$H$_6$O$_2$      & Methyl acetate                  &   -97.90    &     3.74  &     4.18  &     1.42  &      m\\
 C$_3$H$_8$O$_2$      & Dimethoxymethane                &   -83.30    &    -6.95  &   -11.13  &   -20.06  &      f\\
 C$_4$H$_6$O$_2$      & Diacetyl                        &   -78.20    &    -5.27  &    -0.58  &     3.26  &      f\\
 C$_4$H$_1$$_0$O$_2$     & Diethyl peroxide                &   -46.10    &     6.05  &     7.10  &     9.34  &      f\\
 C$_5$H$_8$O$_2$      & Acetylacetone                   &   -90.50    &    -1.20  &     6.16  &     4.86  &      f\\
 C$_6$H$_4$O$_2$      & p-Benzoquinone                  &   -29.30    &    -2.33  &    -3.65  &     4.13  &      f\\
 C$_7$H$_6$O$_2$      & Benzoic acid                    &   -70.10    &     3.83  &     2.31  &     2.04  &      f\\
 C$_7$H$_6$O$_2$      & o-Salicylic acid                &   -70.10    &     3.83  &     2.31  &     2.04  &      f\\
 O$_3$          & Ozone                           &    34.10    &    16.97  &    14.38  &     3.59  &      d\\
 C$_4$H$_2$O$_3$      & Malaic anhydride                &   -95.20    &     5.04  &     6.64  &    18.76  &      f\\
 C$_4$H$_6$O$_3$      & Acetic anhydride                &  -137.10    &     2.01  &     4.42  &     5.35  &      f\\
 C$_2$H$_2$O$_4$      & Oxalic acid                     &  -175.00    &     0.92  &    -0.14  &     2.57  &      f\\
 N           & Nitrogen (++)                   &  1133.90    &  -170.01  &  -160.75  &  -160.97  &     aa\\
 H$_2$N         & Amidogen                        &    45.50    &   -10.09  &    -8.50  &    -7.09  &      d\\
 H$_3$N         & Ammonia                         &   -11.00    &     7.93  &     4.62  &     3.71  &      d\\
 H$_4$N         & Ammonium (+)                    &   155.00    &    -1.60  &     9.63  &    -4.43  &     bb\\
 CN          & Cyanide                         &   104.00    &    23.99  &    25.27  &    10.37  &      d\\
 CHN         & Hydrogen cyanide                &    32.30    &     0.64  &     3.00  &    -1.31  &      d\\
 CH$_4$N        & CH2-NH2 (+)                     &   178.00    &     7.31  &     8.76  &    -1.76  &      l\\
 CH$_4$N        & CH3NH (--)                       &    30.50    &    -8.37  &    -6.41  &     3.28  &      o\\
 CH$_4$N        & CH3-NH.                         &    37.00    &    -9.76  &    -4.31  &    -2.90  &      l\\
 CH$_5$N        & Methylamine                     &    -5.50    &     1.48  &    -1.00  &    -0.64  &      f\\
 C$_2$H$_3$N       & Acetonitrile                    &    20.90    &     2.36  &    -1.70  &    -1.65  &     cc\\
 C$_2$H$_3$N       & Methyl isocyanide               &    35.60    &    19.06  &    24.71  &    14.75  &     cc\\
 C$_2$H$_5$N       & Ethyleneimine (Azirane)         &    30.20    &     1.37  &    -5.15  &     2.89  &      f\\
 C$_2$H$_6$N       & Me2N (--)                        &    24.70    &   -16.89  &   -16.20  &    -2.38  &      o\\
 C$_2$H$_7$N       & Ethylamine                      &   -11.40    &    -1.16  &    -1.88  &    -3.78  &      f\\
 C$_2$H$_7$N       & Dimethylamine                   &    -4.43    &    -3.50  &    -2.25  &    -1.23  &      f\\
 C$_3$H$_3$N       & Acrylonitrile                   &    44.10    &     6.02  &    -0.30  &     0.82  &      f\\
 C$_3$H$_5$N       & Ethyl cyanide                   &    12.10    &     6.41  &     1.62  &     0.88  &      m\\
 C$_3$H$_9$N       & Isopropylamine                  &   -20.00    &     1.20  &     3.60  &     0.69  &      f\\
 C$_3$H$_9$N       & Trimethylamine                  &    -5.67    &    -5.25  &     2.83  &     3.91  &      f\\
 C$_3$H$_9$N       & n-Propylamine                   &   -16.80    &    -1.12  &    -1.47  &    -5.34  &      f\\
 C$_4$H$_5$N       & Pyrrole                         &    25.90    &     1.15  &     6.48  &    13.91  &      f\\
 C$_4$H$_9$N       & Pyrrolidine                     &    -0.80    &   -11.26  &   -15.11  &    -9.70  &      f\\
 C$_4$H$_1$$_1$N      & t-Butylamine                    &   -28.90    &     3.63  &    13.38  &     7.62  &      f\\
 C$_5$H$_5$N       & Pyridine                        &    34.60    &    -4.30  &    -5.86  &    -2.63  &      f\\
 C$_6$H$_7$N       & Aniline                         &    20.80    &     0.42  &     0.83  &    -0.39  &      f\\
 C$_6$H$_1$$_5$N      & Triethylamine                   &   -22.06    &    -4.60  &     7.26  &     5.72  &      f\\
 C$_7$H$_5$N       & Phenyl cyanide                  &    51.50    &     6.89  &     0.38  &     1.79  &      f\\
 NO          & Nitrogen oxide                  &    21.58    &    -6.85  &   -21.75  &   -20.41  &      d\\
 NO          & NO (+)                          &   237.00    &     1.21  &    -6.42  &    -8.86  &      d\\
 CNO         & NCO                             &    38.10    &    -5.77  &    -1.02  &     0.77  &      d\\
 CHNO        & Hydrogen isocyanate             &   -24.30    &     8.98  &    13.47  &     9.11  &      d\\
 C$_3$H$_7$NO      & Dimethylformamide               &   -45.80    &     1.13  &     8.36  &     8.85  &      f\\
\hline
\end{tabular}
\end{center}
\end{table}
\clearpage

\begin{table}
\caption{Comparison of Calculated and Observed $\Delta H_f$ 
for MNDO, AM1, and PM3 (contd.)}
\begin{center}
\compresstable
\begin{tabular}{llrrrrr}
Empirical & Chemical Name & $\Delta H_f$ & \multicolumn{3}{c}{Difference} & \\
Formula   &               & Exp. & PM3 &  MNDO  &  AM1 &     Ref.\\
\hline
 NO$_2$         & Nitrogen dioxide                &     7.91    &    -8.95  &   -12.49  &   -22.92  &      d\\
 NO$_2$         & Nitrogen dioxide (+)            &   233.00    &   -24.65  &     7.60  &   -11.92  &      l\\
 HNO$_2$        & Nitrous acid, trans             &   -18.80    &     3.89  &   -21.89  &   -20.65  &      d\\
 CH$_3$NO$_2$      & Nitromethane                    &   -17.90    &     1.91  &    21.17  &     7.91  &      f\\
 CH$_3$NO$_2$      & Methyl nitrite                  &   -15.80    &     6.63  &   -20.91  &   -16.07  &      f\\
 C$_2$H$_5$NO$_2$     & Nitroethane                     &   -23.50    &     2.58  &    21.35  &     6.62  &      f\\
 C$_2$H$_5$NO$_2$     & Glycine                         &   -93.70    &    -2.32  &    -1.33  &    -7.90  &      f\\
 C$_3$H$_7$NO$_2$     & 1-Nitropropane                  &   -30.00    &     3.15  &    21.81  &     6.31  &      f\\
 C$_3$H$_7$NO$_2$     & 2-Nitropropane                  &   -33.20    &     5.98  &    26.79  &    11.48  &      f\\
 C$_3$H$_7$NO$_2$     & Alanine                         &  -111.40    &    10.23  &    13.35  &     6.36  &      f\\
 C$_4$H$_9$NO$_2$     & 1-Nitrobutane                   &   -34.40    &     2.17  &    21.47  &     3.89  &      f\\
 C$_4$H$_9$NO$_2$     & 2-Nitrobutane                   &   -39.10    &     7.11  &    28.90  &    10.81  &      f\\
 C$_4$H$_9$NO$_2$     & 2-Nitroisobutane                &   -42.20    &     9.85  &    38.84  &    18.07  &     dd\\
 C$_6$H$_5$NO$_2$     & Nitrobenzene                    &    15.40    &    -0.97  &    22.08  &     9.78  &      m\\
 C$_7$H$_7$NO$_2$     & 2-Nitrotoluene                  &     9.30    &    -1.41  &    20.55  &     9.49  &      f\\
 C$_7$H$_7$NO$_2$     & 3-Nitrotoluene                  &     4.10    &     0.86  &    25.83  &    13.37  &     ee\\
 C$_7$H$_7$NO$_2$     & 4-Nitrotoluene                  &     7.40    &    -2.83  &    22.31  &     9.73  &     ee\\
 NO$_3$         & Nitrate radical                 &    17.00    &     5.81  &    28.09  &    16.16  &      d\\
 NO$_3$         & Nitrate anion                   &   -74.70    &   -18.65  &     7.62  &   -14.21  &     aa\\
 HNO$_3$        & Nitric acid                     &   -32.10    &    -5.96  &    14.50  &    -5.43  &      d\\
 CH$_3$NO$_3$      & Methyl nitrate                  &   -29.11    &    -3.37  &    16.60  &    -2.29  &      f\\
 C$_2$H$_5$NO$_3$     & Ethyl nitrate                   &   -36.83    &     0.37  &    18.84  &    -0.53  &      f\\
 C$_2$H$_5$NO$_3$     & Nitroethanol                    &   -75.10    &    13.60  &    29.35  &    10.43  &     ff\\
 N$_2$          & Nitrogen                        &     0.00    &    17.55  &     8.26  &    11.15  &      b\\
 H$_2$N$_2$        & Diazene                         &    36.00    &     1.73  &    -4.17  &    -4.47  &     gg\\
 H$_4$N$_2$        & Hydrazine                       &    22.80    &    -2.17  &    -8.65  &    -9.15  &      d\\
 CH$_2$N$_2$       & Diazomethane                    &    71.00    &   -10.00  &    -3.77  &    -8.40  &      f\\
 CH$_2$N$_2$       & N=N-CH2-                        &    79.00    &    12.65  &    -6.62  &     7.80  &     hh\\
 CH$_6$N$_2$       & Methylhydrazine                 &    22.60    &    -4.74  &    -8.27  &    -5.31  &      f\\
 C$_2$N$_2$        & Cyanogen                        &    73.80    &     3.64  &    -7.25  &    -5.91  &      f\\
 C$_2$H$_8$N$_2$      & 1,1-Dimethylhydrazine           &    20.00    &    -4.95  &    -1.91  &     3.91  &      f\\
 C$_2$H$_8$N$_2$      & 1,2-Dimethylhydrazine           &    22.00    &    -6.47  &    -7.02  &    -0.53  &      f\\
 C$_4$N$_2$        & Dicyanoacetylene                &   126.50    &     1.49  &   -15.16  &    -6.77  &      f\\
 C$_4$H$_2$N$_2$      & Fumaronitrile                   &    81.30    &     4.63  &    -6.62  &    -5.35  &      f\\
 C$_4$H$_4$N$_2$      & Pyridazine                      &    66.50    &   -10.58  &   -22.96  &   -11.25  &      f\\
 C$_4$H$_4$N$_2$      & Pyrimidine                      &    47.00    &    -9.10  &   -12.10  &    -3.16  &      f\\
 C$_4$H$_4$N$_2$      & Pyrazine                        &    46.90    &    -7.65  &    -9.22  &    -2.76  &      f\\
 C$_6$H$_1$$_4$N$_2$     & azo-n-Propane                   &     8.60    &    -2.93  &    -5.84  &     6.03  &     ii\\
 N$_2$O         & Nitrous oxide                   &    19.60    &     5.76  &    11.40  &     8.82  &      d\\
 CH$_4$N$_2$O      & Urea                            &   -58.70    &    17.70  &    19.06  &    14.58  &      f\\
 C$_2$H$_6$N$_2$O$_2$    & N-Nitrodimethylamine            &    -3.20    &     4.39  &    25.45  &    24.81  &      m\\
 C$_6$H$_6$N$_2$O$_2$    & Para nitroaniline               &    16.20    &    -5.64  &    20.62  &     5.22  &      f\\
 N$_2$O$_3$        & Dinitrogen trioxide             &    19.80    &     3.86  &    -6.18  &     1.98  &      d\\
 N$_2$O$_4$        & Dinitrogen tetroxide            &     2.17    &     6.08  &    27.69  &    22.82  &      d\\
 CH$_2$N$_2$O$_4$     & Dinitromethane                  &   -13.30    &     1.26  &    41.12  &    16.11  &     ff\\
 C$_2$H$_4$N$_2$O$_4$    & 1,1-Dinitroethane               &   -24.10    &     6.61  &    47.23  &    21.41  &     ff\\
\hline
\end{tabular}
\end{center}
\end{table}
\clearpage

\begin{table}
\caption{Comparison of Calculated and Observed $\Delta H_f$ 
for MNDO, AM1, and PM3 (contd.)}
\begin{center}
\compresstable
\begin{tabular}{llrrrrr}
Empirical & Chemical Name & $\Delta H_f$ & \multicolumn{3}{c}{Difference} & \\
Formula   &               & Exp. & PM3 &  MNDO  &  AM1 &     Ref.\\
\hline
 C$_2$H$_4$N$_2$O$_4$    & 1,2-Dinitroethane               &   -22.90    &     3.23  &    42.77  &    12.83  &     ff\\
 C$_3$H$_8$O$_2$      & Dimethoxymethane                &   -83.30    &    -6.95  &   -11.13  &   -20.06  &      f\\
 C$_3$H$_6$N$_2$O$_4$    & 1,1-Dinitropropane              &   -25.90    &     3.82  &    45.21  &    16.58  &      f\\
 C$_3$H$_6$N$_2$O$_4$    & 1,3-Dinitropropane              &   -31.60    &     4.97  &    44.41  &    13.28  &     ff\\
 C$_3$H$_6$N$_2$O$_4$    & 2,2-Dinitropropane              &   -27.00    &     3.90  &    50.10  &    21.34  &     ff\\
 C$_4$H$_8$N$_2$O$_4$    & 1,1-Dinitrobutane               &   -34.10    &     6.72  &    48.81  &    17.96  &     ff\\
 C$_4$H$_8$N$_2$O$_4$    & 1,4-Dinitrobutane               &   -38.90    &     6.06  &    45.71  &    11.13  &     ff\\
 C$_6$H$_4$N$_2$O$_4$    & m-Dinitrobenzene                &    11.30    &    -2.23  &    47.24  &    21.67  &      f\\
 C$_6$H$_4$N$_2$O$_4$    & o-Dinitrobenzene                &    20.20    &   -11.13  &    34.49  &    12.77  &     ee\\
 C$_6$H$_4$N$_2$O$_4$    & p-Dinitrobenzene                &    13.30    &    -3.32  &    41.18  &    19.83  &     ee\\
 C$_7$H$_6$N$_2$O$_4$    & 2,4-Dinitrotoluene              &     4.70    &    -2.84  &    44.06  &    21.32  &     ee\\
 C$_7$H$_6$N$_2$O$_4$    & 2,6-Dinitrotoluene              &     9.60    &    -3.37  &    42.08  &    20.31  &     ee\\
 N$_2$O$_5$        & Dinitrogen pentoxide            &     2.70    &   -21.83  &    31.44  &     4.16  &      d\\
 C$_6$H$_4$N$_2$O$_5$    & 4,6-Dinitrophenol               &   -23.40    &   -13.92  &    33.97  &    15.57  &      f\\
 N$_3$          & Azide                           &    99.00    &     6.94  &     3.41  &     8.37  &      x\\
 HN$_3$         & Hydrazoic acid                  &    70.30    &     4.93  &     2.75  &     5.52  &     cc\\
 CHN$_3$O$_6$      & Trinitromethane                 &    -3.20    &    -1.61  &    61.76  &    28.04  &     jj\\
 C$_2$H$_3$N$_3$O$_6$    & 1,1,1-Trinitroethane            &   -12.40    &     2.27  &    68.69  &    33.33  &     ff\\
 C$_3$H$_5$N$_3$O$_6$    & 1,1,1-Trinitropropane           &   -18.40    &     6.81  &    73.64  &    33.77  &     ff\\
 C$_6$H$_3$N$_3$O$_6$    & 1,3,5-Trinitrobenzene           &    14.90    &    -1.92  &    65.36  &    34.91  &     ee\\
 C$_7$H$_5$N$_3$O$_6$    & 2,4,6-Trinitrotoluene           &    12.90    &    -9.77  &    61.56  &    28.28  &      f\\
 C$_7$H$_5$N$_3$O$_7$    & 2,4,6-Trinitroanisole           &    -5.80    &   -19.97  &    47.45  &    18.10  &      f\\
 C$_8$H$_7$N$_3$O$_7$    & 2,4,6-Trinitrophenetole         &   -20.10    &   -10.13  &    56.66  &    27.80  &      f\\
 C$_3$H$_5$N$_3$O$_9$    & Glycerol trinitrate             &   -64.70    &   -13.40  &    57.91  &    -9.28  &      f\\
 CH$_2$N$_4$       & 1-H Tetrazole                   &    79.95    &     6.24  &   -26.18  &    29.63  &      f\\
 CN$_4$O$_8$       & Tetranitromethane               &    18.50    &   -12.24  &    76.30  &    34.38  &     ff\\
 C$_5$H$_8$N$_4$O$_1$$_2$   & Pentaerythritol tetranitrate    &   -92.50    &    -5.95  &    93.73  &    -3.30  &      f\\
 C$_3$H$_6$N$_6$      & Melamine                        &    12.40    &    12.28  &     9.03  &    51.85  &      f\\
 S           & S (--)                           &    16.77    &     3.91  &    29.84  &    11.18  &      d\\
 HS          & Hydrogen sulfide                &    33.30    &     4.91  &     4.00  &     6.82  &      d\\
 HS          & HS (--)                          &   -17.10    &     1.24  &    24.01  &     5.23  &      d\\
 H$_2$S         & Hydrogen sulfide                &    -4.90    &     3.98  &     8.72  &     6.10  &      d\\
 CS          & Carbon sulfide                  &    67.00    &    30.32  &    37.45  &    27.84  &      d\\
 CH$_2$S        & Thioformaldehyde                &    24.30    &    13.27  &     3.52  &     5.63  &     kk\\
 CH$_4$S        & Thiomethanol                    &    -5.40    &    -0.14  &    -1.90  &     1.04  &      f\\
 C$_2$H$_4$S       & Thiirane                        &    19.70    &     9.09  &    -0.88  &    11.00  &      f\\
 C$_2$H$_6$S       & Thioethanol                     &   -11.00    &     2.25  &    -2.42  &     0.36  &      f\\
 C$_2$H$_6$S       & Dimethyl thioether              &    -8.90    &    -2.08  &    -8.19  &    -0.47  &      f\\
 C$_3$H$_6$S       & Thietane                        &    14.60    &    -7.14  &   -19.73  &    -7.41  &      f\\
 C$_3$H$_8$S       & Isopropanthiol                  &   -18.10    &     3.71  &     1.83  &     2.04  &      d\\
 C$_3$H$_8$S       & Methylethylthioether            &   -14.20    &     0.10  &    -8.87  &    -1.42  &     ll\\
 C$_3$H$_8$S       & 1-Propanthiol                   &   -16.20    &     2.09  &    -2.00  &    -1.21  &      f\\
 C$_4$H$_4$S       & Thiophene                       &    27.60    &     3.07  &    -1.18  &    -0.21  &      f\\
 C$_4$H$_6$S       & 2,3-Dihydrothiophene            &    21.80    &    -8.43  &   -18.03  &   -11.67  &     mm\\
 C$_4$H$_6$S       & 2,5-Dihydrothiophene            &    20.90    &    -6.11  &   -16.66  &    -8.29  &     mm\\
 C$_4$H$_8$S       & Tetrahydrothiophene             &    -8.10    &    -2.31  &   -16.11  &    -8.46  &      f\\
\hline
\end{tabular}
\end{center}
\end{table}
\clearpage

\begin{table}
\caption{Comparison of Calculated and Observed $\Delta H_f$ 
for MNDO, AM1, and PM3 (contd.)}
\begin{center}
\compresstable
\begin{tabular}{llrrrrr}
Empirical & Chemical Name & $\Delta H_f$ & \multicolumn{3}{c}{Difference} & \\
Formula   &               & Exp. & PM3 &  MNDO  &  AM1 &     Ref.\\
\hline
 C$_4$H$_8$S       & Cisdimethylthiirane             &     2.70    &    13.19  &     2.03  &    15.06  &     ll\\
 C$_4$H$_8$S       & Transdimethylthiirane           &     0.90    &    14.20  &     3.00  &    15.84  &     ll\\
 C$_4$H$_1$$_0$S      & Butanethiol                     &   -21.10    &     1.59  &    -1.85  &    -3.16  &      d\\
 C$_4$H$_1$$_0$S      & Diethyl thioether               &   -19.89    &     0.58  &    -8.29  &    -2.27  &      f\\
 C$_4$H$_1$$_0$S      & Methylpropylthioether           &   -19.50    &     0.04  &    -8.34  &    -2.88  &     ll\\
 C$_5$H$_1$$_0$S      & Thiacyclohexane                 &   -15.12    &    -0.67  &   -12.97  &    -8.81  &      f\\
 C$_5$H$_1$$_0$S      & Cyclopentanthiol                &   -11.40    &    -3.51  &    -9.98  &    -8.97  &     nn\\
 C$_6$H$_6$S       & Thiophenol                      &    26.90    &     0.70  &    -3.55  &    -1.24  &      f\\
 C$_6$H$_1$$_2$S      & Cyclohexanethiol                &   -22.96    &     1.15  &    -2.92  &    -7.53  &      n\\
 C$_7$H$_8$S       & Toluenethiol                    &    23.60    &     1.35  &    -3.42  &    -1.29  &      f\\
 SO          & Sulfur monoxide (triplet)       &     1.20    &   -14.83  &     2.95  &   -12.73  &      d\\
 CSO         & Carbon oxysulfide               &   -33.85    &    10.09  &    10.93  &     4.87  &      f\\
 C$_2$H$_4$SO      & Thiolacetic acid                &   -43.50    &     4.60  &     2.17  &     4.28  &     hh\\
 C$_2$H$_6$SO      & Dimethyl sulfoxide              &   -36.09    &    -2.66  &    40.27  &    -3.22  &      f\\
 C$_4$H$_1$$_0$SO     & Diethyl sulfoxide               &   -49.10    &     2.46  &    41.41  &    -2.45  &      f\\
 SO$_2$         & Sulfur dioxide                  &   -71.00    &    20.22  &    75.42  &    23.96  &      d\\
 C$_2$H$_6$SO$_2$     & Dimethyl sulfone                &   -89.10    &    12.71  &   142.74  &    18.74  &      f\\
 C$_3$H$_8$SO$_2$     & Methylethyl sulfone             &   -97.60    &    18.94  &   144.61  &    21.75  &      f\\
 C$_4$H$_6$SO$_2$     & Divinyl sulfone                 &   -36.00    &    10.64  &   137.33  &    10.55  &      f\\
 C$_4$H$_1$$_0$SO$_2$    & Diethyl sulfone                 &  -102.50    &    21.56  &   142.99  &    21.16  &      f\\
 SO$_3$         & Sulfur trioxide                 &   -94.60    &   -10.18  &   153.06  &    -2.73  &      d\\
 C$_2$H$_6$SO$_3$     & Dimethyl sulfite                &  -115.50    &   -14.58  &    50.29  &   -23.82  &      f\\
 H$_2$SO$_4$       & Sulfuric acid                   &  -175.70    &   -12.43  &   162.57  &   -10.67  &      d\\
 C$_2$H$_6$SO$_4$     & Dimethyl sulfate                &  -164.10    &    -8.06  &   158.52  &   -10.69  &      f\\
 CHNS        & Hydrogen isothiocyanate         &    30.00    &     9.52  &    13.40  &    -2.65  &      d\\
 C$_2$H$_3$NS      & Methyl isothiocyanate           &    27.10    &     9.03  &     9.82  &     0.55  &     cc\\
 C$_2$H$_3$NS      & Methyl thiocyanate              &    38.33    &    -9.41  &   -14.52  &   -15.58  &     cc\\
 CH$_4$N$_2$S      & Thiourea                        &    -6.00    &    29.90  &    20.76  &    15.84  &     oo\\
 S$_2$          & Sulfur dimer                    &    30.80    &    -2.12  &     3.96  &    -6.43  &      d\\
 H$_2$S$_2$        & Hydrogen disulfide              &     3.71    &     4.93  &     2.77  &     4.83  &     cc\\
 CS$_2$         & Carbon disulfide                &    28.00    &     8.92  &     8.91  &   -10.52  &      d\\
 CH$_4$S$_2$       & CH3SSH                          &    -0.90    &     2.62  &    -3.31  &     2.84  &     kk\\
 C$_2$H$_6$S$_2$      & CH3SCH2SH                       &     0.10    &    -2.80  &   -11.28  &    -3.83  &     kk\\
 C$_2$H$_6$S$_2$      & Ethanedithiol-1,2               &    -2.20    &     3.35  &    -4.16  &    -1.11  &      d\\
 C$_2$H$_6$S$_2$      & 2,3-Dithiabutane                &    -5.60    &     0.78  &    -9.20  &     1.41  &      f\\
 C$_3$H$_8$S$_2$      & Propane-1,3-dithiol             &    -7.00    &     0.58  &    -4.38  &    -4.36  &     nn\\
 C$_4$H$_1$$_0$S$_2$     & (C2H5S)2                        &   -17.80    &     6.79  &    -9.09  &     1.08  &     pp\\
 C$_2$N$_2$S$_2$      & S2(CN)2                         &    82.30    &    -3.99  &   -11.85  &   -18.21  &      f\\
 H$_2$S$_3$        & Hydrogen trisulfide             &     7.29    &    -1.58  &     1.12  &     0.49  &     cc\\
 C$_2$H$_6$S$_3$      & 2,3,4-Trithiapentane            &     0.00    &    -6.92  &   -13.28  &    -4.71  &     qq\\
 C$_3$H$_4$S$_3$      & 1,3-Dithiolan-2-thione          &    22.70    &    17.61  &   -11.36  &    -2.70  &      d\\
 S$_4$          & Sulfur tetramer                 &    32.70    &    14.69  &    13.13  &     0.99  &     aa\\
 H$_2$S$_4$        & Hydrogen tetrasulfide           &    10.57    &   -10.92  &    64.50  &    -7.47  &     cc\\
 H$_2$S$_5$        & Hydrogen pentasulfide           &    13.84    &    -8.42  &    -0.61  &    -6.72  &     cc\\
 S$_8$          & S8                              &    24.00    &    -5.85  &    -0.76  &    -8.74  &      d\\
 F           & Fluoride (--)                    &   -61.00    &    29.77  &    43.87  &    64.44  &      d\\
\hline
\end{tabular}
\end{center}
\end{table}
\clearpage

\begin{table}
\caption{Comparison of Calculated and Observed $\Delta H_f$ 
for MNDO, AM1, and PM3 (contd.)}
\begin{center}
\compresstable
\begin{tabular}{llrrrrr}
Empirical & Chemical Name & $\Delta H_f$ & \multicolumn{3}{c}{Difference} & \\
Formula   &               & Exp. & PM3 &  MNDO  &  AM1 &     Ref.\\
\hline
 HF          & Hydrogen fluoride               &   -65.14    &     2.39  &     5.40  &    -9.14  &      d\\
 CF          & Fluoromethylidyne               &    61.00    &    -6.95  &   -22.41  &   -23.04  &      d\\
 CH$_2$F        & Fluoromethyl (+)                &   200.30    &    -0.05  &   -17.55  &   -19.97  &     rr\\
 CH$_3$F        & Fluoromethane                   &   -56.80    &     2.98  &    -4.13  &    -4.24  &     ss\\
 CH$_3$F        & Trifluoromethane (+)            &   233.30    &    -5.14  &   -10.13  &   -29.35  &     rr\\
 C$_2$HF        & Fluoroacetylene                 &    30.00    &   -11.91  &   -14.37  &   -14.81  &      d\\
 C$_2$H$_3$F       & Fluoroethylene                  &   -32.50    &     3.87  &    -2.05  &    -1.58  &     tt\\
 C$_2$H$_4$F       & CH3CHF (+)                      &   166.00    &     6.83  &    -1.32  &    -8.97  &     uu\\
 C$_2$H$_5$F       & Fluoroethane                    &   -62.90    &     2.65  &    -2.25  &    -3.44  &      f\\
 C$_3$H$_7$F       & 2-Fluoropropane                 &   -69.40    &     2.55  &     2.76  &    -0.42  &      f\\
 C$_6$H$_5$F       & Fluorobenzene                   &   -27.76    &     7.45  &     2.43  &     4.37  &      f\\
 OF          & FO                              &    26.10    &    -4.93  &    -4.38  &    -3.50  &     vv\\
 HOF         & Hypofluorous acid               &   -20.90    &    -8.35  &     2.26  &    -1.71  &     ww\\
 COF         & COF                             &   -42.30    &   -12.73  &    -7.71  &   -13.67  &      g\\
 CHOF        & HCOF                            &   -90.00    &     1.17  &     1.16  &    -2.96  &      d\\
 C$_2$H$_3$OF      & Acetyl fluoride                 &  -106.40    &     7.68  &     9.87  &     7.59  &      f\\
 O$_2$F         & Fluorine dioxide                &     3.00    &     9.87  &    21.12  &     5.96  &      d\\
 C$_7$H$_5$O$_2$F     & p-Fluorobenzoic acid            &  -118.40    &     9.41  &     7.30  &     6.24  &      f\\
 CNF         & Cyanogen fluoride               &     8.60    &    -2.12  &   -10.97  &   -13.00  &      d\\
 NOF         & Nitrosyl fluoride               &   -15.70    &    12.33  &    -9.10  &   -10.82  &      d\\
 NO$_2$F        & Fluorine nitrite                &   -26.00    &     0.35  &    26.60  &     4.61  &      d\\
 NO$_3$F        & Fluorine nitrate                &     2.50    &    -8.64  &    25.43  &    11.15  &      d\\
 SF          & SF                              &    -4.10    &    -7.53  &     4.82  &   -16.06  &      d\\
 SOF         & SOF                             &   -63.30    &   -11.15  &    34.98  &   -16.32  &      g\\
 SO$_2$F        & SO2F                            &  -113.20    &    10.67  &   127.19  &     2.32  &      g\\
 F$_2$          & Fluorine                        &     0.00    &   -21.70  &     7.32  &   -22.49  &      b\\
 CF$_2$         & Difluoromethylene               &   -43.50    &    -5.61  &   -21.73  &   -24.48  &      g\\
 CF$_2$         & Difluoromethylene               &   -45.00    &    -4.11  &   -20.23  &   -22.98  &     xx\\
 CHF$_2$        & Difluoromethyl (+)              &   142.40    &     3.07  &   -10.01  &   -20.56  &     rr\\
 CH$_2$F$_2$       & Difluoromethane                 &  -108.13    &     4.35  &    -3.67  &    -7.99  &      f\\
 CH$_2$F$_2$       & Difluoromethane (+)             &   185.20    &    -4.84  &    -6.86  &   -33.65  &     rr\\
 C$_2$F$_2$        & Difluoroacetylene               &     5.00    &   -16.61  &   -26.03  &   -24.64  &      d\\
 C$_2$H$_2$F$_2$      & gem-Difluoroethylene            &   -80.50    &     7.47  &    -3.18  &    -2.22  &      f\\
 C$_2$H$_3$F$_2$      & CH3CF2 (+)                      &   107.00    &    15.17  &     9.54  &    -1.75  &     uu\\
 C$_2$H$_4$F$_2$      & 1,1-Difluoroethane              &  -118.80    &     6.84  &     5.32  &     0.15  &      f\\
 C$_6$H$_4$F$_2$      & o-Difluorobenzene               &   -70.30    &     7.09  &    -0.36  &     3.76  &      f\\
 C$_6$H$_4$F$_2$      & m-Difluorobenzene               &   -74.00    &    10.58  &     2.99  &     6.01  &      f\\
 C$_6$H$_4$F$_2$      & p-Difluorobenzene               &   -73.30    &     9.92  &     2.18  &     5.23  &      f\\
 OF$_2$         & Difluorine oxide                &     5.90    &   -10.68  &    12.28  &     4.55  &      d\\
 COF$_2$        & Carbonyl fluoride               &  -152.70    &    11.16  &    14.23  &     6.55  &      f\\
 NF$_2$         & NF2 (--)                         &   -29.50    &    -1.52  &   -14.15  &     4.23  &     yy\\
 NF$_2$         & NF2.                            &    10.10    &     1.79  &   -24.96  &   -16.51  &      d\\
 N$_2$F$_2$        & cis-Difluorodiazene             &    16.40    &    11.54  &   -18.70  &     4.36  &      d\\
 N$_2$F$_2$        & trans-Difluorodiazene           &    19.40    &     9.74  &   -17.06  &    11.80  &      d\\
 SF$_2$         & Sulfur difluoride               &   -70.90    &   -20.97  &    17.93  &   -31.18  &      d\\
 SOF$_2$        & Thionyl fluoride                &  -130.00    &    -8.21  &    84.24  &   -26.23  &      d\\
\hline
\end{tabular}
\end{center}
\end{table}
\clearpage

\begin{table}
\caption{Comparison of Calculated and Observed $\Delta H_f$ 
for MNDO, AM1, and PM3 (contd.)}
\begin{center}
\compresstable
\begin{tabular}{llrrrrr}
Empirical & Chemical Name & $\Delta H_f$ & \multicolumn{3}{c}{Difference} & \\
Formula   &               & Exp. & PM3 &  MNDO  &  AM1 &     Ref.\\
\hline
 SO$_2$F$_2$       & Sulfuryl fluoride               &  -181.30    &    -3.03  &   203.20  &   -14.61  &      d\\
 S$_2$F$_2$        & FSSF                            &   -80.40    &     6.56  &    39.11  &    -6.43  &      d\\
 S$_2$F$_2$        & SSF2                            &   -95.94    &    39.83  &   103.88  &    15.80  &      d\\
 CF$_3$         & Trifluoromethyl                    & -112.40    &   -19.72  &   -24.69  &   -30.41  &      d\\
 CF$_3$         & Trifluoromethyl (--)             &  -163.40    &   -15.44  &   -15.42  &   -15.44  &     yy\\
 CF$_3$         & Trifluoromethyl (+)             &    99.30    &     0.23  &     1.56  &   -17.24  &     rr\\
 CHF$_3$        & Trifluoromethane                &  -166.30    &     4.30  &     2.45  &    -6.22  &      f\\
 CHF$_3$        & Trifluoromethane (+)            &   151.90    &    -2.52  &     6.76  &   -30.70  &     rr\\
 C$_2$HF$_3$       & Trifluoroethylene               &  -117.30    &    -4.22  &   -13.85  &   -13.31  &      f\\
 C$_2$H$_2$F$_3$      & CF3CH2 (+)                      &   114.00    &     8.30  &     7.16  &     0.34  &     uu\\
 C$_2$H$_2$F$_3$      & CF3CH2.                         &  -123.60    &    -7.66  &    -6.09  &    -7.83  &     zz\\
 C$_2$H$_2$F$_3$      & CH2F.CF2 (+)                    &    81.00    &    11.67  &     1.37  &   -11.99  &     uu\\
 C$_2$H$_3$F$_3$      & 1,1,1-Trifluoroethane           &  -178.00    &     5.68  &    13.59  &     5.31  &      f\\
 C$_7$H$_5$F$_3$      & Trifluoromethylbenzene          &  -143.20    &     8.17  &    15.57  &     8.71  &      f\\
 C$_2$HO$_2$F$_3$     & Trifluoroacetic acid            &  -255.00    &    10.92  &    17.52  &    12.24  &      f\\
 NF$_3$         & Nitrogen trifluoride            &   -31.60    &     7.17  &    -2.66  &    -8.47  &      d\\
 C$_2$NF$_3$       & Trifluoroacetonitrile           &  -118.40    &     3.30  &     5.15  &    -1.15  &      d\\
 NOF$_3$        & F3NO                            &   -39.00    &    12.33  &    61.71  &    24.35  &      d\\
 SF$_3$         & Sulfur trifluoride              &  -130.00    &    -4.22  &    91.47  &   -25.38  &      g\\
 SOF$_3$        & SOF3                            &  -185.10    &     8.61  &   180.10  &   -21.67  &      g\\
 CF$_4$         & Carbon tetrafluoride            &  -223.30    &    -1.83  &     9.08  &    -2.46  &      f\\
 C$_2$F$_4$        & Tetrafluoroethylene             &  -157.90    &   -10.33  &   -17.09  &   -17.01  &      f\\
 COF$_4$        & Trifluoromethyl hypofluorite    &  -182.80    &    -4.52  &    19.41  &     4.98  &      d\\
 N$_2$F$_4$        & Tetrafluorohydrazine            &    -2.00    &     4.92  &    -8.19  &    14.66  &      d\\
 SF$_4$         & Sulfur tetrafluoride            &  -182.40    &    -2.92  &   135.89  &   -38.97  &      d\\
 SOF$_4$        & SOF4                            &  -235.50    &    -0.83  &   269.01  &   -18.44  &      g\\
 C$_6$HF$_5$       & Pentafluorobenzene              &  -192.50    &     3.78  &    -9.28  &     0.40  &      f\\
 SF$_5$         & Sulfur pentafluoride            &  -217.10    &   -15.42  &   208.11  &   -63.28  &      d\\
 SF$_5$         & Sulfur pentafluoride (--)        &  -291.00    &   -12.02  &   159.34  &   -34.93  &    aaa\\
 C$_2$F$_6$        & Hexafluoroethane                &  -321.20    &     3.34  &    21.52  &     7.89  &      d\\
 C$_6$F$_6$        & Hexafluorobenzene               &  -228.50    &    -0.89  &   -15.06  &    -2.78  &      f\\
 C$_2$OF$_6$       & Dimethyl perfluoroether         &  -368.95    &   -12.11  &    11.41  &   -11.55  &    bbb\\
 C$_3$OF$_6$       & Perfluoroacetone                &  -325.20    &   -14.90  &     3.15  &    -6.51  &    ccc\\
 SF$_6$         & Sulfur hexafluoride             &  -291.40    &   -13.19  &   320.58  &   -39.41  &      d\\
 C$_3$F$_8$        & Perfluoropropane                &  -426.15    &    11.70  &    41.53  &    22.56  &    bbb\\
 C$_4$F$_8$        & Perfluorocyclobutane            &  -369.50    &    -9.79  &     5.66  &     2.12  &      f\\
 C$_3$O$_2$F$_8$      & CF3-O-CF2-O-CF3                       &  -520.60    &   -15.57  &    18.00  &   -15.16  &    bbb\\
 C$_4$F$_1$$_0$       & n-Perfluorobutane               &  -533.89    &    23.37  &    66.53  &    40.56  &    bbb\\
 S$_2$F$_1$$_0$       & S2F10                           &  -493.40    &    82.69  &   544.19  &    10.58  &      d\\
 Cl          & Chlorine, atom                  &    28.99    &     0.00  &     0.00  &     0.00  &      a\\
 Cl          & Chloride (--)                    &   -55.90    &     4.67  &     1.21  &    18.24  &      d\\
 HCl         & Hydrogen chloride               &   -22.10    &     1.63  &     6.84  &    -2.51  &      d\\
 CCl         & Chloromethylidyne               &   111.30    &    -6.06  &    -3.87  &   -10.17  &      g\\
 CHCl        & Chloromethylene                 &    80.00    &     3.16  &     0.94  &    -2.35  &      d\\
 CH$_3$Cl       & Methyl chloride                 &   -20.00    &     5.30  &    -2.52  &     1.03  &      d\\
 C$_2$HCl       & Chloroacetylene                 &    51.10    &    -4.55  &     1.44  &    -3.36  &      d\\
\hline
\end{tabular}
\end{center}
\end{table}
\clearpage

\begin{table}
\caption{Comparison of Calculated and Observed $\Delta H_f$ 
for MNDO, AM1, and PM3 (contd.)}
\begin{center}
\compresstable
\begin{tabular}{llrrrrr}
Empirical & Chemical Name & $\Delta H_f$ & \multicolumn{3}{c}{Difference} & \\
Formula   &               & Exp. & PM3 &  MNDO  &  AM1 &     Ref.\\
\hline
 C$_2$H$_3$Cl      & Chloroethylene                  &     8.60    &     1.10  &    -3.69  &    -2.75  &      d\\
 C$_2$H$_5$Cl      & Chloroethane                    &   -26.80    &     4.71  &    -2.00  &     0.61  &     aa\\
 HOCl        & Hypochlorous acid               &   -17.80    &   -16.50  &     2.12  &    -3.98  &      d\\
 COCl        & COCl                            &   -15.00    &    -1.11  &    -0.60  &    -0.44  &      d\\
 C$_7$H$_5$OCl     & Benzoyl chloride                &   -26.10    &     7.63  &     2.60  &    10.27  &      f\\
 O$_2$Cl        & Chlorine dioxide                &    25.00    &   -23.55  &   110.99  &    80.87  &      d\\
 CNCl        & Cyanogen chloride               &    31.60    &    -0.03  &     1.64  &    -6.97  &      d\\
 NOCl        & Nitrosyl chloride               &    12.36    &    -7.90  &   -16.59  &    -7.71  &      d\\
 NO$_2$Cl       & Nitryl chloride                 &     2.90    &   -15.90  &    14.37  &    11.67  &      d\\
 SCl         & SCl                             &    41.80    &   -13.16  &   -25.35  &   -23.63  &      g\\
 SOCl        & SOCl                            &   -17.40    &   -13.70  &     1.49  &   -20.20  &      g\\
 SO$_2$Cl       & SO2Cl                           &   -66.40    &     9.07  &    92.20  &     5.34  &      g\\
 FCl         & Chlorine fluoride               &   -12.10    &    -9.60  &    20.29  &     1.56  &      d\\
 HFCl        & Hydrogen chloride fluoride(--)   &  -142.00    &     5.31  &    15.82  &    23.61  &    aaa\\
 CH$_2$FCl      & Fluorochloromethane             &   -62.60    &     5.02  &    -5.40  &    -2.75  &      d\\
 COFCl       & Carbonyl fluoride chloride      &  -102.00    &     8.37  &     9.73  &     9.98  &      d\\
 O$_3$FCl       & Perchloryl fluoride             &    -5.12    &    19.63  &   328.34  &   251.58  &      d\\
 CHF$_2$Cl      & Difluorochloromethane           &  -115.60    &     5.87  &     1.16  &     1.33  &      f\\
 F$_3$Cl        & Chlorine trifluoride            &   -38.00    &    15.90  &   116.69  &    58.19  &      d\\
 CF$_3$Cl       & Trifluorochloromethane          &  -169.20    &    -0.08  &     9.56  &     6.51  &      d\\
 F$_5$Cl        & Chlorine pentafluoride          &   -54.00    &    -0.03  &   258.73  &   144.39  &      d\\
 Cl$_2$         & Chlorine                        &     0.00    &   -11.58  &   -10.68  &   -14.17  &      b\\
 HCl$_2$        & Hydrogen dichloride (--)         &  -142.00    &    30.98  &    47.20  &    47.59  &    aaa\\
 CCl$_2$        & Dichloromethylene               &    57.00    &     0.43  &     0.61  &    -8.57  &      d\\
 CH$_2$Cl$_2$      & Dichloromethane                 &   -22.96    &     5.80  &    -5.09  &    -2.91  &      f\\
 C$_2$H$_2$Cl$_2$     & gem-Dichloroethylene            &     0.60    &     3.36  &    -3.27  &    -3.95  &      d\\
 C$_2$H$_2$Cl$_2$     & cis-Dichloroethylene            &     1.00    &     2.96  &    -3.67  &    -4.35  &      d\\
 C$_2$H$_2$Cl$_2$     & trans-Dichloroethylene          &     1.20    &     2.33  &    -4.96  &    -4.67  &      d\\
 C$_2$H$_4$Cl$_2$     & 1,1-Dichloroethane              &   -30.90    &     4.34  &    -1.68  &    -0.24  &     aa\\
 C$_2$H$_4$Cl$_2$     & 1,2-Dichloroethane              &   -31.00    &     6.28  &    -5.43  &    -2.85  &     aa\\
 C$_6$H$_4$Cl$_2$     & o-Dichlorobenzene               &     7.10    &     3.96  &     1.37  &     2.04  &      f\\
 C$_6$H$_4$Cl$_2$     & m-Dichlorobenzene               &     6.10    &     4.02  &     0.38  &     2.01  &      f\\
 C$_6$H$_4$Cl$_2$     & p-Dichlorobenzene               &     5.30    &     4.74  &     1.00  &     2.59  &      f\\
 OCl$_2$        & Chlorine monoxide               &    25.00    &   -41.26  &     6.28  &    -5.54  &      d\\
 COCl$_2$       & Carbonyl chloride               &   -52.60    &     3.50  &    -0.06  &     5.24  &      d\\
 O$_7$Cl$_2$       & Cl2O7                           &    65.00    &   -91.27  &   613.55  &   489.56  &     aa\\
 SCl$_2$        & Sulfur dichloride               &    -4.20    &    -6.75  &   -19.66  &   -22.06  &      d\\
 SOCl$_2$       & Thionyl chloride                &   -50.80    &     3.18  &    28.61  &   -13.53  &     cc\\
 SO$_2$Cl$_2$      & Sulfuryl chloride               &   -86.20    &     5.25  &   130.78  &    16.87  &      d\\
 S$_2$Cl$_2$       & ClSSCl                          &    -4.00    &    -3.68  &   -16.82  &   -20.58  &      d\\
 CHFCl$_2$      & Fluorodichloromethane           &   -67.70    &     5.70  &    -1.52  &     2.45  &    ddd\\
 CF$_2$Cl$_2$      & Difluorodichloromethane         &  -117.50    &     1.42  &     7.30  &    10.45  &      d\\
 CCl$_3$        & Trichloromethyl                 &    21.00    &   -19.46  &   -20.55  &   -25.76  &      g\\
 CHCl$_3$       & Chloroform                      &   -24.66    &     3.76  &    -4.31  &    -4.37  &    ddd\\
 C$_2$HCl$_3$      & Trichloroethylene               &    -2.00    &    -0.37  &    -4.43  &    -6.45  &      d\\
 C$_2$H$_3$Cl$_3$     & 1,1,1-Trichloroethane           &   -35.50    &     3.55  &     4.00  &     3.53  &      f\\
\hline
\end{tabular}
\end{center}
\end{table}
\clearpage

\begin{table}
\caption{Comparison of Calculated and Observed $\Delta H_f$ 
for MNDO, AM1, and PM3 (contd.)}
\begin{center}
\compresstable
\begin{tabular}{llrrrrr}
Empirical & Chemical Name & $\Delta H_f$ & \multicolumn{3}{c}{Difference} & \\
Formula   &               & Exp. & PM3 &  MNDO  &  AM1 &     Ref.\\
\hline
 SCl$_3$        & Sulfur trichloride              &     8.80    &   -27.93  &   -41.31  &   -68.90  &      g\\
 SOCl$_3$       & SOCl3                           &   -47.50    &     0.98  &    49.86  &   -22.29  &      g\\
 CFCl$_3$       & Fluorotrichloromethane          &   -69.00    &     1.64  &     3.74  &     8.06  &      d\\
 CCl$_4$        & Carbon tetrachloride            &   -22.90    &    -3.09  &    -2.62  &    -5.26  &      d\\
 C$_2$Cl$_4$       & Tetrachloroethylene             &    -2.70    &    -5.42  &    -5.33  &    -9.78  &      f\\
 SCl$_4$        & Sulfur tetrachloride            &    -0.70    &   -19.13  &   -32.65  &   -82.03  &      g\\
 SOCl$_4$       & SOCl4                           &   -55.70    &    26.16  &   142.61  &    52.88  &      g\\
 SCl$_5$        & Sulfur pentachloride            &    -8.60    &     1.09  &    43.63  &   -23.07  &      g\\
 C$_2$Cl$_6$       & Hexachloroethane                &   -34.50    &    -2.02  &     7.10  &    -1.35  &      f\\
 C$_6$Cl$_6$       & Hexachlorobenzene               &    -8.60    &    -0.60  &     5.92  &     0.66  &      f\\
 SCl$_6$        & Sulfur hexachloride             &   -19.80    &    30.07  &   138.09  &    54.56  &      g\\
 Br          & Bromine, atom                   &    26.74    &     0.00  &     0.00  &     0.00  &      a\\
 Br          & Bromide (--)                     &   -52.30    &    -3.94  &    14.80  &    31.89  &      d\\
 HBr         & Hydrogen bromide                &    -8.71    &    14.02  &    12.35  &    -1.80  &      d\\
 HBr         & Hydrogen bromide (+)            &   261.10    &    13.61  &    16.77  &   -12.33  &    eee\\
 CBr         & Bromomethylidyne                &   125.90    &    12.48  &     1.62  &     0.87  &      g\\
 CH$_3$Br       & Bromomethane                    &    -9.10    &     7.10  &    -1.27  &     2.90  &      f\\
 C$_2$H$_3$Br      & Bromoethylene                   &    18.71    &     5.05  &    -2.92  &    -0.78  &      f\\
 C$_2$H$_5$Br      & Bromoethane                     &   -15.20    &     3.80  &    -1.86  &     2.05  &      f\\
 C$_3$H$_5$Br      & 3-Bromopropene                  &    10.90    &     4.32  &    -2.10  &     1.45  &    fff\\
 C$_3$H$_7$Br      & 1-Bromopropane                  &   -20.50    &     3.77  &    -1.54  &     0.52  &      f\\
 C$_3$H$_7$Br      & 2-Bromopropane                  &   -23.50    &     2.61  &     2.79  &     5.54  &    ggg\\
 C$_6$H$_5$Br      & Bromobenzene                    &    25.20    &     5.77  &    -1.38  &     1.48  &      f\\
 OBr         & BrO                             &    30.06    &    -9.28  &     5.27  &     5.60  &    eee\\
 HOBr        & Hypobromous acid                &   -20.00    &   -13.95  &    -2.74  &    -4.73  &    hhh\\
 COBr        & COBr                            &    20.50    &   -30.75  &   -28.80  &   -25.61  &      g\\
 C$_2$H$_3$OBr     & Acetyl bromide                  &   -45.60    &     2.03  &     2.33  &    11.23  &      f\\
 C$_7$H$_5$OBr     & Benzoyl bromide                 &   -11.60    &     3.02  &     0.58  &    12.09  &      f\\
 CNBr        & Cyanogen bromide                &    43.30    &    10.32  &    -3.31  &   -10.85  &      f\\
 NOBr        & Nitrosyl bromide                &    19.63    &   -13.04  &   -17.79  &     1.64  &      d\\
 SBr         & SBr                             &    56.10    &    -7.83  &   -26.72  &   -25.61  &      g\\
 SOBr        & SOBr                            &    -4.30    &   -11.92  &     2.17  &   -16.18  &      g\\
 SO$_2$Br       & SO2Br                           &   -52.80    &     9.92  &    89.44  &    11.02  &      g\\
 FBr         & Bromine fluoride                &   -14.00    &    -7.26  &     8.18  &     6.78  &      d\\
 F$_3$Br        & Bromine trifluoride             &   -61.10    &    13.97  &    83.96  &    82.57  &      d\\
 CF$_3$Br       & Bromotrifluoromethane           &  -155.10    &    -2.78  &     8.47  &    10.49  &      d\\
 F$_5$Br        & Bromine pentafluoride           &  -102.50    &    26.63  &   207.34  &   183.61  &      d\\
 ClBr        & Bromine chloride                &     3.50    &    -6.70  &   -12.99  &   -14.13  &      d\\
 ClBr        & Bromine chloride (+)            &   260.00    &   -12.36  &     5.80  &   -13.61  &     aa\\
 CCl$_3$Br      & Trichlorobromomethane           &    -9.30    &    -4.79  &    -5.83  &    -5.41  &      f\\
 Br$_2$         & Bromine                         &     7.40    &    -2.48  &    -9.08  &   -12.68  &      d\\
 Br$_2$         & Bromine (+)                     &   253.50    &     9.49  &    12.53  &    -7.73  &    eee\\
 CBr$_2$        & Dibromomethylene                &    84.30    &    20.63  &     7.77  &     5.71  &      g\\
 CH$_2$Br$_2$      & Dibromomethane                  &    -3.53    &    11.38  &    -1.60  &     2.50  &    iii\\
 COBr$_2$       & Carbonyl bromide                &   -20.10    &    -5.26  &   -11.54  &     2.22  &      f\\
 SBr$_2$        & Sulfur dibromide                &    -3.00    &    27.85  &     2.76  &     0.86  &    jjj\\
\hline
\end{tabular}
\end{center}
\end{table}
\clearpage

\begin{table}
\caption{Comparison of Calculated and Observed $\Delta H_f$ 
for MNDO, AM1, and PM3 (contd.)}
\begin{center}
\compresstable
\begin{tabular}{llrrrrr}
Empirical & Chemical Name & $\Delta H_f$ & \multicolumn{3}{c}{Difference} & \\
Formula   &               & Exp. & PM3 &  MNDO  &  AM1 &     Ref.\\
\hline
 SOBr$_2$       & Thionyl bromide                 &   -11.50    &    -7.15  &    16.12  &   -16.55  &      g\\
 SO$_2$Br$_2$      & Sulfuryl bromide                &   -59.50    &    13.29  &   127.83  &    40.61  &      g\\
 S$_2$Br$_2$       & S2Br2                           &     7.40    &    14.39  &    -5.90  &   -11.83  &    jjj\\
 C$_2$F$_4$Br$_2$     & 1,2-Dibromotetrafluoroethane    &  -189.00    &    -5.35  &    16.70  &    23.66  &    ggg\\
 CBr$_3$        & Tribromomethyl                  &    64.70    &    -0.48  &   -35.85  &   -37.81  &      g\\
 CHBr$_3$       & Bromoform                       &     4.40    &    13.13  &    -1.37  &     1.94  &    iii\\
 SBr$_3$        & Sulfur tribromide               &    50.20    &   -33.82  &   -51.28  &   -76.36  &      g\\
 SOBr$_3$       & SOBr3                           &    -8.60    &    -1.46  &    41.38  &    -8.43  &      g\\
 CBr$_4$        & Carbon tetrabromide             &    35.10    &    -2.21  &   -21.31  &   -19.21  &      g\\
 SBr$_4$        & Sulfur tetrabromide             &    53.00    &   -34.27  &   -42.94  &   -88.51  &      g\\
 SOBr$_4$       & SOBr4                           &    -3.30    &   -18.35  &    26.54  &   -26.47  &      g\\
 SBr$_5$        & Sulfur pentabromide             &    55.90    &   -11.91  &    23.50  &   -37.10  &      g\\
 SBr$_6$        & Sulfur hexabromide              &    58.80    &    19.24  &   107.54  &    30.60  &      g\\
 I           & Iodide (--)                      &   -46.50    &   -18.12  &    40.09  &    44.28  &      d\\
 HI          & Hydrogen iodide                 &     6.30    &    22.50  &     9.44  &     1.63  &      d\\
 CI          & Iodomethylidyne                 &   144.80    &     0.66  &     7.11  &     6.17  &      g\\
 CH$_3$I        & Methyl iodide                   &     3.40    &     6.03  &    -1.51  &     2.25  &      f\\
 C$_2$H$_5$I       & Iodoethane                      &    -2.00    &     4.06  &    -2.50  &     0.90  &      f\\
 C$_3$H$_5$I       & Allyl iodide                    &    22.80    &     4.32  &    -3.32  &    -0.42  &      f\\
 C$_3$H$_5$I       & E-1-Iodo-1-propene              &    22.26    &     3.12  &    -7.83  &    -2.60  &      n\\
 C$_3$H$_5$I       & Z-1-Iodo-1-propene              &    20.66    &     8.54  &    -5.45  &    -0.21  &      n\\
 C$_3$H$_7$I       & 1-Iodopropane                   &    -7.10    &     4.11  &    -2.25  &    -0.78  &      f\\
 C$_3$H$_7$I       & 2-Iodopropane                   &    -9.80    &     4.51  &     2.26  &     4.06  &      f\\
 C$_4$H$_9$I       & 1-Butyl iodide                  &   -12.00    &     3.64  &    -2.09  &    -2.75  &      f\\
 C$_4$H$_9$I       & 2-Iodo-2-methylpropane          &   -17.22    &     4.69  &    10.63  &     8.87  &      n\\
 C$_6$H$_5$I       & Iodobenzene                     &    39.40    &     5.27  &    -7.00  &    -1.35  &      n\\
 C$_6$H$_1$$_1$I      & Iodocyclohexane                 &   -11.94    &     0.17  &    -5.12  &    -8.13  &      n\\
 C$_7$H$_7$I       & o-Iodotoluene                   &    31.74    &     7.03  &    -4.79  &     0.02  &      n\\
 C$_7$H$_7$I       & m-Iodotoluene                   &    31.90    &     3.37  &    -7.20  &    -1.51  &      n\\
 C$_7$H$_7$I       & p-Iodotoluene                   &    29.10    &     6.16  &    -4.49  &     1.17  &      f\\
 C$_7$H$_7$I       & Benzyl iodide                   &    25.10    &    12.45  &     3.47  &     5.70  &      f\\
 OI          & IO                              &    41.84    &   -10.90  &     4.82  &    -4.88  &    eee\\
 COI         & COI                             &    63.50    &   -66.43  &   -62.47  &   -58.78  &      g\\
 C$_2$H$_3$OI      & Acetyl iodide                   &   -30.20    &     0.23  &     3.21  &     9.42  &      n\\
 C$_3$H$_5$OI      & 1-Iodo-2-propanone              &   -31.22    &     5.10  &    -1.63  &     0.13  &      n\\
 C$_7$H$_5$OI      & Benzoyl iodide                  &     2.52    &     5.40  &     2.48  &    11.51  &      n\\
 CNI         & Cyanogen iodide                 &    53.68    &     9.82  &   -14.14  &   -11.14  &      n\\
 NOI         & Nitrosyl iodide                 &    26.80    &    -8.64  &    -5.90  &     5.42  &      d\\
 SI          & SI                              &    73.10    &   -15.07  &   -26.46  &   -24.46  &      g\\
 SOI         & SOI                             &    12.70    &   -13.11  &     6.95  &   -13.53  &      g\\
 SO$_2$I        & SO2I                            &   -34.90    &     7.65  &    64.62  &    10.36  &      g\\
 FI          & Iodine fluoride                 &   -22.65    &    14.65  &    13.35  &    13.54  &      d\\
 CF$_3$I        & Trifluoroiodomethane            &  -140.49    &     2.52  &    12.27  &     7.82  &      n\\
 F$_5$I         & Iodine pentafluoride            &  -200.84    &    -2.10  &   331.69  &   295.27  &      d\\
 F$_7$I         & Iodine heptafluoride            &  -229.70    &     1.96  &   334.86  &   274.06  &      d\\
 ClI         & Iodine chloride                 &     4.57    &     6.21  &   -11.46  &    -9.18  &    ddd\\
\hline
\end{tabular}
\end{center}
\end{table}
\clearpage

\begin{table}
\caption{Comparison of Calculated and Observed $\Delta H_f$ 
for MNDO, AM1, and PM3 (contd.)}
\begin{center}
\compresstable
\begin{tabular}{llrrrrr}
Empirical & Chemical Name & $\Delta H_f$ & \multicolumn{3}{c}{Difference} & \\
Formula   &               & Exp. & PM3 &  MNDO  &  AM1 &     Ref.\\
\hline
 BrI         & Iodine bromide                  &     9.77    &     5.88  &    -2.54  &    -3.77  &      d\\
 I$_2$          & Iodine                          &    14.92    &     5.82  &     6.30  &     4.91  &      d\\
 CI$_2$         & Diiodomethylene                 &   120.40    &     1.16  &   -15.53  &     1.18  &      g\\
 CH$_2$I$_2$       & Diiodomethane                   &    27.00    &     6.49  &   -10.20  &    -5.52  &    eee\\
 C$_2$H$_2$I$_2$      & E-1,2-Diiodoethene              &    49.56    &     5.39  &   -14.37  &    -5.52  &      n\\
 C$_2$H$_2$I$_2$      & Z-1,2-Diiodoethene              &    49.56    &    13.19  &   -14.52  &    -6.09  &      n\\
 C$_2$H$_4$I$_2$      & 1,2-Diiodoethane                &    15.96    &     7.33  &    -4.46  &    -0.25  &      n\\
 C$_3$H$_6$I$_2$      & 1,2-Diiodopropane               &     8.50    &    12.24  &    -1.24  &     1.19  &      f\\
 C$_4$H$_8$I$_2$      & 1,2-Diiodobutane                &     2.70    &    13.38  &     3.61  &     3.49  &      f\\
 C$_6$H$_4$I$_2$      & o-Diiodobenzene                 &    60.16    &    13.57  &   -15.49  &    -6.20  &      n\\
 COI$_2$        & Carbonyl iodide                 &     9.60    &   -11.08  &   -15.15  &    -3.07  &      g\\
 SI$_2$         & Sulfur diiodide                 &    81.90    &   -30.33  &   -52.36  &   -56.22  &      g\\
 SOI$_2$        & Thionyl iodide                  &    21.50    &   -10.40  &    26.53  &   -11.32  &      g\\
 SO$_2$I$_2$       & Sulfuryl iodide                 &   -26.00    &    -4.06  &    51.60  &    -1.22  &      g\\
 S$_2$I$_2$        & S2I2                            &    59.00    &   -13.02  &   -26.75  &   -30.81  &      g\\
 CI$_3$         & Triiodomethyl                   &   117.30    &    -9.78  &   -68.03  &   -52.71  &      g\\
 CHI$_3$        & Iodoform                        &    50.40    &    10.21  &   -18.41  &   -12.60  &    iii\\
 SI$_3$         & Sulfur triiodide                &   100.30    &   -45.04  &   -60.00  &   -83.97  &      g\\
 SOI$_3$        & SOI3                            &    40.40    &   -16.34  &     1.69  &   -33.55  &      g\\
 CI$_4$         & Carbon tetraiodide              &   108.20    &    -5.54  &   -61.38  &   -54.02  &      g\\
 SI$_4$         & Sulfur tetraiodide              &   120.20    &   -50.98  &   -49.07  &  -102.72  &      g\\
 SOI$_4$        & SOI4                            &    60.00    &   -26.92  &   -17.15  &   -32.33  &      g\\
 SI$_5$         & Sulfur pentaiodide              &   130.90    &    -1.25  &    -1.06  &   -68.65  &      g\\
 SI$_6$         & Sulfur hexaiodide               &   158.90    &     8.10  &    48.67  &   -30.82  &      g\\
 C$_2$LiH$_5$      & Ethyl-lithium                  &    13.90    &  &   -25.58  &  &      f\\
 C$_4$LiH$_9$      & n-Butyl-lithium                &    -6.00    &  &   -23.36  &  &      f\\
 Be          & Beryllium, atom                 &    76.96    &     0.00  &     0.00  &     0.00  &      a\\
 BeH         & Beryllium hydride               &    75.60    &     7.17  &   -15.34  &   -10.43  &     aa\\
 BeH         & Beryllium hydride (+)           &   276.40    &   -23.49  &   -37.86  &   -38.02  &      d\\
 BeH$_2$        & Beryllium dihydride             &    30.00    &    20.29  &   -24.41  &   -16.49  &      d\\
 BeO         & Beryllium oxide                 &    32.60    &    20.44  &     5.97  &    37.90  &      d\\
 BeHO        & Beryllium hydroxide             &   -27.40    &     1.05  &     1.20  &    37.46  &      d\\
 BeH$_2$O$_2$      & Beryllium di-hydroxide          &  -161.80    &    20.31  &    25.44  &    85.79  &      d\\
 BeF         & Beryllium fluoride              &   -41.80    &    -8.27  &   -10.92  &    18.16  &     aa\\
 BeF$_2$        & Beryllium difluoride            &  -189.70    &     2.37  &    -2.22  &    54.79  &     aa\\
 BeCl        & Beryllium chloride              &    14.51    &    -8.81  &   -10.81  &   -13.97  &      d\\
 BeCl$_2$       & Beryllium dichloride            &   -86.10    &    -0.39  &    -3.75  &   -13.66  &      d\\
 BeBr        & Beryllium bromide               &    28.70    &    -6.18  &    -0.15  &   -28.34  &      d\\
 BeBr$_2$       & Beryllium dibromide             &   -54.80    &     1.33  &    11.77  &   -51.28  &      d\\
 BeI         & Beryllium iodide                &    40.60    &    -9.34  &     1.31  &    -1.03  &      d\\
 BeI$_2$        & Beryllium diiodide              &   -15.30    &     0.88  &    -8.62  &   -16.06  &      d\\
 Be$_2$         & Beryllium, dimer                &   152.30    &   -15.59  &   -26.89  &   -26.89  &      d\\
 Be$_2$O        & Beryllium oxide                 &   -15.00    &    -4.74  &    40.50  &    80.34  &      d\\
 Be$_2$OF$_2$      & Be(OF)2                         &  -287.90    &     9.43  &     1.80  &   125.89  &      d\\
 Be$_3$O$_3$       & Be3O3                           &  -251.90    &    -3.22  &    71.80  &   206.51  &      d\\
 Mg          & Magnesium, atom                 &    35.00    &     0.00  &  &  &      a\\
\hline
\end{tabular}
\end{center}
\end{table}
\clearpage

\begin{table}
\caption{Comparison of Calculated and Observed $\Delta H_f$ 
for MNDO, AM1, and PM3 (contd.)}
\begin{center}
\compresstable
\begin{tabular}{llrrrrr}
Empirical & Chemical Name & $\Delta H_f$ & \multicolumn{3}{c}{Difference} & \\
Formula   &               & Exp. & PM3 &  MNDO  &  AM1 &     Ref.\\
\hline
 Mg          & Magnesium (+)                   &   213.10    &     4.76  &  &  &     aa\\
 Mg          & Magnesium (++)                  &   561.30    &    -6.21  &  &  &     aa\\
 MgH         & Magnesium hydride               &    40.50    &     1.31  &  &  &      d\\
 C$_1$$_0$MgH$_1$$_0$    & Dicyclopentadienyl magnesium    &    32.60    &    -1.13  &  &  &      f\\
 MgO         & Magnesium oxide                 &    14.00    &    -1.95  &  &  &      d\\
 MgHO        & Magnesium hydroxide             &   -22.00    &   -33.05  &  &  &    mmm\\
 MgH$_2$O$_2$      & Magnesium di-hydroxide          &  -135.00    &     8.24  &  &  &      d\\
 MgN         & Magnesium nitride               &    69.00    &    34.99  &  &  &      d\\
 MgS         & Magnesium sulfide               &    62.00    &   -19.64  &  &  &      d\\
 MgF         & Magnesium fluoride              &   -56.60    &   -14.68  &  &  &      d\\
 MgF$_2$        & Magnesium difluoride            &  -173.00    &    12.31  &  &  &     aa\\
 MgCl        & Magnesium chloride              &   -10.30    &   -22.63  &  &  &      d\\
 MgCl$_2$       & Magnesium dichloride            &   -93.80    &    11.40  &  &  &      d\\
 MgBr        & Magnesium bromide               &    -8.40    &   -13.35  &  &  &      d\\
 MgBr$_2$       & Magnesium dibromide             &   -72.40    &     4.52  &  &  &      d\\
 MgI$_2$        & Magnesium diiodide              &   -41.10    &    -3.48  &  &  &     aa\\
 Mg$_2$         & Magnesium, dimer                &    68.77    &     1.23  &  &  &     aa\\
 Mg$_2$F$_4$       & Magnesium difluoride, dimer     &  -410.70    &     2.28  &  &  &      d\\
 Mg$_2$Cl$_4$      & Magnesium dichloride, dimer     &  -228.10    &    -4.00  &  &  &      d\\
 Mg$_2$Br$_4$      & Magnesium dibromide, dimer      &  -183.50    &     2.33  &  &  &      d\\
 BH          & BH                             &   107.40    &  &   -35.34  &   -25.10  &    nnn\\
 BH          & BH (+)                         &   334.20    &  &   -30.29  &   -46.33  &     aa\\
 BH$_2$         & BH2                            &    45.70    &  &    20.74  &    30.04  &    ooo\\
 BH$_3$         & Borane                         &    23.90    &  &   -12.18  &     2.35  &    nnn\\
 C$_3$BH$_9$       & Trimethyborane                 &   -29.30    &  &   -10.79  &     4.48  &    nnn\\
 C$_6$BH$_1$$_5$      & Triethylborane                 &   -37.70    &  &   -13.45  &    -2.07  &    nnn\\
 C$_9$BH$_2$$_1$      & Tripropylborane                &   -56.60    &  &    -6.96  &    -3.71  &    ppp\\
 C$_1$$_2$BH$_2$$_7$     & Tributylborane                 &   -68.60    &  &    -8.49  &   -10.56  &      f\\
 BO          & BO                             &     6.00    &  &    -7.23  &    -5.19  &    nnn\\
 BH$_3$O        & H2BOH                          &   -69.40    &  &    -8.81  &    -4.02  &    ooo\\
 CBH$_3$O       & BH3CO                          &   -26.60    &  &   -19.58  &    15.49  &    nnn\\
 BO$_2$         & BO2.                           &   -71.70    &  &     3.25  &     0.98  &    nnn\\
 BHO$_2$        & HOBO                           &  -134.30    &  &     1.43  &    -4.18  &     aa\\
 BHO$_2$        & OBOH                           &  -134.20    &  &     1.32  &    -4.28  &    nnn\\
 BH$_3$O$_2$       & HB(OH)2                        &  -153.10    &  &    -2.34  &   -10.62  &    ooo\\
 C$_2$BH$_7$O$_2$     & (MeO)2BH                       &  -138.40    &  &    -3.46  &    -2.18  &    nnn\\
 C$_6$BH$_7$O$_2$     & PhB(OH)2                       &  -151.80    &  &    12.78  &    15.15  &    ooo\\
 BH$_3$O$_3$       & B(OH)3                         &  -237.60    &  &     1.27  &   -14.06  &     aa\\
 C$_3$BH$_9$O$_3$     & Trimethoxyborane               &  -214.90    &  &     4.36  &    -0.97  &    nnn\\
 C$_6$BH$_1$$_5$O$_3$    & Triethoxyborane                &  -239.50    &  &    14.87  &     6.02  &    nnn\\
 BN          & BN                             &   114.00    &  &    25.29  &    18.01  &      d\\
 C$_3$BH$_1$$_2$N     & Trimethyborane-Ammonia adduct  &   -54.10    &  &     1.58  &    -0.46  &    nnn\\
 C$_3$BH$_1$$_2$N     & Trimethyamine-Borane adduct    &   -20.30    &  &    26.45  &    23.98  &    nnn\\
 C$_6$BH$_1$$_8$N     & Trimethyborane-Trimethylamine  &   -52.60    &  &     9.68  &    25.33  &    nnn\\
 C$_6$BH$_1$$_8$N$_3$    & B(N(Me)2)3                     &   -65.90    &  &    33.52  &    31.50  &      f\\
 BS          & Boron sulfide                  &    58.10    &  &    13.97  &    19.48  &    jjj\\
\hline
\end{tabular}
\end{center}
\end{table}
\clearpage

\begin{table}
\caption{Comparison of Calculated and Observed $\Delta H_f$ 
for MNDO, AM1, and PM3 (contd.)}
\begin{center}
\compresstable
\begin{tabular}{llrrrrr}
Empirical & Chemical Name & $\Delta H_f$ & \multicolumn{3}{c}{Difference} & \\
Formula   &               & Exp. & PM3 &  MNDO  &  AM1 &     Ref.\\
\hline
 C$_2$BH$_9$S      & (CH3)2S-BH3                           &   -10.60    &  &    -8.45  &     1.40  &     aa\\
 C$_4$BH$_1$$_3$S     & (CH3CH2)2S-BH3                        &   -30.50    &  &     0.79  &     8.50  &     aa\\
 BS$_2$         & BS2                                   &    28.90    &  &   -15.01  &    -1.71  &     aa\\
 BF          & BF                                    &   -29.20    &  &   -33.92  &   -27.36  &    nnn\\
 BOF         & FBO                                   &  -145.00    &  &     4.82  &    -4.25  &    nnn\\
 BH$_2$O$_2$F      & FB(OH)2                               &  -249.50    &  &     7.97  &    -4.85  &    nnn\\
 CBH$_3$F$_2$      & CH3BF2                                &  -199.00    &  &     7.53  &     6.13  &    nnn\\
 C$_2$BH$_3$F$_2$     & F2BCHCH2                              &  -171.00    &  &     4.40  &     6.73  &    nnn\\
 BOF$_2$        & F2BO                                  &  -200.00    &  &    10.14  &     1.68  &    ddd\\
 BHOF$_2$       & F2BOH                                 &  -260.70    &  &     7.42  &    -4.27  &    nnn\\
 BF$_3$         & Boron trifluoride                     &  -271.60    &  &    10.60  &    -0.59  &    nnn\\
 C$_2$BH$_6$OF$_3$    & (CH3)2O-BF3                           &  -328.20    &  &    15.46  &    -4.49  &    nnn\\
 C$_4$BH$_1$$_0$OF$_3$   & (CH3CH2)2O-BF3                        &  -343.80    &  &    21.35  &     0.16  &     aa\\
 C$_3$BH$_9$NF$_3$    & Trimethyamine-Boron trifluoride       &  -304.40    &  &    40.18  &    32.12  &     aa\\
 BCl         & Boron chloride                        &    35.70    &  &   -26.48  &   -22.21  &    nnn\\
 BOCl        & BOCl                                  &   -75.00    &  &   -17.04  &   -24.29  &    nnn\\
 C$_2$BH$_6$O$_2$Cl   & (MeO)2BCl                             &  -178.20    &  &    12.29  &    12.75  &     aa\\
 BF$_2$Cl       & BClF2                                 &  -212.70    &  &    11.07  &     4.16  &    nnn\\
 BFCl$_2$       & BCl2F                                 &  -154.10    &  &     9.99  &     5.11  &    nnn\\
 BCl$_3$        & Boron trichloride                     &   -96.40    &  &     8.94  &    -0.65  &    nnn\\
 BBr         & BBr                                   &    56.90    &  &   -23.22  &   -19.67  &    nnn\\
 BF$_2$Br       & BBrF2                                 &  -196.00    &  &    12.82  &     5.91  &    ddd\\
 BBr$_3$        & Boron tribromide                      &   -49.10    &  &    15.48  &    -0.98  &    nnn\\
 BI$_3$         & Boron triiodide                       &    17.00    &  &    -8.39  &     2.41  &    nnn\\
 B$_2$          & Boron dimer                           &   198.40    &  &    -5.23  &     8.75  &    nnn\\
 B$_2$H$_4$        & B2H4 D2d                              &    45.70    &  &    63.43  &    80.87  &    qqq\\
 B$_2$H$_4$        & B2H4 (C2v)                            &    45.60    &  &    29.53  &    47.57  &    qqq\\
 B$_2$H$_6$        & Diborane                              &     8.50    &  &   -10.35  &    -3.22  &     aa\\
 B$_2$O$_2$        & BOBO                                  &  -108.60    &  &   -29.02  &   -24.56  &    nnn\\
 B$_2$O$_2$        & O-B-B-O                               &  -109.00    &  &   -28.62  &   -24.16  &    rrr\\
 B$_2$O$_3$        & B2O3                                  &  -201.50    &  &     2.59  &    -5.48  &    nnn\\
 B$_2$H$_2$O$_3$      & B2O3H2                                &  -200.40    &  &    57.48  &    34.16  &    sss\\
 B$_2$H$_4$O$_4$      & B2(OH)4                               &  -315.00    &  &     7.09  &    -0.48  &    ooo\\
 C$_2$B$_2$H$_1$$_1$N    & B2H5N(CH3)2                           &   -27.90    &  &    29.71  &    23.02  &      f\\
 B$_2$S$_2$        & S-B-B-S                               &    36.10    &  &   -35.38  &   -15.63  &     aa\\
 B$_2$F$_4$        & B2F4                                  &  -344.00    &  &     4.76  &   -10.03  &    nnn\\
 B$_2$Cl$_4$       & B2Cl4                                 &  -117.10    &  &     6.91  &    -3.03  &    nnn\\
 B$_3$H$_3$O$_3$      & B3O3H3                                &  -288.80    &  &    17.21  &     2.22  &    nnn\\
 B$_3$H$_3$O$_6$      & B3O3(OH)3                             &  -543.60    &  &    42.53  &     9.79  &    nnn\\
 B$_3$H$_6$N$_3$      & Borazole                              &  -122.20    &  &    -8.92  &   -19.97  &    nnn\\
 B$_3$H$_2$O$_3$F     & B3O3H2F                               &  -382.60    &  &    25.76  &     4.65  &    nnn\\
 B$_3$HO$_3$F$_2$     & B3O3HF2                               &  -475.80    &  &    34.23  &     7.08  &    nnn\\
 B$_3$O$_3$F$_3$      & B3O3F3                                &  -567.70    &  &    42.11  &     9.26  &    nnn\\
 B$_3$H$_2$O$_3$Cl    & B3O3H2Cl                              &  -313.80    &  &    14.68  &     0.15  &    nnn\\
 B$_3$HO$_3$Cl$_2$    & B3O3HCl2                              &  -338.70    &  &    12.63  &    -1.19  &    nnn\\
 B$_3$O$_3$Cl$_3$     & B3O3Cl3                               &  -390.10    &  &    37.61  &    24.87  &    nnn\\
\hline
\end{tabular}
\end{center}
\end{table}
\clearpage

\begin{table}
\caption{Comparison of Calculated and Observed $\Delta H_f$ 
for MNDO, AM1, and PM3 (contd.)}
\begin{center}
\compresstable
\begin{tabular}{llrrrrr}
Empirical & Chemical Name & $\Delta H_f$ & \multicolumn{3}{c}{Difference} & \\
Formula   &               & Exp. & PM3 &  MNDO  &  AM1 &     Ref.\\
\hline
 B$_3$H$_3$N$_3$Cl$_3$   & B3N3Cl3H3                             &  -237.70    &  &    12.66  &   -20.75  &    nnn\\
 Al          & Aluminum, atom                         &    79.49    &   -33.46  &     0.00  &     0.00  &      a\\
 Al          & Al (+)                                 &   218.10    &    61.65  &   -24.23  &    -4.77  &      d\\
 HAl         & AlH                                    &    62.00    &     8.06  &   -16.03  &    -9.11  &      d\\
 C$_3$H$_9$Al      & Trimethylaluminum                      &   -20.90    &    15.18  &   -19.21  &    -6.58  &      f\\
 C$_6$H$_1$$_5$Al     & Triethylaluminum                       &   -39.10    &     4.79  &   -16.71  &    -7.12  &      f\\
 AlO         & AlO                                    &    16.00    &   -23.89  &   -17.79  &    -7.60  &      d\\
 AlO         & AlO   (+)                              &   237.30    &   -34.94  &   -14.26  &    -2.69  &      d\\
 AlO         & AlO   (--)                              &   -64.40    &    13.56  &     8.62  &    11.54  &      d\\
 HAlO        & Al-O-H                                 &   -43.00    &     9.64  &   -18.10  &     2.93  &      d\\
 HAlO        & AlOH  (+)                              &   130.00    &     6.42  &    -8.44  &    27.95  &      d\\
 HAlO        & AlOH  (--)                              &   -55.00    &   -50.68  &    13.91  &    10.26  &      d\\
 HAlO        & H-Al=O                                 &    -8.00    &    10.19  &   -13.71  &   -13.89  &      d\\
 AlO$_2$        & AlO2                                   &   -20.60    &   -16.79  &     6.18  &    16.70  &      d\\
 AlO$_2$        & AlO2  (--)                              &  -116.00    &    -1.09  &    25.51  &    20.53  &      d\\
 HAlO$_2$       & AlO2H                                  &  -110.00    &     4.90  &    16.19  &    30.47  &      d\\
 AlN         & Aluminum nitride                       &   125.00    &   -49.55  &    15.91  &    -7.09  &      d\\
 AlS         & Aluminum sulfide                       &    57.10    &   -25.70  &    -4.52  &    -8.26  &    jjj\\
 AlF         & Aluminum fluoride                      &   -63.50    &    13.37  &   -20.07  &   -14.37  &      d\\
 AlF         & AlF (+)                                &   165.40    &    -5.03  &   -50.58  &   -38.41  &      d\\
 AlOF        & AlFO                                   &  -139.00    &    14.29  &    25.40  &    28.85  &      d\\
 AlF$_2$        & AlF2                                   &  -166.00    &     3.26  &    -5.15  &    -3.01  &      d\\
 AlF$_2$        & AlF2 (+)                               &    22.00    &     7.04  &    -7.83  &    -3.62  &      d\\
 AlF$_2$        & AlF2 (--)                               &  -217.00    &   -12.60  &     7.42  &     6.13  &      d\\
 AlOF$_2$       & AlF2O                                  &  -265.00    &    56.45  &    62.47  &    73.09  &      d\\
 AlOF$_2$       & AlF2O (--)                              &  -311.56    &    25.59  &    53.68  &    54.40  &      d\\
 AlF$_3$        & Aluminum trifluoride                   &  -289.00    &    -2.50  &    -2.34  &     3.15  &      d\\
 AlF$_4$        & AlF4(--)                                &  -476.00    &     6.82  &    35.35  &    48.30  &      d\\
 AlCl        & Aluminum chloride                      &   -12.30    &     6.82  &   -15.55  &    -8.55  &      d\\
 AlCl        & Aluminum chloride (+)                  &   206.00    &    -7.27  &   -31.22  &   -25.91  &      d\\
 AlOCl       & AlClO                                  &   -83.20    &    10.78  &    14.51  &    16.76  &      d\\
 AlFCl       & AlClF                                  &  -117.00    &     0.96  &    -7.76  &    -3.54  &      d\\
 AlFCl       & Aluminum chloride fluoride (+)         &    66.00    &     8.95  &    -2.50  &    -5.25  &      d\\
 AlF$_2$Cl      & AlClF2                                 &  -238.80    &     4.21  &    -2.86  &     2.60  &      d\\
 AlCl$_2$       & Aluminum dichloride                    &   -67.00    &    -1.21  &    -7.58  &    -7.01  &      d\\
 AlCl$_2$       & AlCl2 (+)                              &   115.00    &     9.33  &     3.92  &   -10.82  &      d\\
 AlCl$_2$       & AlCl2 (--)                              &  -115.00    &   -29.99  &   -15.61  &    -4.95  &      d\\
 AlFCl$_2$      & AlCl2F                                 &  -189.00    &    11.02  &    -2.37  &     2.31  &      d\\
 AlCl$_3$       & Aluminum trichloride                   &  -139.72    &    17.63  &    -0.64  &    -0.59  &      d\\
 AlBr$_3$       & Aluminum tribromide                    &   -98.10    &    12.25  &    37.80  &     1.64  &      d\\
 AlI         & AlI                                    &    16.24    &    33.08  &    14.96  &    12.73  &      d\\
 AlI$_3$        & Aluminum triiodide                     &   -46.20    &     6.30  &    57.74  &    16.48  &      d\\
 Al$_2$         & Al2                                    &   116.40    &   -35.71  &    14.94  &     2.30  &      d\\
 Al$_2$O        & Al2O                                   &   -34.70    &     6.12  &   -36.97  &    -4.46  &      d\\
 Al$_2$O        & Al2O  (+)                              &   155.90    &    12.54  &   -46.84  &    -7.56  &      d\\
 Al$_2$O$_2$       & Al2O2                                  &   -94.30    &     6.74  &   -13.49  &    18.90  &      d\\
\hline
\end{tabular}
\end{center}
\end{table}
\clearpage

\begin{table}
\caption{Comparison of Calculated and Observed $\Delta H_f$ 
for MNDO, AM1, and PM3 (contd.)}
\begin{center}
\compresstable
\begin{tabular}{llrrrrr}
Empirical & Chemical Name & $\Delta H_f$ & \multicolumn{3}{c}{Difference} & \\
Formula   &               & Exp. & PM3 &  MNDO  &  AM1 &     Ref.\\
\hline
 Al$_2$O$_2$       & Al2O2  (+)                             &   135.60    &   -63.10  &   -56.69  &    -7.92  &      d\\
 Al$_2$F$_6$       & Al2F6                                  &  -629.45    &    -1.91  &    -2.12  &   -16.53  &      d\\
 Al$_2$Cl$_6$      & Al2Cl6                                 &  -309.70    &    -1.54  &    14.39  &    -9.26  &      d\\
 Al$_2$Br$_6$      & Al2Br6                                 &  -224.00    &    -0.92  &    91.33  &   -24.97  &      d\\
 Al$_2$I$_6$       & Al2I6                                  &  -117.00    &    -0.36  &   124.41  &     9.01  &      d\\
 Ga          & Gallium, atom                          &    65.40    &     0.00  &  &  &      a\\
 Ga          & Gallium (--)                            &    57.00    &   -14.40  &  &  &      d\\
 Ga          & Gallium (+)                            &   205.90    &    -0.40  &  &  &     aa\\
 Ga          & Gallium (++)                           &   680.50    &    18.43  &  &  &     aa\\
 Ga          & Gallium (+++)                          &  1390.20    &    -2.78  &  &  &     aa\\
 GaH         & Gallium hydride                        &    52.70    &    14.84  &  &  &     aa\\
 C$_3$GaH$_9$      & Trimethyl gallium                      &   -10.80    &    34.62  &  &  &     aa\\
 C$_6$GaH$_1$$_5$     & Triethylgallium                        &   -14.70    &   -19.27  &  &  &    ppp\\
 C$_1$$_2$GaH$_2$$_7$    & Tributylgallium                        &   -53.00    &   -20.62  &  &  &    ppp\\
 GaO         & Gallium oxide                          &    66.80    &   -31.47  &  &  &     aa\\
 GaHO        & Gallium hydroxide                      &   -27.40    &    -4.60  &  &  &     aa\\
 GaF         & Gallium fluoride                       &   -60.21    &    14.04  &  &  &     aa\\
 GaCl        & Gallium chloride                       &   -19.10    &    -2.09  &  &  &     aa\\
 GaCl$_3$       & Gallium trichloride                    &  -107.00    &    27.35  &  &  &     aa\\
 GaH$_3$NCl$_3$    & GaCl3-NH3                              &  -150.10    &    21.96  &  &  &     aa\\
 GaBr        & Gallium bromide                        &   -11.70    &   -19.86  &  &  &     aa\\
 GaI         & Gallium iodide                         &     6.91    &   -19.76  &  &  &     aa\\
 Ga$_2$         & Gallium, dimer                         &   104.80    &   -22.24  &  &  &     aa\\
 Ga$_2$S        & Ga2S                                   &     5.00    &    35.20  &  &  &    jjj\\
 Ga$_2$Cl$_6$      & Ga2Cl6                                 &  -233.10    &   -18.82  &  &  &     aa\\
 Ga$_2$I$_6$       & Ga2Br6                                 &   -76.00    &    -0.02  &  &  &     aa\\
 In          & Indium, atom                           &    58.00    &     0.00  &  &  &      a\\
 In          & Indium (+)                             &   187.30    &    13.16  &  &  &     aa\\
 In          & Indium (++)                            &   624.00    &    28.94  &  &  &     aa\\
 In          & Indium (+++)                           &  1272.00    &   -15.43  &  &  &     aa\\
 InH         & Indium hydride                         &    51.50    &     0.92  &  &  &     cc\\
 C$_3$InH$_9$      & Trimethyl indium                       &    40.80    &   -42.95  &  &  &      f\\
 InO         & Indium oxide                           &    92.50    &   -39.53  &  &  &     aa\\
 InHO        & Indium hydroxide                       &   -18.90    &    -4.80  &  &  &     aa\\
 InS         & Indium sulfide                         &    56.90    &    -2.66  &  &  &     aa\\
 InF         & Indium fluoride                        &   -48.60    &    13.74  &  &  &     cc\\
 InCl        & Indium chloride                        &   -18.00    &     4.68  &  &  &     cc\\
 InCl        & Indium chloride (+)                    &   202.90    &    -0.45  &  &  &     aa\\
 InCl$_3$       & Indium trichloride                     &   -89.40    &    16.61  &  &  &     aa\\
 InBr        & Indium bromide                         &   -13.60    &    -3.91  &  &  &     cc\\
 InBr$_3$       & Indium tribromide                      &   -67.40    &     7.49  &  &  &     aa\\
 InI         & Indium iodide                          &     1.80    &    -9.00  &  &  &     cc\\
 InI$_3$        & Indium triiodide                       &   -28.80    &    10.57  &  &  &     aa\\
 In$_2$         & Indium, dimer                          &    91.00    &    -0.75  &  &  &     aa\\
 In$_2$S        & Indium sulfide                         &    15.10    &     5.01  &  &  &     aa\\
 In$_2$Cl$_3$      & In2Cl3                                 &  -103.60    &   -20.38  &  &  &     aa\\
\hline
\end{tabular}
\end{center}
\end{table}
\clearpage

\begin{table}
\caption{Comparison of Calculated and Observed $\Delta H_f$ 
for MNDO, AM1, and PM3 (contd.)}
\begin{center}
\compresstable
\begin{tabular}{llrrrrr}
Empirical & Chemical Name & $\Delta H_f$ & \multicolumn{3}{c}{Difference} & \\
Formula   &               & Exp. & PM3 &  MNDO  &  AM1 &     Ref.\\
\hline
 Tl          & Thallium, atom                         &    43.55    &     0.00  &  &  &      a\\
 Tl          & Thallium (+)                           &   185.90    &    19.15  &  &  &     aa\\
 Tl          & Thallium (++)                          &   658.40    &    -1.53  &  &  &     aa\\
 Tl          & Thallium (+++)                         &  1347.80    &     2.11  &  &  &     aa\\
 TlCl        & Thallium chloride                      &   -16.20    &     2.77  &  &  &     aa\\
 TlCl        & Thallium chloride (+)                  &   209.00    &     2.60  &  &  &     aa\\
 TlBr        & Thallium bromide                       &    -9.00    &   -10.14  &  &  &     aa\\
 TlBr        & Thallium bromide (+)                   &   203.30    &    11.48  &  &  &     aa\\
 TlI         & Thallium iodide                        &     1.70    &   -14.37  &  &  &     aa\\
 Tl$_2$Cl$_2$      & Thallium chloride dimer                &   -49.40    &     0.76  &  &  &     aa\\
 Si          & Silicon, atom                          &   108.39    &     0.00  &     0.00  &     0.00  &      a\\
 HSi         & SiH                                    &    86.30    &     8.27  &     3.92  &     3.51  &     aa\\
 H$_2$Si        & Silylene (singlet)                     &    61.10    &    11.69  &     3.22  &     6.73  &    ttt\\
 H$_2$Si        & Silylene (triplet)                     &     6.50    &    -9.43  &    -2.74  &   -30.68  &    uuu\\
 H$_3$Si        & Silyl (--)                              &    14.00    &   -16.80  &    32.45  &   -15.82  &    vvv\\
 H$_3$Si        & Silyl (+)                              &   234.10    &   -10.81  &   -43.35  &   -11.21  &    vvv\\
 H$_3$Si        & Silyl                                  &    46.40    &    -3.47  &    -9.55  &   -20.08  &    www\\
 H$_4$Si        & Silane                                 &     8.20    &     4.27  &    -7.02  &    -4.08  &      d\\
 CH$_5$Si       & Methylsilyl                            &    30.50    &    -7.27  &   -11.56  &   -21.52  &    www\\
 CH$_6$Si       & Methylsilane                           &    -7.80    &     4.16  &    -5.86  &    -3.00  &      n\\
 C$_2$H$_6$Si      & Vinylsilane                            &    -1.90    &    21.83  &     8.19  &    13.51  &     aa\\
 C$_2$H$_7$Si      & Dimethylsilyl                          &    14.30    &   -10.42  &   -13.62  &   -21.81  &    www\\
 C$_2$H$_8$Si      & Ethylsilane                            &   -15.00    &     4.79  &    -6.71  &    -1.52  &    xxx\\
 C$_2$H$_8$Si      & Dimethylsilane                         &   -20.00    &    -0.79  &    -9.28  &    -5.68  &      n\\
 C$_3$H$_9$Si      & Trimethylsilyl                         &    -0.80    &   -14.25  &   -16.93  &   -22.20  &    www\\
 C$_3$H$_1$$_0$Si     & Trimethylsilane                        &   -37.40    &     0.10  &    -6.99  &    -1.98  &      n\\
 C$_4$H$_1$$_2$Si     & Diethylsilane                          &   -43.60    &    10.94  &    -0.52  &     7.19  &     oo\\
 C$_4$H$_1$$_2$Si     & Tetramethylsilane                      &   -55.70    &     1.95  &    -3.48  &     3.54  &    yyy\\
 C$_5$H$_1$$_2$Si     & 1,1-Dimethylsilacyclobutane            &   -33.70    &    -1.95  &   -14.75  &    -3.54  &      n\\
 C$_6$H$_1$$_6$Si     & Triethylsilane                         &   -39.50    &   -15.98  &   -24.63  &   -15.29  &     oo\\
 C$_8$H$_2$$_0$Si     & Tetraethylsilane                       &   -64.40    &   -13.60  &   -17.73  &    -7.53  &     oo\\
 SiO         & Silicon monoxide                       &   -23.90    &    -2.13  &     1.26  &    21.82  &      g\\
 C$_3$H$_1$$_0$SiO    & Trimethylsilicon hydroxide             &  -119.40    &     3.90  &    -2.55  &     9.06  &      f\\
 SiO$_2$        & Silicon dioxide                        &   -73.00    &   -15.93  &    50.06  &     5.51  &      d\\
 SiF         & Silicon fluoride                       &     1.70    &   -22.57  &   -30.47  &   -27.81  &     aa\\
 H$_3$SiF       & Fluorosilane                           &   -84.50    &     7.09  &   -11.95  &    -1.31  &    zzz\\
 C$_4$H$_1$$_2$SiF    & SiMe4F (--) C3v symmetry                &  -147.50    &    25.02  &    33.75  &    24.07  &    aaa\\
 SiOF        & SiOF                                   &  -136.10    &    28.55  &    56.09  &    30.12  &      g\\
 SiF$_2$        & Silicon difluoride                     &  -141.20    &   -13.69  &   -23.70  &   -13.44  &   aaaa\\
 H$_2$SiF$_2$      & Difluorosilane                         &  -189.00    &    13.82  &    -3.55  &     7.52  &   aaaa\\
 SiOF$_2$       & SiOF2                                  &  -231.00    &     1.63  &    42.33  &    10.26  &      d\\
 SiF$_3$        & Trifluorosilyl                         &  -245.00    &   -15.33  &    -3.88  &   -22.78  &    www\\
 HSiF$_3$       & Trifluorosilane                        &  -287.00    &     6.82  &     1.91  &     6.09  &      d\\
 SiF$_4$        & Silicon tetrafluoride                  &  -386.00    &    -4.58  &    15.55  &     3.99  &   aaaa\\
 SiF$_5$        & SiF5 (--)                               &  -507.10    &     2.73  &    17.63  &     3.59  &    aaa\\
 SiCl        & Silicon chloride                       &    45.30    &   -15.54  &   -15.64  &   -18.03  &     aa\\
\hline
\end{tabular}
\end{center}
\end{table}
\clearpage

\begin{table}
\caption{Comparison of Calculated and Observed $\Delta H_f$ 
for MNDO, AM1, and PM3 (contd.)}
\begin{center}
\compresstable
\begin{tabular}{llrrrrr}
Empirical & Chemical Name & $\Delta H_f$ & \multicolumn{3}{c}{Difference} & \\
Formula   &               & Exp. & PM3 &  MNDO  &  AM1 &     Ref.\\
\hline
 H$_3$SiCl      & Chlorosilane                           &   -32.40    &     4.77  &   -11.55  &    -5.29  &   aaaa\\
 C$_2$H$_7$SiCl    & Chlorodimethylsilane                   &   -69.90    &     6.14  &    -3.17  &    -0.11  &     oo\\
 C$_3$H$_9$SiCl    & Chlorotrimethylsilane                  &   -84.60    &     2.98  &    -2.84  &    -0.25  &      f\\
 SiOCl       & SiOCl                                  &   -86.70    &    27.67  &    44.15  &    28.57  &      g\\
 SiF$_4$Cl      & SiF4Cl (--)                             &  -465.30    &     0.08  &    17.46  &     7.96  &    aaa\\
 SiCl$_2$       & Silicon dichloride                     &   -40.60    &    -8.85  &    -5.71  &    -6.10  &   aaaa\\
 H$_2$SiCl$_2$     & Dichlorosilane                         &   -75.30    &     5.79  &    -8.18  &    -5.44  &   aaaa\\
 CH$_4$SiCl$_2$    & Dichloromethylsilane                   &   -96.00    &     7.08  &    -1.45  &    -2.49  &      f\\
 C$_2$H$_6$SiCl$_2$   & Dichlorodimethylsilane                 &  -109.50    &     1.48  &    -1.76  &    -5.22  &      n\\
 SiOCl$_2$      & SiOCl2                                 &  -167.70    &    46.47  &    73.57  &    49.35  &      g\\
 SiCl$_3$       & Trichlorosilyl                         &   -76.00    &   -18.86  &   -13.50  &   -39.98  &    www\\
 HSiCl$_3$      & Trichlorosilane                        &  -119.30    &     6.58  &     1.42  &    -5.87  &   aaaa\\
 CH$_3$SiCl$_3$    & Trichloromethylsilane                  &  -131.20    &    -1.68  &     0.56  &   -12.49  &      n\\
 SiCl$_4$       & Silicon tetrachloride                  &  -158.40    &     1.97  &    10.76  &   -12.52  &      d\\
 SiCl$_5$       & SiCl5 (--)                              &  -237.20    &   -17.47  &   -19.33  &   -30.42  &    aaa\\
 SiBr        & Silicon bromide                        &    50.00    &    -9.03  &     7.81  &    -3.02  &     aa\\
 H$_3$SiBr      & Bromosilane                            &   -15.30    &    -0.73  &    -2.75  &    -5.83  &   bbbb\\
 C$_3$H$_9$SiBr    & Trimethylbromosilane                   &   -70.00    &     1.22  &     7.60  &     3.51  &      f\\
 SiOBr       & SiOBr                                  &   -71.40    &    24.24  &    42.77  &    31.06  &      g\\
 SiBr$_2$       & Silicon dibromide                      &    -9.60    &   -17.84  &    20.81  &     3.63  &   aaaa\\
 H$_2$SiBr$_2$     & Dibromosilane                          &   -43.20    &    -3.98  &    11.13  &    -2.62  &   aaaa\\
 SiOBr$_2$      & SiOBr2                                 &  -137.40    &    43.31  &    85.94  &    60.01  &      g\\
 SiBr$_3$       & Silicon tribromide                     &   -56.10    &    -4.67  &    41.12  &    -9.60  &      g\\
 HSiBr$_3$      & Tribromosilane                         &   -72.50    &    -7.06  &    30.08  &     2.27  &   bbbb\\
 SiBr$_4$       & Silicon tetrabromide                   &   -99.30    &    -8.65  &    48.91  &     4.81  &     aa\\
 SiI         & Silicon iodide                         &    76.40    &    -5.44  &    15.53  &     4.11  &      g\\
 H$_3$SiI       & Iodosilane                             &    -0.50    &     1.34  &    11.52  &     5.40  &   aaaa\\
 SiOI        & SiOI                                   &   -53.30    &   -19.59  &    56.17  &    76.73  &      g\\
 SiI$_2$        & Silicon diiodide                       &    22.00    &    -3.98  &    60.11  &    35.53  &   aaaa\\
 H$_2$SiI$_2$      & Diiodosilane                           &    -9.10    &     5.55  &    32.05  &    14.06  &      d\\
 SiOI$_2$       & SiOI2                                  &   -99.40    &    49.17  &    89.66  &    80.71  &      g\\
 SiI$_3$        & Silicon triiodide                      &     0.50    &     6.58  &    44.85  &    11.84  &      g\\
 HSiI$_3$       & Triiodosilane                          &   -17.80    &     8.78  &    51.86  &    21.69  &   aaaa\\
 SiI$_4$        & Silicon tetraiodide                    &   -26.40    &    12.14  &    68.74  &    27.99  &      d\\
 Si$_2$         & Silicon dimer                          &   140.90    &    -5.17  &    78.91  &    -1.40  &      g\\
 Si$_2$         & Silicon dimer                          &   140.90    &    -5.17  &    78.91  &    -1.40  &      g\\
 H$_6$Si$_2$       & Disilane                               &    17.10    &     0.75  &     5.28  &    -0.96  &   cccc\\
 C$_6$H$_1$$_8$Si$_2$    & Hexamethyldisilane                     &   -85.80    &     2.58  &    12.00  &    19.58  &      n\\
 C$_6$H$_1$$_8$Si$_2$O   & Hexamethyldisiloxane                   &  -185.80    &     1.36  &    -8.80  &    17.82  &      f\\
 C$_6$H$_1$$_9$Si$_2$N   & Hexamethyldisilazane                   &  -113.90    &    -7.02  &    -9.80  &    -4.17  &      f\\
 Si$_2$Cl$_6$      & Hexachlorodisilane                     &  -243.50    &    13.75  &    30.78  &     1.88  &   cccc\\
 Si$_2$Br$_6$      & Hexabromodisilane                      &  -182.80    &    18.10  &   116.84  &    38.91  &      g\\
 Si$_3$         & Silicon trimer                         &   152.20    &     0.56  &    14.94  &    32.68  &      g\\
 H$_8$Si$_3$       & Si3H8                                  &    28.90    &    -6.58  &     3.27  &    -3.99  &     aa\\
 Ge          & Germanium, atom                        &    89.50    &   -27.71  &     0.00  &     0.00  &      a\\
 GeH$_4$        & Germane                                &    21.70    &    10.36  &    -8.25  &     7.32  &     aa\\
\hline
\end{tabular}
\end{center}
\end{table}
\clearpage

\begin{table}
\caption{Comparison of Calculated and Observed $\Delta H_f$ 
for MNDO, AM1, and PM3 (contd.)}
\begin{center}
\compresstable
\begin{tabular}{llrrrrr}
Empirical & Chemical Name & $\Delta H_f$ & \multicolumn{3}{c}{Difference} & \\
Formula   &               & Exp. & PM3 &  MNDO  &  AM1 &     Ref.\\
\hline
 CGe         & Germanium carbide                      &   151.00    &    27.61  &    73.43  &    52.55  &     aa\\
 C$_3$GeH$_9$      & Trimethylgermanium                     &     2.20    &   -14.31  &    -3.96  &    -3.52  &   dddd\\
 C$_3$GeH$_9$      & Trimethylgermanium (+)                 &   165.00    &    11.41  &    -2.37  &    15.44  &   dddd\\
 C$_4$GeH$_1$$_2$     & Tetramethylgermanium                   &   -32.00    &    -2.33  &    -0.82  &     5.77  &   dddd\\
 C$_7$GeH$_1$$_8$     & t-Butyltrimethylgermane                &   -55.70    &     4.35  &    16.59  &     9.66  &   dddd\\
 C$_8$GeH$_2$$_0$     & Tetraethylgermanium                    &   -38.60    &   -17.40  &   -18.72  &   -21.93  &    ppp\\
 C$_1$$_2$GeH$_2$$_8$    & Tetrapropylgermanium                   &   -54.20    &   -24.25  &   -21.40  &   -33.37  &    ppp\\
 C$_2$$_4$GeH$_2$$_0$    & Tetraphenylgermanium                   &   106.50    &     2.67  &   -21.55  &   -15.34  &    ppp\\
 GeO         & Germanium oxide                        &   -11.04    &     5.12  &    11.85  &    20.24  &     aa\\
 C$_5$GeH$_1$$_4$O    & Ethoxytrimethylgermane                 &   -87.80    &    -3.55  &    -0.60  &    -2.28  &   dddd\\
 C$_4$GeH$_1$$_2$O$_4$   & Tetramethoxygermanium                  &  -230.40    &   -15.58  &    -3.23  &   -22.38  &     oo\\
 C$_5$GeH$_1$$_5$N    & Dimethylaminetrimethylgermane          &   -29.10    &   -12.42  &    -3.99  &     1.13  &   dddd\\
 GeF         & Germanium fluoride                     &    -8.00    &     4.66  &    -8.40  &   -11.74  &     aa\\
 GeF$_2$        & Germanium difluoride                   &  -121.00    &    13.42  &    -2.85  &     0.44  &     aa\\
 GeF$_4$        & Germanium tetrafluoride                &  -284.40    &    -5.85  &    10.14  &    21.04  &   eeee\\
 GeCl        & Germanium chloride                     &    37.10    &    -9.42  &   -19.66  &   -17.46  &     aa\\
 C$_3$GeH$_9$Cl    & Trimethylchlorogermane                 &   -63.80    &     3.65  &     1.70  &     5.11  &   dddd\\
 GeCl$_2$       & Germanium dichloride                   &   -42.00    &    10.20  &     7.45  &    -1.99  &     aa\\
 GeCl$_4$       & Germanium tetrachloride                &  -118.50    &    -9.18  &    -7.27  &   -20.20  &     aa\\
 GeCl$_4$       & Germanium tetrachloride (+)            &   156.80    &   -23.99  &    34.57  &     4.27  &     aa\\
 GeBr        & Germanium bromide                      &    56.30    &   -14.02  &   -13.38  &   -13.06  &     aa\\
 C$_3$GeH$_9$Br    & Trimethylbromogermane                  &   -53.10    &     5.27  &    14.82  &    12.03  &   dddd\\
 GeBr$_2$       & Germanium dibromide                    &   -15.00    &    10.92  &    16.33  &     5.50  &     aa\\
 GeBr$_4$       & Germanium tetrabromide                 &   -71.70    &     5.02  &    37.70  &     2.23  &     aa\\
 GeI$_2$        & Germanium diiodide                     &    11.20    &    16.81  &    49.16  &    21.63  &     aa\\
 GeI$_4$        & Germanium tetraiodide                  &   -13.60    &     6.88  &    70.71  &    17.24  &     aa\\
 Ge$_2$         & Germanium, dimer                       &   113.10    &   -17.81  &    77.58  &   -30.34  &     aa\\
 Ge$_2$H$_6$       & Digermane                              &    38.80    &     6.78  &     8.11  &     0.32  &     aa\\
 C$_6$Ge$_2$H$_1$$_8$    & GeMe3-GeMe3                            &   -62.50    &    10.43  &    38.49  &    21.23  &   ffff\\
 C$_6$Ge$_2$H$_1$$_8$O   & Bis(trimethylgermanium) oxide          &  -127.00    &     4.42  &     8.80  &    15.64  &   dddd\\
 Ge$_3$H$_8$       & Trigermane                             &    54.20    &     4.20  &    21.02  &   -12.48  &     aa\\
 Sn          & Tin, atom                              &    72.20    &     0.00  &     0.00  &  &      a\\
 SnH         & Tin hydride                            &    49.00    &    14.25  &     0.71  &  &   gggg\\
 SnH$_4$        & Tin tetrahydride (stannane)            &    38.90    &    -3.42  &    -8.40  &  &     aa\\
 C$_2$SnH$_8$      & Dimethylstannane                       &    21.00    &    -8.67  &   -15.24  &  &     aa\\
 C$_3$SnH$_1$$_0$     & Trimethyltin hydride                   &     5.00    &    -4.90  &   -10.84  &  &     aa\\
 C$_4$SnH$_1$$_2$     & Diethyltin dihydride                   &    11.00    &    -6.23  &   -13.82  &  &     aa\\
 C$_4$SnH$_1$$_2$     & Tetramethyltin                         &    -4.50    &    -8.35  &   -12.38  &  &     aa\\
 C$_5$SnH$_1$$_4$     & Ethyltrimethyltin                      &    -7.00    &    -9.49  &   -13.82  &  &    ppp\\
 C$_6$SnH$_1$$_6$     & Propyltrimethyltin                     &   -11.20    &   -10.42  &   -14.13  &  &    ppp\\
 C$_7$SnH$_1$$_8$     & t-Butyltrimethyltin                    &   -16.00    &    -8.22  &     0.02  &  &    ppp\\
 C$_8$SnH$_2$$_0$     & Tetraethyltin                          &   -10.90    &   -16.43  &   -21.12  &  &     aa\\
 C$_9$SnH$_1$$_4$     & Phenyltrimethylstannane                &    26.70    &    -3.33  &    -6.80  &  &    ppp\\
 C$_2$$_4$SnH$_2$$_0$    & Tetraphenyltin                         &   114.39    &    14.36  &    17.61  &  &      f\\
 SnO         & Tin oxide                              &     3.61    &     0.92  &     9.61  &  &     aa\\
 C$_5$SnH$_1$$_5$N    & Dimethylaminetrimethyltin              &    -4.30    &    -4.90  &     7.03  &  &   dddd\\
\hline
\end{tabular}
\end{center}
\end{table}
\clearpage

\begin{table}
\caption{Comparison of Calculated and Observed $\Delta H_f$ 
for MNDO, AM1, and PM3 (contd.)}
\begin{center}
\compresstable
\begin{tabular}{llrrrrr}
Empirical & Chemical Name & $\Delta H_f$ & \multicolumn{3}{c}{Difference} & \\
Formula   &               & Exp. & PM3 &  MNDO  &  AM1 &     Ref.\\
\hline
 SnS         & Tin sulfide                            &    28.50    &     9.23  &    -1.42  &  &     aa\\
 SnF         & Tin fluoride                           &    -9.00    &    -8.48  &   -11.43  &  &   gggg\\
 SnF$_2$        & Tin difluoride                         &  -116.00    &     4.56  &     1.36  &  &   gggg\\
 SnCl        & Tin chloride                           &    16.00    &    -8.43  &   -21.55  &  &   gggg\\
 C$_3$SnH$_9$Cl    & Trimethyltin chloride                  &   -46.40    &     6.45  &     2.25  &  &   dddd\\
 C$_6$SnH$_1$$_5$Cl   & Triethyltin chloride                   &   -46.20    &    -4.25  &   -10.28  &  &    ppp\\
 SnCl$_2$       & Tin dichloride                         &   -56.40    &     0.88  &   -25.75  &  &   gggg\\
 C$_2$SnH$_6$Cl$_2$   & Dimethyltin dichloride                 &   -71.00    &    19.30  &     4.78  &  &   hhhh\\
 CSnH$_3$Cl$_3$    & Methyltin trichloride                  &   -99.80    &    16.16  &    11.60  &  &    ppp\\
 C$_2$SnH$_5$Cl$_3$   & Ethyltin trichloride                   &  -102.00    &    14.35  &     8.64  &  &    ppp\\
 SnCl$_4$       & Tin tetrachloride                      &  -112.70    &    11.47  &    12.67  &  &     aa\\
 SnBr        & Tin bromide                            &    24.00    &    -9.45  &    -7.85  &  &   gggg\\
 C$_3$SnH$_9$Br    & Trimethyltin bromide                   &   -33.60    &    -1.42  &     8.60  &  &   iiii\\
 C$_1$$_2$SnH$_2$$_7$B   & Tri-n-butyltin bromide                 &   -65.70    &   -11.12  &     0.69  &  &      f\\
 SnBr$_2$       & Tin dibromide                          &   -29.00    &   -16.75  &    -9.20  &  &   gggg\\
 SnBr$_4$       & Tin tetrabromide                       &   -75.20    &    14.93  &    44.67  &  &     aa\\
 SnI         & Tin iodide                             &    36.00    &    -6.57  &     0.43  &  &   gggg\\
 C$_3$SnH$_9$I     & Trimethyltin iodide                    &   -19.70    &    -0.83  &    10.86  &  &   dddd\\
 SnI$_2$        & Tin diiodide                           &     2.00    &   -17.23  &     3.40  &  &   gggg\\
 C$_6$GeSnH$_1$$_8$   & GeMe3-SnMe3                            &   -39.70    &     7.48  &    10.05  &  &   dddd\\
 Pb          & Lead, atom                             &    46.62    &     0.00  &     0.00  &  &      a\\
 Pb          & Lead (+)                               &   219.00    &    -1.09  &    -0.64  &  &     aa\\
 PbH         & Lead hydride                           &    56.50    &    -2.32  &   -15.77  &  &    ddd\\
 PbH$_4$        & Lead tetrahydride (plumbane)           &    59.70    &    -0.89  &     3.39  &  &   jjjj\\
 C$_3$PbH$_9$      & Trimethyllead                          &    46.66    &    15.35  &   -10.64  &  &   ffff\\
 C$_3$PbH$_9$      & Trimethyllead (+)                      &   200.07    &     9.83  &    11.79  &  &   ffff\\
 C$_4$PbH$_1$$_2$     & Tetramethyllead                        &    32.50    &    -2.60  &    -6.45  &  &     aa\\
 C$_7$PbH$_1$$_8$     & t-Butyltrimethyllead                   &     6.92    &     7.04  &    19.50  &  &   ffff\\
 C$_8$PbH$_2$$_0$     & Tetraethyllead                         &    26.19    &   -17.64  &   -15.64  &  &     aa\\
 PbO         & Lead oxide                             &    16.80    &     4.51  &    12.31  &  &      x\\
 PbS         & Lead sulfide                           &    31.50    &     8.40  &     0.46  &  &      d\\
 PbF         & Lead fluoride                          &   -19.18    &    -1.78  &    -3.40  &  &      d\\
 PbF$_2$        & Lead difluoride                        &  -104.00    &    14.35  &     9.83  &  &      d\\
 PbCl        & Lead chloride                          &     3.60    &    -1.95  &   -18.89  &  &      d\\
 PbCl        & Lead chloride (+)                      &   178.20    &    -3.79  &   -18.44  &  &      d\\
 PbCl$_2$       & Lead dichloride                        &   -41.60    &     1.86  &   -35.40  &  &      d\\
 PbCl$_2$       & Lead dichloride (+)                    &   195.10    &   -15.62  &    -6.18  &  &      d\\
 PbCl$_4$       & Lead tetrachloride (est.)              &   -69.20    &     7.32  &    19.19  &  &     aa\\
 PbBr        & Lead bromide                           &    16.95    &   -10.00  &   -10.94  &  &      d\\
 PbBr$_2$       & Lead dibromide                         &   -24.95    &    -5.75  &    -8.91  &  &      d\\
 PbI         & Lead iodide                            &    25.75    &   -12.30  &     0.35  &  &      d\\
 PbI$_2$        & Lead diiodide                          &    -0.76    &   -16.45  &     9.49  &  &      d\\
 Pb$_2$         & Lead, dimer                            &    79.50    &   -34.17  &    -6.95  &  &      d\\
 C$_6$Pb$_2$H$_1$$_8$    & Hexamethyldiplumbane                   &    38.71    &    10.90  &     6.67  &  &   ffff\\
 P           & Phosphorus (++)                        &   775.10    &  -142.10  &   -62.55  &   -63.01  &     aa\\
 HP          & Phosphinidene                          &    60.60    &    12.83  &    27.87  &    15.04  &      d\\
\hline
\end{tabular}
\end{center}
\end{table}
\clearpage

\begin{table}
\caption{Comparison of Calculated and Observed $\Delta H_f$ 
for MNDO, AM1, and PM3 (contd.)}
\begin{center}
\compresstable
\begin{tabular}{llrrrrr}
Empirical & Chemical Name & $\Delta H_f$ & \multicolumn{3}{c}{Difference} & \\
Formula   &               & Exp. & PM3 &  MNDO  &  AM1 &     Ref.\\
\hline
 H$_2$P         & Phosphino                              &    30.10    &    -0.80  &     1.32  &     4.07  &      d\\
 H$_3$P         & Phosphine                              &     1.30    &    -1.08  &     2.63  &     8.88  &      d\\
 CP          & Carbon phosphide                       &   107.53    &    12.09  &    22.12  &    23.58  &      d\\
 CHP         & Methinophosphine                       &    35.80    &    10.64  &     6.40  &    11.57  &      d\\
 CH$_5$P        & Methylphosphine                        &    -7.00    &    -2.48  &    -7.73  &     5.80  &   kkkk\\
 C$_2$H$_7$P       & Ethylphosphine                         &   -12.00    &     0.30  &    -9.12  &     8.10  &   kkkk\\
 C$_2$H$_7$P       & Dimethylphosphine                      &   -15.00    &    -4.38  &   -16.55  &     3.33  &   kkkk\\
 C$_3$H$_9$P       & Trimethylphosphine                     &   -22.50    &    -7.36  &   -25.78  &     0.46  &      f\\
 C$_4$H$_1$$_1$P      & Diethylphosphine                       &   -25.00    &     1.64  &   -20.08  &     6.62  &   kkkk\\
 C$_6$H$_1$$_5$P      & Triethylphosphine                      &   -11.80    &   -24.95  &   -53.06  &   -18.30  &      f\\
 PO          & Phosphorus oxide                       &    -2.90    &   -16.55  &   -18.16  &   -13.58  &      g\\
 C$_3$H$_9$PO      & Trimethylphosphine oxide               &  -102.20    &    19.41  &    59.25  &     0.59  &      f\\
 PO$_2$         & Phosphorus dioxide                     &   -71.00    &    -5.71  &    23.93  &    13.18  &      g\\
 PO$_3$         & PO3(--)                                 &  -190.00    &    -7.23  &    49.56  &   -13.09  &   llll\\
 HPO$_3$        & HPO3                                   &  -135.00    &   -42.27  &    29.84  &   -33.62  &   llll\\
 CH$_5$PO$_3$      & Methylphosphonic acid                  &  -240.50    &    31.46  &    88.41  &    13.74  &      f\\
 C$_2$H$_7$PO$_3$     & Ethylphosphonic acid                   &  -239.40    &    29.75  &    80.23  &    10.55  &      f\\
 C$_3$H$_9$PO$_3$     & Trimethyl phosphite                    &  -168.30    &   -14.58  &   -34.63  &   -15.01  &      f\\
 C$_6$H$_1$$_5$PO$_3$    & Triethyl phosphite                     &  -195.90    &   -12.72  &   -25.58  &    -9.19  &      f\\
 C$_6$H$_1$$_5$PO$_4$    & Triethyl phosphate                     &  -284.50    &    32.94  &    74.91  &     8.13  &      f\\
 NP          & Phosphorus nitride                     &    25.04    &     7.86  &     8.87  &     7.46  &      d\\
 PF          & Phosphorus fluoride                    &   -20.80    &     0.37  &    10.88  &     1.10  &      g\\
 POF         & POF                                    &  -111.80    &   -12.17  &    -2.42  &     0.51  &      g\\
 C$_2$H$_6$PO$_2$F    & Methyl methylphosphonofluoridate       &  -197.30    &   -10.32  &    51.33  &   -20.81  &   mmmm\\
 C$_3$H$_8$PO$_2$F    & Ethyl methylphosphonofluoridate        &  -205.80    &    -6.57  &    53.80  &   -18.61  &   mmmm\\
 C$_4$H$_1$$_0$PO$_2$F   & i-Propyl methylphosphonofluoridate     &  -214.60    &     0.71  &    63.03  &    -8.76  &   mmmm\\
 C$_4$H$_1$$_0$PO$_2$F   & n-Propyl methylphosphonofluoridate     &  -210.21    &    -7.26  &    53.80  &   -20.95  &   mmmm\\
 C$_5$H$_1$$_2$PO$_2$F   & i-Propyl ethylphosphonofluoridate      &  -219.75    &     2.13  &    59.27  &   -12.24  &   mmmm\\
 C$_5$H$_1$$_2$PO$_2$F   & s-Butyl methylphosphonofluoridate      &  -220.10    &    -1.15  &    63.08  &   -15.66  &   mmmm\\
 C$_5$H$_1$$_2$PO$_2$F   & n-Butyl methylphosphonofluoridate      &  -215.14    &    -7.67  &    54.02  &   -22.83  &   mmmm\\
 C$_6$H$_1$$_4$PO$_2$F   & Neopentyl methylphosphonofluoridate    &  -224.16    &    -2.51  &    70.28  &   -13.87  &   mmmm\\
 PF$_2$         & Phosphorus difluoride                  &  -119.00    &   -25.44  &   -19.00  &   -14.24  &      g\\
 POF$_2$        & POF2                                   &  -213.60    &    24.67  &    61.67  &    39.37  &      g\\
 CH$_3$POF$_2$     & Methylphosphonodifluoride              &  -233.23    &     7.90  &    86.01  &     5.55  &   mmmm\\
 PF$_3$         & Phosphorus trifluoride                 &  -229.00    &   -23.17  &    -0.32  &     0.01  &      d\\
 POF$_3$        & Phosphorus oxyfluoride                 &  -289.50    &    -8.21  &    89.96  &    -3.31  &     aa\\
 PF$_4$         & Phosphorus tetrafluoride               &  -287.90    &   -16.05  &    53.22  &     6.78  &      g\\
 PF$_4$         & Phosphorus tetrafluoride (--)           &  -325.00    &   -28.60  &    18.07  &    26.20  &    aaa\\
 PF$_5$         & Phosphorus pentafluoride               &  -381.10    &    -5.83  &   132.28  &     1.70  &      d\\
 PF$_6$         & Phosphorus hexafluoride (--)            &  -522.00    &    13.46  &   152.31  &    17.53  &    aaa\\
 PCl         & Phosphorus chloride                    &    25.60    &     3.32  &    10.79  &    -4.29  &      g\\
 POCl        & POCL                                   &   -64.70    &   -11.71  &   -10.11  &    -0.41  &      g\\
 PCl$_2$        & Phosphorus dichloride                  &   -21.30    &   -18.80  &   -28.08  &   -26.29  &      g\\
 POCl$_2$       & POCl2                                  &  -109.90    &    15.05  &    33.91  &    24.79  &      g\\
 CH$_3$POCl$_2$    & Methylphosphonodichloride              &  -124.10    &    -4.79  &    48.66  &     9.27  &   mmmm\\
 PCl$_3$        & Phosphorus trichloride                 &   -69.00    &   -19.48  &   -27.45  &   -20.06  &      d\\
\hline
\end{tabular}
\end{center}
\end{table}
\clearpage

\begin{table}
\caption{Comparison of Calculated and Observed $\Delta H_f$ 
for MNDO, AM1, and PM3 (contd.)}
\begin{center}
\compresstable
\begin{tabular}{llrrrrr}
Empirical & Chemical Name & $\Delta H_f$ & \multicolumn{3}{c}{Difference} & \\
Formula   &               & Exp. & PM3 &  MNDO  &  AM1 &     Ref.\\
\hline
 POCl$_3$       & Phosphorus oxychloride                 &  -132.80    &    -7.40  &    53.17  &    14.82  &     aa\\
 PSCl$_3$       & Phosphorus thiochloride                &   -91.00    &    30.77  &    62.61  &    58.69  &      d\\
 PCl$_4$        & Phosphorus tetrachloride               &   -80.50    &   -22.33  &   -22.75  &   -29.57  &      g\\
 PCl$_5$        & Phosphorus pentachloride               &   -86.10    &   -25.51  &    44.13  &    14.27  &      d\\
 PBr         & Phosphorus bromide                     &    43.00    &    -8.14  &   -13.58  &   -15.41  &    ddd\\
 POBr        & POBr                                   &   -50.20    &    -9.70  &    -3.70  &    10.62  &      g\\
 PBr$_2$        & Phosphorus dibromide                   &     6.70    &    -6.27  &   -15.98  &   -11.11  &      g\\
 POBr$_2$       & POBr2                                  &   -78.30    &    14.94  &    34.09  &    38.98  &      g\\
 PBr$_3$        & Phosphorus tribromide                  &   -34.90    &     6.73  &    -3.23  &    11.53  &      d\\
 POBr$_3$       & Phosphorus oxybromide                  &   -97.00    &    16.75  &    68.32  &    67.95  &      d\\
 PSBr$_3$       & Phosphorus thiobromide                 &   -67.20    &    59.36  &    85.63  &   111.94  &      d\\
 PBr$_4$        & Phosphorus tetrabromide                &   -17.40    &   -16.14  &   -21.64  &    -8.52  &      g\\
 PBr$_5$        & Phosphorus pentabromide                &   -11.00    &   -16.11  &    42.74  &    63.99  &      g\\
 PI          & Phosphorus iodide                      &    54.60    &    -3.58  &    -0.85  &    -3.15  &      g\\
 POI         & POI                                    &   -33.40    &   -10.23  &     8.79  &    17.69  &      g\\
 PI$_2$         & Phosphorus diiodide                    &    41.30    &    -4.74  &    -4.79  &    -6.36  &      g\\
 POI$_2$        & POI2                                   &   -40.10    &     4.90  &    19.49  &    33.02  &      g\\
 PI$_3$         & Phosphorus triiodide                   &    25.10    &     8.01  &     1.18  &     0.72  &      g\\
 POI$_3$        & Phosphorus oxyiodide                   &   -39.70    &    31.42  &    71.31  &    75.86  &      g\\
 PI$_4$         & Phosphorus tetraiodide                 &    60.20    &   -15.53  &   -36.12  &   -33.68  &      g\\
 PI$_5$         & Phosphorus pentaiodide                 &    97.70    &    -0.37  &   -16.01  &   -42.18  &      g\\
 C$_6$BH$_1$$_8$P     & Trimethylborane-Trimethylphosphine      &  -68.30    &  &   -20.22  &    -4.15  &     aa\\
 BH$_3$PF$_3$      & Phosphorus trifluoride-Borane          &  -204.10    &  &   -13.16  &   -30.50  &     aa\\
 C$_3$BH$_9$PF$_3$    & Trimethyphosphine-Boron trifluoride    &  -312.60    &  &     2.34  &    30.85  &     aa\\
 P$_2$          & Phosphorus dimer                       &    42.80    &   -10.81  &    -1.74  &   -18.23  &      g\\
 H$_4$P$_2$        & P2H4                                   &     5.00    &    -8.72  &    -7.88  &     1.58  &     aa\\
 P$_4$          & Phosphorus tetramer                    &    31.10    &    11.77  &     5.11  &    19.28  &      g\\
 P$_4$O$_6$        & Phosphorus trioxide                    &  -529.23    &    18.15  &     7.98  &   207.56  &      d\\
 P$_4$O$_1$$_0$       & Phosphorus pentoxide                   &  -694.09    &   -18.66  &   262.21  &    -7.10  &      d\\
 As          & Arsenic, atom                          &    72.30    &     0.00  &  &  &      a\\
 As          & Arsenic (++)                           &   731.30    &   -11.02  &  &  &     aa\\
 AsH$_3$        & Arsine                                 &    15.88    &    -3.21  &  &  &     aa\\
 C$_3$AsH$_9$      & Trimethylarsine                        &     2.80    &   -17.70  &  &  &      f\\
 C$_6$AsH$_1$$_5$     & Triethylarsine                         &    13.40    &   -33.84  &  &  &      f\\
 C$_1$$_8$AsH$_1$$_5$    & Triphenylarsine                        &    97.60    &    10.11  &  &  &    ppp\\
 C$_3$AsH$_9$O$_3$    & Trimethyl arsenite                     &  -131.10    &    -8.24  &  &  &      f\\
 C$_6$AsH$_1$$_5$O$_3$   & Triethyl arsenite                      &  -156.80    &     1.52  &  &  &      f\\
 C$_9$AsH$_2$$_1$O$_3$   & Tri-n-propyl arsenite                  &  -173.70    &     2.53  &  &  &      f\\
 AsN         & Arsenic nitride                        &    46.91    &    23.23  &  &  &     aa\\
 AsS         & Arsenic sulfide                        &    48.50    &    -2.10  &  &  &    jjj\\
 AsF$_3$        & Arsenic trifluoride                    &  -187.80    &    -2.40  &  &  &     aa\\
 AsF$_5$        & Arsenic pentafluoride                  &  -295.60    &     7.77  &  &  &    jjj\\
 AsCl        & Arsenic chloride                       &    10.50    &     5.18  &  &  &      g\\
 AsCl$_3$       & Arsenic trichloride                    &   -62.50    &    -7.71  &  &  &     aa\\
 AsBr$_3$       & Arsenic tribromide                     &   -31.57    &    11.30  &  &  &    jjj\\
 AsI$_3$        & Arsenic triiodide                      &     6.90    &    20.38  &  &  &      g\\
\hline
\end{tabular}
\end{center}
\end{table}
\clearpage

\begin{table}
\caption{Comparison of Calculated and Observed $\Delta H_f$ 
for MNDO, AM1, and PM3 (contd.)}
\begin{center}
\compresstable
\begin{tabular}{llrrrrr}
Empirical & Chemical Name & $\Delta H_f$ & \multicolumn{3}{c}{Difference} & \\
Formula   &               & Exp. & PM3 &  MNDO  &  AM1 &     Ref.\\
\hline
 As$_2$         & Arsenic, dimer                         &    53.11    &    -7.46  &  &  &     aa\\
 As$_3$         & Arsenic, trimer                        &    62.48    &     6.53  &  &  &    jjj\\
 As$_4$         & Arsenic, tetramer                      &    34.39    &     0.05  &  &  &     aa\\
 As$_4$O$_6$       & Arsenic trioxide                       &  -289.00    &     1.11  &  &  &     aa\\
 Zn          & Zinc, atom                             &    31.17    &     0.00  &     0.00  &     0.00  &      a\\
 Zn          & Zinc (+)                               &   249.46    &   -14.09  &    -9.83  &    -5.21  &     aa\\
 Zn          & Zinc (++)                              &   665.10    &    -2.37  &    55.11  &    64.34  &     aa\\
 ZnH         & Zinc hydride                           &    54.40    &     2.48  &   -11.24  &    -3.45  &      a\\
 CZnH$_3$       & Methylzinc (+)                         &   215.10    &    13.77  &    12.59  &     9.33  &     aa\\
 C$_2$ZnH$_6$      & Dimethylzinc                           &    12.67    &    -4.49  &     7.18  &     7.14  &     aa\\
 C$_2$ZnH$_6$      & Dimethylzinc (+)                       &   221.70    &    12.28  &    31.49  &    11.85  &     aa\\
 C$_4$ZnH$_1$$_0$     & Diethylzinc                            &    12.10    &    -6.53  &     0.83  &     2.01  &     aa\\
 C$_6$ZnH$_1$$_4$     & Di-n-propylzinc                        &    -2.90    &     1.38  &     7.39  &     5.27  &      f\\
 C$_8$ZnH$_1$$_8$     & Di-n-butylzinc                         &   -11.90    &     0.20  &     7.10  &     0.67  &      f\\
 ZnS         & Zinc sulfide                           &    48.30    &     3.18  &    30.71  &    30.04  &    jjj\\
 ZnCl$_2$       & Zinc dichloride                        &   -63.60    &    10.77  &    14.89  &     9.02  &     aa\\
 ZnCl$_2$       & Zinc dichloride (+)                    &   207.70    &     0.09  &    28.58  &    11.72  &     aa\\
 Cd          & Cadmium, atom                          &    26.72    &     0.00  &  &  &      a\\
 C$_2$CdH$_6$      & Dimethylcadmium                        &    25.80    &     4.77  &  &  &      f\\
 C$_2$CdH$_6$      & Dimethylcadmium (+)                    &   223.20    &    -5.06  &  &  &     aa\\
 C$_4$CdH$_1$$_0$     & Diethylcadmium                         &    25.50    &     0.37  &  &  &      f\\
 Hg          & Mercury, atom                          &    14.69    &     0.00  &     0.00  &     0.00  &      a\\
 Hg          & Mercury (+)                            &   256.83    &    14.70  &   -34.37  &   -31.33  &     aa\\
 Hg          & Mercury (++)                           &   690.84    &    -9.71  &   -11.56  &    -5.48  &     aa\\
 HgH         & Mercury hydride                        &    57.20    &    -9.14  &   -19.42  &    -1.27  &     aa\\
 CHgH$_3$       & Methylmercury                          &    40.00    &     5.88  &   -11.08  &    -1.86  &     aa\\
 C$_2$HgH$_6$      & Dimethylmercury                        &    22.30    &     6.07  &   -12.14  &     5.07  &     aa\\
 C$_2$HgH$_6$      & Dimethylmercury (+)                    &   233.90    &     6.09  &    -4.07  &    -4.50  &     aa\\
 C$_4$HgH$_1$$_0$     & Diethylmercury                         &    17.80    &    -2.07  &    -5.69  &     1.93  &     aa\\
 C$_6$HgH$_1$$_4$     & Diisopropylmercury                     &     9.70    &    -7.06  &    13.11  &     6.06  &      f\\
 C$_6$HgH$_1$$_4$     & Di-n-propylmercury                     &     8.20    &    -2.66  &    -3.40  &    -0.48  &      f\\
 C$_6$HgH$_1$$_4$     & Di-n-propylmercury                     &     8.20    &    -2.66  &    -3.40  &    -0.48  &      f\\
 C$_8$HgH$_1$$_8$     & Di-n-butylmercury                      &    -7.80    &     2.64  &     3.38  &     1.89  &      f\\
 C$_8$HgH$_1$$_8$     & Diisobutylmercury                      &    -9.20    &    -1.14  &    19.31  &     9.80  &      f\\
 C$_1$$_2$HgH$_1$$_0$    & Diphenylmercury                        &    93.80    &     4.48  &     7.62  &     7.24  &      f\\
 C$_2$HgN$_2$      & Mercuric cyanide                       &    91.00    &     2.08  &   -21.94  &    -6.77  &     aa\\
 HgF         & Mercury fluoride                       &     1.00    &    -8.24  &   -25.47  &   -11.68  &     aa\\
 HgF$_2$        & Mercury difluoride                     &   -70.20    &    19.15  &     3.33  &    16.01  &    ddd\\
 HgCl        & Mercury chloride                       &    20.10    &   -15.22  &   -26.40  &   -22.77  &     aa\\
 CHgH$_3$Cl     & Methylmercuric chloride                &   -12.50    &     9.10  &    -5.51  &    -0.04  &     aa\\
 C$_2$HgH$_5$Cl    & Ethylmercuric chloride                 &   -15.00    &     5.15  &    -1.76  &    -1.40  &     aa\\
 HgCl$_2$       & Mercury dichloride                     &   -34.97    &     2.30  &    -1.97  &    -9.87  &    ddd\\
 HgBr        & Mercury bromide                        &    24.90    &   -14.19  &   -10.12  &   -28.40  &    ddd\\
 CHgH$_3$Br     & Methylmercuric bromide                 &    -4.40    &     3.32  &     7.18  &    -6.61  &     aa\\
 C$_2$HgH$_5$Br    & Ethylmercuric bromide                  &    -7.20    &    -0.57  &    11.13  &    -7.74  &     aa\\
 HgBr$_2$       & Mercury dibromide                      &   -20.40    &    -6.54  &    23.49  &   -23.55  &    ddd\\
\hline
\end{tabular}
\end{center}
\end{table}
\clearpage

\begin{table}
\caption{Comparison of Calculated and Observed $\Delta H_f$ 
for MNDO, AM1, and PM3 (contd.)}
\begin{center}
\compresstable
\begin{tabular}{llrrrrr}
Empirical & Chemical Name & $\Delta H_f$ & \multicolumn{3}{c}{Difference} & \\
Formula   &               & Exp. & PM3 &  MNDO  &  AM1 &     Ref.\\
\hline
 CHgH$_3$I      & Methylmercuric iodide                  &     5.21    &     4.28  &     8.27  &    15.94  &     aa\\
 C$_2$HgH$_5$I     & Ethylmercuric iodide                   &     3.30    &    -0.46  &    11.11  &    13.80  &     aa\\
 HgI$_2$        & Mercury diiodide                       &    -4.10    &    -0.13  &    25.21  &    23.20  &     aa\\
 Hg$_2$         & Mercury, dimer                         &    26.00    &    -0.11  &     4.22  &     2.04  &     aa\\
 Hg$_2$         & Mercury, dimer (+)                     &   244.30    &    -6.86  &   -27.62  &   -23.63  &     aa\\
 Sb          & Antimony, atom                         &    63.20    &     0.00  &  &  &    jjj\\
 Sb          & Antimony (+)                           &   263.50    &     8.49  &  &  &     aa\\
 Sb          & Antimony (++)                          &   646.00    &   -29.16  &  &  &     aa\\
 SbH$_3$        & Stibine                                &    34.70    &    23.37  &  &  &     aa\\
 C$_3$SbH$_9$      & Trimethylstibine                       &     7.70    &    -8.08  &  &  &    ppp\\
 C$_6$SbH$_1$$_5$     & Triethylstibine                        &    10.40    &   -33.48  &  &  &    ppp\\
 C$_1$$_8$SbH$_1$$_5$    & Triphenylstibine                       &   104.10    &    -5.75  &  &  &    ppp\\
 SbO         & Antimony oxide                         &    47.70    &   -11.77  &  &  &     aa\\
 SbN         & Antimony nitride                       &    63.70    &    38.82  &  &  &     aa\\
 SbF         & Antimony fluoride                      &   -11.30    &    10.62  &  &  &     aa\\
 SbCl        & Antimony chloride                      &    -6.20    &    19.51  &  &  &     aa\\
 SbOCl       & Antimony oxychloride                   &   -25.50    &    18.39  &  &  &     aa\\
 SbCl$_2$       & Antimony dichloride                    &   -18.50    &   -15.82  &  &  &     aa\\
 SbCl$_3$       & Antimony trichloride                   &   -75.00    &     2.55  &  &  &     aa\\
 SbCl$_5$       & Antimony pentachloride                 &   -94.20    &     2.56  &  &  &     aa\\
 SbBr$_3$       & Antimony tribromide                    &   -46.50    &     9.99  &  &  &     aa\\
 InSb        & Indium antimonide                      &    82.30    &    -2.22  &  &  &     aa\\
 Sb$_2$         & Antimony, dimer                        &    56.30    &    14.14  &  &  &     aa\\
 InSb$_2$       & Indium diantimonide                    &    75.00    &   -16.43  &  &  &     aa\\
 Sb$_4$         & Antimony, tetramer                     &    49.00    &     2.46  &  &  &     aa\\
 Se          & Selenium, atom                         &    54.30    &     0.00  &  &  &      a\\
 SeH$_2$        & Hydrogen selenide                      &     7.10    &    15.62  &  &  &     aa\\
 C$_4$SeH$_1$$_0$     & Diethylselenium                        &   -13.70    &   -23.69  &  &  &      f\\
 SeO         & Selenium oxide                         &    14.80    &     5.55  &  &  &      g\\
 SeO$_2$        & Selenium dioxide                       &   -53.90    &    24.21  &  &  &     cc\\
 SeOF$_2$       & Seleninyl difluoride                   &  -121.60    &   -12.65  &  &  &      g\\
 SeF$_4$        & Selenium tetrafluoride                 &  -194.00    &   -15.49  &  &  &      g\\
 SeF$_6$        & Selenium hexafluoride                  &  -267.00    &     0.23  &  &  &     cc\\
 CSe$_2$        & Carbon diselenide                      &    61.40    &     3.36  &  &  &      f\\
 GeTe        & Germanium telluride                    &    42.00    &    25.53  &  & &     cc\\
 SnTe        & Tin telluride                          &    38.40    &    15.16  &  & &     cc\\
 AsTe        & Arsenic telluride                      &    54.70    &     6.39  &  &  &    jjj\\
 ZnTe        & Zinc telluride                         &    61.00    &   -12.74  &  &  &    jjj\\
 Te$_2$         & Tellurium, dimer                       &    38.00    &    13.31  &  &  &    jjj\\
 Te$_2$O$_2$       & Tellurium oxide dimer                  &   -26.00    &   -18.47  &  &  &    jjj\\
 Te$_2$F$_1$$_0$      & Te2F10                                 &  -554.40    &    10.58  &  &  &    jjj\\
 Bi          & Bismuth, atom                          &    50.10    &     0.00  &  &  &      a\\
 Bi          & Bismuth (+)                            &   219.10    &    18.81  &  &  &     aa\\
 Bi          & Bismuth (++)                           &   606.10    &   -17.27  &  &  &     aa\\
 C$_3$BiH$_9$      & Trimethylbismuth                       &    46.10    &    -3.07  &  &  &      f\\
 C$_6$BiH$_1$$_5$     & Triethylbismuth                        &    51.60    &   -26.08  &  &  &      f\\
 C$_1$$_8$BiH$_1$$_5$    & Triphenylbismuth                       &   138.60    &    -2.92  &  &  &      f\\
 BiF         & Bismuth fluoride                       &    -7.02    &    14.10  &  &  &    jjj\\
\hline
\end{tabular}
\end{center}
\end{table}
\clearpage

\begin{table}
\caption{Comparison of Calculated and Observed $\Delta H_f$ 
for MNDO, AM1, and PM3 (contd.)}
\begin{center}
\compresstable
\begin{tabular}{llrrrrr}
Empirical & Chemical Name & $\Delta H_f$ & \multicolumn{3}{c}{Difference} & \\
Formula   &               & Exp. & PM3 &  MNDO  &  AM1 &     Ref.\\
\hline
 BiCl        & Bismuth chloride                       &     6.00    &    11.76  &  &  &    jjj\\
 BiCl$_3$       & Bismuth trichloride                    &   -63.50    &    20.87  &  &  &     aa\\
 BiSe        & Bismuth selenide                       &    42.00    &    12.55  &  &  &     aa\\
 Bi$_2$         & Bismuth, dimer                         &    52.50    &     2.68  &  &  &     aa\\
\end{tabular}
\end{center}
\end{table}

\begin{center} References \end{center}

\begin{description}
\item{   a: } ``CRC Handbook of Chemistry and Physics'', 60th Edition, R.\ C.\ Weast, (Ed.),
       CRC Press, Boca Raton, FL, 1980.
  
\item{   b: } Element in its standard state.
  
\item{   c: } R.\ D.\ Levin, S.\ G.\ Lias, ``Ionization Potentials and Appearance Potential
       Measurements'', 1971-1981, Natl.\ Stand.\ Ref.\ Data Ser., Natl.\ Bur.\
       Stand.\ 71, (1982), Cat.\ No.\ C13.48:71.
  
\item{   d: } M.\ W.\ Chase, C.\ A.\ Davies, J.\ R.\ Downey, D.\ R.\ Frurip, R.\ A.\ McDonald,
       A.\ N.\ Syverud, JANAF Thermochemical Tables, Third Edition,
       J.\ Phys.\ Chem.\ Ref.\ Data 14, Suppl.\ 1 (1985).
  
\item{   e: } M.\ T.\ Bowers, ``Gas Phase Ion Chemistry'', Vol.\ 2 (Academic Press, New York,
       (1979).
  
\item{   f: } J.\ O.\ Cox, G.\ Pilcher, ``Thermochemistry of Organic and Organometallic
       Compounds'', Academic Press, New York, N.Y., 1970.
  
\item{   g: } G.\ Ditter, U.\ Niemann, Philips J.\ Res., 37, 1 (1982).
  
\item{   h: } F.\ P.\ Lossing, Can J.\ Chem., 49, 357 (1971).
  
\item{   i: } H.\ Halim, N.\ Heinrich, W.\ Koch, J.\ Schmidt, G.\ Frenking, J.\ Comp.\ Chem.,
       7, 93 (1986), references therein.
  
\item{   j: } S.\ Schroder, W.\ Thiel, J.\ Am.\ Chem.\ Soc., 107, 4422 (1985). ``BEST'' ab-initio
       involving at least MP2 correction to 6-31G* wavefunctions.
  
\item{   k: } J.\ C.\ Traeger, R.\ G.\ McLoughlin, J.\ Am.\ Chem.\ Soc., 103, 3647 (1981).
  
\item{   l: } J.\ L.\ Franklin, J.\ G.\ Dillard, H.\ M.\ Rosenstock, J.\ T.\ Herron,
       K.\ Draxl, F.\ H.\ Field, Natl.\ Stand.\ Ref.\ Data Ser., Natl.\ Bur.\
       Stand., No.\ 26 (1969).
  
\item{   m: } D.\ R.\ Stull, E.\ F.\ Westrum, Jr., G.\ C.\ Sinke, ``The Chemical Thermodynamics
       of Organic Compounds'', Wiley, New York, N.Y., 1969.
  
\item{   n: } J.\ B.\ Pedley, B.\ S.\ Iseard, ``CATCH Tables of Silicon Compounds'', University
       of Sussex, 1972
  
\item{   o: } J.\ E.\ Bartmess, R.\ T.\ McIver, ``Gas Phase Ion Chemistry'', Academic Press,
       New York, 1979, Vol.\ II.
  
\item{   p: } R.\ B.\ Turner, P.\ Goebel, B.\ J.\ Mallon, W.\ von E.\ Doering, J.\ F.\ Coburn,
       M.\ Pomerantz, J.\ Am.\ Chem.\ Soc., 90, 4315 (1968).
  
\item{   q: } W.\ D.\ Good, R.\ T.\ Moore, A.\ G.\ Osborn, D.\ R.\ Douslin, J.\ Chem.\
       Thermodyn., 6, 303 (1974).
  
\item{   r: } J.\ M.\ Harris, S.\ G.\ Shafer, S.\ D.\ Worley, J.\ Comput.\ Chem., 3, 208 (1982).
  
\item{   s: } F.\ P.\ Lossing, A.\ Maccoll, Can.\ J.\ Chem., 54, 990 (1976).
  
\item{   t: } J.\ L.\ Abboud, W.\ J.\ Hehre, R.\ W.\ Taft, J.\ Am.\ Chem.\ Soc., 94, 6072
       (1972).
  
\item{   u: } G.\ Vincow, H.\ J.\ Dauben, F.\ R.\ Hunter, W.\ V.\ Volland, J.\ Am.\ Chem.\ Soc.,
       91, 2823 (1969).
  
\item{   v: } P.\ v.\ R.\ Schleyer, J.\ E.\ Williams, K.\ R.\ Blanchard, J.\ Am.\ Chem.\ Soc.,
       92, 3277, (1970).
  
\item{   w: } Average value from two determinations: M.\ Mansson, N.\ Rapport, E.\ F.\ Westrum
       J.\ Am.\ Chem.\ Soc., 92, 7296 (1970); R.\ S.\ Butler, A.\ S.\ Carson,
       P.\ G.\ Laye, W.\ V.\ Steele, J.\ Chem.\ Thermodyn.\ 3, 277 (1971).
  
\item{   x: } M.\ W.\ Chase, J.\ L.\ Curnutt, A.\ T.\ Hu, H.\ Prophet, A.\ N.\ Syverud, L.\ C.\ Walke
       J.\ Phys.\ Chem.\ Ref.\ Data, 3, 311, (1974).
  
\item{   y: } J.\ Chao, B.\ J.\ Zwolinski, J.\ Phys.\ Chem.\ Ref.\ Data, 5, 319 (1976).
  
\item{   z: } G.\ Herzberg, ``Molecular Spectra and Molecular Structure I.\ Spectra of
       Diatomic Molecules'', 2nd ed, Van Nostrand, New York, N.Y., 1950.
  
\item{  aa: } D.\ D.\ Wagman, W.\ H.\ Evans, V.\ B.\ Parker, R.\ H.\ Schumm, I.\ Halow,
       S.\ M.\ Bailey, K.\ L.\ Churney, R.\ L.\ Nuttall, J.\ Phys.\ Chem.\
       Ref.\ Data Suppl.\ 11, 2 (1982).
  
\item{  bb: } P.\ Kebarle, R.\ Tamdagni, H.\ Kirooka, T.\ B.\ McMahon, Int.\ J.\ Mass Spectrom.\
       Ion Phys., 19 71 (1976).
  
\item{  cc: } D.\ D.\ Wagman, W.\ H.\ Evans, V.\ B.\ Parker, T.\ Hawlow, S.\ M.\ Bailey,
       R.\ H.\ Schumm, Natl.\ Bur.\ Stand.\ (U.S.), Tech.\ Note, No 270-3 (1968)
       and ``Errata in NBS Technical Note 270-8'' (1981).
  
\item{  dd: } Yu.\ K.\ Knobel, E.\ A.\ Miroschnichenko, Yu.\ A.\ Lebedev, Dokl.\ Akad.\ Nauk
       SSSR, 190, 348 (1970).
  
\item{  ee: } D.\ Hwang, M.\ Tamura, T.\ Yoshida, N.\ Tanaka, F.\ Hosoya, J.\ Energetic
       Matls, (1989) in press.
  
\item{  ff: } B.\ I.\ Istomin, V.\ Palm, Reakts.\ Sposobnosi Org.\ Soedin., 10, 583 (1973).
  
\item{  gg: } C.\ Willis, F.\ P.\ Lossing, R.\ A.\ Back, Can.\ J.\ Chem.\ 54, 1 (1967). Note:
       S.\ N.\ Foner, R.\ L.\ Hudson, J.\ Chem.\ Phys., 68, 3162 (1978) give the
       heat of formation of diazine as 50.7 +/- 2 Kcal/mol.
  
\item{  hh: } S.\ W.\ Benson, F.\ R.\ Cruickshank, D.\ M.\ Golden, G.\ R.\ Haugen, H.\ E.\ O'Neal,
       A.\ S.\ Rodgers, R.\ Shaw, R.\ Walsh, Chem.\ Rev., 69, 279 (1969).
  
\item{  ii: } P.\ S.\ Engel, J.\ L.\ Wood, J.\ A.\ Sweet, M.\ L.\ Margrave, J.\ Am.\ Chem.\ Soc.\
       96, 2381 (1974).
  
\item{  jj: } G.\ A.\ Carpenter, M.\ F.\ Zimmer, E.\ E.\ Baroody, R.\ A.\ Robb, J.\ Chem.\ Eng.\
       Data, 15, 553 (1970).
  
\item{  kk: } J.\ J.\ Grabowski, L.\ Zhang, J.\ Am.\ Chem.\ Soc., 111, 1193 (1989).
  
\item{  ll: } H.\ Mackle, P.\ A.\ G.\ O'Hare, Tetrahedron, 19, 961 (1963).
  
\item{  mm: } J.\ V.\ Davis, S.\ Sunner, Acta Chem.\ Scand., 16, 1870 (1962).
  
\item{  nn: } M.\ Mansson, S.\ Sunner, Acta Chem.\ Scand., 16, 1863 (1962).
  
\item{  oo: } J.\ B.\ Pedley, G.\ Rylance, ``Sussex-N.P.L.\ Computer Analysed Thermochemical
       Data: Organic and Organometallic Compounds'', Sussex University, 1977.
  
\item{  pp: } H.\ Mackle, R.\ T.\ B.\ McLean, Trans.\ Faraday Soc., 59, 669 (1963).
  
\item{  qq: } B.\ G.\ Hobrock, R.\ W.\ Kiser, J.\ Phys.\ Chem., 67, 1283 (1963).
  
\item{  rr: } R.\ J.\ Blint, T.\ B.\ McMahon, J.\ L.\ Beauchamp, J.\ Am.\ Chem.\ Soc., 96, 1269
       (1974).
  
\item{  ss: } A.\ S.\ Rodgers, J.\ Chao, R.\ C.\ Wilhoit, B.\ J.\ Zwolinski, J.\ Phys.\ Chem.\
       Ref.\ Data, 3, 117 (1974).
  
\item{  tt: } R.\ F.\ Lake, H.\ Thompson, Proc.\ R.\ Soc.\ London, Ser.\ A, 315, 323 (1970).
  
\item{  uu: } D.\ P.\ Ridge, J.\ Am.\ Chem.\ Soc., 97, 5670 (1975).
  
\item{  vv: } J.\ Berkowitz, P.\ M.\ Dehmer, W.\ A.\ Chupka, J.\ Chem.\ Phys., 59, 925 (1973).
  
\item{  ww: } J.\ A.\ Pople, L.\ A.\ Curtiss, J.\ Chem.\ Phys.\ 90, 2833 (1989). Ref.\ gives -19.9
       Kcal/mol for calculated heat of formation of HOF at 0K.
  
\item{  xx: } J.\ Vogt, J.\ L.\ Beauchamp, J.\ Am.\ Chem.\ Soc., 97, 6682 (1975).
  
\item{  yy: } P.\ W.\ Harland, J.\ L.\ Franklin, J.\ Chem.\ Phys., 61, 1621 (1974).
  
\item{  zz: } E.\ C.\ Wu, A.\ S.\ Rodgers, J.\ Phys.\ Chem.; 78, 2315 (1974).
  
\item{ aaa: } J.\ W.\ Larson, T.\ B.\ McMahon, J.\ Am.\ Chem.\ Soc., 107, 766 (1985).
  
\item{ bbb: } Private Communication by Dr Dave Dixon, DuPont.
  
\item{ ccc: } A.\ S.\ Gordon, Int.\ J.\ Chem.\ Kinet., 4, 541 (1972).
  
\item{ ddd: } D.\ R.\ Stull, H.\ Prophet, Natl.\ Stand., Ref.\ Data Ser.\ (U.S., Natl.\ Bur.\
       Stand.) NSRDS-NBS 37, 1971.
  
\item{ eee: } NBS Technical Note 270-3, Jan.\ (1968).
  
\item{ fff: } D.\ C.\ Frost, C.\ A.\ McDowell, D.\ A.\ Vroom, J.\ Chem.\ Phys., 46, 4255 (1967).
  
\item{ ggg: } Natl.\ Bur.\ Stand.\ Selected Values of Chemical and Thermodynamic Properties,
       Rossini, 1st Feb.\ (1952).
  
\item{ hhh: } R.\ L.\ De Kock, B.\ R.\ Higginson, D.\ R.\ Lloyd, A.\ Breeze,
       D.\ W.\ J.\ Cruickshank, D.\ R.\ Armstrong, Mol.\ Phys., 24, 1059 (1972).
  
\item{ iii: } S.\ A.\ Kudchadker, A.\ P.\ Kudchadker, J.\ Phys.\ Chem.\ Ref.\ Data,
       4, 457 (1975).
  
\item{ jjj: } I.\ Barin, O.\ Knacke, O.\ Kubaschewski, ``Thermochemical Properties of
       Inorganic Compounds'', Springer-Verlag, Berlin, 1977.
  
\item{ kkk: } ``Spectroscopic Properties of Inorganic and Organometallic Compounds'',
       Royal Society of Chemistry, Spottiswoode Ballantyne Ltd., Colchester
       and London, 1980, Vol.\ 14.
  
\item{ lll: } C.\ H.\ Wu, J.\ Chem.\ Phys., 91, 546 (1989).
  
\item{ mmm: } L.\ Operti, E.\ C.\ Tews, T.\ J.\ Macmahon, B.\ S.\ Freiser, J.\ Am.\ Chem.\ Soc.,
       111, 9152 (1989).
  
\item{ nnn: } M.\ D.\ Harmony, V.\ W.\ Laurie, R.\ L.\ Kuczkowsky, R.\ H.\ Schwendeman,
       D.\ A.\ Ramsay, F.\ J.\ Lovas, W.\ J.\ Lafferty, A.\ G.\ Maki, J.\ Phys.\
       Chem.\ Ref.\ Data 8, 3 (1979).
  
\item{ ooo: } M.\ F.\ Guest, J.\ B.\ Pedley, M.\ Horn, J.\ Chem.\ Thermodyn., 1, 345 (1969).
  
\item{ ppp: } G.\ M.\ Kolyakova, I.\ B.\ Rabinovich, E.\ N.\ Zoria, Dokl.\ Akad.\ Nauk
       SSSR, 209, 616 (1973).
  
\item{ qqq: } L.\ A.\ Curtiss, J.\ A.\ Pople, J.\ Chem.\ Phys., 90, 4314 (1989).
  
\item{ rrr: } R.\ Thomson, F.\ W.\ Dalby, Can J.\ Phys., 47, 1155 (1969).
  
\item{ sss: } L.\ Barton, S.\ K.\ Watson, R.\ F.\ Porter, J.\ Phys.\ Chem., 69, 3160 (1965).
  
\item{ ttt: }  A.\ J.\ Vanderwielen, M.\ A.\ Ring, H.\ E.\ O'Neal, J.\ Am.\ Chem.\ Soc., 97,
       993 (1975).
  
\item{ uuu: } M.\ M.\ Heaton, J.\ Chem.\ Phys., 67, 5396 (1977).
  
\item{ vvv: } M.\ R.\ Nimlos, G.\ B.\ Ellison, J.\ Am.\ Chem.\ Soc., 108, 6522-6529 (1986).
  
\item{ www: } R.\ Walsh, Acc.\ Chem.\ Res., 14, 246 (1981)
  
\item{ xxx: } W.\ C.\ Steele, L.\ D.\ Nichols, F.\ G.\ A.\ Stone, J.\ Am.\ Chem.\ Soc., 84,
       4441 (1962)
  
\item{ yyy: } W.\ V.\ Steele, J.\ Chem.\ Thermo., 15, 595 (1983).
  
\item{ zzz: } H.\ B.\ Schlegel, J.\ Phys.\ Chem., 88, 6254 (1984).
  
\item{aaaa: } R.\ Walsh, J.\ Chem.\ Soc.; Faraday I, 79, 2233 (1983).
  
\item{bbbb: } M.\ Farber, R.\ D.\ Srivastava, Chem.\ Phys.\ Lett., 60, 216, (1979).
  
\item{cccc: } P.\ Potzinger, A.\ Ritter, J.\ Krause, Z.\ Naturforsch., Teil A, 30, 347,
       (1975).
  
\item{dddd: } J.\ C.\ Baldwin, M.\ F.\ Lappert, J.\ B.\ Pedley, J.\ S.\ Poland,
       J.\ Chem.\ Soc., Dalton Trans., 1943 (1972).
  
\item{eeee: } CODATA, J.\ Chem.\ Thermod., 10, 903 (1978).
  
\item{ffff: } M.\ F.\ Lappert, J.\ B.\ Pedley, J.\ Simpson, T.\ R.\ Spalding, J.\ Organomet.\
       Chem., 29, 195 (1971).
  
\item{gggg: } ``Gmelins Handbuch der anorganischen Chemie'', Zinn, Thiel C1,
       Verlag Chemie, GMBH, Weinheim, 1972.
  
\item{hhhh: } G.\ A.\ Nash, H.\ A.\ Skinner, W.\ F.\ Stack, Trans.\ Faraday Soc.,
      61, 640 (1965).
  
\item{iiii: } H.\ A.\ Skinner, Adv.\ Organomet.\ Chem., 2, 49 (1964).
  
\item{jjjj: } F.\ E.\ Saalfeld, H.\ Svec, J.\ Inorg.\ Chem., 2, 46 (1963).
  
\item{kkkk: } Y.\ Wada, R.\ W.\ Kiser, J.\ Phys.\ Chem., 68, 2290 (1964).
  
\item{llll: } M.\ Henchman, A.\ A.\ Viggiano, J.\ F.\ Paulson, A.\ Freeman, J.\ Wormhoudt,
       J.\ Am.\ Chem.\ Soc., 107, 1253 (1985).
  
\item{mmmm: } E.\ C.\ Penski, E.\ S.\ Domalski, CRDEC-TR-87063, (1987).
\end{description}


\section{Comp. of Calc'd and Exp'l Geometries for MNDO, AM1, and PM3}

The following Table lists over 1100 experimental bond lengths, angles,  and
dihedrals or torsion angles, and the errors in the prediction of these
quantities for each method.

Reference geometry data are taken from many sources, mainly from X-ray and
microwave structures. No differentiation is made between these various sources,
although microwave structures are preferred to X-ray. Further, no
differentiation is made between various definitions of bond lengths. Thus the
bond length corresponding to the bottom of the harmonic well, $r_o$, is not
distinguished from the equilibrium bond length, or from the bond length
corresponding to the system in which the zero point energy is included. For
semiempirical methods, the average error in bond lengths is large relative to
the differences between the bond lengths obtained using the various
definitions.

\subsection*{Notes on the Table}
\begin{itemize}
\item The AM1 results for P$_4$O$_6$ and P$_4$O$_{10}$ are incorrect. The
experimental geometry was used to start the calculation.  In order to generate
the lowest energy geometry, the systems must first be distorted.
\item For salicylaldoxim, the bonds reported are the internal hydrogen bond and
a non-bonding O-N interaction.
\item Several 180$^{\circ}$ angles are reported. Most of these correspond to
symmetry-defined angles.  In some instances, e.g., MgCl$_2$, the calculated
angle is incorrect.  The protocols for this Chapter require {\em all} such
results to be reported.
\item The PM3 prediction for the bond-length in HgI is in error by over 10\AA .
HgI is predicted, by PM3, to consist of a mercury and an iodine atom.  The
fault can be traced to the incorrect PM3 prediction for the ground state of the
iodine atom. Removal of this one error would lower the average errors in Hg
bond lengths from 0.586\AA\ to 0.054\AA , and the average error in I bond
lengths from 0.371\AA\ to 0.091\AA .  These, in turn, result in the average
bond-length error in PM3 dropping from 0.045\AA .  If that were done, then PM3
would become the most accurate method for the prediction of bond-lengths. 
Again, the protocols involved preclude this being done at this time.
\end{itemize}

\begin{table}
\caption{\label{geotab}Comparison of Calculated and Observed Geometries 
for MNDO, AM1, and PM3}
\begin{center}
\compresstable
\begin{tabular}{llrrrrrr}
 Empirical  & Chemical Name &  Geometric &  Exp. & \multicolumn{3}{c}{Errors} & \\
  Formula   &               &  Variable &        & PM3  & MNDO  &  AM1 & Ref.\\
\hline
 H$_2$          & Hydrogen                           &HH            &     0.741   &    -0.042 &    -0.078 &    -0.064 &     a \\
 CH$_2$         & Methylene, singlet                 &CH            &     1.110   &    -0.018 &    -0.019 &    -0.007 &     a \\
             &                                    &HCH         &     102.4   &       1.3 &       8.7 &       8.1   &       \\
 CH$_2$         & Methylene, triplet                 &CH            &     1.029   &     0.033 &     0.023 &     0.034 &     a \\
             &                                    &HCH         &     136.0   &       9.1 &      13.7 &      12.4   &       \\
 CH$_4$         & Methane                            &CH            &     1.094   &    -0.007 &     0.010 &     0.018 &     b \\
 C$_2$          & Carbon, dimer                      &CC            &     1.242   &    -0.053 &    -0.073 &    -0.078 &     a \\
 C$_2$H$_2$        & Acetylene                          &CC            &     1.203   &    -0.013 &    -0.008 &    -0.008 &     b \\
             &                                    &CH            &     1.060   &     0.004 &    -0.009 &     0.001 &       \\
 C$_2$H$_4$        & Ethylene                           &CC            &     1.339   &    -0.017 &    -0.004 &    -0.013 &     b \\
             &                                    &CH            &     1.086   &     0.000 &     0.003 &     0.012 &       \\
             &                                    &HCC         &     121.2   &       1.9 &       2.0 &       1.5   &       \\
 C$_2$H$_6$        & Ethane                             &CC            &     1.536   &    -0.032 &    -0.015 &    -0.036 &     b \\
             &                                    &CH            &     1.091   &     0.007 &     0.018 &     0.026 &       \\
             &                                    &HCC         &     110.9   &       0.7 &       0.3 &      -0.2   &       \\
 C$_3$H$_4$        & Allene                             &CC            &     1.308   &    -0.012 &    -0.002 &    -0.010 &     c \\
             &                                    &CH            &     1.087   &    -0.001 &     0.003 &     0.013 &       \\
             &                                    &HCC         &     120.9   &       1.3 &       1.9 &       1.2   &       \\
 C$_3$H$_4$        & Cyclopropene                       &C2C3          &     1.509   &    -0.025 &     0.003 &    -0.020 &     d \\
             &                                    &C1C2          &     1.296   &     0.018 &     0.032 &     0.023 &       \\
             &                                    &C1H           &     1.072   &     0.001 &    -0.010 &    -0.003 &       \\
             &                                    &HC1C2       &     149.9   &       1.6 &       1.7 &       2.0   &       \\
 C$_3$H$_4$        & Propyne                            &C2C1          &     1.206   &    -0.014 &    -0.009 &    -0.009 &     e \\
             &                                    &C1H           &     1.056   &     0.008 &    -0.005 &     0.004 &       \\
             &                                    &C3C3          &     1.459   &    -0.026 &    -0.014 &    -0.032 &       \\
             &                                    &C3H           &     1.105   &    -0.007 &     0.006 &     0.016 &       \\
             &                                    &HCC         &     111.0   &      -0.3 &       0.0 &      -0.5   &       \\
 C$_3$H$_6$        & Cyclopropane                       &CC            &     1.510   &    -0.011 &     0.016 &    -0.009 &     f \\
             &                                    &CH            &     1.089   &     0.006 &     0.007 &     0.015 &       \\
 C$_3$H$_6$        & Propene                            &C=C           &     1.336   &    -0.008 &     0.004 &    -0.005 &     g \\
             &                                    &C-C           &     1.501   &    -0.021 &    -0.005 &    -0.025 &       \\
             &                                    &CCC         &     124.3   &      -0.9 &       2.6 &       0.0   &       \\
             &                                    &C3H           &     1.085   &     0.013 &     0.024 &     0.033 &       \\
             &                                    &HC3C2       &     111.2   &       1.7 &       1.9 &       0.7   &       \\
             &                                    &C2H           &     1.090   &     0.006 &     0.006 &     0.014 &       \\
             &                                    &HC2C1       &     119.0   &       1.8 &       0.3 &       1.9   &       \\
             &                                    &HC1           &     1.081   &     0.005 &     0.008 &     0.017 &       \\
             &                                    &HC1C2       &     121.5   &       1.1 &       0.8 &       0.8   &       \\
 C$_3$H$_8$        & Propane                            &CC            &     1.526   &    -0.014 &     0.004 &    -0.019 &     g \\
             &                                    &CCC         &     112.4   &      -0.6 &       3.0 &      -0.5   &       \\
             &                                    &C2H           &     1.115   &    -0.007 &     0.000 &     0.007 &       \\
             &                                    &HC2C1       &     109.5   &       0.3 &      -0.7 &      -0.1   &       \\
             &                                    &C1H           &     1.096   &     0.001 &     0.014 &     0.021 &       \\
             &                                    &HC1C2       &     111.8   &      -0.4 &      -1.5 &      -1.3   &       \\
\hline
\end{tabular}
\end{center}
\end{table}
\clearpage

\begin{table}
\caption{\label{geotabb}Comparison of Calculated and Observed Geometries for 
MNDO, AM1, and PM3 (contd.)}
\begin{center}
\compresstable
\begin{tabular}{llrrrrrr}
 Empirical  & Chemical Name &  Geometric &  Exp. & \multicolumn{3}{c}{Errors} & \\
  Formula   &               &  Variable &        & PM3  & MNDO  &  AM1 & Ref.\\
\hline
 C$_4$H$_2$        & Diacetylene                        &C1C2          &     1.205   &    -0.012 &    -0.006 &    -0.006 &     h \\
             &                                    &C2C3          &     1.376   &    -0.005 &    -0.008 &    -0.020 &       \\
             &                                    &CH            &     1.046   &     0.019 &     0.004 &     0.014 &       \\
 C$_4$H$_4$        & CH2=C=C=CH2                        &CH            &     1.083   &     0.004 &     0.007 &     0.017 &     i \\
             &                                    &C1C2          &     1.318   &    -0.017 &    -0.008 &    -0.016 &       \\
             &                                    &C2C3          &     1.283   &    -0.016 &    -0.013 &    -0.017 &       \\
 C$_4$H$_4$        & Vinylacetylene                     &C3C4          &     1.341   &    -0.009 &     0.004 &    -0.005 &     j \\
             &                                    &C2C3          &     1.431   &    -0.017 &    -0.014 &    -0.026 &       \\
             &                                    &C2C3C4      &     123.1   &      -0.5 &       2.3 &       1.0   &       \\
             &                                    &C1C2          &     1.208   &    -0.015 &    -0.010 &    -0.010 &       \\
 C$_4$H$_6$        & Bicyclobutane                      &C1C2          &     1.498   &     0.008 &     0.029 &     0.012 &     k \\
             &                                    &C1C3          &     1.497   &    -0.016 &     0.040 &    -0.002 &       \\
             &                                    &C2C3C1C4    &     121.7   &      -1.7 &       1.0 &       0.3   &       \\
             &                                    &C1H           &     1.071   &     0.012 &     0.003 &     0.008 &       \\
             &                                    &C2H           &     1.093   &     0.003 &     0.005 &     0.012 &       \\
 C$_4$H$_6$        & 2-Butyne                           &C2C3          &     1.213   &    -0.020 &    -0.013 &    -0.015 &     j \\
             &                                    &C1C2          &     1.467   &    -0.035 &    -0.023 &    -0.042 &       \\
             &                                    &CH            &     1.115   &    -0.017 &    -0.004 &     0.006 &       \\
             &                                    &HCC         &     110.7   &       0.0 &       0.3 &      -0.1   &       \\
 C$_4$H$_6$        & 1,3-Butadiene                      &C1C2          &     1.344   &    -0.013 &     0.000 &    -0.009 &     l \\
             &                                    &C2C3          &     1.467   &    -0.011 &    -0.002 &    -0.016 &       \\
             &                                    &CCC         &     122.9   &      -0.5 &       2.8 &       0.6   &       \\
 C$_4$H$_8$        & 1-Butene                           &C2C3          &     1.347   &    -0.019 &    -0.006 &    -0.016 &     m \\
             &                                    &C1C2          &     1.508   &    -0.019 &    -0.003 &    -0.024 &       \\
             &                                    &CCC         &     123.8   &      -1.2 &       1.6 &      -0.4   &       \\
 C$_4$H$_8$        & Cyclobutane                        &CC            &     1.548   &    -0.006 &     0.001 &    -0.005 &     n \\
             &                                    &C1C2C4C3    &     153.0   &      27.0 &      27.1 &      26.9   &       \\
             &                                    &CH            &     1.105   &    -0.005 &     0.000 &     0.005 &       \\
 C$_4$H$_8$        & Isobutene                          &C1C2          &     1.330   &     0.003 &     0.018 &     0.006 &     o \\
             &                                    &C2C3          &     1.507   &    -0.020 &     0.002 &    -0.023 &       \\
             &                                    &C1C2C3      &     122.4   &      -0.3 &      -0.5 &       0.0   &       \\
 C$_4$H$_1$$_0$       & n-Butane                           &C1C2          &     1.533   &    -0.021 &    -0.002 &    -0.026 &     e \\
             &                                    &C2C3          &     1.533   &    -0.013 &     0.006 &    -0.019 &       \\
             &                                    &CCC         &     112.8   &      -1.2 &       2.0 &      -1.2   &       \\
 C$_4$H$_1$$_0$       & Isobutane                          &CC            &     1.525   &    -0.005 &     0.016 &    -0.011 &     p \\
 C$_5$H$_8$        & 1,4-Pentadiene  C1                 &C=C           &     1.339   &    -0.011 &     0.001 &    -0.008 &     q \\
             &                                    &C-C           &     1.511   &    -0.022 &    -0.005 &    -0.027 &       \\
             &                                    &C-C=C       &     115.5   &       7.6 &      11.1 &       8.4   &       \\
             &                                    &C-C-C       &     113.1   &       1.3 &      -0.4 &       1.2   &       \\
 C$_5$H$_8$        & 1,4-Pentadiene  C2                 &C=C           &     1.339   &    -0.011 &     0.001 &    -0.008 &     q \\
             &                                    &C-C           &     1.511   &    -0.021 &    -0.005 &    -0.025 &       \\
             &                                    &C-C=C       &     115.5   &       7.6 &      11.2 &       8.3   &       \\
             &                                    &C-C-C       &     108.9   &       1.9 &       3.7 &       2.9   &       \\
 C$_5$H$_8$        & 1,4-Pentadiene  Cs                 &C=C           &     1.339   &    -0.011 &     0.001 &    -0.008 &     q \\
             &                                    &C-C           &     1.511   &    -0.021 &    -0.005 &    -0.025 &       \\
             &                                    &C-C=C       &     115.5   &       7.6 &      11.1 &       8.4   &       \\
             &                                    &C-C-C       &     108.9   &       2.0 &       3.7 &       2.8   &       \\
 C$_5$H$_1$$_2$       & Neopentane                         &CC            &     1.539   &    -0.012 &     0.015 &    -0.018 &     f \\
             &                                    &CH            &     1.120   &    -0.022 &    -0.011 &    -0.004 &       \\
             &                                    &HCC         &     110.0   &       1.3 &       1.7 &       0.3   &       \\
\hline
\end{tabular}
\end{center}
\end{table}
\clearpage

\begin{table}
\caption{\label{geotabc}Comparison of Calculated and Observed Geometries for 
MNDO, AM1, and PM3 (contd.)}
\begin{center}
\compresstable
\begin{tabular}{llrrrrrr}
 Empirical  & Chemical Name &  Geometric &  Exp. & \multicolumn{3}{c}{Errors} & \\
  Formula   &               &  Variable &        & PM3  & MNDO  &  AM1 & Ref.\\
\hline
 C$_6$H$_6$        & Benzene                            &CC            &     1.399   &    -0.008 &     0.008 &    -0.004 &     r \\
             &                                    &CH            &     1.084   &     0.011 &     0.006 &     0.016 &       \\
 C$_6$H$_6$        & Fulvene                            &C3C4          &     1.476   &    -0.005 &     0.001 &     0.000 &     s \\
             &                                    &C2C3          &     1.355   &     0.000 &     0.011 &     0.008 &       \\
             &                                    &C1C2          &     1.470   &     0.008 &     0.021 &     0.013 &       \\
             &                                    &C1C6          &     1.349   &    -0.018 &    -0.004 &    -0.017 &       \\
 C$_6$H$_1$$_0$       & Cyclohexene                        &C1C2          &     1.335   &    -0.001 &     0.011 &     0.002 &     t \\
             &                                    &C2C3          &     1.504   &    -0.017 &     0.000 &    -0.020 &       \\
             &                                    &C3C4          &     1.515   &     0.006 &     0.026 &     0.002 &       \\
             &                                    &C4C5          &     1.550   &    -0.031 &    -0.011 &    -0.036 &       \\
             &                                    &C5C4C2C1    &      21.8   &       6.0 &      -0.5 &       5.4   &       \\
 C$_6$H$_1$$_2$       & Cyclohexane                        &CC            &     1.536   &    -0.015 &     0.005 &    -0.021 &     u \\
             &                                    &CCC         &     111.4   &      -0.4 &       3.3 &       0.0   &       \\
             &                                    &CCCC        &      46.3   &       9.7 &      -0.6 &       8.7   &       \\
 H$_2$O         & Water                              &OH            &     0.957   &    -0.006 &    -0.014 &     0.004 &     b \\
             &                                    &HOH         &     104.5   &       3.2 &       2.3 &      -1.0   &       \\
 CO          & Carbon monoxide                    &CO            &     1.128   &     0.007 &     0.035 &     0.043 &     v \\
 CH$_2$O        & Formaldehyde                       &CO            &     1.208   &    -0.006 &     0.009 &     0.019 &     w \\
             &                                    &CH            &     1.116   &    -0.025 &    -0.010 &    -0.006 &       \\
             &                                    &HCO         &     121.7   &       0.1 &       1.8 &       0.5   &       \\
 CH$_4$O        & Methanol                           &CO            &     1.425   &    -0.030 &    -0.034 &    -0.014 &     x \\
             &                                    &CH            &     1.094   &     0.003 &     0.025 &     0.025 &       \\
             &                                    &HCO         &     108.5   &       3.6 &       3.8 &       2.4   &       \\
             &                                    &OH            &     0.945   &     0.004 &     0.002 &     0.019 &       \\
             &                                    &COH         &     107.0   &       0.5 &       4.6 &       0.2   &       \\
 C$_2$H$_2$O       & Ketene                             &CO            &     1.161   &     0.014 &     0.023 &     0.032 &     y \\
             &                                    &CC            &     1.314   &    -0.006 &     0.005 &    -0.007 &       \\
             &                                    &CH            &     1.083   &     0.001 &     0.002 &     0.012 &       \\
             &                                    &HCC         &     118.7   &       3.3 &       3.0 &       2.7   &       \\
 C$_2$H$_4$O       & Acetaldehyde                       &C1-C2         &     1.501   &    -0.002 &     0.016 &    -0.011 &     z \\
             &                                    &C2-O          &     1.210   &     0.000 &     0.011 &     0.022 &       \\
             &                                    &O-C2-C1     &     123.9   &      -0.6 &       1.1 &      -0.4   &       \\
             &                                    &C2-H          &     1.114   &    -0.012 &    -0.002 &     0.000 &       \\
             &                                    &C1-C2-H     &     117.5   &      -0.7 &      -3.5 &      -2.2   &       \\
 C$_2$H$_6$O       & Dimethyl ether                     &CC            &     1.410   &    -0.004 &    -0.014 &     0.007 &    aa \\
             &                                    &COC         &     111.3   &       2.8 &       8.7 &       1.6   &       \\
 C$_3$H$_4$O       & Acrolein                           &C3C2          &     1.335   &    -0.005 &     0.008 &    -0.001 &    bb \\
             &                                    &C2C1          &     1.478   &     0.001 &     0.007 &    -0.010 &       \\
             &                                    &CCC         &     121.0   &       2.6 &       6.3 &       2.2   &       \\
             &                                    &CO            &     1.208   &     0.003 &     0.016 &     0.026 &       \\
             &                                    &OCC         &     124.0   &       0.0 &       1.5 &       0.1   &       \\
 C$_3$H$_6$O       & Acetone                            &C=O           &     1.222   &    -0.006 &     0.005 &     0.013 &    cc \\
             &                                    &C-C           &     1.507   &    -0.002 &     0.020 &    -0.012 &       \\
             &                                    &C-C=O       &     121.4   &       0.9 &      -0.1 &       0.9   &       \\
 C$_4$H$_4$O       & Furan                              &CO            &     1.362   &     0.016 &     0.005 &     0.033 &    dd \\
             &                                    &CCO         &     106.6   &       0.3 &       1.0 &       0.0   &       \\
             &                                    &C3C2          &     1.361   &     0.012 &     0.029 &     0.019 &       \\
             &                                    &CCC         &     110.7   &      -0.5 &      -0.4 &      -0.6   &       \\
\hline
\end{tabular}
\end{center}
\end{table}
\clearpage

\begin{table}
\caption{\label{geotabd}Comparison of Calculated and Observed Geometries for 
MNDO, AM1, and PM3 (contd.)}
\begin{center}
\compresstable
\begin{tabular}{llrrrrrr}
 Empirical  & Chemical Name &  Geometric &  Exp. & \multicolumn{3}{c}{Errors} & \\
  Formula   &               &  Variable &        & PM3  & MNDO  &  AM1 & Ref.\\
\hline
 O$_2$          & Oxygen, triplet state              &OO            &     1.216   &    -0.047 &    -0.082 &    -0.131 &     v \\
 H$_2$O$_2$        & Hydrogen peroxide                  &OO            &     1.475   &     0.006 &    -0.180 &    -0.177 &    ee \\
        &                                    &OH            &     0.950   &    -0.005 &     0.011 &     0.034 &       \\
        &                                    &HOO         &      94.8   &       1.8 &      12.5 &      11.5   &       \\
        &                                    &HOOH        &     119.8   &      60.4 &      36.8 &       1.7   &       \\
        &                                    &HOOH        &     119.8   &      60.4 &      36.8 &       1.7   &    ee \\
 CO$_2$         & Carbon dioxide                     &CO            &     1.162   &     0.019 &     0.024 &     0.027 &     b \\
 CH$_2$O$_2$       & Formic acid                        &C=O           &     1.202   &     0.009 &     0.025 &     0.028 &    ff \\
        &                                    &C-O           &     1.343   &     0.001 &     0.011 &     0.014 &       \\
        &                                    &OCO         &     124.9   &      -7.7 &      -4.3 &      -7.2   &       \\
        &                                    &OH            &     0.972   &    -0.019 &    -0.023 &    -0.001 &       \\
        &                                    &HOC         &     106.3   &       5.5 &      10.0 &       4.3   &       \\
        &                                    &CH            &     1.097   &    -0.002 &     0.008 &     0.006 &       \\
        &                                    &HC-O        &     124.1   &       6.3 &       2.6 &       6.0   &       \\
 C$_2$H$_2$O$_2$      & trans Glyoxal                      &CO            &     1.207   &     0.000 &     0.013 &     0.022 &    bb \\
        &                                    &CC            &     1.525   &     0.001 &     0.004 &    -0.016 &       \\
        &                                    &CCO         &     121.2   &      -0.7 &       0.8 &      -0.2   &       \\
 C$_2$H$_4$O$_2$      & Methyl formate                     &C=O           &     1.200   &     0.008 &     0.024 &     0.029 &    gg \\
        &                                    &C-O           &     1.334   &     0.022 &     0.023 &     0.028 &       \\
        &                                    &O-C=O       &     125.9   &      -5.3 &      -3.9 &      -6.7   &       \\
        &                                    &O-CH3         &     1.437   &    -0.024 &    -0.033 &    -0.009 &       \\
        &                                    &C-O-C       &     114.8   &       4.8 &      10.8 &       2.5   &       \\
 C$_6$H$_4$O$_2$      & p-Benzoquinone                     &C1C2          &     1.477   &     0.010 &     0.024 &     0.002 &    hh \\
        &                                    &C2C3          &     1.322   &     0.013 &     0.027 &     0.016 &       \\
        &                                    &CCC         &     121.1   &       0.5 &       1.0 &       0.8   &       \\
        &                                    &CO            &     1.222   &    -0.005 &     0.004 &     0.014 &       \\
 O$_3$          & Ozone                              &O-O           &     1.198   &     0.025 &    -0.007 &    -0.038 &     b \\
        &                                    &O-O-O       &     114.9   &      -0.8 &       2.7 &       6.0   &       \\
 H$_3$N         & Ammonia                            &NH            &     1.012   &    -0.013 &    -0.005 &    -0.014 &     a \\
        &                                    &HNH         &     106.7   &       1.4 &      -1.4 &       2.4   &       \\
 CN          &           Cyanide (+)              &CN            &     1.290   &     0.065 &    -0.149 &    -0.164 &     a \\
 CN          & Cyanide                            &CN            &     1.175   &    -0.018 &    -0.022 &    -0.027 &     a \\
 CHN         & Hydrogen cyanide                   &CN            &     1.154   &     0.002 &     0.006 &     0.006 &    ii \\
        &                                    &CH            &     1.063   &     0.007 &    -0.008 &     0.006 &       \\
 CH$_5$N        & Methylamine                        &CN            &     1.474   &    -0.005 &    -0.014 &    -0.041 &    jj \\
        &                                    &NH            &     1.011   &    -0.012 &    -0.003 &    -0.010 &       \\
        &                                    &HNC         &     112.0   &      -2.2 &      -2.2 &      -0.9   &       \\
        &                                    &HNH         &     105.9   &       2.8 &      -0.3 &       3.0   &       \\
 C$_2$H$_3$N       & Acetonitrile                       &CC            &     1.458   &    -0.018 &    -0.006 &    -0.019 &    kk \\
        &                                    &CH            &     1.104   &    -0.006 &     0.006 &     0.016 &       \\
        &                                    &HCC         &     109.5   &       0.9 &       1.1 &       0.6   &       \\
        &                                    &CN            &     1.157   &     0.002 &     0.005 &     0.006 &       \\
 C$_2$H$_3$N       & Methyl isocyanide                  &CN-           &     1.424   &     0.009 &     0.000 &    -0.029 &    kk \\
        &                                    &CH            &     1.101   &    -0.004 &     0.014 &     0.024 &       \\
        &                                    &HCN         &     109.1   &       0.6 &       1.1 &       1.0   &       \\
        &                                    &-CN           &     1.166   &     0.015 &     0.025 &     0.015 &       \\
\hline
\end{tabular}
\end{center}
\end{table}
\clearpage

\begin{table}
\caption{\label{geotabe}Comparison of Calculated and Observed Geometries for 
MNDO, AM1, and PM3 (contd.)}
\begin{center}
\compresstable
\begin{tabular}{llrrrrrr}
 Empirical  & Chemical Name &  Geometric &  Exp. & \multicolumn{3}{c}{Errors} & \\
  Formula   &               &  Variable &        & PM3  & MNDO  &  AM1 & Ref.\\
\hline
 C$_2$H$_7$N       & Dimethylamine                      &C-N           &     1.462   &     0.012 &     0.000 &    -0.024 &    ll \\
             &                                    &N-H           &     1.019   &    -0.021 &    -0.010 &    -0.016 &       \\
             &                                    &H-N-C       &     108.9   &       0.6 &       0.5 &       1.5   &       \\
             &                                    &C-N-C       &     112.2   &       1.1 &       5.3 &       2.1   &       \\
 C$_3$H$_9$N       & Trimethylamine                     &CN            &     1.451   &     0.029 &     0.014 &    -0.006 &    mm \\
             &                                    &CNC         &     110.9   &       1.3 &       5.2 &       2.1   &       \\
 C$_4$H$_5$N       & Pyrrole                            &CN            &     1.370   &     0.027 &     0.028 &     0.022 &     l \\
             &                                    &CNC         &     107.7   &       2.0 &       2.0 &       1.1   &       \\
             &                                    &C3C2          &     1.382   &     0.008 &     0.013 &     0.020 &       \\
             &                                    &CCC         &     109.8   &      -2.8 &      -2.4 &      -1.4   &       \\
             &                                    &C4C3          &     1.417   &    -0.027 &    -0.022 &    -0.015 &       \\
 NO          & Nitrogen oxide                     &NO            &     1.151   &    -0.024 &    -0.029 &    -0.037 &     a \\
 CHNO        & Hydrogen isocyanate                &NH            &     0.987   &    -0.001 &     0.014 &     0.000 &     a \\
             &                                    &CN            &     1.207   &     0.044 &     0.043 &     0.025 &       \\
             &                                    &CNH         &     128.1   &      -6.3 &     -10.3 &      -2.7   &       \\
             &                                    &CO            &     1.171   &     0.011 &     0.014 &     0.031 &       \\
             &                                    &OCN         &     180.0   &       0.0 &       0.0 &       0.0   &       \\
 CH$_3$NO       & Formamide                          &CN            &     1.376   &     0.015 &     0.013 &    -0.009 &    nn \\
             &                                    &NH            &     1.002   &    -0.012 &    -0.010 &    -0.016 &       \\
             &                                    &CH            &     1.102   &    -0.001 &     0.006 &     0.012 &       \\
             &                                    &CO            &     1.193   &     0.027 &     0.034 &     0.050 &       \\
             &                                    &OCN         &     123.8   &      -6.1 &      -2.9 &      -1.8   &       \\
 NO$_2$         & Nitrogen dioxide                   &NO            &     1.197   &    -0.016 &    -0.023 &    -0.038 &     a \\
             &                                    &ONO         &     136.0   &       1.7 &      -2.8 &       0.5   &       \\
 HNO$_2$        & Nitrous acid (cis)                 &N-O           &     1.460   &    -0.122 &    -0.163 &    -0.169 &     a \\
             &                                    &N=O           &     1.200   &    -0.025 &    -0.031 &    -0.038 &       \\
             &                                    &ONO         &     114.0   &      -0.7 &       3.1 &       2.6   &       \\
             &                                    &OH            &     0.980   &    -0.020 &    -0.017 &     0.003 &       \\
             &                                    &HON         &     103.0   &       6.8 &      16.7 &      12.5   &       \\
 HNO$_2$        & Nitrous acid (trans)               &N-O           &     1.460   &    -0.077 &    -0.148 &    -0.141 &     a \\
             &                                    &N=O           &     1.200   &    -0.033 &    -0.034 &    -0.042 &       \\
             &                                    &ONO         &     118.0   &      -8.9 &      -4.7 &      -5.2   &       \\
             &                                    &OH            &     0.980   &    -0.029 &    -0.022 &    -0.005 &       \\
             &                                    &HON         &     105.0   &      -0.4 &       5.0 &       2.0   &       \\
 C$_7$H$_7$NO$_2$     & Salicylaldoxime                    &N$\cdots$H    &     1.834   &     0.013 &     0.450 &     0.303 &    oo \\
             &                                    &O$\cdots$N    &     2.626   &     0.058 &     0.350 &     0.298 &       \\
 HNO$_3$        & Nitric acid                        &N=O           &     1.206   &    -0.003 &     0.005 &    -0.011 &     a \\
             &                                    &O=N=O       &     130.0   &       2.7 &      -3.5 &      -0.9   &       \\
             &                                    &N-O           &     1.405   &     0.005 &    -0.065 &    -0.072 &       \\
             &                                    &OH            &     0.960   &    -0.007 &     0.002 &     0.022 &       \\
             &                                    &NOH         &     102.0   &       7.0 &      12.0 &       7.7   &       \\
 N$_2$          & Nitrogen                           &NN            &     1.094   &     0.004 &     0.010 &     0.012 &     v \\
 H$_4$N$_2$        & Hydrazine                          &NN            &     1.449   &    -0.009 &    -0.052 &    -0.071 &     a \\
             &                                    &NH            &     1.022   &    -0.021 &    -0.001 &    -0.008 &       \\
             &                                    &HNN         &     112.0   &      -5.5 &      -4.8 &      -4.4   &       \\
             &                                    &HNNH        &      90.0   &      90.8 &      90.3 &      90.2   &       \\
\hline
\end{tabular}
\end{center}
\end{table}
\clearpage

\begin{table}
\caption{\label{geotabf}Comparison of Calculated and Observed Geometries for 
MNDO, AM1, and PM3 (contd.)}
\begin{center}
\compresstable
\begin{tabular}{llrrrrrr}
 Empirical  & Chemical Name &  Geometric &  Exp. & \multicolumn{3}{c}{Errors} & \\
  Formula   &               &  Variable &        & PM3  & MNDO  &  AM1 & Ref.\\
\hline
 C$_2$N$_2$        & Cyanogen                           &CN            &     1.154   &     0.005 &     0.008 &     0.008 &     b \\
             &                                    &CC            &     1.389   &    -0.007 &    -0.011 &    -0.005 &       \\
 C$_2$H$_6$N$_2$      & Dimethyldiazene                    &NN            &     1.254   &    -0.026 &    -0.032 &    -0.030 &     b \\
             &                                    &CN            &     1.474   &    -0.007 &     0.000 &    -0.022 &       \\
             &                                    &CNN         &     111.9   &       7.4 &       5.0 &       7.9   &       \\
 C$_4$H$_4$N$_2$      & Pyrimidine                         &C4C5          &     1.393   &     0.003 &     0.016 &     0.015 &    pp \\
             &                                    &N3C4          &     1.350   &     0.004 &     0.003 &     0.000 &       \\
             &                                    &N3C4C5      &     121.2   &      -0.5 &       0.1 &       1.1   &       \\
             &                                    &C2N3          &     1.328   &     0.029 &     0.029 &     0.033 &       \\
             &                                    &C2H           &     1.082   &     0.016 &     0.016 &     0.028 &       \\
             &                                    &C5H           &     1.087   &     0.007 &     0.000 &     0.008 &       \\
             &                                    &C6H           &     1.079   &     0.017 &     0.016 &     0.026 &       \\
 N$_2$O         & Nitrous oxide                      &NN            &     1.128   &    -0.004 &     0.000 &     0.000 &     a \\
             &                                    &NO            &     1.184   &     0.013 &    -0.003 &    -0.009 &       \\
 H$_2$N$_2$O$_2$      & NH2-NO2                            &NN            &     1.427   &     0.004 &    -0.020 &    -0.060 &    qq \\
             &                                    &NO            &     1.206   &     0.005 &     0.003 &    -0.002 &       \\
             &                                    &NH            &     1.005   &    -0.005 &     0.013 &    -0.003 &       \\
             &                                    &ONO         &     130.1   &      -3.0 &      -6.0 &      -6.1   &       \\
 N$_2$O$_3$        & Dinitrogen trioxide                &NN            &     2.080   &    -0.383 &    -0.550 &    -0.498 &     a \\
             &                                    &NO            &     1.120   &     0.031 &     0.018 &     0.008 &       \\
             &                                    &NNO         &     110.0   &       6.7 &       6.1 &       6.2   &       \\
             &                                    &NO'           &     1.180   &     0.022 &     0.028 &     0.019 &       \\
             &                                    &O'NO'       &     134.0   &      -4.6 &      -9.1 &      -9.2   &       \\
 N$_2$O$_4$        & Dinitrogen tetroxide               &NN            &     1.750   &     0.009 &    -0.135 &     0.068 &     a \\
             &                                    &NO            &     1.180   &     0.015 &     0.009 &    -0.008 &       \\
             &                                    &ONN         &     113.1   &       1.5 &       2.2 &       1.1   &       \\
 N$_3$          & Azide                              &NN            &     1.181   &    -0.008 &    -0.007 &    -0.004 &     a \\
 C$_3$H$_3$N$_3$      & s-Triazine                         &CN            &     1.338   &     0.020 &     0.019 &     0.026 &    rr \\
             &                                    &NCN         &     126.8   &      -5.2 &      -3.4 &      -1.1   &       \\
 CHN$_3$O$_6$      & Trinitromethane                    &C-N           &     1.505   &     0.078 &     0.053 &     0.055 &    ss \\
             &                                    &N-C-N       &     110.7   &      -4.0 &      -1.1 &      -3.9   &       \\
             &                                    &N=O           &     1.219   &    -0.018 &    -0.014 &    -0.027 &       \\
 H$_2$S         & Hydrogen sulfide                   &HS            &     1.328   &    -0.038 &    -0.029 &    -0.005 &     a \\
             &                                    &HSH         &      92.2   &       1.3 &       5.8 &       3.3   &       \\
 CS          & Carbon sulfide                     &CS            &     1.534   &    -0.087 &    -0.050 &    -0.105 &    tt \\
 CH$_2$S        & Thioformaldehyde                   &CS            &     1.611   &    -0.072 &    -0.074 &    -0.099 &    nn \\
             &                                    &CH            &     1.093   &     0.002 &     0.000 &     0.013 &       \\
             &                                    &HCS         &     121.6   &       4.5 &       2.4 &       3.7   &       \\
 CH$_4$S        & Thiomethanol                       &CS            &     1.818   &    -0.018 &    -0.101 &    -0.064 &    uu \\
             &                                    &SH            &     1.329   &    -0.022 &    -0.027 &    -0.008 &       \\
             &                                    &HSC         &     100.3   &      -0.4 &       2.0 &      -0.8   &       \\
             &                                    &HCSH        &     180.0   &       0.0 &       0.0 &       0.0   &       \\
 C$_2$H$_6$S       & Dimethyl thioether                 &CS            &     1.802   &     0.000 &    -0.079 &    -0.050 &    ss \\
             &                                    &CSC         &      98.9   &       3.2 &       9.4 &       3.0   &       \\
             &                                    &HCS         &     106.6   &       3.8 &       3.1 &       2.6   &       \\
             &                                    &H'CS        &     106.6   &       0.0 &       0.0 &       0.0   &       \\
\hline
\end{tabular}
\end{center}
\end{table}
\clearpage

\begin{table}
\caption{\label{geotabg}Comparison of Calculated and Observed Geometries for 
MNDO, AM1, and PM3 (contd.)}
\begin{center}
\compresstable
\begin{tabular}{llrrrrrr}
 Empirical  & Chemical Name &  Geometric &  Exp. & \multicolumn{3}{c}{Errors} & \\
  Formula   &               &  Variable &        & PM3  & MNDO  &  AM1 & Ref.\\
\hline
 C$_4$H$_4$S       & Thiophene                          &CS            &     1.714   &     0.011 &    -0.035 &    -0.042 &    vv \\
             &                                    &CCS         &      92.2   &      -0.8 &       1.4 &       1.6   &       \\
             &                                    &C3C2          &     1.370   &    -0.004 &     0.005 &     0.007 &       \\
             &                                    &CCC         &     111.5   &       0.6 &       0.4 &       0.1   &       \\
 CSO         & Carbon oxysulfide                  &CO            &     1.159   &     0.017 &     0.022 &     0.042 &    ww \\
             &                                    &CS            &     1.559   &    -0.055 &    -0.049 &    -0.101 &       \\
 SO$_2$         & Sulfur dioxide                     &SO            &     1.432   &     0.010 &     0.044 &    -0.003 &     a \\
             &                                    &OSO         &     119.5   &     -13.4 &     -12.7 &     -11.6   &       \\
 SO$_3$         & Sulfur trioxide                    &SO            &     1.430   &    -0.046 &     0.061 &    -0.080 &     a \\
 H$_2$SO$_4$       & Sulfuric acid                      &S-O           &     1.550   &     0.118 &     0.079 &     0.072 &     a \\
             &                                    &OH            &     0.970   &    -0.023 &    -0.023 &    -0.015 &       \\
             &                                    &SOH         &     105.0   &      12.7 &      12.2 &      15.9   &       \\
             &                                    &S=O           &     1.420   &    -0.013 &     0.095 &    -0.062 &       \\
 NS          & Sulfur nitride                     &SN            &     1.495   &    -0.044 &    -0.055 &    -0.105 &     a \\
 C$_2$H$_3$NS      & Methyl isothiocyanate              &CS            &     1.597   &    -0.099 &    -0.087 &    -0.125 &    xx \\
             &                                    &C=N           &     1.192   &     0.039 &     0.026 &     0.029 &       \\
             &                                    &C-N           &     1.479   &    -0.039 &    -0.043 &    -0.075 &       \\
             &                                    &C-N=C       &     141.6   &      -2.0 &       0.4 &      -0.4   &       \\
 C$_2$N$_2$S       & Sulfur dicyanide                   &CN            &     1.157   &     0.007 &     0.007 &     0.012 &    yy \\
             &                                    &CS            &     1.701   &    -0.037 &    -0.071 &    -0.083 &       \\
             &                                    &NCS         &     170.0   &       6.4 &       6.9 &       6.5   &       \\
             &                                    &CSC         &      98.4   &       3.2 &       5.1 &       4.8   &       \\
 S$_2$          & Sulfur dimer                       &SS            &     1.889   &    -0.032 &    -0.114 &    -0.053 &     a \\
 H$_2$S$_2$        & H2S2                               &SS            &     2.055   &    -0.022 &    -0.132 &     0.052 &    zz \\
             &                                    &SH            &     1.327   &    -0.017 &    -0.023 &    -0.001 &       \\
             &                                    &HSS         &      91.3   &      12.0 &      11.2 &       7.4   &       \\
             &                                    &HSSH        &      90.5   &       1.9 &       9.0 &       8.9   &       \\
 CS$_2$         & Carbon disulfide                   &CS            &     1.553   &    -0.072 &    -0.061 &    -0.094 &    ss \\
 C$_2$H$_6$S$_2$      & 2,3-Dithiabutane                   &CS            &     1.810   &    -0.006 &    -0.085 &    -0.060 &   aaa \\
             &                                    &SS            &     2.038   &    -0.001 &    -0.102 &     0.072 &       \\
             &                                    &CSS         &     102.8   &       6.4 &       5.2 &       3.4   &       \\
             &                                    &CSSC        &      84.7   &       5.5 &      23.1 &       9.8   &       \\
             &                                    &CS            &     1.810   &    -0.006 &    -0.085 &    -0.060 &    ss \\
             &                                    &SS            &     2.038   &    -0.001 &    -0.102 &     0.072 &       \\
             &                                    &SSC         &     102.8   &       5.8 &       5.2 &       3.2   &       \\
             &                                    &CSSC        &      84.7   &       5.5 &      23.1 &       9.8   &       \\
 S$_6$          & S6                                 &SS            &     2.057   &    -0.009 &    -0.110 &     0.052 &   bbb \\
             &                                    &SSS         &     102.2   &       5.3 &       2.9 &       0.3   &       \\
             &                                    &SSSS        &      74.5   &      -9.9 &      -5.1 &      -0.6   &       \\
 S$_8$          & S8                                 &SS            &     2.048   &    -0.075 &    -0.113 &    -0.042 &   ccc \\
             &                                    &SSS         &     107.9   &       8.1 &       0.4 &       3.7   &       \\
             &                                    &SSSS        &      98.6   &     -11.0 &      -0.2 &      -4.4   &       \\
 HF          & Hydrogen fluoride                  &HF            &     0.917   &     0.021 &     0.039 &    -0.091 &   ddd \\
 CF          & Fluoromethylidyne                  &CF            &     1.266   &    -0.008 &    -0.003 &    -0.008 &     a \\
 CHF         & Fluoromethylene                    &CH            &     1.121   &    -0.021 &    -0.001 &     0.006 &     a \\
             &                                    &CF            &     1.314   &    -0.030 &    -0.029 &    -0.023 &       \\
             &                                    &FCH         &     101.6   &       4.0 &       9.5 &       9.0   &       \\
\hline
\end{tabular}
\end{center}
\end{table}
\clearpage

\begin{table}
\caption{\label{geotabh}Comparison of Calculated and Observed Geometries for 
MNDO, AM1, and PM3 (contd.)}
\begin{center}
\compresstable
\begin{tabular}{llrrrrrr}
 Empirical  & Chemical Name &  Geometric &  Exp. & \multicolumn{3}{c}{Errors} & \\
  Formula   &               &  Variable &        & PM3  & MNDO  &  AM1 & Ref.\\
\hline
 CH$_3$F        & Fluoromethane                      &CH            &     1.098   &    -0.006 &     0.019 &     0.023 &   eee \\
             &                                    &CF            &     1.382   &    -0.032 &    -0.033 &    -0.007 &       \\
             &                                    &FCH         &     108.5   &       1.2 &       0.8 &       1.0   &       \\
 C$_2$HF        & Fluoroacetylene                    &H-C           &     1.053   &     0.012 &    -0.004 &     0.006 &   fff \\
             &                                    &C-C           &     1.198   &    -0.010 &    -0.005 &    -0.004 &       \\
             &                                    &C-F           &     1.279   &     0.020 &    -0.002 &     0.018 &       \\
 C$_2$H$_3$F       & Fluoroethylene                     &CC            &     1.333   &     0.000 &     0.018 &     0.007 &   ggg \\
             &                                    &CH(g)         &     1.076   &     0.017 &     0.023 &     0.028 &       \\
             &                                    &CCH(g)      &     127.7   &      -1.6 &      -4.7 &      -3.9   &       \\
             &                                    &CH(t)         &     1.085   &     0.000 &     0.002 &     0.011 &       \\
             &                                    &CCH(t)      &     123.9   &      -2.6 &      -3.0 &      -3.0   &       \\
             &                                    &CH(c)         &     1.090   &    -0.004 &    -0.003 &     0.006 &       \\
             &                                    &CCH(c)      &     121.4   &       2.3 &       3.1 &       1.5   &       \\
             &                                    &CF            &     1.348   &    -0.010 &    -0.024 &     0.003 &       \\
             &                                    &FCC         &     121.0   &       1.1 &       2.3 &       2.2   &       \\
 C$_3$H$_3$F       & Fluoroallene                       &C1C2          &     1.301   &     0.009 &     0.019 &     0.012 &   hhh \\
             &                                    &C1H           &     1.083   &     0.011 &     0.016 &     0.022 &       \\
             &                                    &HC1C2       &     124.3   &       0.5 &      -1.4 &      -1.4   &       \\
             &                                    &C1F           &     1.360   &    -0.020 &    -0.035 &    -0.007 &       \\
             &                                    &FC1C2       &     121.9   &       0.4 &       1.0 &       1.8   &       \\
             &                                    &C2C3          &     1.309   &    -0.015 &    -0.006 &    -0.014 &       \\
             &                                    &C3H           &     1.086   &     0.001 &     0.005 &     0.015 &       \\
             &                                    &HC3C2       &     120.8   &       1.5 &       2.1 &       1.5   &       \\
 CNF         & Cyanogen fluoride                  &CN            &     1.159   &     0.000 &     0.001 &     0.006 &   fff \\
             &                                    &CF            &     1.262   &     0.035 &     0.011 &     0.045 &       \\
 NOF         & Nitrosyl fluoride                  &NF            &     1.520   &    -0.153 &    -0.215 &    -0.153 &     a \\
             &                                    &NO            &     1.130   &     0.032 &     0.031 &     0.018 &       \\
             &                                    &FNO         &     110.2   &       1.4 &       3.6 &       2.1   &       \\
 CH$_3$SO$_3$F     & Methyl fluorosulfate               &S=O           &     1.410   &    -0.005 &     0.101 &    -0.051 &   iii \\
             &                                    &O=S=O       &     124.4   &      -0.6 &      -2.0 &      -0.5   &       \\
             &                                    &S-F           &     1.545   &    -0.009 &     0.064 &    -0.038 &       \\
             &                                    &F-S=O       &     106.8   &       1.6 &       0.4 &       2.3   &       \\
             &                                    &S-O           &     1.558   &     0.125 &     0.075 &     0.084 &       \\
             &                                    &O-S=O       &     109.5   &       0.5 &       0.3 &      -1.6   &       \\
             &                                    &C-O           &     1.420   &    -0.019 &    -0.017 &    -0.012 &       \\
             &                                    &C-O-S       &     116.5   &       7.1 &       9.5 &       7.1   &       \\
 F$_2$          & Fluorine                           &FF            &     1.412   &    -0.062 &    -0.146 &     0.015 &     a \\
 H$_2$F$_2$        & Hydrogen fluoride dimer            &HF            &     0.920   &     0.019 &     0.036 &    -0.093 &     a \\
             &                                    &H'F           &     1.870   &    -0.126 &     1.199 &     0.430 &       \\
             &                                    &H'FH        &     108.0   &      39.3 &      36.1 &     -18.3   &       \\
 CF$_2$         & Difluoromethylene                  &CF            &     1.300   &    -0.002 &     0.004 &     0.012 &     a \\
             &                                    &FCF         &     104.9   &       1.4 &       3.4 &       1.1   &       \\
 OF$_2$         & F2O                                &OF            &     1.412   &    -0.034 &    -0.131 &    -0.058 &     a \\
             &                                    &FOF         &     103.2   &      -2.2 &       5.9 &      -0.7   &       \\
 COF$_2$        & Carbonyl difluoride                &CO            &     1.174   &     0.025 &     0.045 &     0.046 &     a \\
             &                                    &CF            &     1.312   &     0.010 &     0.004 &     0.016 &       \\
             &                                    &FCO         &     126.0   &      -1.4 &      -1.9 &      -1.5   &       \\
\hline
\end{tabular}
\end{center}
\end{table}
\clearpage

\begin{table}
\caption{\label{geotabi}Comparison of Calculated and Observed Geometries for 
MNDO, AM1, and PM3 (contd.)}
\begin{center}
\compresstable
\begin{tabular}{llrrrrrr}
 Empirical  & Chemical Name &  Geometric &  Exp. & \multicolumn{3}{c}{Errors} & \\
  Formula   &               &  Variable &        & PM3  & MNDO  &  AM1 & Ref.\\
\hline
 SF$_2$         & Sulfur difluoride                  &SF            &     1.592   &    -0.032 &    -0.020 &    -0.036 &     a \\
             &                                    &FSF         &      98.2   &      -1.7 &       1.4 &       0.8   &       \\
 CSF$_2$        & Thiocarbonyl difluoride            &CS            &     1.589   &     0.011 &    -0.014 &    -0.029 &    ss \\
             &                                    &CF            &     1.315   &     0.023 &     0.006 &     0.031 &       \\
             &                                    &FCS         &     126.4   &       2.5 &      -0.4 &       2.4   &       \\
 SOF$_2$        & Thionyl fluoride                   &SO            &     1.412   &     0.055 &     0.068 &     0.021 &     a \\
             &                                    &SF            &     1.585   &    -0.010 &     0.023 &    -0.037 &       \\
             &                                    &FSO         &     106.8   &      -5.8 &      -4.0 &      -4.2   &       \\
             &                                    &FSF         &      92.8   &       1.0 &       4.2 &       3.7   &       \\
 SO$_2$F$_2$       & Sulfuryl fluoride                  &SF            &     1.530   &     0.017 &     0.080 &    -0.015 &     a \\
             &                                    &FSF         &      96.1   &      -0.8 &       1.6 &       1.2   &       \\
             &                                    &SO            &     1.405   &    -0.005 &     0.103 &    -0.046 &       \\
             &                                    &OSO         &     124.0   &       2.5 &       0.9 &       0.9   &       \\
 S$_2$F$_2$        & FSSF                               &SF            &     1.635   &    -0.051 &    -0.063 &    -0.071 &     a \\
             &                                    &SS            &     1.888   &     0.121 &     0.077 &     0.243 &       \\
             &                                    &FSS         &     108.3   &       3.7 &      -1.8 &       0.8   &       \\
             &                                    &FSSF        &      87.9   &      -0.2 &      -0.4 &       1.7   &       \\
 S$_2$F$_2$        & SSF2                               &SS            &     1.860   &     0.059 &     0.034 &    -0.027 &     a \\
             &                                    &SF            &     1.598   &    -0.006 &     0.003 &    -0.047 &       \\
             &                                    &FSS         &     107.5   &       7.0 &       1.4 &      12.4   &       \\
             &                                    &FSF         &      92.5   &      -1.6 &       4.0 &       1.0   &       \\
 CHF$_3$        & Trifluoromethane                   &CH            &     1.098   &     0.013 &     0.038 &     0.032 &     a \\
             &                                    &CF            &     1.333   &     0.013 &     0.020 &     0.035 &       \\
             &                                    &FCH         &     110.3   &       2.8 &       1.2 &       2.7   &       \\
 NF$_3$         & Nitrogen trifluoride               &NF            &     1.371   &    -0.017 &    -0.056 &    -0.011 &     a \\
             &                                    &FNF         &     102.2   &       2.8 &       4.0 &       0.4   &       \\
 C$_2$NF$_3$       & Trifluoroacetonitrile              &CC            &     1.461   &     0.026 &     0.037 &     0.025 &     a \\
             &                                    &CF            &     1.335   &     0.015 &     0.020 &     0.036 &       \\
             &                                    &CCF         &     111.4   &       2.0 &       0.5 &       2.1   &       \\
             &                                    &CN            &     1.153   &     0.002 &     0.006 &     0.006 &       \\
 CF$_4$         & Carbon tetrafluoride               &CF            &     1.321   &     0.016 &     0.026 &     0.037 &   jjj \\
 C$_2$F$_4$        & Tetrafluoroethylene                &CC            &     1.311   &     0.044 &     0.070 &     0.057 &   ggg \\
             &                                    &CF            &     1.319   &     0.007 &    -0.001 &     0.021 &       \\
             &                                    &FCC         &     123.8   &       1.3 &       0.4 &       1.6   &       \\
 SF$_4$         & Sulfur tetrafluoride               &SF            &     1.545   &     0.051 &     0.052 &     0.001 &     a \\
             &                                    &FSF         &     101.5   &      19.7 &       0.6 &       2.4   &       \\
             &                                    &SF'           &     1.646   &    -0.024 &     0.025 &    -0.073 &       \\
             &                                    &FSF'        &      87.8   &      -6.9 &       0.6 &      -0.8   &       \\
 CNSOF$_5$      & Pentafluoro(isocyanato)sulfur      &S-F           &     1.567   &    -0.004 &     0.089 &    -0.007 &   iii \\
             &                                    &S-N           &     1.668   &     0.085 &     0.061 &     0.000 &       \\
             &                                    &N=C           &     1.234   &     0.011 &     0.026 &    -0.022 &       \\
             &                                    &S-N=C       &     124.9   &      27.9 &      10.7 &      55.1   &       \\
             &                                    &C=O           &     1.179   &    -0.005 &    -0.006 &     0.015 &       \\
             &                                    &N=C=O       &     173.8   &       0.6 &      -3.9 &       6.1   &       \\
 C$_2$F$_6$        & C2F6                               &C-C           &     1.545   &     0.062 &     0.129 &     0.077 &   kkk \\
             &                                    &C-F           &     1.314   &     0.027 &     0.032 &     0.049 &       \\
             &                                    &F-C-C       &     109.8   &       1.7 &       0.9 &       2.3   &       \\
\hline
\end{tabular}
\end{center}
\end{table}
\clearpage

\begin{table}
\caption{\label{geotabj}Comparison of Calculated and Observed Geometries for 
MNDO, AM1, and PM3 (contd.)}
\begin{center}
\compresstable
\begin{tabular}{llrrrrrr}
 Empirical  & Chemical Name &  Geometric &  Exp. & \multicolumn{3}{c}{Errors} & \\
  Formula   &               &  Variable &        & PM3  & MNDO  &  AM1 & Ref.\\
\hline
 SF$_6$         & Sulfur hexafluoride                &SF            &     1.564   &    -0.003 &     0.091 &    -0.024 &     a \\
 C$_3$F$_8$        & Perfluoropropane                   &C-C           &     1.551   &     0.050 &     0.121 &     0.063 &   kkk \\
             &                                    &C-C-C       &     115.7   &      -4.9 &      -1.7 &      -5.6   &       \\
             &                                    &C-F           &     1.314   &     0.029 &     0.033 &     0.050 &       \\
             &                                    &F-C-C       &     110.7   &       1.2 &       0.6 &       1.8   &       \\
 C$_3$O$_2$F$_8$      & CF3-O-CF2-O-CF3                    &C1-O          &     1.358   &     0.035 &     0.049 &     0.049 &   kkk \\
             &                                    &O-C2          &     1.357   &     0.036 &     0.046 &     0.043 &       \\
             &                                    &C1-O-C2     &     123.3   &      -4.4 &       7.6 &      -1.5   &       \\
             &                                    &C1-F          &     1.305   &     0.027 &     0.038 &     0.047 &       \\
 C$_4$F$_1$$_0$       & Perfluoro-n-butane                 &C1-C2         &     1.554   &     0.053 &     0.123 &     0.063 &   kkk \\
             &                                    &C2-C3         &     1.554   &     0.039 &     0.112 &     0.051 &       \\
             &                                    &C-C-C       &     115.0   &      -3.9 &      -1.1 &      -5.1   &       \\
             &                                    &C1-F          &     1.314   &     0.028 &     0.033 &     0.050 &       \\
             &                                    &F-C1-C2     &     108.5   &       2.5 &       1.6 &       3.2   &       \\
 S$_2$F$_1$$_0$       & S2F10                              &S-S           &     2.210   &     0.680 &     3.395 &     1.819 &     a \\
             &                                    &S-F           &     1.560   &     0.021 &     2.399 &    -0.006 &       \\
 C$_5$O$_2$F$_1$$_2$     & CF3-O-CF2-O-CF(CF3)-O-CF3          &C1-O          &     1.355   &     0.042 &     0.057 &     0.057 &   kkk \\
             &                                    &C1-F          &     1.305   &     0.026 &     0.036 &     0.046 &       \\
             &                                    &F-C1-O      &     107.0   &      -4.5 &      -1.1 &      -3.2   &       \\
 HCl         & Hydrogen chloride                  &HCl           &     1.275   &    -0.007 &     0.073 &     0.009 &     a \\
 CHCl        & Chloromethylene                    &CH            &     1.120   &    -0.020 &    -0.020 &    -0.010 &     a \\
             &                                    &CCl           &     1.689   &    -0.135 &     0.050 &    -0.042 &       \\
             &                                    &ClCH        &     103.4   &      12.1 &       6.0 &       7.7   &       \\
 CH$_3$Cl       & Chloromethane                      &CCl           &     1.781   &    -0.017 &     0.014 &    -0.040 &     a \\
             &                                    &CH            &     1.096   &    -0.002 &     0.006 &     0.016 &       \\
             &                                    &HCCl        &     110.9   &      -1.0 &      -2.8 &      -2.6   &       \\
 C$_2$HCl       & Chloroacetylene                    &H-C           &     1.055   &     0.009 &    -0.004 &     0.005 &   lll \\
             &                                    &C-C           &     1.203   &    -0.009 &    -0.010 &    -0.007 &       \\
             &                                    &C-Cl          &     1.637   &    -0.065 &     0.024 &    -0.039 &       \\
 OCl         & Chlorine monoxide                  &ClO           &     1.546   &     0.002 &     0.073 &     0.089 &     a \\
 NOCl        & NOCl                               &ClN           &     1.950   &    -0.186 &    -0.167 &    -0.219 &     a \\
             &                                    &NO            &     1.170   &    -0.014 &    -0.033 &    -0.033 &       \\
             &                                    &CNCl        &     114.0   &       4.9 &       3.5 &       5.3   &       \\
 NO$_2$Cl       & NO2Cl                              &ClN           &     1.830   &    -0.012 &    -0.020 &    -0.060 &     a \\
             &                                    &NOCl          &     1.210   &    -0.013 &    -0.019 &    -0.024 &       \\
 FCl         & Chlorine fluoride                  &ClF           &     1.628   &    -0.045 &     0.022 &     0.019 &     a \\
 O$_3$FCl       & ClO3F                              &ClF           &     1.630   &     0.060 &     0.105 &     0.051 &     a \\
             &                                    &ClO           &     1.460   &    -0.007 &     0.270 &     0.328 &       \\
             &                                    &OClF        &      95.2   &       7.6 &      11.0 &       3.0   &       \\
 CHF$_2$Cl      & Chlorodifluoromethane              &CH            &     1.090   &     0.018 &     0.036 &     0.037 &   mmm \\
             &                                    &CCl           &     1.740   &     0.082 &     0.100 &     0.070 &       \\
             &                                    &ClCH        &     107.0   &       2.9 &      -2.1 &      -1.8   &       \\
             &                                    &CF            &     1.350   &    -0.004 &    -0.009 &     0.019 &       \\
             &                                    &FCCl        &     110.5   &      -0.3 &      -0.3 &       1.9   &       \\
             &                                    &FCClH       &     120.0   &       2.5 &       1.0 &       2.0   &       \\
 F$_3$Cl        & Chlorine trifluoride C2v           &ClF           &     1.598   &     0.073 &     0.101 &     0.085 &     a \\
             &                                    &ClF'          &     1.698   &    -0.027 &     0.001 &    -0.015 &       \\
             &                                    &FClF'       &      87.5   &      32.5 &      32.5 &      32.5   &       \\
\hline
\end{tabular}
\end{center}
\end{table}
\clearpage

\begin{table}
\caption{\label{geotabk}Comparison of Calculated and Observed Geometries for 
MNDO, AM1, and PM3 (contd.)}
\begin{center}
\compresstable
\begin{tabular}{llrrrrrr}
 Empirical  & Chemical Name &  Geometric &  Exp. & \multicolumn{3}{c}{Errors} & \\
  Formula   &               &  Variable &        & PM3  & MNDO  &  AM1 & Ref.\\
\hline
 Cl$_2$         & Chlorine                           &ClCl          &     1.986   &     0.049 &     0.010 &    -0.068 &     a \\
 CH$_2$Cl$_2$      & Dichloromethane                    &CCl           &     1.772   &    -0.014 &     0.014 &    -0.031 &   jjj \\
             &                                    &ClCCl       &     111.8   &      -3.9 &      -0.6 &       1.1   &       \\
             &                                    &CH            &     1.103   &    -0.001 &     0.000 &     0.010 &       \\
 OCl$_2$        & Cl2O                               &ClO           &     1.701   &    -0.001 &    -0.018 &     0.032 &     a \\
             &                                    &ClOCl       &     110.8   &      -1.6 &       2.2 &       0.3   &       \\
 COCl$_2$       & Carbonyl chloride                  &CO            &     1.166   &     0.032 &     0.034 &     0.056 &     a \\
             &                                    &CCl           &     1.746   &    -0.009 &     0.014 &    -0.027 &       \\
             &                                    &ClCO        &     124.3   &      -0.1 &      -0.4 &      -1.0   &       \\
 SCl$_2$        & Sulfur dichloride                  &SCl           &     2.015   &     0.016 &    -0.044 &    -0.057 &     a \\
             &                                    &ClSCl       &     102.7   &      -1.1 &       3.6 &       3.6   &       \\
 SOCl$_2$       & Thionyl chloride                   &SO            &     1.443   &     0.035 &     0.025 &     0.007 &   nnn \\
             &                                    &SCl           &     2.076   &     0.005 &    -0.039 &    -0.069 &       \\
             &                                    &ClSO        &     106.3   &      -1.6 &       0.4 &       1.8   &       \\
 SO$_2$Cl$_2$      & Sulfuryl chloride                  &S=O           &     1.418   &    -0.017 &     0.089 &    -0.024 &   iii \\
             &                                    &S-Cl          &     2.012   &     0.063 &     0.044 &     0.032 &       \\
             &                                    &O=S-Cl      &     108.0   &       0.7 &       1.3 &       1.3   &       \\
 S$_2$Cl$_2$       & ClSSCl                             &SCl           &     2.057   &    -0.014 &    -0.081 &    -0.098 &     a \\
             &                                    &SS            &     1.931   &     0.033 &    -0.011 &    -0.014 &       \\
             &                                    &ClSSCl      &     108.2   &       5.0 &       0.0 &       8.9   &       \\
 CF$_2$Cl$_2$      & Dichlorodifluoromethane            &CCl           &     1.770   &     0.038 &     0.055 &     0.037 &     a \\
             &                                    &ClCCl       &     108.5   &      -1.9 &      -1.7 &      -1.8   &       \\
             &                                    &CF            &     1.330   &     0.015 &     0.007 &     0.040 &       \\
             &                                    &FCCl        &     109.8   &       1.6 &       0.8 &       2.1   &       \\
 CHCl$_3$       & Chloroform                         &CCl           &     1.782   &    -0.029 &     0.000 &    -0.034 &   jjj \\
             &                                    &ClCH        &     107.5   &       2.9 &       1.1 &       0.2   &       \\
 CFCl$_3$       & Trichlorofluoromethane             &CF            &     1.330   &     0.019 &    -0.003 &     0.046 &     a \\
             &                                    &CCl           &     1.760   &     0.019 &     0.046 &     0.026 &       \\
 CCl$_4$        & Carbon tetrachloride               &CCl           &     1.760   &    -0.013 &     0.022 &     0.000 &   jjj \\
 C$_2$Cl$_6$       & Hexachloroethane                   &CC            &     1.550   &    -0.038 &     0.016 &     0.007 &     a \\
             &                                    &CCl           &     1.740   &     0.014 &     0.050 &     0.020 &       \\
             &                                    &ClCC        &     109.0   &       1.2 &       2.4 &       0.7   &       \\
 SCl$_6$        & Sulfur hexachloride                &SCl           &     2.030   &     0.110 &     0.112 &     0.118 &   ooo \\
 HBr         & Hydrogen bromide                   &HBr           &     1.415   &     0.056 &     0.025 &     0.006 &   ppp \\
 CH$_3$Br       & Bromomethane                       &CBr           &     1.933   &     0.018 &    -0.055 &    -0.028 &    ss \\
             &                                    &CH            &     1.086   &     0.004 &     0.016 &     0.024 &       \\
             &                                    &HCBr        &     107.7   &       0.7 &       0.8 &       1.1   &       \\
 C$_2$HBr       & Bromoacetylene                     &H-C           &     1.055   &     0.010 &    -0.004 &     0.006 &   qqq \\
             &                                    &C-C           &     1.204   &    -0.016 &    -0.009 &    -0.005 &       \\
             &                                    &C-Br          &     1.792   &    -0.007 &    -0.051 &    -0.049 &       \\
 C$_2$HOBr      & Bromoketene                        &C-Br          &     1.880   &    -0.003 &    -0.063 &    -0.042 &   rrr \\
             &                                    &C-H           &     1.082   &     0.005 &     0.001 &     0.015 &       \\
             &                                    &H-C-Br      &     124.4   &      -6.9 &      -8.9 &      -8.6   &       \\
             &                                    &C=C           &     1.316   &    -0.005 &     0.006 &    -0.003 &       \\
             &                                    &Br-C=C      &     118.4   &      -2.5 &       3.0 &       5.4   &       \\
             &                                    &C=O           &     1.161   &     0.011 &     0.020 &     0.027 &       \\
\hline
\end{tabular}
\end{center}
\end{table}
\clearpage

\begin{table}
\caption{\label{geotabl}Comparison of Calculated and Observed Geometries for 
MNDO, AM1, and PM3 (contd.)}
\begin{center}
\compresstable
\begin{tabular}{llrrrrrr}
 Empirical  & Chemical Name &  Geometric &  Exp. & \multicolumn{3}{c}{Errors} & \\
  Formula   &               &  Variable &        & PM3  & MNDO  &  AM1 & Ref.\\
\hline
 C$_2$H$_3$OBr     & Acetyl bromide                     &CC            &     1.516   &    -0.039 &    -0.001 &    -0.026 &    ss \\
             &                                    &CBr           &     1.973   &    -0.007 &    -0.087 &    -0.026 &       \\
             &                                    &BrCC        &     111.0   &      -4.7 &       2.0 &       2.3   &       \\
             &                                    &CO            &     1.183   &     0.002 &     0.026 &     0.042 &       \\
             &                                    &CCO         &     127.1   &       7.2 &       0.2 &      -3.0   &       \\
 CNBr        & Cyanogen bromide                   &BrC           &     1.789   &     0.007 &    -0.046 &    -0.028 &     a \\
             &                                    &CN            &     1.158   &    -0.003 &     0.003 &     0.006 &       \\
 NOBr        & BrNO                               &BrN           &     2.140   &    -0.252 &    -0.271 &    -0.218 &    ss \\
             &                                    &NO            &     1.146   &     0.001 &    -0.007 &    -0.011 &       \\
             &                                    &BrNO        &     114.5   &       6.3 &       4.4 &       8.0   &       \\
 FBr         & BrF                                &BrF           &     1.756   &     0.018 &    -0.029 &     0.021 &     a \\
 F$_3$Br        & Bromine trifluoride                &BrF           &     1.721   &     0.065 &     0.036 &     0.087 &     a \\
             &                                    &BrF'          &     1.806   &    -0.020 &    -0.049 &     0.010 &       \\
             &                                    &FBrF'       &      86.2   &      33.8 &      33.8 &      -4.8   &       \\
 CF$_3$Br       & Trifluorobromomethane              &CBr           &     1.909   &     0.051 &     0.029 &     0.134 &     a \\
             &                                    &CF            &     1.328   &     0.007 &     0.019 &     0.039 &       \\
             &                                    &FCBr        &     110.3   &       0.5 &       0.8 &       3.4   &       \\
 F$_5$Br        & Bromine pentafluoride              &BrF(ax)       &     1.697   &     0.058 &     0.069 &     0.126 &   iii \\
             &                                    &BrF(eq)       &     1.768   &     0.006 &     0.003 &     0.031 &       \\
 SF$_5$Br       & Sulfur bromide pentafluoride       &S-F(ax)       &     1.520   &     0.019 &     0.125 &     0.022 &   iii \\
             &                                    &S-F(eq)       &     1.646   &    -0.105 &     0.020 &    -0.101 &       \\
             &                                    &S-Br          &     2.133   &     0.293 &     0.224 &     0.461 &       \\
 ClBr        & Bromine chloride                   &BrCl          &     2.136   &     0.040 &    -0.056 &    -0.072 &     a \\
 CH$_2$ClBr     & Bromochloromethane                 &C-Cl          &     1.768   &    -0.050 &     0.015 &    -0.035 &   sss \\
             &                                    &C-Br          &     1.930   &     0.009 &    -0.058 &    -0.014 &       \\
             &                                    &Br-C-Cl     &     112.3   &      -9.8 &      -0.7 &       1.6   &       \\
             &                                    &C-H           &     1.074   &     0.026 &     0.028 &     0.038 &       \\
 Br$_2$         & Bromine                            &BrBr          &     2.283   &     0.160 &    -0.115 &    -0.099 &     a \\
 CH$_2$Br$_2$      & Dibromomethane                     &CH            &     1.079   &     0.015 &     0.023 &     0.031 &    ss \\
             &                                    &HCH         &     113.6   &      -2.4 &      -2.8 &      -3.5   &       \\
             &                                    &CBr           &     1.927   &    -0.015 &    -0.059 &    -0.025 &       \\
             &                                    &BrCH        &     106.5   &       6.0 &       2.1 &       1.7   &       \\
 C$_4$O$_2$Br$_2$     & 1,2-Dibromocyclobutene-3,4-dione   &C=C2          &     1.356   &    -0.013 &     0.008 &     0.015 &   ttt \\
             &                                    &C2-C3         &     1.518   &    -0.005 &    -0.001 &     0.000 &       \\
             &                                    &C=C2-C3     &      94.4   &       0.1 &      -0.9 &      -1.2   &       \\
             &                                    &C-Br          &     1.831   &    -0.004 &    -0.044 &    -0.028 &       \\
             &                                    &Br-C=C      &     133.0   &      -0.1 &       2.4 &       2.5   &       \\
             &                                    &C-O           &     1.185   &     0.006 &     0.020 &     0.024 &       \\
             &                                    &O-C3-C2     &     135.0   &       1.7 &       1.6 &       1.1   &       \\
 SOBr$_2$       & Thionyl bromide                    &S=O           &     1.449   &    -0.002 &     0.025 &     0.016 &   iii \\
             &                                    &S-Br          &     2.255   &     0.066 &    -0.120 &    -0.048 &       \\
             &                                    &Br-S-O      &     107.6   &      -4.2 &       0.5 &       3.4   &       \\
             &                                    &Br-S-Br       &     2.255   &     0.066 &    -0.120 &    -0.048 &       \\
 C$_2$Br$_4$       & Tetrabromoethylene                 &CC            &     1.362   &    -0.065 &    -0.020 &    -0.018 &    ss \\
             &                                    &CBr           &     1.881   &    -0.043 &    -0.060 &    -0.024 &       \\
             &                                    &BrCC        &     122.4   &       6.2 &       1.3 &       0.0   &       \\
 SBr$_6$        & Sulfur hexabromide                 &SBr           &     2.190   &     0.238 &     0.052 &     0.108 &   ooo \\
\hline
\end{tabular}
\end{center}
\end{table}
\clearpage

\begin{table}
\caption{\label{geotabm}Comparison of Calculated and Observed Geometries for 
MNDO, AM1, and PM3 (contd.)}
\begin{center}
\compresstable
\begin{tabular}{llrrrrrr}
 Empirical  & Chemical Name &  Geometric &  Exp. & \multicolumn{3}{c}{Errors} & \\
  Formula   &               &  Variable &        & PM3  & MNDO  &  AM1 & Ref.\\
\hline
 HI          & Hydrogen iodide                    &HI            &     1.609   &     0.068 &    -0.042 &    -0.022 &   ppp \\
 CH$_3$I        & Iodomethane                        &CH            &     1.084   &     0.009 &     0.020 &     0.025 &    ss \\
             &                                    &CI            &     2.132   &    -0.104 &    -0.117 &    -0.082 &       \\
             &                                    &HCH         &     111.2   &      -1.2 &      -2.8 &      -1.4   &       \\
 C$_2$HI        & Iodoacetylene                      &H-C           &     1.056   &     0.009 &    -0.004 &     0.006 &   uuu \\
             &                                    &C-C           &     1.206   &    -0.018 &    -0.007 &    -0.008 &       \\
             &                                    &C-I           &     1.989   &    -0.084 &    -0.106 &    -0.077 &       \\
 FI          & Iodine fluoride                    &IF            &     1.906   &    -0.017 &    -0.004 &    -0.025 &     a \\
 CF$_3$I        & Trifluoroiodomethane               &CI            &     2.130   &    -0.078 &    -0.005 &     0.045 &     a \\
             &                                    &CF            &     1.332   &     0.008 &     0.022 &     0.037 &       \\
             &                                    &FCI         &     110.6   &       1.5 &       1.9 &       3.5   &       \\
 F$_5$I         & Iodine pentafluoride               &IF(ax)        &     1.844   &     0.023 &     0.139 &     0.087 &   vvv \\
             &                                    &IF(eq)        &     1.869   &     0.013 &     0.088 &     0.029 &       \\
 F$_7$I         & Iodine heptafluoride               &IF(ax)        &     1.760   &     1.034 &     0.731 &     0.866 &   www \\
             &                                    &IF(eq)        &     1.860   &     0.042 &     0.218 &     0.114 &       \\
 ClI         & Iodine chloride                    &ICl           &     2.327   &    -0.135 &    -0.065 &    -0.109 &     a \\
 BrI         & Iodine bromide                     &IBr           &     2.485   &     0.076 &    -0.135 &    -0.131 &     a \\
 I$_2$          & Iodine                             &II            &     2.666   &     0.002 &    -0.151 &    -0.128 &     a \\
 BeH         & Beryllium hydride (+)              &Be-H          &     1.312   &    -0.039 &    -0.065 &    -0.045 &     a \\
 BeH         & Beryllium hydride                  &Be-H          &     1.343   &    -0.036 &    -0.053 &    -0.033 &     a \\
 BeO         & Beryllium oxide                    &Be-O          &     1.331   &    -0.027 &     0.004 &     0.071 &     a \\
 BeS         & Beryllium sulfide                  &Be-S          &     1.742   &    -0.090 &    -0.118 &    -0.057 &     a \\
 BeF         & Beryllium fluoride                 &Be-F          &     1.361   &     0.012 &     0.098 &     0.121 &     a \\
 BeF$_2$        & Beryllium difluoride               &Be-F          &     1.400   &     0.002 &     0.060 &     0.090 &     a \\
 C$_5$BeH$_5$Cl    & Cyclopentadienylberyllium chloride &C-H           &     1.090   &    -0.006 &    -0.005 &     0.000 &   xxx \\
             &                                    &Be-ring       &     1.485   &     0.198 &     0.068 &     0.262 &       \\
             &                                    &Be-Cl         &     1.839   &    -0.068 &     0.117 &     0.017 &       \\
 BeCl$_2$       & Beryllium dichloride               &Be-Cl         &     1.770   &    -0.026 &     0.143 &     0.067 &     a \\
 BeBr$_2$       & Beryllium dibromide                &Be-Br         &     1.910   &    -0.096 &     0.102 &    -0.013 &     a \\
 BeI         & Beryllium iodide                   &Be-I          &     2.132   &     0.090 &     0.023 &     0.006 &     a \\
 BeI$_2$        & Beryllium diiodide                 &Be-I          &     2.120   &    -0.051 &     0.001 &    -0.019 &     a \\
 MgH         & Magnesium hydride                  &Mg-H          &     1.730   &    -0.043 &  &  &     a \\
 MgO         & Magnesium oxide                    &Mg-O          &     1.749   &     0.031 &  &  &     a \\
 MgS         & Magnesium sulfide                  &Mg-S          &     2.143   &     0.215 &  &  &     a \\
 MgF         & Magnesium fluoride                 &Mg-F          &     1.750   &     0.004 &  &  &     a \\
 MgF$_2$        & Magnesium difluoride               &Mg-F          &     1.771   &    -0.009 &  &  &   yyy \\
 MgCl        & Magnesium chloride                 &Mg-Cl         &     2.199   &    -0.324 &  &  &     a \\
 MgCl$_2$       & Magnesium dichloride               &Mg-Cl         &     2.186   &    -0.299 &  &  &   yyy \\
             &                                    &Cl-Mg-Cl    &     180.0   &      23.7 &  &    &       \\
 MgBr        & Magnesium bromide                  &Mg-Br         &     2.360   &    -0.006 &  &  &     a \\
 MgBr$_2$       & Magnesium dibromide                &Mg-Br         &     2.340   &     0.013 &  &  &     a \\
 MgI$_2$        & Magnesium diiodide                 &Mg-I          &     2.520   &    -0.100 &  &  &     a \\
 Mg$_2$         & Magnesium, dimer                   &Mg-Mg       &       3.9   &       3.1 &  &    &     a \\
 BH          & BH                                 &BH            &     1.236   &  &    -0.058 &    -0.016 &   xxx \\
 BH$_2$         & BH2                                &BH            &     1.180   &  &    -0.022 &     0.016 &   aaa \\
             &                                    &HBH         &     131.0   &  &      -4.6 &      -3.0   &       \\
 C$_3$BH$_9$       & Trimethylboron                     &B-C           &     0.850   &  &     0.000 &     0.000 &   ooo \\
\hline
\end{tabular}
\end{center}
\end{table}
\clearpage

\begin{table}
\caption{\label{geotabn}Comparison of Calculated and Observed Geometries for 
MNDO, AM1, and PM3 (contd.)}
\begin{center}
\compresstable
\begin{tabular}{llrrrrrr}
 Empirical  & Chemical Name &  Geometric &  Exp. & \multicolumn{3}{c}{Errors} & \\
  Formula   &               &  Variable &        & PM3  & MNDO  &  AM1 & Ref.\\
\hline
 BO          & BO                                 &BO            &     1.204   &  &    -0.034 &    -0.036 &   xxx \\
 CBH$_3$O       & BH3CO                              &C-B           &     1.534   &  &    -0.039 &    -0.004 &   iii \\
             &                                    &B-H           &     1.222   &  &    -0.046 &    -0.017 &       \\
             &                                    &C-B-H       &     103.8   &  &       3.0 &      -0.5   &       \\
             &                                    &C-O           &     1.135   &  &     0.028 &     0.036 &       \\
 C$_3$BH$_9$O$_3$     & Trimethoxyborane                   &B-O           &     1.380   &  &    -0.008 &    -0.020 &   ooo \\
 CBH$_3$S       & CH3-B=S                            &B-C           &     1.535   &  &    -0.025 &    -0.029 &   iii \\
             &                                    &C-H           &     1.109   &  &     0.002 &     0.003 &       \\
             &                                    &H-C-B       &     110.3   &  &       0.1 &      -1.7   &       \\
             &                                    &B=S           &     1.603   &  &    -0.118 &    -0.131 &       \\
 C$_6$BH$_5$F$_2$     & PhBF2                              &B-F           &     1.330   &  &    -0.007 &    -0.020 &   bbb \\
             &                                    &F-B-F       &     116.0   &  &      -3.7 &      -3.6   &       \\
             &                                    &B-C           &     1.550   &  &     0.007 &     0.000 &       \\
 C$_2$B$_4$H$_6$      & C2B4H6                             &C-C           &     1.540   &  &     0.055 &     0.010 &   ggg \\
             &                                    &C-B3          &     1.627   &  &     0.023 &    -0.004 &       \\
             &                                    &C-B4          &     1.605   &  &     0.072 &     0.043 &       \\
             &                                    &B3-B4         &     1.721   &  &     0.031 &     0.003 &       \\
             &                                    &B4-B6         &     1.752   &  &     0.006 &    -0.013 &       \\
 B$_6$H$_1$$_0$       & B6H10                              &B1-B2         &     1.757   &  &     0.043 &    -0.029 &   jjj \\
             &                                    &B2-B3         &     1.755   &  &    -0.001 &    -0.058 &       \\
             &                                    &BBB         &      58.7   &  &       0.6 &       1.5   &       \\
             &                                    &BB            &     1.808   &  &     0.096 &     0.021 &       \\
             &                                    &BBB         &      58.7   &  &       0.6 &       1.5   &       \\
 HAl         & AlH                                &AlH           &     1.648   &     0.015 &    -0.222 &    -0.186 &     a \\
 AlO         & AlO                                &AlO           &     1.618   &     0.003 &    -0.143 &    -0.068 &     a \\
 AlF         & Aluminum fluoride                  &AlF           &     1.654   &    -0.002 &    -0.094 &    -0.096 &     a \\
 AlF$_3$        & Aluminum trifluoride               &AlF           &     1.630   &     0.015 &    -0.041 &    -0.052 &     a \\
 AlF$_4$        & AlF4(--)                            &AlF            &     1.690   &    -0.002 &    -0.041 &    -0.070 &     a \\
 AlCl        & Aluminum chloride                  &AlCl           &     2.130   &    -0.183 &    -0.055 &    -0.294 &     a \\
 AlCl$_3$       & Aluminum trichloride               &AlCl           &     2.060   &    -0.094 &     0.005 &    -0.186 &     a \\
 AlBr        & Aluminum bromide                   &AlBr           &     2.295   &    -0.003 &    -0.093 &    -0.031 &     a \\
 AlBr$_3$       & Aluminum tribromide                &AlBr           &     2.270   &    -0.395 &    -0.095 &    -0.026 &     a \\
 AlI$_3$        & Aluminum triiodide                 &AlI            &     2.499   &    -0.012 &    -0.174 &    -0.111 &     a \\
 Al$_2$         & Al2                                &AlAl           &     2.467   &     0.073 &    -0.175 &    -0.060 &     a \\
 Al$_2$O        & Al2O                               &AlO            &     1.730   &    -0.053 &    -0.124 &    -0.065 &     a \\
 GaH         & Gallium hydride                    &Ga-H           &     1.663   &    -0.015 &  &  &   ppp \\
 GaF         & Gallium fluoride                   &Ga-F           &     1.774   &     0.009 &  &  &   ppp \\
 GaF$_3$        & Gallium trifluoride                &Ga-F           &     1.725   &    -0.012 &  &  &   yyy \\
 GaCl        & Gallium chloride                   &Ga-Cl          &     2.202   &     0.104 &  &  &   ppp \\
 GaH$_3$NCl$_3$    & Gallium trichloride-ammonia        &Ga-N           &     2.057   &     0.358 &  &  &   iii \\
             &                                    &Ga-Cl          &     2.142   &     0.233 &  &  &       \\
             &                                    &Cl-Ga-Cl     &     116.4   &       4.0 &  &    &       \\
 GaCl$_4$       & GaCl4(--)                           &Ga-Cl          &     2.170   &     0.046 &  &  &   kkk \\
 GaBr        & Gallium bromide                    &Ga-Cl          &     2.352   &     0.042 &  &  &   ppp \\
 GaH$_3$NBr$_3$    & Gallium tribromide-ammonia         &Ga-N           &     2.081   &    -0.230 &  &  &   iii \\
             &                                    &Ga-Br          &     2.288   &     0.048 &  &  &       \\
             &                                    &Br-Ga-Br     &     116.1   &      -1.7 &  &    &       \\
\hline
\end{tabular}
\end{center}
\end{table}
\clearpage

\begin{table}
\caption{\label{geotabo}Comparison of Calculated and Observed Geometries for 
MNDO, AM1, and PM3 (contd.)}
\begin{center}
\compresstable
\begin{tabular}{llrrrrrr}
 Empirical  & Chemical Name &  Geometric &  Exp. & \multicolumn{3}{c}{Errors} & \\
  Formula   &               &  Variable &        & PM3  & MNDO  &  AM1 & Ref.\\
\hline
 GaI         & Gallium iodide                     &Ga-I           &     2.575   &    -0.036 &  &  &   ppp \\
 GaI$_3$        & Gallium triiodide                  &Ga-I           &     2.458   &     0.119 &  &  &   yyy \\
 Ga$_2$O        & Gallium(I) oxide                   &Ga-O           &     1.824   &    -0.028 &  &  &   iii \\
             &                                    &Ga-O-Ga      &     142.9   &      37.1 &  &    &       \\
 Ga$_2$H$_4$Cl$_2$    & Ga2Cl2H4                           &Ga-Ga          &     3.241   &     0.583 &  &  &   lll \\
             &                                    &Ga-Cl          &     2.349   &     0.059 &  &  &       \\
             &                                    &Ga-H           &     1.559   &     0.039 &  &  &       \\
 Ga$_2$Cl$_6$      & Ga2Cl6                             &Ga-Cl(b)       &     2.300   &     0.061 &  &  &   ooo \\
             &                                    &Ga-Cl(t)       &     2.100   &    -0.211 &  &  &       \\
 Ga$_2$Br$_6$      & Ga2Br6                             &Ga-Br(b)       &     2.450   &    -0.016 &  &  &   ooo \\
             &                                    &Ga-Br(t)       &     2.250   &    -0.017 &  &  &       \\
 InH         & Indium hydride                     &In-H           &     1.838   &    -0.104 &  &  &   ppp \\
 InF         & Indium fluoride                    &In-F           &     1.985   &     0.001 &  &  &   ppp \\
 InCl        & Indium chloride                    &In-Cl          &     2.401   &     0.002 &  &  &   ppp \\
 InBr        & Indium bromide                     &In-Br          &     2.543   &    -0.253 &  &  &   ppp \\
 InI         & Indium iodide                      &In-I           &     2.729   &    -0.019 &  &  &   mmm \\
 InI$_3$        & Indium triiodide                   &In-I           &     2.641   &     0.004 &  &  &   yyy \\
 In$_2$O        & Indium(I) oxide                    &In-O           &     2.020   &    -0.021 &  &  &   iii \\
             &                                    &In-O-In      &     145.0   &      35.0 &  &    &       \\
 C$_5$TlH$_5$      & Cyclopentadienyl thallium          &Tl-C           &     2.705   &     0.007 &  &  &   xxx \\
 TlF         & Thallium fluoride (TlF)            &Tl-F           &     2.084   &     0.050 &  &  &   ooo \\
 TlCl        & Thallium chloride                  &Tl-Cl          &     2.485   &     0.004 &  &  &   xxx \\
 TlBr        & Thallium bromide                   &Tl-Br          &     2.618   &    -0.059 &  &  &   xxx \\
 TlI         & Thallium iodide                    &Tl-I           &     2.814   &    -0.090 &  &  &   xxx \\
 Tl$_2$F$_2$       & Thallium fluoride dimer            &Tl-F           &     2.290   &    -0.005 &  &  &   iii \\
             &                                    &F-Tl-F       &      90.0   &      23.1 &  &    &       \\
 HSi         & SiH                                &SiH            &     1.520   &    -0.015 &    -0.146 &    -0.066 &     a \\
 H$_2$Si        & Silylene (singlet)                 &SiH            &     1.519   &    -0.006 &    -0.139 &    -0.062 &   nnn \\
             &                                    &HSiH         &      92.1   &       2.8 &       5.2 &       8.9   &       \\
 H$_4$Si        & Silane                             &SiH            &     1.481   &     0.007 &    -0.105 &    -0.020 &     a \\
 CH$_6$Si       & Methylsilane                       &SiH            &     1.485   &     0.008 &    -0.106 &    -0.022 &    ss \\
             &                                    &HSiH         &     108.3   &      -0.1 &       0.5 &      -0.4   &       \\
             &                                    &SiC            &     1.867   &     0.001 &    -0.064 &    -0.058 &       \\
 C$_4$H$_1$$_2$Si     & Tetramethylsilane                  &SiC            &     1.875   &     0.015 &    -0.060 &    -0.046 &    ss \\
 SiN         & Silicon nitride                    &SiN            &     1.572   &    -0.108 &    -0.092 &    -0.087 &     a \\
 CH$_6$SiS      & CH3-S-SiH3                         &C-S            &     1.819   &    -0.019 &    -0.111 &    -0.058 &   ooo \\
             &                                    &S-Si           &     2.134   &     0.105 &    -0.070 &     0.176 &       \\
             &                                    &C-S-Si       &      98.3   &       3.9 &      16.8 &      10.5   &       \\
             &                                    &Si-H           &     1.481   &     0.009 &    -0.112 &    -0.020 &       \\
             &                                    &C-H            &     1.091   &     0.007 &     0.016 &     0.026 &       \\
 CH$_5$SiF      & CH3-SiH2F                          &Si-F           &     1.602   &     0.001 &    -0.004 &     0.021 &   ppp \\
             &                                    &Si-C           &     1.845   &     0.028 &    -0.038 &    -0.043 &       \\
             &                                    &C-Si-F       &     109.2   &      -0.1 &       0.1 &       2.8   &       \\
             &                                    &Si-H           &     1.478   &     0.025 &    -0.097 &    -0.019 &       \\
             &                                    &C-Si-H       &     112.4   &      -2.3 &      -1.5 &      -3.1   &       \\
 SiF$_2$        & Difluorosilylene                   &SiF            &     1.591   &    -0.016 &    -0.013 &     0.021 &     a \\
             &                                    &FSiF         &     101.0   &      -5.7 &      -4.1 &      -4.0   &       \\
\hline
\end{tabular}
\end{center}
\end{table}
\clearpage

\begin{table}
\caption{\label{geotabp}Comparison of Calculated and Observed Geometries for 
MNDO, AM1, and PM3 (contd.)}
\begin{center}
\compresstable
\begin{tabular}{llrrrrrr}
 Empirical  & Chemical Name &  Geometric &  Exp. & \multicolumn{3}{c}{Errors} & \\
  Formula   &               &  Variable &        & PM3  & MNDO  &  AM1 & Ref.\\
\hline
 HSiF$_3$       & Trifluorosilane                    &SiH            &     1.447   &     0.061 &    -0.072 &    -0.007 &     a \\
             &                                    &SiF            &     1.562   &     0.028 &     0.031 &     0.047 &       \\
             &                                    &FSiH         &     110.6   &       1.8 &       3.0 &       1.2   &       \\
 SiF$_4$        & Tetrafluorosilane                  &SiF            &     1.552   &     0.028 &     0.032 &     0.052 &     a \\
 SiCl        & Chlorosilylidyne                   &SiCl           &     2.063   &    -0.117 &     0.009 &    -0.077 &     a \\
 SiCl$_2$       & Dichlorosilylene                   &ClSiCl       &     109.7   &      -7.8 &      -4.2 &      -5.3   &   qqq \\
 CH$_3$SiCl$_3$    & Trichloromethylsilane              &Si-C           &     1.848   &    -0.002 &    -0.053 &    -0.060 &   iii \\
             &                                    &Si-Cl          &     2.026   &     0.038 &     0.065 &     0.035 &       \\
             &                                    &C-Si-Cl      &     110.3   &       1.0 &       1.5 &       0.9   &       \\
 SiCl$_4$       & Silicon tetrachloride              &SiCl           &     2.017   &     0.024 &     0.063 &     0.022 &     a \\
 SiF$_3$Br      & Trifluorobromosilane               &Si-Br          &     2.156   &    -0.282 &     0.063 &     0.139 &   rrr \\
             &                                    &Si-F           &     1.559   &     0.013 &     0.026 &     0.048 &       \\
             &                                    &F-Si-Br      &     110.4   &      -2.2 &       1.8 &       1.8   &       \\
 SiBr$_4$       & Silicon tetrabromide               &SiBr           &     2.150   &    -0.354 &     0.040 &     0.093 &     a \\
 H$_3$SiI       & Iodosilane                         &SiI            &     2.437   &    -0.425 &    -0.051 &    -0.003 &     a \\
             &                                    &SiH            &     1.486   &     0.006 &    -0.116 &    -0.020 &       \\
             &                                    &HSiI         &     108.5   &      -0.5 &      -0.1 &       1.3   &       \\
 SiI$_4$        & Silicon tetraiodide                &SiI            &     2.430   &     0.037 &    -0.097 &    -0.005 &     a \\
 Si$_2$         & Silicon dimer                      &SiSi           &     2.246   &     0.050 &    -0.260 &    -0.458 &     a \\
 H$_6$Si$_2$       & Disilane                           &SiSi           &     2.331   &     0.063 &    -0.166 &     0.081 &   sss \\
             &                                    &SiH            &     1.492   &    -0.004 &    -0.118 &    -0.026 &       \\
             &                                    &HSiSi        &     110.3   &      -0.8 &       2.3 &      -0.7   &       \\
 H$_1$$_0$Si$_5$      & Cyclopentasilane                   &Si-Si          &     2.342   &     0.040 &    -0.179 &     0.050 &   iii \\
 GeH$_4$        & Germane                            &Ge-H           &     1.527   &    -0.022 &    -0.046 &     0.019 &   ttt \\
 CGeH$_6$       & Methylgermane                      &Ge-C           &     1.945   &     0.010 &    -0.018 &     0.042 &    ss \\
             &                                    &Ge-H           &     1.529   &    -0.024 &    -0.045 &     0.016 &       \\
             &                                    &C-H            &     1.083   &     0.007 &     0.024 &     0.022 &       \\
 C$_2$GeH$_8$      & Ethylgermane                       &Ge-C           &     1.949   &     0.008 &    -0.008 &     0.055 &   xxx \\
             &                                    &C-C            &     1.545   &    -0.052 &    -0.025 &    -0.056 &       \\
             &                                    &Ge-C-C       &     112.2   &      -7.0 &       5.0 &      -1.0   &       \\
             &                                    &Ge-H           &     1.522   &    -0.016 &    -0.039 &     0.023 &       \\
             &                                    &H-Ge-C       &     109.7   &      -0.1 &       0.2 &      -2.2   &       \\
 C$_2$GeH$_8$      & Dimethylgermane                    &Ge-H           &     1.532   &    -0.027 &    -0.046 &     0.012 &   uuu \\
             &                                    &H-Ge-H       &     108.7   &       0.0 &       0.7 &       5.2   &       \\
             &                                    &Ge-C           &     1.950   &     0.007 &    -0.019 &     0.034 &       \\
             &                                    &C-Ge-C       &     110.0   &      -0.4 &       0.8 &      -3.3   &       \\
 C$_3$GeH$_1$$_0$     & Trimethylgermane                   &Ge-H           &     1.522   &    -0.017 &    -0.034 &     0.021 &   xxx \\
             &                                    &Ge-C           &     1.947   &     0.012 &    -0.013 &     0.035 &       \\
             &                                    &H-Ge-C       &     109.3   &       0.3 &      -0.5 &       1.6   &       \\
 C$_4$GeH$_1$$_2$     & Tetramethylgermanium               &Ge-C           &     1.945   &     0.015 &    -0.006 &     0.036 &   vvv \\
             &                                    &C-H            &     1.120   &    -0.030 &    -0.013 &    -0.014 &       \\
 GeO         & Germanium oxide                    &Ge-O           &     1.625   &     0.012 &    -0.055 &    -0.002 &   ppp \\
 CGeH$_3$N      & Cyanogermane                       &Ge-C           &     1.919   &    -0.084 &    -0.060 &    -0.031 &   xxx \\
             &                                    &C-N            &     1.155   &     0.003 &     0.011 &     0.008 &       \\
 CGeH$_3$NO     & Germyl isocyanate                  &Ge-H           &     1.532   &    -0.017 &    -0.048 &     0.007 &   www \\
             &                                    &Ge-N           &     1.831   &     0.010 &    -0.001 &    -0.030 &       \\
 GeH$_3$N$_3$      & Germylazide                        &Ge-H           &     1.497   &     0.019 &    -0.012 &     0.057 &   xxx \\
             &                                    &Ge-N           &     1.866   &    -0.035 &     0.010 &    -0.008 &       \\
\hline
\end{tabular}
\end{center}
\end{table}
\clearpage

\begin{table}
\caption{\label{geotabq}Comparison of Calculated and Observed Geometries for 
MNDO, AM1, and PM3 (contd.)}
\begin{center}
\compresstable
\begin{tabular}{llrrrrrr}
 Empirical  & Chemical Name &  Geometric &  Exp. & \multicolumn{3}{c}{Errors} & \\
  Formula   &               &  Variable &        & PM3  & MNDO  &  AM1 & Ref.\\
\hline
 GeS         & Germanium sulfide                  &Ge-S           &     2.012   &    -0.039 &    -0.087 &     0.020 &   ppp \\
 CGeH$_3$NS     & Germyl isothiocyanate              &Ge-H           &     1.520   &    -0.002 &    -0.037 &     0.021 &   iii \\
             &                                    &Ge-N           &     1.817   &     0.015 &     0.028 &    -0.010 &       \\
             &                                    &H-Ge-N       &     106.9   &       3.6 &       1.0 &       1.3   &       \\
             &                                    &C-N            &     1.144   &     0.069 &     0.054 &     0.054 &       \\
 GeF         & Germanium fluoride                 &Ge-F           &     1.750   &    -0.039 &    -0.046 &    -0.093 &   yyy \\
 GeH$_3$F       & Fluorogermane                      &Ge-F           &     1.734   &     0.004 &     0.003 &    -0.019 &   iii \\
             &                                    &Ge-H           &     1.523   &    -0.011 &    -0.039 &     0.017 &       \\
             &                                    &F-Ge-H       &     106.0   &       3.9 &       2.6 &       2.3   &       \\
 CGeH$_5$F      & Methylgermanium fluoride dihydride &Ge-F           &     1.739   &    -0.003 &     0.003 &    -0.023 &   ppp \\
             &                                    &Ge-C           &     1.927   &     0.034 &     0.004 &     0.044 &       \\
             &                                    &C-Ge-F       &     106.0   &       1.5 &       1.7 &      -1.7   &       \\
             &                                    &Ge-H           &     1.523   &    -0.012 &    -0.038 &     0.015 &       \\
             &                                    &C-Ge-H       &     113.9   &      -4.3 &      -2.5 &      -3.1   &       \\
 GeF$_2$        & Germanium difluoride               &Ge-F           &     1.732   &    -0.037 &    -0.017 &    -0.062 &   yyy \\
             &                                    &F-Ge-F       &      97.2   &      14.5 &      -2.0 &      -1.7   &       \\
 C$_2$GeH$_6$F$_2$    & Dimethylgermanium difluoride       &Ge-F           &     1.739   &    -0.015 &     0.004 &    -0.029 &   zzz \\
             &                                    &F-Ge-F       &     105.4   &      -1.7 &      -3.7 &      -4.2   &       \\
             &                                    &Ge-C           &     1.928   &     0.027 &     0.008 &     0.038 &       \\
             &                                    &C-Ge-F       &     107.4   &       2.3 &       2.3 &       1.8   &       \\
 CGeH$_3$F$_3$     & Trifluoromethylgermane             &Ge-C           &     1.904   &     0.046 &     0.032 &     0.065 &   iii \\
             &                                    &Ge-F           &     1.714   &     0.000 &     0.024 &    -0.007 &       \\
             &                                    &F-Ge-C       &     113.2   &       0.0 &       0.8 &       1.3   &       \\
 GeH$_3$Cl      & Chlorogermane                      &Ge-Cl          &     2.150   &     0.046 &     0.098 &    -0.018 &    ss \\
             &                                    &Ge-H           &     1.537   &    -0.033 &    -0.060 &     0.009 &       \\
             &                                    &H-Ge-H       &     111.0   &      -0.9 &       1.3 &       1.1   &       \\
 C$_3$GeH$_9$Cl    & Trimethylchlorogermane             &Ge-Cl          &     2.170   &     0.033 &     0.088 &    -0.026 &     A \\
             &                                    &Ge-C           &     1.940   &     0.011 &    -0.010 &     0.040 &       \\
             &                                    &C-Ge-Cl      &     106.6   &       0.5 &       0.0 &       1.4   &       \\
 GeF$_3$Cl      & Chlorotrifluorogermane             &Ge-Cl          &     2.067   &     0.067 &     0.189 &     0.105 &   xxx \\
             &                                    &Ge-F           &     1.688   &     0.014 &     0.040 &     0.015 &       \\
             &                                    &F-Ge-F       &     107.5   &       0.1 &      -0.7 &      -2.0   &       \\
 GeCl$_2$       & Germanium dichloride               &Ge-Cl          &     2.186   &    -0.198 &     0.027 &    -0.089 &   yyy \\
             &                                    &Cl-Ge-Cl     &     100.4   &      79.6 &       4.3 &      13.6   &       \\
 C$_2$GeH$_6$Cl$_2$   & Dimethylgermanium dichloride       &Ge-Cl          &     2.143   &     0.039 &     0.099 &    -0.004 &     B \\
             &                                    &Cl-Ge-Cl     &     105.0   &       1.9 &       1.1 &       3.3   &       \\
             &                                    &Ge-C           &     1.928   &     0.014 &    -0.002 &     0.054 &       \\
             &                                    &C-Ge-Cl      &     108.0   &       0.8 &       0.7 &       0.9   &       \\
 GeHCl$_3$      & Trichlorogermane                   &Ge-Cl          &     2.114   &     0.040 &     0.108 &     0.015 &   xxx \\
 CGeH$_3$Cl$_3$    & Trichloromethylgermane             &Ge-Cl          &     2.135   &     0.028 &     0.097 &     0.001 &   xxx \\
             &                                    &C-Ge-Cl      &     106.0   &       4.6 &       5.6 &       4.2   &       \\
 GeCl$_4$       & Germanium tetrachloride            &Ge-Cl          &     2.113   &     0.037 &     0.112 &     0.024 &     C \\
 GeH$_3$Br      & Bromogermane                       &Ge-Br          &     2.297   &     0.035 &     0.069 &    -0.041 &   iii \\
             &                                    &Ge-H           &     1.527   &    -0.026 &    -0.051 &     0.025 &       \\
             &                                    &H-Ge-Br      &     106.3   &       1.6 &       0.7 &       3.2   &       \\
 C$_3$GeH$_9$Br    & Bromotrimethylgermane              &Ge-Br          &     2.323   &     0.039 &     0.051 &    -0.055 &   xxx \\
             &                                    &Ge-C           &     1.936   &     0.012 &    -0.006 &     0.055 &       \\
             &                                    &Br-Ge-C      &     106.3   &       0.1 &       0.6 &       4.2   &       \\
\hline
\end{tabular}
\end{center}
\end{table}
\clearpage

\begin{table}
\caption{\label{geotabr}Comparison of Calculated and Observed Geometries for 
MNDO, AM1, and PM3 (contd.)}
\begin{center}
\compresstable
\begin{tabular}{llrrrrrr}
 Empirical  & Chemical Name &  Geometric &  Exp. & \multicolumn{3}{c}{Errors} & \\
  Formula   &               &  Variable &        & PM3  & MNDO  &  AM1 & Ref.\\
\hline
 GeBr$_2$       & Germanium dibromide                &Ge-Br          &     2.337   &    -0.015 &    -0.016 &    -0.157 &     D \\
             &                                    &Br-Ge-Br     &     101.2   &      11.3 &       4.2 &      78.7   &       \\
 CGeH$_3$Br$_3$    & Tribromomethylgermane              &Ge-C           &     1.889   &     0.047 &     0.036 &     0.133 &   iii \\
             &                                    &Ge-Br          &     2.276   &     0.056 &     0.075 &    -0.010 &       \\
             &                                    &C-Ge-Br      &     111.6   &      -1.0 &      -1.4 &      -5.8   &       \\
             &                                    &C-H            &     1.120   &    -0.028 &    -0.014 &    -0.015 &       \\
 GeBr$_4$       & Germanium tetrabromide             &Ge-Br          &     2.272   &     0.041 &     0.067 &    -0.003 &     E \\
 GeH$_3$I       & Iodogermane                        &Ge-I           &     2.508   &    -0.036 &     0.023 &    -0.075 &   xxx \\
 GeI$_4$        & Germanium tetraiodide              &Ge-I           &     2.500   &    -0.032 &    -0.003 &    -0.090 &   ooo \\
 GeH$_6$Si      & Germylsilane                       &Ge-Si          &     2.357   &     0.047 &    -0.017 &     0.002 &   xxx \\
             &                                    &Ge-H           &     1.529   &    -0.031 &    -0.047 &     0.020 &       \\
             &                                    &Si-H           &     1.483   &     0.005 &    -0.104 &    -0.025 &       \\
 Ge$_2$H$_6$       & Digermane                          &GeGe           &     2.403   &    -0.010 &     0.121 &    -0.037 &    ss \\
             &                                    &GeH            &     1.541   &    -0.038 &    -0.059 &     0.007 &       \\
             &                                    &HGeH         &     106.4   &       4.0 &       1.4 &       2.0   &       \\
 C$_6$Ge$_2$H$_1$$_8$O   & Bis(trimethylgermanium) oxide      &Ge-O           &     1.770   &     0.016 &    -0.016 &     0.079 &     F \\
             &                                    &Ge-O-Ge      &     141.0   &     -15.8 &      38.9 &     -19.4   &       \\
             &                                    &Ge-C           &     1.980   &    -0.005 &    -0.038 &    -0.011 &       \\
 SnH$_4$        & Tin tetrahydride (stannane)        &Sn-H           &     1.701   &     0.000 &    -0.115 &  &     G \\
 CSnH$_6$       & Methyltin trihydride               &Sn-C           &     2.140   &     0.047 &    -0.083 &  &   iii \\
             &                                    &Sn-H           &     1.708   &    -0.007 &    -0.122 &  &       \\
             &                                    &H-Sn-C       &     109.4   &       0.7 &       1.2 &    &       \\
 C$_2$SnH$_8$      & Dimethyltin dihydride              &Sn-H           &     1.680   &     0.019 &    -0.092 &  &     H \\
             &                                    &Sn-C           &     2.150   &     0.027 &    -0.091 &  &       \\
             &                                    &H-Sn-C       &     108.0   &       1.8 &       1.4 &    &       \\
             &                                    &C-Sn-C       &     104.8   &       4.5 &       7.6 &    &       \\
 C$_3$SnH$_1$$_0$     & Trimethyltin hydride               &Sn-H           &     1.705   &    -0.008 &    -0.117 &  &     I \\
             &                                    &Sn-C           &     2.147   &     0.018 &    -0.086 &  &       \\
             &                                    &H-Sn-C       &     111.5   &      -1.8 &      -3.6 &    &       \\
 C$_4$SnH$_1$$_2$     & Tetramethyltin                     &Sn-C           &     2.134   &     0.014 &    -0.071 &  &     J \\
 SnO         & Tin oxide                          &SnO            &     1.833   &     0.006 &    -0.084 &  &   ppp \\
 SnS         & Tin sulfide                        &SnS            &     2.209   &    -0.071 &    -0.208 &  &   ppp \\
 SnH$_3$Cl      & Tin chloride trihydride            &Sn-Cl          &     2.327   &     0.069 &    -0.018 &  &   xxx \\
 SnCl$_2$       & Tin dichloride                     &Sn-Cl          &     2.346   &    -0.006 &    -0.075 &  &   yyy \\
             &                                    &Cl-Sn-Cl     &      99.0   &      -0.1 &       3.5 &    &       \\
 C$_2$SnH$_6$Cl$_2$   & Dimethyltin dichloride             &Sn-Cl          &     2.327   &     0.037 &    -0.020 &  &     H \\
             &                                    &Cl-Sn-Cl     &     106.2   &      -0.3 &      -2.1 &    &       \\
             &                                    &Sn-C           &     2.109   &     0.000 &    -0.034 &  &       \\
             &                                    &S-Sn-Cl      &     108.5   &      -0.2 &      -0.9 &    &       \\
 SnCl$_4$       & Tin tetrachloride                  &Sn-Cl          &     2.280   &     0.075 &     0.004 &  &     K \\
 SnH$_3$Br      & Tin bromide trihydride             &Sn-Br          &     2.469   &    -0.016 &    -0.068 &  &    ss \\
             &                                    &Sn-H           &     1.767   &    -0.075 &    -0.178 &  &       \\
             &                                    &H-Sn-H       &     112.8   &       1.2 &      -0.4 &    &       \\
 C$_3$SnH$_9$Br    & Trimethyltin bromide               &Sn-Br          &     2.490   &    -0.039 &    -0.074 &  &     L \\
             &                                    &Sn-C           &     2.170   &    -0.056 &    -0.105 &  &       \\
\hline
\end{tabular}
\end{center}
\end{table}
\clearpage

\begin{table}
\caption{\label{geotabs}Comparison of Calculated and Observed Geometries for 
MNDO, AM1, and PM3 (contd.)}
\begin{center}
\compresstable
\begin{tabular}{llrrrrrr}
 Empirical  & Chemical Name &  Geometric &  Exp. & \multicolumn{3}{c}{Errors} & \\
  Formula   &               &  Variable &        & PM3  & MNDO  &  AM1 & Ref.\\
\hline
 SnBr$_2$       & Tin dibromide                      &Sn-Br          &     2.512   &    -0.108 &    -0.149 &  &   yyy \\
             &                                    &Br-Sn-Br     &     100.0   &       2.5 &       4.2 &    &       \\
 SnBr$_4$       & Tin tetrabromide                   &Sn-Br          &     2.440   &     0.000 &    -0.055 &  &     M \\
 SnH$_3$I       & Tin iodide trihydride              &Sn-I           &     2.674   &    -0.051 &    -0.139 &  &   xxx \\
 C$_3$SnH$_9$I     & Trimethyltin iodide                &Sn-I           &     2.720   &    -0.052 &    -0.168 &  &     L \\
 SnI$_2$        & Tin diiodide                       &Sn-I           &     2.706   &    -0.062 &    -0.210 &  &   yyy \\
             &                                    &I-Sn-I       &     103.8   &       5.7 &       3.3 &    &       \\
 PbH         & Lead hydride                       &Pb-H           &     1.839   &    -0.110 &    -0.181 &  &     a \\
 C$_4$PbH$_1$$_2$     & Tetramethyllead                    &Pb-C           &     2.240   &    -0.054 &    -0.069 &  &     N \\
             &                                    &C-H            &     1.080   &     0.013 &     0.022 &  &       \\
             &                                    &H-C-Pb       &     104.6   &       6.0 &       4.5 &    &       \\
 PbO         & Lead oxide                         &Pb-O           &     1.920   &     0.016 &    -0.038 &  &     O \\
 PbS         & Lead sulfide                       &Pb-S           &     2.290   &    -0.138 &    -0.179 &  &     O \\
 PbF         & Lead fluoride                      &Pb-F           &     2.058   &    -0.030 &    -0.063 &  &     a \\
 PbF$_2$        & Lead difluoride                    &Pb-F           &     2.033   &    -0.007 &    -0.038 &  &     P \\
             &                                    &F-Pb-F       &      97.2   &      -8.3 &      -5.8 &    &       \\
 PbCl        & Lead chloride                      &Pb-Cl          &     2.180   &     0.203 &     0.198 &  &     a \\
 PbCl$_2$       & Lead dichloride                    &Pb-Cl          &     2.444   &     0.014 &    -0.064 &  &     Q \\
             &                                    &Cl-Pb-Cl     &      98.3   &       1.3 &       2.5 &    &       \\
 PbCl$_4$       & Lead tetrachloride                 &Pb-Cl          &     2.430   &    -0.154 &    -0.049 &  &     a \\
 PbBr        & Lead bromide                       &Pb-Br          &     2.546   &     0.017 &    -0.080 &  &     a \\
 PbBr$_2$       & Lead dibromide                     &Pb-Br          &     2.597   &    -0.032 &    -0.127 &  &     P \\
             &                                    &Br-Pb-Br     &      99.2   &       3.4 &       2.7 &    &       \\
 PbI         & Lead iodide                        &Pb-I           &     2.736   &     0.037 &    -0.155 &  &     a \\
 PbI$_2$        & Lead diiodide                      &Pb-I           &     2.804   &    -0.026 &    -0.206 &  &     P \\
             &                                    &I-Pb-I       &      99.7   &       8.2 &       4.4 &    &       \\
 C$_6$Pb$_2$H$_1$$_8$    & Hexamethyldiplumbane               &Pb-Pb          &     2.880   &    -0.012 &    -0.110 &  &     R \\
             &                                    &Pb-C           &     2.250   &    -0.050 &    -0.073 &  &       \\
             &                                    &Pb-Pb-C      &     109.5   &       2.8 &       2.2 &    &       \\
 H$_3$P         & Phosphine                          &PH             &     1.420   &    -0.096 &    -0.080 &    -0.057 &     a \\
             &                                    &HPH          &      93.8   &       3.3 &       2.3 &       2.6   &       \\
 CP          & Carbon phosphide                   &CP             &     1.562   &    -0.173 &    -0.145 &    -0.029 &     a \\
 CHP         & Methinophosphide                   &CP             &     1.542   &    -0.133 &    -0.114 &    -0.132 &     a \\
             &                                    &HC             &     1.067   &     0.001 &    -0.010 &    -0.003 &       \\
 CH$_5$P        & Methylphosphine                    &CP             &     1.858   &     0.007 &    -0.108 &    -0.132 &    ss \\
             &                                    &PH             &     1.423   &    -0.087 &    -0.080 &    -0.059 &       \\
             &                                    &HPC          &      97.5   &       2.5 &       3.3 &       2.5   &       \\
             &                                    &HPH          &      93.4   &       3.5 &       2.6 &       2.8   &       \\
 C$_3$H$_9$P       & Trimethylphosphine                 &CP             &     1.843   &     0.029 &    -0.081 &    -0.118 &    ss \\
             &                                    &CPC          &      98.9   &       1.7 &       7.9 &       3.2   &       \\
 C$_5$H$_5$P       & Phosphole (?)                      &C1C2           &     1.384   &    -0.006 &     0.013 &    -0.005 &    ss \\
             &                                    &CP             &     1.733   &    -0.042 &    -0.092 &    -0.133 &       \\
             &                                    &C2C3           &     1.413   &    -0.034 &    -0.016 &    -0.034 &       \\
 PO          & Phosphorus oxide                   &PO             &     1.476   &    -0.018 &    -0.053 &    -0.039 &     a \\
 NP          & Phosphorus nitride                 &PN             &     1.491   &    -0.077 &    -0.093 &    -0.109 &     a \\
 CH$_2$PF       & CH2=P-F                            &P-F            &     1.598   &    -0.030 &    -0.049 &    -0.052 &     S \\
             &                                    &C=P            &     1.644   &    -0.071 &    -0.064 &    -0.106 &       \\
\hline
\end{tabular}
\end{center}
\end{table}
\clearpage

\begin{table}
\caption{\label{geotabt}Comparison of Calculated and Observed Geometries for 
MNDO, AM1, and PM3 (contd.)}
\begin{center}
\compresstable
\begin{tabular}{llrrrrrr}
 Empirical  & Chemical Name &  Geometric &  Exp. & \multicolumn{3}{c}{Errors} & \\
  Formula   &               &  Variable &        & PM3  & MNDO  &  AM1 & Ref.\\
\hline
 PF$_3$         & Phosphorus trifluoride             &PF             &     1.570   &    -0.012 &    -0.014 &    -0.027 &     a \\
             &                                    &FPF          &      97.8   &      -2.0 &       1.1 &       0.2   &       \\
 POF$_3$        & Phosphous oxyfluoride              &PF             &     1.520   &     0.009 &     0.034 &     0.006 &     a \\
             &                                    &FPF          &     102.5   &      -2.7 &      -0.6 &      -0.8   &       \\
             &                                    &PO             &     1.450   &     0.001 &     0.036 &     0.001 &       \\
 PSF$_3$        & Phosphorus thiofluoride            &PF             &     1.530   &     0.009 &     0.027 &     0.000 &     a \\
             &                                    &FPF          &     100.3   &      -5.2 &      -0.9 &      -1.8   &       \\
             &                                    &PS             &     1.870   &     0.064 &     0.106 &     0.004 &       \\
 PF$_5$         & Phosphorus pentafluoride           &PF(ax)         &     1.577   &    -0.024 &     0.025 &    -0.028 &     a \\
             &                                    &PF(eq)         &     1.534   &    -0.006 &     0.039 &     0.001 &       \\
 PCl$_3$        & Phosphorus trichloride             &PCl            &     2.039   &     0.025 &    -0.050 &    -0.120 &     a \\
             &                                    &ClPCl        &     100.3   &      -0.6 &       4.9 &       5.1   &       \\
 PCl$_5$        & Phosphorus pentachloride           &PCl(ax)        &     2.190   &    -0.097 &    -0.078 &    -0.115 &     a \\
             &                                    &PCl(eq)        &     2.040   &     0.012 &    -0.007 &    -0.069 &       \\
 PBr$_3$        & Phosphorus tribromide              &PBR            &     2.220   &    -0.070 &    -0.131 &    -0.119 &    ss \\
             &                                    &BrPBr        &     101.0   &       0.3 &       4.7 &       5.8   &       \\
 POBr$_3$       & Phosphorus oxybromide              &P=O            &     1.455   &    -0.048 &     0.021 &     0.025 &   iii \\
             &                                    &P-Br           &     2.174   &    -0.059 &    -0.046 &    -0.014 &       \\
             &                                    &O=P-Br       &     114.2   &       2.3 &      -0.4 &       1.0   &       \\
 PSBr$_3$       & Phosphorus thiotribromide          &P=S            &     1.894   &    -0.053 &    -0.042 &    -0.011 &   iii \\
             &                                    &P-Br           &     2.193   &    -0.102 &    -0.077 &    -0.040 &       \\
             &                                    &S=P-Br       &     116.2   &      12.0 &      -1.5 &      -1.0   &       \\
 BH$_3$PF$_3$      & PH3-BF3                            &P-B           &     1.836   &  &     2.386 &     0.321 &   xxx \\
             &                                    &P-H           &     1.207   &  &     0.132 &     0.158 &       \\
             &                                    &H-P-H       &     115.1   &  &     -18.8 &     -16.6   &       \\
 GeH$_3$PSF$_2$    & Difluoro(germylthio)phosphine      &Ge-S           &     2.256   &     0.031 &    -0.072 &     0.058 &   iii \\
             &                                    &P-S            &     2.115   &    -0.013 &    -0.163 &    -0.009 &       \\
             &                                    &Ge-S-P       &      99.0   &      26.4 &      41.3 &      17.6   &       \\
             &                                    &Ge-H           &     1.538   &    -0.032 &    -0.061 &     0.006 &       \\
             &                                    &S-Ge-H       &     110.0   &       3.5 &      -3.2 &      -1.2   &       \\
             &                                    &P-F            &     1.590   &    -0.035 &    -0.032 &    -0.052 &       \\
             &                                    &S-P-F        &      99.9   &      13.4 &       5.4 &       9.3   &       \\
 Ge$_3$H$_9$P      & Trigermylphosphine                 &P-Ge           &     2.306   &    -0.070 &    -0.159 &    -0.204 &   iii \\
             &                                    &Ge-P-Ge      &      95.7   &      13.5 &      24.3 &      24.3   &       \\
             &                                    &Ge-H           &     1.490   &     0.016 &    -0.009 &     0.031 &       \\
             &                                    &P-Ge-H       &     110.3   &       5.6 &      -3.0 &      -7.8   &       \\
 P$_2$          & Phosphorus dimer                   &PP             &     1.894   &    -0.179 &    -0.200 &    -0.271 &     a \\
 Ge$_2$H$_6$P$_2$F$_2$   & 1,1-Difluoro-2,2-digermyldiphosphan&P-F            &     1.581   &    -0.032 &    -0.028 &    -0.052 &   iii \\
             &                                    &P-P            &     2.177   &    -0.061 &    -0.209 &    -0.288 &       \\
             &                                    &P-P-F        &      98.9   &       8.1 &       7.1 &       0.6   &       \\
             &                                    &Ge-P           &     2.320   &    -0.110 &    -0.141 &    -0.166 &       \\
             &                                    &Ge-P-F       &      95.7   &      33.6 &      31.9 &      30.5   &       \\
             &                                    &Ge-P-Ge      &      98.6   &      12.3 &      19.8 &      12.3   &       \\
             &                                    &Ge-H           &     1.512   &     0.000 &    -0.034 &     0.018 &       \\
 P$_4$          & Phosphorus tetramer                &PP             &     2.210   &    -0.013 &    -0.158 &    -0.170 &     a \\
 P$_4$O$_6$        & Phosphorus trioxide                &PO             &     1.650   &     0.058 &    -0.046 &     0.020 &     a \\
             &                                    &OPO          &      99.0   &      -2.5 &      -3.0 &      -2.3   &       \\
\hline
\end{tabular}
\end{center}
\end{table}
\clearpage

\begin{table}
\caption{\label{geotabu}Comparison of Calculated and Observed Geometries for 
MNDO, AM1, and PM3 (contd.)}
\begin{center}
\compresstable
\begin{tabular}{llrrrrrr}
 Empirical  & Chemical Name &  Geometric &  Exp. & \multicolumn{3}{c}{Errors} & \\
  Formula   &               &  Variable &        & PM3  & MNDO  &  AM1 & Ref.\\
\hline
 P$_4$O$_1$$_0$       & Phosphorus pentoxide               &P-O            &     1.600   &     0.090 &     0.006 &     0.004 &     a \\
             &                                    &O-P-O        &     101.0   &      -4.7 &      -3.1 &      14.2   &       \\
 AsH$_3$        & Arsine                             &As-H           &     1.513   &     0.007 &  &  &     T \\
             &                                    &H-As-H       &      92.1   &       2.1 &  &    &       \\
 C$_3$AsN$_3$      & Arsenic tricyanide                 &As-C           &     1.900   &    -0.031 &  &  &   ooo \\
             &                                    &C-As-C       &      92.0   &       6.3 &  &    &       \\
             &                                    &As-C-N       &     171.0   &       4.9 &  &    &       \\
 AsF$_3$        & Arsenic trifluoride                &As-F           &     1.706   &     0.000 &  &  &   ooo \\
             &                                    &F-As-F       &      96.2   &      -0.3 &  &    &       \\
 AsF$_5$        & Arsenic pentafluoride              &As-F(ax)       &     1.711   &    -0.026 &  &  &    ss \\
             &                                    &As-F(eq)       &     1.656   &     0.005 &  &  &       \\
 C$_3$AsF$_9$      & Triperfluoromethylarsine           &As-C           &     2.053   &     0.028 &  &  &   ooo \\
             &                                    &C-As-C       &     100.0   &      -0.5 &  &    &       \\
 AsCl$_3$       & Arsenic trichloride                &As-Cl          &     2.161   &     0.002 &  &  &   ooo \\
             &                                    &Cl-As-Cl     &      98.7   &       1.1 &  &    &       \\
 AsBr$_3$       & Arsenic tribromide                 &As-Br          &     2.323   &    -0.008 &  &  &   iii \\
             &                                    &Br-As-Br     &      99.8   &       0.5 &  &    &       \\
 AsI$_3$        & Arsenic triiodide                  &As-I           &     2.550   &    -0.041 &  &  &   ooo \\
             &                                    &I-As-I       &     100.2   &       4.6 &  &    &       \\
 AsH$_9$Si$_3$     & Trisilylarsine                     &As-Si          &     2.353   &     0.018 &  &  &   iii \\
             &                                    &Si-As-Si     &      94.1   &      -3.3 &  &    &       \\
             &                                    &Si-H           &     1.470   &     0.024 &  &  &       \\
 C$_2$ZnH$_6$      & Dimethylzinc                       &Zn-C           &     1.930   &     0.007 &    -0.046 &    -0.031 &     U \\
             &                                    &C-Zn-C       &     180.0   &       0.0 &       0.6 &      -0.1   &       \\
 C$_4$ZnH$_1$$_0$     & Diethylzinc                        &Zn-C           &     1.950   &     0.017 &    -0.046 &    -0.022 &     U \\
             &                                    &C-C-Zn       &     114.5   &     -15.8 &       3.9 &       0.1   &       \\
             &                                    &C-Zn-C       &     180.0   &       0.2 &       0.1 &       0.0   &       \\
 C$_6$ZnH$_8$      & Cyclopentadienylmethylzinc         &C-C            &     1.438   &    -0.003 &     0.012 &     0.000 &     V \\
             &                                    &C-Zn           &     2.280   &     0.061 &    -0.023 &     0.078 &       \\
             &                                    &Zn-C(H3)       &     1.903   &     0.043 &    -0.025 &    -0.023 &       \\
 C$_6$ZnH$_1$$_4$     & Di-n-propylzinc                    &C-C-C        &     113.6   &      -0.8 &       1.8 &      -1.7   &     U \\
             &                                    &Zn-C           &     1.952   &     0.021 &    -0.044 &    -0.020 &       \\
             &                                    &Zn-C-C       &     114.5   &     -13.6 &       3.8 &       0.6   &       \\
             &                                    &C-Zn-C       &     180.0   &       0.0 &       0.0 &       0.0   &       \\
 ZnF$_2$        & Zinc difluoride                    &Zn-F           &     1.742   &    -0.002 &    -0.062 &     0.006 &   yyy \\
             &                                    &F-Zn-F       &     180.0   &       0.0 &       0.1 &       0.0   &       \\
 ZnCl$_2$       & Zinc dichloride                    &Zn-Cl          &     2.062   &     0.002 &     0.053 &     0.005 &   yyy \\
             &                                    &Cl-Zn-Cl     &     180.0   &       0.0 &       0.0 &       0.0   &       \\
 ZnBr$_2$       & Zinc dibromide                     &Zn-Br          &     2.204   &    -0.107 &     0.031 &    -0.093 &   yyy \\
             &                                    &Br-Zn-Br     &     180.0   &       0.0 &       0.0 &       0.0   &       \\
 ZnI$_2$        & Zinc diiodide                      &Zn-I           &     2.401   &     0.003 &    -0.019 &    -0.055 &   yyy \\
             &                                    &I-Zn-I       &     180.0   &       0.0 &      -0.1 &       0.0   &       \\
 C$_2$CdH$_6$      & Dimethylcadmium                    &Cd-C           &     2.112   &    -0.077 &  &  &   xxx \\
 C$_4$CdN$_4$S$_4$    & Cd(NCS)3(SCN) (=)                  &Cd-N           &     2.240   &    -0.022 &  &  &     W \\
             &                                    &Cd-S           &     2.570   &    -0.001 &  &  &       \\
 CdF$_2$        & Cadmium difluoride                 &Cd-F           &     1.970   &    -0.004 &  &  &   ooo \\
 CdCl$_2$       & Cadmium dichloride                 &Cd-Cl          &     2.210   &     0.015 &  &  &   ooo \\
\hline
\end{tabular}
\end{center}
\end{table}
\clearpage

\begin{table}
\caption{\label{geotabv}Comparison of Calculated and Observed Geometries for 
MNDO, AM1, and PM3 (contd.)}
\begin{center}
\compresstable
\begin{tabular}{llrrrrrr}
 Empirical  & Chemical Name &  Geometric &  Exp. & \multicolumn{3}{c}{Errors} & \\
  Formula   &               &  Variable &        & PM3  & MNDO  &  AM1 & Ref.\\
\hline
 CdBr$_2$       & Cadmium dibromide                  &Cd-Br          &     2.394   &    -0.034 &  &  &     X \\
 CdI$_2$        & Cadmium diiodide                   &Cd-I           &     2.550   &     0.038 &  &  &   ooo \\
 HgH         & Mercury hydride                    &Hg-H           &     1.740   &    -0.044 &    -0.190 &    -0.069 &     a \\
 HgO         & Mercury oxide                      &Hg-O           &     1.840   &     0.001 &     0.042 &     0.222 &   www \\
 C$_2$HgH$_3$N     & Methylmercuric cyanide             &Hg-CN          &     2.082   &    -0.055 &    -0.148 &    -0.070 &     Y \\
             &                                    &Hg-CH3         &     2.051   &     0.056 &    -0.069 &    -0.005 &       \\
             &                                    &C-N            &     1.141   &     0.016 &     0.030 &     0.026 &       \\
 HgF         & Mercury fluoride                   &Hg-F           &     1.890   &     0.010 &    -0.019 &     0.003 &   www \\
 HgF$_2$        & Mercury difluoride                 &Hg-F           &     1.960   &    -0.032 &    -0.082 &    -0.052 &   www \\
 C$_2$HgF$_6$      & Ditrifluoromethyl mercury          &Hg-C           &     2.101   &    -0.003 &     0.162 &     0.110 &     Z \\
             &                                    &C-Hg-C       &     180.0   &       0.0 &       0.3 &       0.2   &       \\
 HgCl        & Mercury chloride                   &Hg-Cl          &     2.230   &    -0.036 &     0.049 &     0.010 &     a \\
 CHgH$_3$Cl     & Methylmercuric chloride            &Hg-C           &     2.052   &     0.060 &    -0.070 &    -0.003 &   xxx \\
             &                                    &Hg-Cl          &     2.285   &    -0.031 &     0.006 &    -0.031 &       \\
 HgCl$_2$       & Mercury dichloride                 &Hg-Cl          &     2.252   &    -0.007 &     0.016 &    -0.013 &    AA \\
 HgBr        & Mercury bromide                    &Hg-Br          &     2.330   &    -0.116 &     0.038 &    -0.101 &   www \\
 CHgH$_3$Br     & Methylmercuric bromide             &Hg-C           &     2.062   &     0.028 &    -0.079 &    -0.012 &   iii \\
             &                                    &Hg-Br          &     2.405   &    -0.114 &    -0.021 &    -0.148 &       \\
             &                                    &C-H            &     1.095   &    -0.008 &     0.013 &     0.013 &       \\
             &                                    &Hg-C-H       &     109.6   &      -0.1 &       0.1 &      -1.8   &       \\
 HgBr$_2$       & Mercury dibromide                  &Hg-Br          &     2.440   &    -0.215 &    -0.077 &    -0.190 &    BB \\
 HgI         & Mercury iodide                     &Hg-I           &     2.490   &    10.156 &    -0.024 &     0.036 &   www \\
 CHgH$_3$I      & Methylmercuric iodide              &Hg-C           &     2.069   &     0.008 &    -0.082 &    -0.015 &   iii \\
             &                                    &Hg-I           &     2.588   &    -0.087 &    -0.109 &    -0.094 &       \\
 HgI$_2$        & Mercury diiodide                   &Hg-I           &     2.554   &    -0.080 &    -0.079 &    -0.058 &    CC \\
 SbH$_3$        & Stibine                            &Sb-H           &     1.707   &    -0.005 &  &  &   ooo \\
             &                                    &H-Sb-H       &      91.3   &       1.1 &  &    &       \\
 SbF$_5$        & Antimony pentafluoride             &Sb-F(ax)       &     2.430   &    -0.455 &  &  &   ooo \\
             &                                    &Sb-F(eq)       &     2.310   &    -0.348 &  &  &       \\
 C$_3$SbF$_9$      & Triperfluoromethylstibine          &Sb-C           &     2.202   &     0.007 &  &  &   ooo \\
             &                                    &C-Sb-C       &     100.0   &      -0.8 &  &    &       \\
 SbCl$_3$       & Antimony trichloride               &Sb-Cl          &     2.323   &    -0.003 &  &  &   iii \\
             &                                    &Cl-Sb-Cl     &      97.1   &      -0.1 &  &    &       \\
 SbCl$_5$       & Antimony pentachloride             &Sb-Cl(ax)      &     2.338   &     0.035 &  &  &   yyy \\
             &                                    &Sb-Cl(eq)      &     2.277   &     0.073 &  &  &       \\
 SbBr$_3$       & Antimony tribromide                &Sb-Br          &     2.490   &    -0.019 &  &  &   ooo \\
             &                                    &Br-Sb-Br     &      98.0   &       0.4 &  &    &       \\
 SbH$_9$Si$_3$     & Trisilylstibine                    &Sb-Si          &     2.555   &    -0.022 &  &  &   iii \\
             &                                    &Si-Sb-Si     &      89.0   &       5.8 &  &    &       \\
             &                                    &Si-H           &     1.470   &     0.018 &  &  &       \\
 Sb$_2$         & Antimony, dimer                    &Sb-Sb          &     2.590   &    -0.289 &  &  &    DD \\
 SeH$_2$        & Hydrogen selenide                  &SeH            &     1.460   &     0.010 &  &  &   ooo \\
             &                                    &HSeH         &      91.0   &       2.6 &  &    &       \\
 CSe         & Selenium carbide                   &Se-C           &     1.676   &    -0.085 &  &  &   ppp \\
 CSeH$_4$       & Methylselenium hydride             &CSe            &     1.959   &    -0.012 &  &  &    ss \\
             &                                    &CH             &     1.088   &     0.005 &  &  &       \\
             &                                    &SeH            &     1.473   &    -0.003 &  &  &       \\
\hline
\end{tabular}
\end{center}
\end{table}
\clearpage

\begin{table}
\caption{\label{geotabw}Comparison of Calculated and Observed Geometries for 
MNDO, AM1, and PM3 (contd.)}
\begin{center}
\compresstable
\begin{tabular}{llrrrrrr}
 Empirical  & Chemical Name &  Geometric &  Exp. & \multicolumn{3}{c}{Errors} & \\
  Formula   &               &  Variable &        & PM3  & MNDO  &  AM1 & Ref.\\
\hline
 C$_2$SeH$_6$      & Ethyl selanol (anti)               &Se-H           &     1.440   &     0.026 &  &  &    EE \\
             &                                    &Se-C           &     1.962   &    -0.006 &  &  &       \\
             &                                    &H-Se-C       &      93.5   &       6.4 &  &    &       \\
             &                                    &Se-C-C       &     108.7   &      -8.5 &  &    &       \\
 C$_2$SeH$_6$      & Dimethylselenium                   &Se-C           &     1.943   &     0.005 &  &  &    FF \\
             &                                    &C-Se-C       &      96.2   &       4.6 &  &    &       \\
 C$_4$SeH$_4$      & Selenophene                        &Se-C2          &     1.855   &     0.032 &  &  &   xxx \\
             &                                    &C5-Se-C2     &      87.8   &       0.1 &  &    &       \\
             &                                    &C2-C3          &     1.369   &    -0.025 &  &  &       \\
             &                                    &Se-C2-C3     &     111.6   &      -1.5 &  &    &       \\
             &                                    &C2-H           &     1.070   &     0.014 &  &  &       \\
             &                                    &Se-C2-H2     &     121.7   &      -0.6 &  &    &       \\
 C$_4$SeH$_8$      & Tetrahydroselenophene              &Se-C2          &     1.963   &    -0.004 &  &  &   xxx \\
             &                                    &C5-Se-C2     &      90.7   &       1.5 &  &    &       \\
             &                                    &C2-C3          &     1.549   &    -0.043 &  &  &       \\
             &                                    &Se-C2-C3     &     104.0   &      -0.1 &  &    &       \\
 SeO         & Selenium oxide                     &Se-O           &     1.663   &    -0.063 &  &  &    GG \\
 SeO$_2$        & Selenium dioxide                   &SeO            &     1.608   &    -0.003 &  &  &   ppp \\
             &                                    &OSeO         &     113.8   &      -7.2 &  &    &       \\
 SeO$_3$        & Selenium trioxide                  &Se-O           &     1.688   &    -0.150 &  &  &   iii \\
 CSeHN       & Isoselenocyanic acid (Se=C=N-H)    &Se=C           &     1.719   &    -0.079 &  &  &   iii \\
             &                                    &C=N            &     1.191   &     0.005 &  &  &       \\
             &                                    &Se=C=N       &     175.0   &       0.0 &  &    &       \\
             &                                    &N-H            &     0.990   &    -0.014 &  &  &       \\
             &                                    &C=N-H        &     140.0   &      15.1 &  &    &       \\
 CSeS        & Carbon sulfide selenide            &C-S            &     1.533   &    -0.081 &  &  &   xxx \\
             &                                    &C-Se           &     1.695   &    -0.110 &  &  &       \\
 CSeF$_2$       & Selenocarbonyl difluoride          &C=Se           &     1.743   &     0.002 &  &  &   iii \\
             &                                    &C-F            &     1.314   &     0.006 &  &  &       \\
             &                                    &F-C-F        &     107.5   &      -2.2 &  &    &       \\
 SeOF$_2$       & Selenyl fluoride                   &SeO            &     1.576   &     0.049 &  &  &    ss \\
             &                                    &SeF            &     1.730   &    -0.006 &  &  &       \\
             &                                    &FSeO         &     104.8   &      -3.4 &  &    &       \\
             &                                    &FSeF         &      92.2   &       2.3 &  &    &       \\
 SeO$_2$F$_2$      & Selenoyl fluoride                  &Se-F           &     1.685   &     0.020 &  &  &   iii \\
             &                                    &F-Se-F       &      94.1   &       1.0 &  &    &       \\
             &                                    &Se-O           &     1.575   &    -0.017 &  &  &       \\
             &                                    &O-Se-O       &     126.2   &       3.7 &  &    &       \\
 SeF$_4$        & Selenium tetrafluoride             &SeF            &     1.770   &    -0.034 &  &  &   ooo \\
             &                                    &FSeF         &     169.2   &     -28.9 &  &    &       \\
             &                                    &SeF'           &     1.680   &     0.028 &  &  &       \\
             &                                    &F'SF'        &     100.5   &      11.5 &  &    &       \\
 CSeNOF$_5$     & Pentafluoro(isocyanato)selenium    &Se-F           &     1.677   &     0.032 &  &  &   iii \\
             &                                    &Se-N           &     1.789   &     0.053 &  &  &       \\
             &                                    &N=C            &     1.260   &     0.007 &  &  &       \\
             &                                    &Se-N=C       &     116.9   &      12.3 &  &    &       \\
             &                                    &C=O            &     1.187   &    -0.020 &  &  &       \\
             &                                    &N=C=O        &     172.9   &      -0.6 &  &    &       \\
\hline
\end{tabular}
\end{center}
\end{table}
\clearpage

\begin{table}
\caption{\label{geotabx}Comparison of Calculated and Observed Geometries for 
MNDO, AM1, and PM3 (contd.)}
\begin{center}
\compresstable
\begin{tabular}{llrrrrrr}
 Empirical  & Chemical Name &  Geometric &  Exp. & \multicolumn{3}{c}{Errors} & \\
  Formula   &               &  Variable &        & PM3  & MNDO  &  AM1 & Ref.\\
\hline
 SeF$_6$        & Selenium hexafluoride              &SeF            &     1.685   &     0.005 &  &  &    HH \\
 C$_2$SeF$_6$      & Diperfluoromethyl selenide         &Se-C           &     1.960   &     0.047 &  &  &   ooo \\
             &                                    &C-Se-C       &     104.0   &      -2.8 &  &    &       \\
 SeCl$_2$       & Selenium dichloride                &Se-Cl          &     2.157   &     0.007 &  &  &   iii \\
             &                                    &Cl-Se-Cl     &      99.6   &       0.0 &  &    &       \\
 SeOCl$_2$      & Selenyl chloride                   &Se=O           &     1.614   &     0.020 &  &  &   iii \\
             &                                    &Se-Cl          &     2.205   &    -0.005 &  &  &       \\
             &                                    &O=Se-Cl      &     106.0   &      -3.0 &  &    &       \\
             &                                    &Cl-Se-Cl     &      96.9   &       1.2 &  &    &       \\
 In$_2$Se       & Indium(I) selenide                 &In-Se          &     2.650   &    -0.070 &  &  &   iii \\
             &                                    &In-Se-In     &     113.0   &      66.3 &  &    &       \\
 SeSi        & Silicon selenide                   &Se-Si          &     2.058   &    -0.084 &  &  &   xxx \\
 SeH$_6$Si$_2$     & Disilyl selenide                   &Se-Si          &     2.270   &     0.123 &  &  &   ooo \\
             &                                    &Si-Se-Si     &      97.0   &      -0.1 &  &    &       \\
 GeSe        & Germanium selenide                 &Ge-Se          &     2.135   &    -0.208 &  &  &   xxx \\
 SnSe        & Tin selenide                       &Sn-Se          &     2.326   &    -0.002 &  &  &   xxx \\
 PbSe        & Lead selenide                      &Pb-Se          &     2.402   &    -0.041 &  &  &   xxx \\
 SeHPF$_2$      & Difluorophosphine selenide         &P=Se           &     2.026   &     0.062 &  &  &   iii \\
             &                                    &P-F            &     1.557   &    -0.006 &  &  &       \\
             &                                    &Se=P-F       &     116.8   &      -1.0 &  &    &       \\
             &                                    &P-H            &     1.422   &    -0.158 &  &  &       \\
             &                                    &Se=P-H       &     118.6   &      -0.3 &  &    &       \\
             &                                    &F-P-F        &      98.1   &      -4.3 &  &    &       \\
 C$_2$Se$_2$H$_6$     & Me-Se-Se-Me                        &Se-C           &     1.950   &    -0.005 &  &  &   ooo \\
             &                                    &Se-Se          &     2.330   &     0.040 &  &  &       \\
             &                                    &Se-Se-C      &      99.0   &       1.2 &  &    &       \\
 Se$_2$OF$_1$$_0$     & Bis(pentafluoroselenium)oxide      &Se-O           &     1.679   &     0.048 &  &  &   iii \\
             &                                    &Se-O-Se      &     142.4   &     -16.4 &  &    &       \\
             &                                    &Se-F(eq)       &     1.683   &     0.008 &  &  &       \\
             &                                    &Se-F(ax)       &     1.665   &     0.031 &  &  &       \\
 C$_7$GeSe$_2$H$_1$   & Tetramethyldiselenagermacyclohexane&Se-C         &     -10.8   &      -4.2 &  &    &    II \\
             &                                    &Ge-Se        &     108.7   &       2.8 &  &    &       \\
             &                                    &Ge-Se-C      &     116.3   &      -2.1 &  &    &       \\
             &                                    &Se-Ge-Se     &    -114.7   &       0.5 &  &    &       \\
             &                                    &Ge-C'        &     100.2   &      -0.4 &  &    &       \\
             &                                    &Ge-C''        &     119.2   &      -0.5 &  &    &       \\
             &                                    &C'-Ge-C''     &     -24.2   &      -4.5 &  &    &       \\
 TeH$_2$        & Hydrogen telluride                 &Te-H           &     1.658   &     0.017 &  &  &    JJ \\
             &                                    &H-Te-H       &      90.3   &      -2.0 &  &    &       \\
 TeO$_2$        & Tellurium dioxide                  &Te-O           &     1.830   &    -0.128 &  &  &   iii \\
 CTeNOF$_5$     & Pentafluoro(isocyanato)tellurium   &Te-F           &     1.826   &    -0.002 &  &  &   iii \\
             &                                    &Te-N           &     1.859   &    -0.045 &  &  &       \\
             &                                    &Te-N=C       &     126.5   &      53.6 &  &    &       \\
 TeF$_6$        & Tellurium hexafluoride             &TeF            &     1.815   &     0.001 &  &  &   iii \\
 TeCl$_4$       & Tellurium tetrachloride            &Te-Cl          &     2.330   &     0.072 &  &  &    tt \\
 TeBr$_2$       & Tellurium dibromide                &Te-Br          &     2.510   &     0.000 &  &  &   ooo \\
             &                                    &Br-Te-Br     &      98.0   &       1.5 &  &    &       \\
\hline
\end{tabular}
\end{center}
\end{table}
\clearpage

\begin{table}
\caption{\label{geotaby}Comparison of Calculated and Observed Geometries for 
MNDO, AM1, and PM3 (contd.)}
\begin{center}
\compresstable
\begin{tabular}{llrrrrrr}
 Empirical  & Chemical Name &  Geometric &  Exp. & \multicolumn{3}{c}{Errors} & \\
  Formula   &               &  Variable &        & PM3  & MNDO  &  AM1 & Ref.\\
\hline
 TeBr$_4$       & Tellurium tetrabromide             &Te-Br          &     2.680   &    -0.149 &  &  &    tt \\
 In$_2$Te       & Indium(I) telluride                &In-Te          &     2.840   &    -0.004 &  &  &   iii \\
             &                                    &In-Te-In     &      99.0   &       6.0 &  &    &       \\
 GeTe        & Germanium telluride                &Ge-Te          &     2.340   &    -0.338 &  &  &   ppp \\
 SnTe        & Tin telluride                      &Sn-Te          &     2.523   &     0.049 &  &  &   ppp \\
 PbTe        & Lead telluride                     &Pb-Te          &     2.595   &     0.141 &  &  &   xxx \\
 C$_9$TeH$_1$$_3$PS   & MeOPh-Te-S-P(S)(OMe)2              &Te-C           &     2.114   &    -0.027 &  &  &    KK \\
             &                                    &Te-S           &     2.444   &     0.234 &  &  &       \\
             &                                    &C-Te-S       &      94.9   &      10.8 &  &    &       \\
             &                                    &S-P            &     2.051   &     0.052 &  &  &       \\
             &                                    &P-S-Te       &     103.3   &     -13.9 &  &    &       \\
             &                                    &P=S            &     1.933   &     0.061 &  &  &       \\
             &                                    &S=P-S        &     107.9   &      -1.4 &  &    &       \\
 Te$_2$         & Tellurium, dimer                   &Te-Te          &     2.560   &     0.145 &  &  &   ppp \\
 Te$_2$OF$_1$$_0$     & Bis(pentafluorotellurium)oxide     &Te-O           &     1.832   &    -0.057 &  &  &   iii \\
             &                                    &Te-O-Te      &     145.5   &      10.4 &  &    &       \\
             &                                    &Te-Feq         &     1.820   &    -0.009 &  &  &       \\
             &                                    &Te-Fax         &     1.799   &     0.015 &  &  &       \\
             &                                    &Fax-Te-Feq   &      89.9   &       4.4 &  &    &       \\
 C$_3$BiH$_9$      & Trimethylbismuth                   &Bi-C           &     2.270   &    -0.004 &  &  &   ooo \\
             &                                    &C-Bi-C       &      96.7   &       0.3 &  &    &       \\
 BiCl$_3$       & Bismuth trichloride                &Bi-Cl          &     2.425   &    -0.006 &  &  &   yyy \\
             &                                    &Cl-Bi-Cl     &      97.3   &       2.7 &  &    &       \\
 BiBr$_3$       & Bismuth tribromide                 &Bi-Br          &     2.630   &    -0.026 &  &  &   ooo \\
             &                                    &Br-Bi-Br     &     100.0   &      -1.4 &  &    &       \\
\hline
\end{tabular}
\end{center}
\end{table}
\clearpage

\begin{center} {bf References} \end{center}

\begin{description}
\item{   a: } M.\ W.\ Chase, C.\ A.\ Davies, J.\ R.\ Downey, D.\ R.\ Frurip, R.\ A.\ McDonald,
       A.\ N.\ Syverud, JANAF Thermochemical Tables, Third Edition,
       J.\ Phys.\ Chem.\ Ref.\ Data 14, Suppl.\ 1 (1985).
  
\item{   b: } G.\ Herzberg, ``Molecular Spectra and Molecular Structure III.\ Electronic
       Spectra and Electronic Structure of Polyatomic Molecules'',
       Van Nostrand, New York, N.Y., 1966.
  
\item{   c: } A.\ G.\ Maki, R.\ A.\ Toth, J.\ Mol.\ Spectr., 17, 136 (1965).
  
\item{   d: } W.\ M.\ Stigliani, V.\ W.\ Laurie, J.\ C.\ Li, J.\ Chem.\ Phys., 62, 1890 (1975).
  
\item{   e: } R.\ A.\ Bonham, L.\ S.\ Bartell, J.\ Am.\ Chem.\ Soc., 81, 3491 (1959).
  
\item{   f: } O.\ Bastiansen, F.\ N.\ Fritsch, K.\ Hedberg, Acta.\ Crystallogr., 17, 538
       (1964).
  
\item{   g: } D.\ R.\ Lide, D.\ Christensen, J.\ Chem.\ Phys., 35, 1374 (1961).
  
\item{   h: } J.\ H.\ Callomon, B.\ P.\ Stoicheff, Can.\ J.\ Phys., 35, 373 (1957).
  
\item{   i: } A.\ Almenningen, O.\ Bastiansen, M.\ Traetteberg, Acta Chem.\ Scand., 15, 1557
       (1961).
  
\item{   j: } T.\ Fukuyama, K.\ Kuchitsu, Y.\ Morino, Bull.\ Chem.\ Soc.\ Jap., 42, 379 (1969).
  
\item{   k: } K.\ W.\ Cox, M.\ D.\ Harmony, G.\ Nelson, K.\ B.\ Wiburg, J.\ Chem.\ Phys., 50,
       1976 (1969).
  
\item{   l: } W.\ Haugen, M.\ Traetteberg, Acta Chem.\ Scand., 20, 1226 (1966).
  
\item{   m: } A.\ Almenningen, I.\ M.\ Anfinsen, A.\ Haaland, Acta Chem.\ Scand., 24, 43
       (1970).
  
\item{   n: } S.\ Meiboom, L.\ C.\ Snyder, J.\ Chem.\ Phys., 52, 3857 (1970).
  
\item{   o: } L.\ R.\ H.\ Scharpen, V.\ W.\ Laurie, J.\ Chem.\ Phys., 39, 1972 (1963).
  
\item{   p: } D.\ R.\ Lide, J.\ Chem.\ Phys., 33, 1519 (1960).
  
\item{   q: } B.\ W.\ McCelland, K.\ Hedberg, J.\ Am.\ Chem.\ Soc., 109, 7304 (1987)
  
\item{   r: } K.\ Tamagawa, T.\ Iijima, M.\ Kimura, J.\ Mol.\ Struct., 30, 243 (1976).
  
\item{   s: } P.\ A.\ Baron, R.\ D.\ Brown, F.\ R.\ Burden, P.\ J.\ Domaille, J.\ E.\ Kent,
       J.\ Mol.\ Spectrosc., 43, 401 (1970).
  
\item{   t: } J.\ F.\ Chiang, S.\ H.\ Bauer, J.\ Am.\ Chem.\ Soc., 91, 1898 (1969).
  
\item{   u: } O.\ Bastiansen, L.\ Fernholt, H.\ M.\ Seip, H.\ Kambara, K.\ Kuchitsu, J.\ Mol.
       Struct., 18, 163 (1973).
  
\item{   v: } G.\ Herzberg, ``Molecular Spectra and Molecular Structure I.\ Spectra of
       Diatomic Molecules'', 2nd ed, Van Nostrand, New York, N.Y., 1950.
  
\item{   w: } K.\ Takagi, T.\ Oka, J.\ Phys.\ Soc.\ Jap., 18, 1174 (1963).
  
\item{   x: } R.\ M.\ Lees, J.\ G.\ Baker, J.\ Chem.\ Phys., 48, 5299 (1968).
  
\item{   y: } A.\ P.\ Cox, L.\ F.\ Thomas, J.\ Sheridan, Spectrochim.\ Acta, 15, 542 (1959).
  
  
\item{  aa: } U.\ Blukis, P.\ H.\ Kasai, R.\ J.\ Myers, J.\ Chem.\ Phys., 38, 2753 (1963).
  
\item{  bb: } K.\ Kuchitsu, T.\ Fukuyama, Y.\ Marino, J.\ Mol.\ Struct., 1, 463 (1968).
  
\item{  cc: } R.\ Nelson, L.\ Pierce, J.\ Mol.\ Spectrosc.\ 18, 344 (1965).
  
\item{  dd: } B.\ Bak, D.\ Christensen, W.\ B.\ Dixon, L.\ Hansen-Nygarrd,
       J.\ Rastrup-Anderson, M.\ Shottlander, J.\ Mol.\ Spectrosc., 9, 124 (1962).
  
\item{  ee: } W.\ C.\ Oelfke, W.\ Gordy, J.\ Chem.\ Phys., 51, 5336 (1969).
  
\item{  ff: } G.\ H.\ Kwei, R.\ F.\ Curl, J.\ Chem.\ Phys., 32, 1592 (1960).
  
\item{  gg: } R.\ F.\ Curl, Jr.\ J.\ Chem.\ Phys., 30, 1529 (1959).
  
\item{  hh: } J.\ Trotter, Acta Crystallogr., 13, 86 (1960).
  
\item{  ii: } D.\ G.\ Lister, J.\ K.\ Tyler, J.\ H.\ Hog, N.\ W.\ Larson, J.\ Mol.\ Struct., 23,
       253 (1974).
  
\item{  jj: } D.\ R.\ Lide, J.\ Chem.\ Phys., 27, 343 (1957).
  
\item{  kk: } C.\ C.\ Costain, J.\ Chem.\ Phys., 29, 864 (1958).
  
\item{  ll: } J.\ E.\ Wollrab, V.\ W.\ Laurie, J.\ Chem.\ Phys., 48, 5058 (1968).
  
\item{  mm: } J.\ E.\ Wollrab, V.\ W.\ Laurie, J.\ Chem.\ Phys., 51, 1580 (1969).
  
\item{  nn: } D.\ R.\ Johnson, F.\ J.\ Lovas, W.\ H.\ Kirchhoff, J.\ Phys.\ Chem.\ Ref.\ Data,
       1, 1011 (1972).
  
\item{  oo: } J.\ O.\ Morley, J.\ Chem.\ Soc.\ Perkin Trans.\ II, (1987), (in press).
  
\item{  pp: } S.\ Cradock, P.\ B.\ Liescheski, D.\ W.\ H.\ Rankin, H.\ E.\ Robertson, J.\ Am.
       Chem.\ Soc., 110, 2758 (1988).
  
\item{  qq: } J.\ K.\ Tyler, J.\ Mol.\ Spectrosc., 11, 39 (1963).
  
\item{  rr: } J.\ E.\ Lancaster, B.\ J.\ Stoicheff, Can.\ J.\ Phys., 34, 1016 (1965).
  
\item{  ss: } J.\ H.\ Callomon, E.\ Hirota, K.\ Kuchitsu, W.\ J.\ Lafferty, A.\ G.\ Maki,
       C.\ S.\ Pote, ``Structure Data on Free Polyatomic Molecules'',
       Landolt-Bornstein, New Series, Group II, Vol.\ 7, Springer, Berlin (1976).
  
\item{  tt: } L.E.\ Sutton, ``Tables of Interatomic Distances and Configurations in
       Molecules and Ions'', Special Publication No.\ 11 + 18, Chem.\ Soc.,
       London (1958), (1965).
  
\item{  uu: } R.\ W.\ Kilb, J.\ Chem.\ Phys., 23, 1736 (1955).
  
\item{  vv: } B.\ Bak, D.\ Christensen, L.\ Hansen-Nygarrd, J.\ Rastrup-Anderson,
       J.\ Mol.\ Spectrosc., 7, 58 (1961).
  
\item{  ww: } J.\ K.\ G.\ Watson, J.\ Mol.\ Spectrosc., 48, 479 (1973).
  
\item{  xx: } D.\ W.\ W.\ Anderson, D.\ W.\ H.\ Rankin, A.\ Robertson, J.\ Mol.\ Struct., 14
       385 (1972).
  
\item{  yy: } P.\ Rosmus, H.\ Stafast, H.\ Bock, Chem.\ Phys.\ Lett., 34, 275 (1975).
  
\item{  zz: } G.\ Winnewisser, M.\ Winnewisser, W.\ Gordy, J.\ Chem.\ Phys., 49, 3465 (1968).
  
\item{ aaa: } R.\ Sutter, H.\ Driezler, F.\ Z.\ Rudolph, Naturforschung A, 20 1676 (1965).
  
\item{ bbb: } J.\ Donohue, A.\ Caron, E.\ Goldish, J.\ Am.\ Chem.\ Soc., 83, 3748 (1961).
  
\item{ ccc: } A.\ Caron, J.\ Donohue, Acta Crystallogr., 18, 562 (1965).
  
\item{ ddd: } G.\ A.\ Kuipers, D.\ F.\ Smith, A.\ N.\ Nielsen, J.\ Chem.\ Phys., 25, 275 (1956).
  
\item{ eee: } J.\ L.\ Duncan, J.\ Mol.\ Struct., 6, 447 (1970).
  
\item{ fff: } J.\ K.\ Tyler, J.\ Sheridan, Trans.\ Faraday Soc., 59, 2661 (1963).
  
\item{ ggg: } J.\ L.\ Carlos, R.\ R.\ Karl, S.\ H.\ Bauer, J.\ Chem.\ Soc., Faraday Trans.\ 2,
       177 (1974).
  
\item{ hhh: } T.\ Ogata, K.\ Fujii, M.\ Yoshikawa, F.\ Hirota, J.\ Am.\ Chem.\ Soc., 109,
       7639 (1987).
  
\item{ iii: } J.\ H.\ Callomon, E.\ Hirota, T.\ Iijima, K.\ Kuchitsu, W.\ J.\ Lafferty,
       ``Structure Data on Free Polyatomic Molecules'', Landolt-Bornstein,
       New Series, Group II, Vol.\ 15, Springer, Berlin (1977).
  
\item{ jjj: } A.\ S.\ Rodgers, J.\ Chao, R.\ C.\ Wilhoit, B.\ J.\ Zwolinski, J.\ Phys.\ Chem.
       Ref.\ Data, 3, 117 (1974).
  
\item{ kkk: } Private Communication by Dr Dave Dixon, DuPont.
  
\item{ lll: } H.\ Jones, M.\ Takami, J.\ Sheridan, Z.\ Naturforsch., A33, 156 (1978).
  
\item{ mmm: } S.\ S.\ Chen, A.\ C.\ Wilhoit, B.\ J.\ Zwolinski, J.\ Phys.\ Chem.\ Ref.\ Data 5,
       571 (1975).
  
\item{ nnn: } I.\ Hargittai, Acta Chem.\ (Budapest), 49, 351 (1969).
  
\item{ ooo: } ``Structural Inorganic Chemistry'', A.\ F.\ Wells, Clarendon Press, Oxford, 1984
  
\item{ ppp: } K.\ P.\ Huber, G.\ Herzberg, ``Molecular Spectra and Molecular Structure'',
       IV.\ ``Constants for Diatomic Molecules'', Van Nostrand, Reinhold,
       New York, (1979).
  
\item{ qqq: } H.\ Jones, J.\ Sheridan, O.\ Stiefvater, Z.\ Naturforsch., A32, 866 (1977).
  
\item{ rrr: } N.\ P.\ C.\ Westwood, W.\ Lewis-Bevan, M.\ C.\ L.\ Gerry,
       J.\ Mol.\ Spectr., 136, 93 (1989).
  
\item{ sss: } Y.\ Niide, I.\ Ohkoshi, J.\ Mol.\ Spectr., 136, 17 (1989).
  
\item{ ttt: } K.\ Hagen, B.\ Lunelli, J.\ Phys.\ Chem., 93, 1326 (1989).
  
\item{ uuu: } U.\ Anderson, N.\ Heineking, H.\ Dreizler, J.\ Mol.\ Struct.\ 137, 296 (1989).
  
\item{ vvv: } N.\ N.\ Greenwood, A.\ Earnshaw, ``Chemistry of the Elements'', Pergamon Press,
       Oxford, p.976 (1984).
  
\item{ www: } D.\ R.\ Stull, H.\ Prophet, Natl.\ Stand., Ref.\ Data Ser.\ (U.S., Natl.\ Bur.
       Stand.) NSRDS-NBS 37, 1971.
  
\item{ xxx: } M.\ D.\ Harmony, V.\ W.\ Laurie, R.\ L.\ Kuczkowsky, R.\ H.\ Schwendeman,
       D.\ A.\ Ramsay, F.\ J.\ Lovas, W.\ J.\ Lafferty, A.\ G.\ Maki, J.\ Phys.
       Chem.\ Ref.\ Data 8, 3 (1979).
  
\item{ yyy: } M.\ Hargittai, Coord.\ Chem.\ Rev., 91, 35 (1988).
  
\item{ zzz: } L.\ J.\ Guggenberger, R.\ E.\ Rundle, J.\ Am.\ Chem.\ Soc., 86, 5344 (1964).
  
\item{aaaa: } G.\ Herzberg, J.\ W.\ C.\ Johns, Proc.\ R.\ Soc.\ London, Ser.\ A., 298, 142 (1967).
  
\item{bbbb: } D.\ Christen, D.\ G.\ Lister, J.\ Sheridom, J.\ Chem.\ Soc., Faraday Trans.
      2, 70, 1953 (1974).
  
\item{cccc: } S.\ G.\ W.\ Ginn, J.\ K.\ Kennedy, J.\ Overend, J.\ Chem.\ Phys., 48, 1571 (1968).
  
\item{dddd: } L.\ A.\ Curtiss, J.\ A.\ Pople, J.\ Chem.\ Phys., 90, 4314 (1989).
  
\item{eeee: } C.\ H.\ Chang, R.\ F.\ Porter, S.\ H.\ Bauer, Inorg.\ Chem., 8, 1689 (1969).
  
\item{ffff: } W.\ Harshbarger, G.\ Lee, R.\ F.\ Porter, S.\ H.\ Bauer, Inorg.\ Chem.\ 8, 1683
       (1969).
  
\item{gggg: } R.\ A.\ Beaudet, R.\ L.\ Poynter, J.\ Chem.\ Phys., 53, 1899 (1970).
  
\item{hhhh: } D.\ Schwock, A.\ B.\ Burg, R.\ A.\ Beaudet, Inorg.\ Chem.\ 16, 3219 (1977).
  
\item{iiii: } E.\ Switkes, I.\ R.\ Epstein, J.\ A.\ Tossell, R.\ M.\ Stevens, W.\ N.\ Lipscomb,
       J.\ Am.\ Chem.\ Soc., 92, 3837 (1970).
  
\item{jjjj: } nI.\ R.\ Epstein, J.\ A.\ Tossell, E.\ Switkes, I.\ R.\ M.\ Stevens,
       W.\ N.\ Lipscomb, Inorg.\ Chem., 10, 171, (1971).
  
  
\item{llll: } M.\ J.\ Goode, A.\ J.\ Downs, C.\ R.\ Pulham, D.\ W.\ H.\ Rankin, H.\ E.\ Robertson,
       J.\ Chem.\ Soc.\ Chem.\ Comm., 768 (1988).
  
\item{mmmm: } S.\ N.\ Vempati, W.\ E.\ Jones, J.\ Mol.\ Spect., 127, 232 (1988).
  
\item{nnnn: } J.\ H.\ Meadows, H.\ F.\ Schaefer III, J.\ Am.\ Chem.\ Soc., 98, 4383 (1976).
  
\item{oooo: } J.\ Nakagawa, Y.\ Shiko, M.\ Hayashi, J.\ Mol.\ Spect., 122, 1 (1987).
  
\item{pppp: } M.\ Hayashi, S.\ Kaminaka, M.\ Fujitake, S.\ Miyazaki, J.\ Mol.\ Spect.,
      135, 289 (1989).
  
\item{qqqq: } R.\ W.\ Davis, M.\ C.\ L.\ Gerry, J.\ Mol.\ Spectrosc.\ 60, 117 (1976).
  
\item{rrrr: } A.\ P.\ Cox, I.\ C.\ Ewart, T.\ R.\ Gayton, J.\ Mol.\ Spect., 125, 76 (1987).
  
\item{ssss: } K.\ C.\ Shotton, A.\ G.\ Lee, W.\ J.\ Jones, J.\ Raman Spectrosc., 1, 243 (1973).
  
\item{tttt: } R.\ A.\ Eades, D.\ A.\ Dixon, J.\ Chem.\ Phys., 72, 3309 (1980).
  
\item{uuuu: } J.\ Durig, M.\ M.\ Chen, Y.\ S.\ Li, J.\ Phys.\ Chem., 77, 227 (1973), and
      E.\ C.\ Thomas, V.\ W.\ Laurie, J.\ Chem.\ Phys., 50, 3512 (1969).
  
\item{vvvv: } J.\ L.\ Hencher, F.\ J.\ Mistoe, Can.\ J.\ Chem., 53, 3542 (1975).
  
\item{wwww: } J.\ D.\ Murdoch, D.\ W.\ H.\ Rankin, B.\ Beagley, J.\ Mol.\ Struct.,
       31, 291, (1976).
  
\item{xxxx: } P.\ Groner, G.\ M.\ Attia, A.\ B.\ Mohamad, J.\ F.\ Sullivan,
       Y.\ S.\ Li, J.\ R.\ Durig, J.\ Chem.\ Phys., 91, 1434 (1989).
  
\item{yyyy: } S.\ Saito, Y.\ Endo, E.\ Hirota, J.\ Mol.\ Spect., 116, 499 (1986).
  
\item{zzzz: } J.\ E.\ Drake, R.\ T.\ Hemmings, J.\ L.\ Hencher, F.\ M.\ Mustoe, Q.\ Shem,
       J.\ Chem.\ Soc., Dalton Trans., 394, (1976).
  
\item{   A: } J.\ Durig, K.\ L.\ Hellams, J.\ Mol.\ Struct., 29, 349 (1975).
  
\item{   B: } J.\ E.\ Drake, H.\ L.\ Hencher, Q.\ Shem, Can.\ J.\ Chem., 55, 1104 (1977).
  
\item{   C: } Y.\ Morino, Y.\ Nakamura, T.\ Iijima, J.\ Chem.\ Phys., 32, 643 (1960).
  
\item{   D: } G.\ Schultz, J.\ Tremmel, I.\ Hargittai, N.\ D.\ Kagramanov, A.\ K.\ Maltsev,
       O.\ M.\ Nefedov, J.\ Mol.\ Struct., 82, 107 (1982).
  
\item{   E: } G.\ G.\ B.\ Souza, J.\ D.\ Wieser, J.\ Mol.\ Struct., 25, 442 (1975).
  
\item{   F: } L.\ V.\ Vilkov, N.\ A.\ Tarensenko, Zh.\ Strukt.\ Khim., 10, 102 (1969).
  
\item{   G: } G.\ R.\ Wilkinson, M.\ K.\ Wilson, J.\ Chem.\ Phys., 25, 784 (1956).
  
\item{   H: } B.\ Beagley, K.\ McAloon, J.\ M.\ Freeman, Acta Crystallogr.,
       Sect.\ B, B30, 444 (1974).
  
\item{   I: } H.\ C.\ Clark, S.\ G.\ Furnival, J.\ T.\ Kwon, Can.\ J.\ Chem., 41, 2889 (1963).
  
\item{   J: } H.\ Fujii, M.\ Kimura, Bull.\ Chem.\ Soc.\ Japan, 44, 2643 (1970)
  
\item{   K: } H.\ Fujii, M.\ Kimura, Bull.\ Chem.\ Soc.\ Japan, 43, 1933 (1970)
  
\item{   L: } H.\ A.\ Skinner, L.\ E.\ Sutton, Trans.\ Farad.\ Soc., 44, 164 (1944).
  
\item{   M: } E.\ Coop, L.\ E.\ Sutton, J.\ Chem.\ Soc., 1269 (1938).
  
\item{   N: } T.\ Oyamada, T.\ Iijima, M.\ Kimura, Bull.\ Chem.\ Soc.\ Japan, 44, 2638 (1971).
  
\item{   O: } F.\ J.\ Lovas, E.\ Tiemann, J.\ Phys.\ Chem.\ Ref.\ Data 3, 609 (1974).
  
\item{   P: } A.\ V.\ Demidov, A.\ G.\ Gershikov, E.\ Z.\ Zasorin, V.\ P.\ Spiridonov,
       A.\ A.\ Ivanov, J.\ Struct.\ Chem., 24, 9 (1983).
  
\item{   Q: } I.\ Hargittai, J.\ Mol.\ Struct., 100, 129 (1983).
  
\item{   R: } H.\ A.\ Skinner, L.\ E.\ Sutton, Trans.\ Farad.\ Soc., 36, 1209 (1940).
  
\item{   S: } O.\ I.\ Osman, B.\ J.\ Whitaker, N.\ P.\ C.\ Simmons, D.\ R.\ M.\ Walton,
      J.\ F.\ Nixon, H.\ W.\ Krojo, J.\ Mol.\ Spect., 137, 373 (1989).
  
\item{   T: } G.\ A.\ McRae, M.\ C.\ L.\ Gerry, M.\ Wong, I.\ Ozier, E.\ A.\ Cohen, J.\ Mol.
       Spect., 123, 321 (1987).
  
\item{   U: } A.\ Almenningen, T.\ U.\ Halgaker, A.\ Haaland, S.\ Samdal,
       Acta Chem.\ Scand., Part A.\ A36, 159 (1982).
  
\item{   V: } A.\ Haaland, S.\ Samdal, R.\ Seip, J.\ Organomet.\ Chem., 153, 187 (1978).
  
\item{   W: } K.\ Ozutsumi, T.\ Takamuka, S.\ Ishiguko, H.\ Ohtaki, Bull.\ Chem.\ Soc.\ Japan,
       62, 1875 (1989).
  
\item{   X: } V.\ M.\ Petrov, A.\ N.\ Utkin, G.\ V.\ Girichev, A.\ A.\ Ivanov, Zh.\ Strukt.
       Khim., 26, 52 (1985).
  
\item{   Y: } N.\ Iwasaki, Bull.\ Chem.\ Soc.\ Japan, 49, 2735 (1976).
  
\item{   Z: } H.\ Oberhammer, J.\ Mol.\ Struct., 48, 389 (1978).
  
\item{  AA: } K.\ Kashiwabara, S.\ Konaka, N.\ M.\ Kimura, Bull.\ Chem.\ Soc.\ Japan,
       46, 410 (1973).
  
\item{  BB: } P.\ W.\ Allen, L.\ E.\ Sutton, Acta.\ Crystallogr.\ 3, 46 (1950).
  
\item{  CC: } V.\ P.\ Spiridonov, A.\ G.\ Gershikov, B.\ S.\ Butayev,
       J.\ Mol.\ Struct., 52, 53 (1979).
  
\item{  DD: } K.\ Balasubramanian, J.\ Li, J.\ Mol.\ Spect., 135, 169 (1989).
  
\item{  EE: } J.\ Nakagawa, H.\ Okutani, M.\ Hayashi, J.\ Mol.\ Spectrosc., 94, 410 (1982).
  
\item{  FF: } J.\ F.\ Beecher, J.\ Mol.\ Spectrosc., 21, 414 (1966).
  
\item{  GG: } E.\ H.\ Fink, K.\ D.\ Setzer, J.\ Mol.\ Spect., 125, 66 (1987).
  
\item{  HH: } L.\ S.\ Bartell, A.\ Jin, J.\ Mol.\ Struct., 118, 47 (1984).
  
\item{  II: } S.\ Tomoda, M.\ Shimoda, M.\ Sanami, Y.\ Takeuchi, Y.\ Litaka, J.\ Chem.\ Soc.
       Chem.\ Comm., 1304 (1989).
  
\item{  JJ: } N.\ K.\ Moncur, P.\ D.\ Willson, T.\ H.\ Edwards, J.\ Mol.\ Spectrosc., 52,
       380 (1974).
  
\item{  KK: } S.\ Husebye, K.\ Maartmann-Moe, O.\ Mikalsen, Acta Chem. Scand., 43, 868 (1989)
\end{description}

