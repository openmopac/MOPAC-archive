\section{Use of \comp{SADDLE} Calculation}
\index{SADDLE!description|(}
The \comp{SADDLE} technique is used for locating a transition state, given two
geometries, one on each side of the transition state. In order for the
\comp{SADDLE} technique to work, the Z-matrix must be specified as follows:

\begin{itemize}
\item The first geometry, defining one geometry is defined as usual.   If
symmetry data is supplied, it should follow the first geometry.  After the
geometry, or geometry and symmetry data, there should be a blank line to
indicate the end of the data
\item The second geometry should then be specified. There must be a one-to-one
correspondence of the atoms in the second geometry to those of the first
geometry.
\end{itemize} 

From this specification, it follows that if two molecules react to form one
molecule, then the first geometry must contain all the atoms of the two
molecules. The easiest way of defining such a geometry is to define one
molecule, then have an unusually long bond-length from one atom in the first
molecule to the first atom in the second molecule.  The two molecules together
form the first geometry.  Likewise, if a molecule decomposes, e.g.\
C$_2$H$_5$OH $\rightarrow$ C$_2$H$_4$ + H$_2$O, every atom in the product must
be defined in the same order as the atoms in the reactant.

An example of a data-set for a \comp{SADDLE} calculation, modeling the
\index{Ethyl radical} ethyl radical hydrogen migration from one methyl group to
the other is given in Figure~\ref{c2h5s}. \index{Data! for ethyl radical
(SADDLE calcn.)}

\begin{figure}
\begin{makeimage}
\end{makeimage}
\begin{verbatim}
Line  1:   UHF  SADDLE
Line  2:   ETHYL RADICAL HYDROGEN MIGRATION
Line  3:
Line  4:    C   0.000000 0    0.000000 0    0.000000 0   0 0 0
Line  5:    C   1.479146 1    0.000000 0    0.000000 0   1 0 0
Line  6:    H   1.109475 1  111.328433 1    0.000000 0   2 1 0
Line  7:    H   1.109470 1  111.753160 1  120.288410 1   2 1 3
Line  8:    H   1.109843 1  110.103163 1  240.205278 1   2 1 3
Line  9:    H   1.082055 1  121.214083 1   38.110989 1   1 2 3
Line 10:    H   1.081797 1  121.521232 1  217.450268 1   1 2 3
Line 11:    O   0.000000 0    0.000000 0    0.000000 0   0 0 0
Line 12:    C   0.000000 0    0.000000 0    0.000000 0   0 0 0
Line 13:    C   1.479146 1    0.000000 0    0.000000 0   1 0 0
Line 14:    H   1.109475 1  111.328433 1    0.000000 0   2 1 0
Line 15:    H   1.109470 1  111.753160 1  120.288410 1   2 1 3
Line 16:    H   2.109843 1   30.103163 1  240.205278 1   2 1 3
Line 17:    H   1.082055 1  121.214083 1   38.110989 1   1 2 3
Line 18:    H   1.081797 1  121.521232 1  217.450268 1   1 2 3
Line 19:    O   0.000000 0    0.000000 0    0.000000 0   0 0 0
Line 20:
\end{verbatim}
\caption{\label{c2h5s} Example of data for \comp{SADDLE} calculation}
\end{figure}

Details of the mathematics of \comp{SADDLE} appeared  in  print  in  1984 (M.\
J.\ S.\ Dewar,  E.\ F.\ Healy,  J.\ J.\ P.\ Stewart,  {\em J.\ Chem.\ Soc.\ 
Faraday Trans.\ II}, 3, 227, (1984)), so only a  superficial  description 
will  be given here.

The main steps in the saddle calculation are as follows:
\begin{enumerate}
\item The heats of formation of both systems are calculated.

\item A vector $R$ of length $3N$ defining the difference  between  the two
geometries in Cartesian coordinates is calculated.  The scalar of the vector is
called the \comp{BAR}, and represents the distance between the two geometries.

\item  The \comp{BAR}  is  reduced  by  some fraction, normally about 5 to 15
percent. Normally, the default step is used, but this can be changed by use of
\comp{BAR=$n.nn$}, where $n.nn$ is the fraction.  \comp{BAR=0.15} is the
default.

\item  The geometry of lower energy is identified; call this $G$.

\item Geometry $G$ is optimized, subject to  the  constraint  that  it 
maintains  a constant distance $P$ from the other geometry.

\item If the newly-optimized geometry is higher in  energy  than  the other 
geometry,  and the last two steps involved the same geometry  moving,  make 
the  other geometry $G$ without modifying $P$, and go to 5.

\item Otherwise go back to 2.
\end{enumerate}

The mechanism of 5 involves the coordinates of the moving  geometry being 
perturbed  by  an  amount equal to the product of the discrepancy between the
calculated and required $P$ and the vector $R$.

\comp{SADDLE} works with Cartesian coordinates, so before the calculation
starts, the two geometries are superimposed as much as possible.  This is done
as follows:
\begin{enumerate}
\item Both geometries are converted into Cartesian coordinates.

\item Both geometries are centered about the origin of Cartesian space.

\item One geometry is  rotated  until  the  difference  vector  is  a minimum 
---  this  minimum  is  within  1 degree of the absolute bottom.

\item The \comp{SADDLE} calculation then proceeds as described above.
\end{enumerate}

The two geometries must be related by a continuous  deformation  of the  
coordinates. For this, internal  coordinates  are unsuitable in that while 
bond  lengths  and  bond  angles  are \index{Dihedral angles!ambiguities in|ff}
unambiguously  defined (being both positive), the dihedral angles can be either
positive or  negative.   Clearly  300  degrees  could  equally  well  be
specified  as  $-60$ degrees.  A wrong choice of dihedral would mean that
instead  of  the  desired  reaction  vector  being  used,  a  completely
incorrect vector was used, with disastrous results.

To prevent this, a \comp{SADDLE} calculation will always convert coordinates
into Cartesian before starting the run.  If symmetry is to be used, then the
geometry must be supplied in Cartesian coordinates, because internal symmetry
relations are not meaningful here.

\subsection{How to escape from a hilltop}
\index{Hilltop, escaping from}
\index{SIGMA}\index{NLLSQ}\index{TS}
A  particularly  irritating  phenomenon  sometimes  occurs  when  a transition 
state is being refined.  A rough estimate of the geometry of the transition
state has been obtained by either a  \comp{SADDLE}  or  reaction path or by
good guesswork.  This geometry is then  refined by \comp{TS}, \comp{SIGMA} or
by \comp{NLLSQ}, and the system characterized by a force calculation.  Remember
that \comp{NLLSQ} is preferred over \comp{SIGMA} when the GNORM is large, so
\comp{NLLSQ} is probably the method of choice, if for any reason \comp{TS} does
not work.    It  is  at this  point  that  things  often go wrong.  Instead of
only one negative force constant, two or more are found.  In the past, the 
recommendation has been to abandon the work and to go on to something less
masochistic. It is possible, however, to  systematically  progress  from  a 
multiple maximum to the desired transition state.  The technique used will now
be described.

If a multiple maximum is identified, most likely one negative force constant 
corresponds  to  the  reaction  coordinate,  in which case the objective  is 
to  render  the  other  force  constants  positive.   The associated  normal 
mode  eigenvalues are complex, but in the output are printed as negative
frequencies, and for the sake of simplicity will  be
\index{Frequencies!imaginary}\index{DRAW} described  as  negative  vibrations. 
Use a graphical user interface program to display the negative vibrations, 
and  identify  which  mode  corresponds  to  the   reaction coordinate.  This
is the one we need to retain.

Hitherto, simple motion in the direction of  the  other  modes  has \index{DRC}
proved  difficult.   However the DRC provides a convenient mechanism for
automatically following a normal coordinate.  Pick the  largest  of  the
negative  modes to be annihilated, and run the DRC along that mode until a
minimum is reached.  At that point,  refine  the  geometry  once  more using 
\comp{TS}  and  repeat  the  procedure  until  only one negative mode exists.

To be on the safe  side,  after  each  \comp{DRC}+\comp{TS}   sequence  do  the
\comp{DRC}+\comp{TS}  operation  again,  but use the  negative of the initial
normal coordinate to start the trajectory.  After both  stationary  points  are
reached,  choose  the  lower  point  as  the starting point for the next
elimination.  The lower point is chosen  because  the  transition  state
wanted  is  the  highest  point  on  the  lowest  energy path connecting
reactants to products.  Sometimes the two points will have equal energy: this 
is normally a consequence of both trajectories leading to the same point or
symmetry equivalent points.

After  all  spurious  negative  modes  have  been  eliminated,  the remaining 
normal  mode  corresponds to the reaction coordinate, and the transition state
has been located.

This technique is relatively rapid, and relies on starting  from  a stationary 
point to begin each trajectory.  If any other point is used, the trajectory
will  not  be  even  roughly  simple  harmonic.   If,  by mistake,  the
reaction coordinate is selected, then the potential energy will  drop  to 
that  of  either  the  reactants  or  products,   which, incidentally,  forms a
handy criterion for selecting the spurious modes: if the potential energy only
drops by  a  small  amount,  and  the  time evolution  is  roughly  simple 
harmonic,  then  the  mode is one of the spurious modes.  If there is any doubt
as to whether a minimum is in the vicinity  of  a stationary point, allow the
trajectory to continue until one complete cycle is executed.  At that point
the  geometry  should  be near to the initial geometry.

Superficially, a line-search might appear more attractive than  the relatively 
expensive  DRC.   However,  a line-search in Cartesian space will normally not
locate the minimum in a mode.  An obvious  example  is the mode corresponding
to a methyl rotation.

\subsection*{Keyword Sequences to be Used}
\begin{enumerate}
\item To locate the starting stationary point  given  an  approximate
transition state: \comp{TS}

\item To define the normal modes: \comp{FORCE ISOTOPE}

At this point, copy all the files to a second filename, for use later.

\item Given vibrational frequencies of $-654$, $-123$,  234,  and  456,
identify  {\em via} a GUI the normal coordinate mode,  say the  $-654$ mode. 
Eliminate the second mode by:

\comp{IRC=2 DRC T=30M RESTART LARGE}

Use is made of the FORCE restart file.

\item Identify  the  minimum  in  the  potential  energy  surface  by
inspection or using the ``grep'' command, of form:

\comp{grep '\%'  $<$Filename$>$.out }

\item Edit out of the output file the data file corresponding to  the lowest
point, and refine the geometry using: \comp{TS}

\item Repeat the last three steps but for the negative of the  normal mode, 
using  the  copied files.  The keywords for the first of the two jobs are:
\comp{IRC=--2 DRC T=30M RESTART LARGE}


\item Repeat the last four steps  as  often  as  there  are  spurious
modes.

\item Finally, carry out a DRC to confirm that the  transition  state does, in
fact, connect the reactants and products.  The drop in potential energy 
should  be  monotonic.   If  you  are  unsure whether  this  last  operation
will work successfully, do it at any time you have a stationary point.  If it
fails at the  very start,  then we are back where we were before---give up
and go home!!
\end{enumerate}

\index{SADDLE!description|)}
