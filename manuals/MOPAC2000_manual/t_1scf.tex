\section{An SCF Calculation}\label{1scf}\index{SCF1@{1SCF!worked example}}
\index{SCF!worked example}
Semiempirical calculations can be run on computers using readily available
programs such as MOPAC. It is possible to use MOPAC for research without having
any knowledge of its workings. This does not imply any failing on the part of
the researcher; after all, it is possible to write extensive computer programs
while having little knowledge of how a computer works. However, in order to
more efficiently use MOPAC, a more than casual knowledge of the theory involved
is desirable. In the following section the details of a very simple calculation
will be described. This calculation can be carried out as a `black box'
calculation, but the following exercise may help to satisfy the intellectual
curiosity of users of semiempirical methods regarding the mechanics of carrying
out an SCF calculation.  \index{MNDO|ff}\index{CNDO/2}

The MNDO method will be used because it is the oldest of the ``NDO''  methods.
The CNDO/2 method is very similar, and the example we will look at will
emphasize the similarity. The system to be examined is a regular hexagon of
hydrogen atoms in which the H--H distance is 0.98316 \AA ngstrom. Of course, a
regular hexagon of hydrogen atoms is not a stable system; the only reason we
are using it here is to demonstrate the working of an SCF calculation. The
optimized geometry was obtained by defining all bond lengths to be equal,
constraining all bond angles to be 120 degrees and defining the system as being
planar. We will need various reference data in order to follow \index{BLOCK
data|ff} the calculation. MOPAC contains a large data-set, BLOCK.F, of atomic
and diatomic parameters for all the elements which have been parameterized. By
reference to this source file we find that, for hydrogen:
$$
\begin{array}{lll}
  G_{ss}    = <\varphi_s\varphi_s|1/r|\varphi_s\varphi_s> &=& \ \ 12.848 {\rm eV} \\
  U_{ss}    = <\varphi_s|H|\varphi_s>               &=& -11.906276 {\rm eV }\\
  \xi_s                                       &=& \ \ \ \ 1.331967 {\rm Bohr} \\
  \beta_s                                     &=& \ \ -6.989064 {\rm eV} \\
  E_{atom}                                    &=& \ \ 52.102 {\rm kcal/mol}
\end{array}
$$
This exercise is designed to allow the reader to
reproduce each step. All that is needed in order to follow
the working is a hand calculator.

\begin{center}
Interatomic Distance Matrix (\AA) \nopagebreak\\
\begin{tabular}{ccccccc} \hline
 Atom &   1    &  2   &   3    &  4    &  5  &    6 \\ \hline
   1  &  0.0000& \\
   2  &  0.9832& 0.0000 \\
   3  &  1.7029& 0.9832& 0.0000 \\
   4  &  1.9663& 1.7029& 0.9832& 0.0000 \\
   5  &  1.7029& 1.9663& 1.7029& 0.9832& 0.0000 \\
   6  &  0.9832& 1.7029& 1.9663& 1.7029& 0.9832& 0.0000 \\ \hline
\end{tabular}
\end{center}

The overlap integral of two Slater orbitals between two
\index{Slater orbitals!overlap of}\index{Overlap!integrals!between two hydrogen atoms}
hydrogen atoms is particularly simple:
$$
<\varphi|\varphi> = (e^{-a})(\frac{a^2}{3}+a+1)
$$
where $a=\xi R/a_o$.

 At the optimum H-H distance of 0.9831571\AA, this yields
an overlap integral of 0.4643. The nearest-neighbor
one-electron integral is thus

$$
H(1,2) = S_{1,2}(\beta_s + \beta_s)/2 = -3.2457 eV.
$$
In general, overlap integrals are more complicated and also
involve angular components, but the principles involved are
the same. You may want to check other off-diagonal terms in
the one-electron matrix, or you may accept the results given
here.

\begin{center}
One-electron matrix (eV) \nopagebreak \\
\begin{tabular}{rrrrrrr} \hline
 Atom & 1 & 2 & 3 & 4 & 5 & 6 \\ \hline
 1 & -51.7124 \\
 2 &  -3.2457& -51.7124 \\
 3 & -1.0970& -3.2457& -51.7124 \\
 4 & -0.6992& -1.0970& -3.2457& -51.7124 \\
 5 & -1.0970& -0.6992& -1.0970& -3.2457& -51.7124 \\
 6& -3.2457& -1.0970& -0.6992& -1.0970& -3.2457& -51.7124 \\ \hline
\end{tabular}
\end{center}

\index{One-electron integrals}
On-diagonal one-electron integrals are more complicated than the off-diagonal
terms. The one-electron energy of an electron in an atomic orbital is the sum
of its kinetic energy and stabilization due to the positive nucleus of its own
atom, U$_{ss}$ or U$_{pp}$, plus the stabilization due to all the other nuclei
in the system. Each electron on a hydrogen atom experiences a stabilization due
to the five other unipositive nuclei in the system. Within semiempirical theory
the electron-nuclear interaction is related to the electron-electron
interaction via
$$
E_{e,n}=-Z_n<\varphi_s\varphi_s|\varphi_s\varphi_s>.
$$
Given the two-electron two-center integral matrix the calculation of the
diagonal terms of the one-electron matrix is straightforward:
$$
 H_{n,n} = -11.9063 - 2(9.6585) - 2(7.0635) -6.3622 = -51.7124.
$$
For interactions between an atomic orbital and a non-hydrogen atom there will
be ten terms; these arise from all permutations of the basis set, $s$, $p_x$,
$p_y$, $p_z$ with the atomic orbital under the neglect of differential overlap
approximation. The ten integrals are $<$ii$|ss>$, $<$ii$|sp_x>$,
$<$ii$|p_xp_x>$, $<$ii$|sp_y>$, $<$ii$|p_xp_y>$,$ <$ii$|p_yp_y>$,
$<$ii$|sp_z>$, $<$ii$|p_xp_z>$, $<$ii$|p_yp_z>$, and $<$ii$|p_zp_z>$.

\begin{center}
Two-Electron Integrals (eV) \nopagebreak \\
\begin{tabular}{rrrrrrr} \hline
 Atom & 1 & 2 & 3 & 4 & 5 & 6 \\ \hline
 1 & 12.8480 \\
 2 & 9.6585& 12.8480 \\
 3 & 7.0635& 9.6585& 12.8480 \\
 4 & 6.3622& 7.0732& 9.6585& 12.8480 \\
 5 & 7.0635& 6.3622& 7.0732& 9.6585& 12.8480 \\
 6 & 9.6585& 7.0635& 6.3622& 7.0732& 9.6585& 12.8480 \\ \hline
\end{tabular}
\end{center}

\subsection{Starting density matrix}
\index{Density matrix!starting}
The density matrix is necessary in order to calculate the Fock matrix, but, in
turn, the Fock matrix is necessary in order to calculate the density matrix. To
break this impasse, a guessed density matrix is used. The guess is very crude:
all off-diagonal matrix elements are set to zero, and all on-diagonal terms on
any atom are set equal to the core charge of that atom divided by the number of
atomic orbitals. Our starting guess for H$_6$ consists of a unit matrix.

Each iteration of the SCF calculation consists of assembling a Fock matrix from
the one-electron matrix, the two-electron integrals, and the density matrix,
diagonalizing it to obtain the eigenvectors, and finally reassembling the
density matrix. At some point the change in density matrix drops below a preset
limit. When this happens we say that the field is self-consistent. We will now
carry out these steps for the H$_6$ system.

\subsection{Assembly of the starting Fock matrix}
\index{Fock matrix!starting}
In the first iteration this is particularly simple, as there are no
off-diagonal terms in the density matrix. Only the on-diagonal terms are
affected. Each on-diagonal term in the Fock matrix F$_{aa}$ is modified by the
electrostatic field of all the electrons in the system except the electron or
fraction of an electron in the atomic orbital $\varphi_a$. Consider F(1,1). The
total initial population of $\varphi_1$ is 1.0, composed of equal amounts of
$\alpha$ and $\beta$ electron density. An electron in $\varphi_1$ would
therefore experience the electrostatic repulsion of half an electron. An
electron cannot repel itself; however, it will be repelled by its partner
electron of opposite spin.

In addition, each electron will be affected, normally repelled, by the
electrostatic field of all the electrons on all the other atoms. Each atom has
one electron, so the total energy of an electron, i.e., the diagonal Fock
matrix element, is given by:
$$
 F(1,1) = -51.7124 + \frac{1}{2}(12.848) + 2(9.6585 + 7.0635) + 6.3622.
$$

The Fock matrix is obtained by adding the two-electron
terms to the one electron matrix. The elements of the Fock
matrix represent the sum of the one and two electron
interactions. For the system of six hydrogen atoms, this
has the following form:
\begin{center}
Initial Fock Matrix (eV) \nopagebreak \\
\begin{tabular}{ccccccc} \hline
 Atom &   1    &  2   &   3    &  4    &  5  &    6 \\ \hline
 1& -5.4823  \\
 2& -3.2457& -5.4823  \\
 3& -1.0970& -3.2457& -5.4823  \\
 4& -0.6992& -1.0970& -3.2457& -5.4823  \\
 5& -1.0970& -0.6992& -1.0970& -3.2457& -5.4823  \\
 6& -3.2457& -1.0970& -0.6992& -1.0970& -3.2457& -5.4823  \\ \hline
\end{tabular}
\end{center}

\subsection{Diagonalization of the Fock matrix}\label{mo}
The Fock matrix is then diagonalized to yield the
\index{Eigenvalues}\index{Eigenvectors|ff}\index{Molecular orbitals|ff}
following set of eigenvalues, or one-electron energies,  and eigenvectors, or
molecular orbitals:
\begin{center}
\begin{tabular}{rrrrrrrr} \hline
Energy Level & \multicolumn{6}{c}{Molecular Orbital Coefficients} \\ \cline{2-7}
            & 1 & 2 & 3 & 4 & 5 & 6 \\ \hline
6 -0.4857  &0.4082& -0.4082&  0.4082& -0.4082&  0.4082& -0.4082  \\
5 -1.8388  &0.5774& -0.2887& -0.2887&  0.5774& -0.2887& -0.2887  \\
4 -1.8388  &0.0000&  0.5000& -0.5000&  0.0000&  0.5000& -0.5000  \\
3 -6.9317  &0.5774&  0.2887& -0.2887& -0.5774& -0.2887&  0.2887  \\
2 -6.9317  &0.0000&  0.5000&  0.5000&  0.0000& -0.5000& -0.5000  \\
1 -14.8670 &0.4082&  0.4082&  0.4082&  0.4082&  0.4082&  0.4082  \\ \hline
\end{tabular}
\end{center}
These form a normalized, orthogonal set. Under the NDDO approximation,
overlaps between different atomic orbitals are ignored, i.e.,
$<\varphi_i|\varphi_j>=\delta(i,j)$, so instead of
$$
<\psi_i|\psi_j> = \sum_{\lambda}\sum_{\sigma}c_{\lambda i}c_{\sigma j}<\varphi_{\lambda}|\varphi_{\sigma}>
$$
we have
$$
<\psi_i|\psi_j> = \sum_{\lambda}c_{\lambda i}c_{\lambda j} = \delta(i,j).
$$

\subsection{Exercises involving eigenvectors}
In the following exercises `verify' means using a hand calculator. They are
intended to confirm understanding of the theory involved. Work though one or
more examples to confirm the validity of the statement that follows.
\begin{enumerate}
\item Verify that the eigenvectors are normalized.
\item Verify that the eigenvectors are orthogonal to each other.
\item Verify that the eigenvalues are correct.
\item Verify that the eigenvectors diagonalize the Fock matrix.
\item Verify that the diagonal sum rule is obeyed; i.e., that the sum of the
eigenvalues is equal to the sum of the diagonal matrix elements (the trace) of
the Fock matrix.
\end{enumerate}

\subsection{Iterating density matrix}
\index{Density matrix!iterating}
The density matrix is then reformed using the occupied set of eigenvectors,
i.e., the lowest three levels. This yields:
\begin{center}
Density Matrix (eV) \nopagebreak \\
\begin{tabular}{rrrrrrr} \hline
 Atom &   1    &  2   &   3    &  4    &  5  &    6 \\ \hline
 1& 1.0000  \\
 2& 0.6667& 1.0000  \\
 3& 0.0000& 0.6667& 1.0000  \\
 4& -0.3333& 0.0000& 0.6667& 1.0000  \\
 5& 0.0000& -0.3333& 0.0000& 0.6667& 1.0000  \\
 6& 0.6667& 0.0000 &-0.3333& 0.0000& 0.6667& 1.0000  \\ \hline
\end{tabular}
\end{center}
 Verify that the density matrix is correct.

\subsection{Iterating Fock matrix}
\index{Fock matrix!iterating}
The second Fock matrix can then be constructed using this density matrix. The
on-diagonal terms are identical to those in the first Fock matrix, since the
atomic orbital electron densities are unchanged, but the off-diagonal terms are
now changed. The off-diagonal terms are modified to allow for exchange
interactions. (Note that not all exchange terms are stabilizing.)

Let us evaluate the matrix element F(1,2):
$$
  F(1,2) = -3.2457 - \frac{1}{2}(0.6667)(9.6583){\rm eV}.
$$
The second Fock matrix is thus:
\begin{center}
Second Fock Matrix (eV) \nopagebreak \\
\begin{tabular}{rrrrrrr} \hline
 Atom &   1    &  2   &   3    &  4    &  5  &    6 \\ \hline
 1& -5.4823  \\
 2& -6.4652& -5.4823  \\
 3& -1.0970& -6.4652& -5.4823  \\
 4& +0.3611& -1.0970& -6.4652 &-5.4823  \\
 5& -1.0970& +0.3611& -1.0970& -6.4652& -5.4823  \\
 6& -6.4652& -1.0970 &+0.3611& -1.0970 &-6.4652& -5.4823  \\ \hline
\end{tabular}
\end{center}

Diagonalization of this matrix yields the same set of eigenvectors as we had
initially. In general, several iterations are necessary in order to obtain an
SCF; however, a few systems exist for which symmetry restrictions on the form
of the eigenvectors allow them to achieve an SCF in one iteration. Hexagonal
H$_6$ is one such system. Although the eigenvectors are the same, the
eigenvalues obviously have to be different.

Exercise: Verify that the SCF energy levels of H$_6$ are -20.2457, -11.2116,
-11.2116, 2.4411, 2.4411, and 4.8929 eV.

Once an SCF is achieved we need to calculate the heat of formation.

\subsection{Calculation of heat of formation}
\index{$\Delta H_f$!worked example}\index{Heat of Formation!worked example}

The heat of formation is defined as:
$$
 \Delta H_f = E_{elect} + E_{nuc}-E_{isol}+E_{atom},
$$
where $E_{elect}$ is the electronic energy, $E_{nuc}$ is the
nuclear-nuclear repulsion energy, $-E_{isol}$ is the energy
required to strip all the valence electrons off all the
atoms in the system, and $E_{atom}$ is the total heat of
atomization of all the atoms in the system.

$E_{elect}$ is calculated from  $\frac{1}{2}{\bf P(H + F)}$, or
$$
E_{elect} = \frac{1}{2}\sum_{\lambda=1}^6\sum_{\sigma=1}^6P_{\lambda\sigma}(H_{\lambda\sigma} + F_{\lambda\sigma}).
$$
Using the data we have already derived, we can calculate $E_{elect}$ as
\begin{eqnarray}
   E_{elect} & = &  3(+1.0000)(-51.7124 + -5.4823)   \nonumber  \\
             &   &+ 6(+0.6667)( -3.2457 + -6.4652)   \nonumber  \\
             &   &+ 3(-0.3333)( -0.6992 + -0.3611)   \nonumber
\end{eqnarray}
or
$$
   E_{elect}  = -210.0898 {\rm eV}.
$$
$E_{nuc}$ is a relatively straightforward calculation, and is
equal to 130.2902eV. The total energy of the system is thus
-79.7996 eV.

We are now ready to calculate the $\Delta H_f$. As the total energy and
$E_{isol}$ are in eV, we must first convert them into kcal/mol:
$$
 \Delta H_f = 23.061( -79.7996 + 71.4377) + 6(52.1020){\rm \ kcal/mol}
$$
or
$$
\Delta H_f= 119.780{\rm  kcal/mol}.
$$
It is convenient to combine $E_{isol}$ and $E_{atom}$ together, to simplify
this calculation. In order to convert any total energy $(E_{elect} + E_{nuc})$
into a $\Delta H_f$, the following operation must be performed:
$$
\Delta H_f = 23.061(E_{elect} + E_{nuc}  -\sum_i E_{(isol-atom)}  )
$$
in which the index $i$ is over all atoms in the system.

Users of MOPAC may wish to verify this calculation for a system of their own
choice. To facilitate this, the data in Table~\ref{eisol} may prove useful.

\begin{table}
\caption{\label{eisol}Values of $E_{(isol-atom)}$ }
\begin{center}
\begin{tabular}{lrrrr} \hline
Element  & \multicolumn{4}{c}{$E_{(isol-atom)}$ (eV/atom)} \\ \cline{2-5}
& \multicolumn{1}{c}{MINDO/3} & \multicolumn{1}{c}{MNDO} &
\multicolumn{1}{c}{AM1} & \multicolumn{1}{c}{PM3} \\
\hline
Hydrogen            &-14.764312&   -14.165588& -13.655739& -15.332633  \\
Lithium             && -6.793583  \\
Beryllium           &&-27.541992  \\
Boron               &-67.584394   &-70.200344 &-69.601659  \\
Carbon            & -126.880346  &-127.910952&-128.226140&-118.640263  \\
Nitrogen           &-192.410048 & -207.466249&-207.307791&-162.513823  \\
Oxygen &-309.652672&-320.451178 &-318.682192&-291.924879  \\
Fluorine& -475.817831&-477.502913&-483.109715&-438.336301  \\
Aluminum& &  -47.931017 & &-50.311708  \\
Silicon &-95.576505&-87.539565&-83.701885&-72.488357  \\
Phosphorus &-154.270388&-156.236921&-124.436836&-121.236135  \\
Sulfur& -231.996798&-228.891710  &&-186.333060  \\
Chlorine& -347.185366&-354.374768&-373.455532&-316.452049  \\
Zinc  &  &-31.231065  \\
Germanium& &  -80.129955  \\
Bromine&  & -347.840783&-353.473742&-353.699430  \\
Tin  & &-95.454929  \\
Iodine&  &-341.704860&-347.970786&-289.422586  \\
Mercury & &-29.456154  \\
Lead & &-107.856099  \\ \hline
\end{tabular}
\end{center}
\end{table}

These numbers may be used in conjunction with the semiempirical electronic and
nuclear energies to calculate the heat of formation.
