\subsection{Capped Bonds}\label{cb}\index{Capped bonds|ff}
\index{Cb}
Sometimes the system being studied is too large to be calculated using MOPAC. 
When only a part of the system is of interest, a mechanism exists to allow only
that part to be calculated, and to ignore or not consider the rest of the
system.  Capped bonds are used to satisfy valency requirements.

The procedure for using capped bonds (Cb) is as follows:
\begin{enumerate}
\item Identify all atoms which are important to the calculation.  For an
enzyme, this would be the residues of the active site, for example.
\item Identify bonds which would be broken in order to isolate the atoms of
interest. Make sure these are single bonds, and try to ensure that there is not
more than one  broken bond on any atom.  Examples of ``good'' broken bonds are:
CH$_2$--CH$_2$, CH$_2$--NH, NH--NH.
\item Attach a Cb to each atom which has a broken bond.  The Cb should be
positioned in the direction of the atom which has deleted, and should have a
bond-length of 1.7\AA , exactly. Do not mark the Cb bond length for
optimization. If two broken bonds exist on an atom, use two Cb, but make the
Cb-atom-Cb angle 109.471221$^\circ$, exactly, and do not let it optimize.
\end{enumerate}

A Cb behaves like a monovalent atom, but always has a zero charge.  In other
words, a Cb has a core charge of +1, and always has an electron population of
1.0.  Cbs  have one orbital, and so can be regarded as being hydrogen-like.

Capped bonds are different from hydrogen atoms, however, in that they have a
large $\beta$-value.  The $\beta$ value is used in the calculation of the
one-electron two-center integral. Because the $\beta$ value is so large, the
difference in electronegativity of the Cb and the atom it is attached to, $A$,
becomes negligible.   Therefore the bonding M.O. consists of $1/\sqrt{2}({\rm
Cb}+A)$.  From this, it follows that the bonding M.O. contributes 1.0 electrons
to the Cb.  

In addition to the bonding M.O., there is an antibonding M.O.  This M.O.\ is of
form $1/\sqrt{2}({\rm Cb}-A)$, and is of very high energy.

To prevent capped bonds from forming bonds to all nearby atoms, overlaps from
capped bonds to all atoms further away than 1.8\AA\ are set to zero.  Because
of this, the coupling between capped bonds and all atoms, other than $A$, is
zero. Of course, $A$ can interact with other atoms, once it has satisfied the
demands of the attached capped bond.

Because of the  huge $\beta$-value, the energy of a Cb-atom bond is very large.
To prevent this interfering with the calculation, when the electronic energy is
calculated, contributions from Cb are ignored.  For this reason it is important
that the bond-length for the Cb should not be optimized.

The M.O.\ energy levels due to Cb-type bonds are also enormous.  Before the
M.O.\ energy levels are printed in the normal output, energy levels arising
from Cb-atom bonds are first set to zero.

The electronic behavior of capped bonds can easily be studied by use of
\comp{1SCF} \comp{DENSITY} \comp{VECTORS}.
