\subsection{Energies of Isolated Atoms}\index{EISOL}\index{EHEAT}\index{ATHEAT}
The $\Delta H_f$ calculated by semiempirical methods is defined as the  energy
in kcal.mol$^{-1}$ required to form one mole of the system in the gas phase at
298K from its elements in their standard state:
$$
\Delta H_f = E_{elect} + E_{nuc} + \sum_AE_{isol}(A) +\sum_AE_{atom}(A)
$$
In order to calculate $\Delta H_f$, the quantity $E_{isol}$ must be 
determined; this is the energy required to form the isolated atom from its 
valence electrons:
$$
E_{isol}(A)=E_{\rm neutral\ atom}(A) -E_{\rm nucleus}(A)-
E_{\rm valence\ electrons}(A)
$$
In the calculation of $E_{elect}$, the energy of valence electrons is  defined
as zero, likewise in calculating $E_{nuc}$, the energy of the  isolated nucleus
is defined as zero, therefore the calculation of $E_{isol}$ simplifies to the
calculation of $E_{\rm neutral\ atom}$.

The energy of $E_{eisol}$ is the energy released when the valence  electrons
are added to the nucleus.  For example, for the hydrogen atom, this would be 
$U_{ss}$.  For poly-electronic atoms, the electron-electron  interactions must
be included, in addition to the one-electron contributions. Most elements have
open shell ground states, and for these systems, the  nature of the state is
important.

For all main group elements, that is, elements with valence shell
configurations of the form $ns^anp^b$, other than the alkali metals, the value
of $E_{isol}$ is  given by:
$$
E_{isol} =aU_{ss}+bU_{pp}+(a-1)G_{ss}+a.bG_{sp}+(b(b-1))/2G_{p2}-bH_{sp}-
cH_{pp}
$$
in which $c=min(b(b-1)/2,(6-b).(5-b)/2)$.  Except for the $H_{pp}$ term, all
the contributions to $E_{isol}$ are obvious.  Non-zero $H_{pp}$ terms occur
when there are two or more unpaired electrons in the ground state, in  which
case there is an exchange stabilization that is otherwise absent.

Because $H_{pp}$ is usually written as $1/2(G_{pp}-G_{p2})$, the expression 
for systems with 2 to 4 $p$ electrons is recast as:
$$
E_{isol} =aU_{ss}+bU_{pp}+(a-1)G_{ss}+a.bG_{sp}+((b(b-1))/2+c/2)G_{p2}-
(a-1)bH_{sp}-c/2G_{pp},
$$
or
$$
E_{isol} =2U_{ss}+bU_{pp}+G_{ss}+2.bG_{sp}+((b(b-1))/2+c/2)G_{p2}-
(a-1)bH_{sp}-c/2G_{pp}.
$$
For the alkali metals, the equation for $E_{isol}$ is the same as that for 
hydrogen.

For the transition metals, the coefficients for the $d-d$ interactions are more
complicated.

The general form for $E_{isol}$ for a transition metal of configuration
$s^md^n$s, in which there are $m_a$ $\alpha$ $s$-electrons and $m_b$ $\beta$ 
$s$-electrons, and $n_a$ $\alpha$ $d$-electrons and $n_b$ $\beta$ 
$d$-electrons, and the total angular quantum number is $L$, is:
\begin{eqnarray} \nonumber
E_{isol}&=&mU_{ss}+nU_{dd}+
(m(m-1))/2G_{ss}+m.nG_{sd}-(m_an_a+m_bn_b)H_{sd}\\ \nonumber
&& +(n(n-1))/2\frac{G_{dd}^0}{5} +(-4(n_a^2+n_b^2)+13n-3/2(L(L+1)))\frac{G_{dd}^2}{49}\\ \nonumber 
&& +(-(n_a^2+n_b^2)/2-9/2n+5/6(L(L+1)))\frac{G_{dd}^4}{49}. \nonumber
\end{eqnarray}
As might be imagined, derivation of this expression is by no means  obvious,
particularly the terms for $G_{dd}^2$ and $G_{dd}^4$.  Interested readers are
referred to Racah's paper in {\it Phys Rev}, {\bf 61}, 186 (1942). In this,
Racah derived an expression for the $d$ orbital energy of the ground  state in
terms of three quantities, $A$, $B$, and $L$, the total angular momentum:
$$
<\! ^{n+1}L|H|^{n+1}L>=\frac{1}{2}n(n-1)(A-8B)+\frac{3}{2}[6n-L(L+1)]B.
$$

The quantities $A$ and $B$, and a third quantity, $C$, not used here, are
related  to the $G_k$ as follows:
\begin{eqnarray} \nonumber
A&=&G_{dd}^0-49G_{dd}^2\\ \nonumber
B&=&G_{dd}^2-5G_{dd}^4\\ \nonumber
C&=&35G_{dd}^4 \\ \nonumber
\end{eqnarray}
Using Racah's equation, derivation of  $E_{isol}$ is straightforward.  In texts
on transition metal ion theory, the quantities $G_{dd}^0$, $G_{dd}^0$, and 
$G_{dd}^0$ are usually represented by the symbols $F_0$, $F_2$, and $F_4$,
respectively.  However, care should be exercised when reading these texts:
sometimes other quantities, $F^0$, $F^2$, and $F^4$ are used.  The relationship
between these three sets of symbols is as follows:
\begin{eqnarray}\nonumber
G_{dd}^0  =  F_0 &=& F^0 \\ \nonumber
G_{dd}^2  =  F_2 &=& \frac{1}{49}F^2 \\ \nonumber
G_{dd}^4  =  F_4 &=& \frac{1}{441}F^4 \\ \nonumber
\end{eqnarray}


Because the coefficients for the two electron terms are so complicated,  values
for all elements likely to be parameterized for semiempirical  methods are
presented in Table~\ref{confs}.  From this table, the values of some 
coefficients are readily derived.  Thus for the $s$-$d$ coulomb integral,
$G_{sd}$, the coefficient is simply the number of $s$ electrons times the
number of $d$ electrons.  One $s$-$d$ exchange integral, $H_{sd}$, exists for 
each electron in the $s$ shell for which there is an electron of the  same spin
in the $d$ shell.  For elements with two $s$ electrons, this is simply the
number of $d$ electrons, for elements with one $s$ electron, the Aufbau
principle indicates that the $d$ shell with higher occupancy has the same spin
as that of the $s$ electron.  Finally, the coefficients for the  simple $d$-$d$
repulsion integral, $G_{dd}^0$, are given by the number of possible $d$-$d$
interactions.

Note also that there are no elements with both $p$ and $d$ valence electrons,
therefore terms of the type $G_{pd}$ are not necessary.

\begin{table}
\caption{\label{confs}Two Electron Energy Contributions to EISOL for Atoms in 
their Ground States}
\begin{center}
\compresstable
\begin{tabular}{cclcccccccccccc}\hline
\multicolumn{2}{c}{Element}   &  Orbital & State&  &
G$_{ss}$ & G$_{sp}$&H$_{sp}$& G$_{pp}$ &G$_{p2}$ &G$_{sd}$&
H$_{sd}$&G$_{dd}^0$&G$_{dd}^2$& G$_{dd}^4$\\ 
%                       Gss  Gsp Hsp  Gpp   Gp2  Gsd Hsd Gdd0 Gdd2 Gdd4
&   & Config.& & Mult.: & 1  & 1  &-1 & -1/2&1/2 &  1& -1/5 &1 &-1/49&-1/49\\
\hline
1&H  & $1s^1    $ &$^2S$&&    &    &   &     &    &   &   &    &    &     \\
2&He & $1s^2    $ &$^1S$&& 1  &    &   &     &    &   &   &    &    &     \\
3&Li & $2s^1    $ &$^2S$&&    &    &   &     &    &   &   &    &    &     \\
4&Be & $2s^2    $ &$^1S$&& 1  &    &   &     &    &   &   &    &    &     \\
5&B  & $2s^22p^1$ &$^2P$&& 1  & 2  & 1 &     &    &   &   &    &    &     \\
6&C  & $2s^22p^2$ &$^3P$&& 1  & 4  & 2 &  1  &  3 &   &   &    &    &     \\
7&N  & $2s^22p^3$ &$^4S$&& 1  & 6  & 3 &  3  &9   &   &   &    &    &     \\
8&O  & $2s^22p^4$ &$^3P$&& 1  & 8  & 4 &  1  &13  &   &   &    &    &     \\
9&F  & $2s^22p^5$ &$^2P$&& 1  & 10 & 5 &     &20  &   &   &    &    &     \\
10&Ne & $2s^22p^6$ &$^1S$&& 1  & 12 & 6 &     &30  &   &   &    &    &     \\
11&Na & $3s^1    $ &$^2S$&&    &    &   &     &    &   &   &    &    &     \\
12&Mg & $3s^2    $ &$^1S$&& 1  &    &   &     &    &   &   &    &    &     \\
13&Al & $3s^23p^1$ &$^2P$&& 1  & 2  & 1 &     &    &   &   &    &    &     \\
14&Si & $3s^23p^2$ &$^3P$&& 1  & 4  & 2 &  1  &3   &   &   &    &    &     \\
15&P  & $3s^23p^3$ &$^4S$&& 1  & 6  & 3 &  3  &9   &   &   &    &    &     \\
16&S  & $3s^23p^4$ &$^3P$&& 1  & 8  & 4 &  1  &13  &   &   &    &    &     \\
17&Cl & $3s^23p^5$ &$^2P$&& 1  & 10 & 5 &     &20  &   &   &    &    &     \\
18&Ar & $3s^23p^6$ &$^1S$&& 1  & 12 & 6 &     &30  &   &   &    &    &     \\
19&K  & $4s^1    $ &$^2S$&&    &    &   &     &    &   &   &    &    &     \\
20&Ca & $4s^2    $ &$^1S$&& 1  &    &   &     &    &   &   &    &    &     \\
21&Sc & $4s^23d^1$ &$^2D$&& 1  &    &   &     &    & 2 & 1 &    &    &     \\
22&Ti & $4s^23d^2$ &$^3F$&& 1  &    &   &     &    & 4 & 2 & 1  & 8  & 1   \\
23&V  & $4s^23d^3$ &$^4F$&& 1  &    &   &     &    & 6 & 3 & 3  & 15 & 8   \\
24&Cr & $4s^13d^5$ &$^7S$&&    &    &   &     &    & 5 & 5 & 10 & 35 & 35  \\
25&Mn & $4s^23d^5$ &$^6S$&& 1  &    &   &     &    &10 & 5 & 10 & 35 & 35  \\
26&Fe & $4s^23d^6$ &$^5D$&& 1  &    &   &     &    &12 & 6 & 15 & 35 & 35  \\
27&Co & $4s^23d^7$ &$^4F$&& 1  &    &   &     &    &14 & 7 & 21 & 43 & 36  \\
28&Ni & $4s^23d^8$ &$^3F$&& 1  &    &   &     &    &16 & 8 & 28 & 50 & 43  \\
29&Cu & $4s^13d^{10}$&$^2S$&&  &    &   &     &    &10 & 5 & 45 & 70 & 70  \\
30&Zn & $4s^2     $&$^1S$&& 1  &    &   &     &    &   &   &    &    &     \\
31&Ga & $4s^24p^1$ &$^2P$&& 1  & 2  & 1 &     &    &   &   &    &    &     \\
32&Ge & $4s^24p^2$ &$^3P$&& 1  & 4  & 2 &  1  &3   &   &   &    &    &     \\
33&As & $4s^24p^3$ &$^4S$&& 1  & 6  & 3 &  3  &9   &   &   &    &    &     \\
34&Se & $4s^24p^4$ &$^3P$&& 1  & 8  & 4 &  1  &13  &   &   &    &    &     \\
35&Br & $4s^24p^5$ &$^2P$&& 1  & 10 & 5 &     &20  &   &   &    &    &     \\
36&Kr & $4s^24p^6$ &$^1S$&& 1  & 12 & 6 &     &30  &   &   &    &    &     \\
37&Rb & $5s^1    $ &$^2S$&&    &    &   &     &    &   &   &    &    &     \\
38&Sr & $5s^2    $ &$^1S$&& 1  &    &   &     &    &   &   &    &    &     \\
39&Y  & $5s^24d^1$ &$^2D$&& 1  &    &   &     &    & 2 & 1 &    &    &     \\
40&Zr & $5s^24d^2$ &$^3F$&& 1  &    &   &     &    & 4 & 2 & 1  & 8  & 1   \\
41&Nb & $5s^14d^4$ &$^6D$&&    &    &   &     &    & 4 & 4 & 6  & 21 & 21  \\
42&Mo & $5s^14d^5$ &$^7S$&&    &    &   &     &    & 5 & 5 & 10 & 35 & 35  \\
43&Tc & $5s^24d^5$ &$^6S$&& 1  &    &   &     &    &10 & 5 & 10 & 35 & 35  \\
44&Ru & $5s^14d^7$ &$^5F$&&    &    &   &     &    & 7 & 5 & 21 & 43 & 36  \\
45&Rh & $5s^14d^8$ &$^4F$&&    &    &   &     &    & 8 & 5 & 28 & 50 & 43  \\
46&Pd & $5s^04d^{10}$&$^1S$&&  &    &   &     &    &   &   & 45 & 70 & 70  \\
47&Ag & $5s^14d^{10}$&$^2S$&&  &    &   &     &    &10 & 5 & 45 & 70 & 70  \\
\hline
\end{tabular}
\end{center}
\end{table}

\begin{table}
\caption{Two Electron Energy Contributions to EISOL for Atoms in their Ground 
States}
\begin{center}
\compresstable
\begin{tabular}{cclcccccccccccc} \hline
\multicolumn{2}{c}{Element}  &  Orbital & State& &
G$_{ss}$ & G$_{sp}$&H$_{sp}$& G$_{pp}$ &G$_{p2}$ &G$_{sd}$&
H$_{sd}$&G$_{dd}^0$&G$_{dd}^2$& G$_{dd}^4$\\ 
%                     Gss  Gsp Hsp  Gpp   Gp2  Gsd Hsd Gdd0 Gdd2 Gdd4
&   & Config.& & Mult.: & 1  & 1  &-1 & -1/2&1/2 &  1& -1/5 &1 &-1/49&-1/49\\
\hline
48&Cd & $5s^2     $&$^1S$&& 1  &    &   &     &    &   &   &    &    &     \\ 
49&In & $5s^25p^1$ &$^2P$&& 1  & 2  & 1 &     &    &   &   &    &    &     \\
50&Sn & $5s^25p^2$ &$^3P$&& 1  & 4  & 2 &  1  &3   &   &   &    &    &     \\
51&Sb & $5s^25p^3$ &$^4S$&& 1  & 6  & 3 &  3  &9   &   &   &    &    &     \\
52&Te & $5s^25p^4$ &$^3P$&& 1  & 8  & 4 &  1  &13  &   &   &    &    &     \\
53&I  & $5s^25p^5$ &$^2P$&& 1  & 10 & 5 &     &20  &   &   &    &    &     \\
54&Xe & $5s^25p^6$ &$^1S$&& 1  & 12 & 6 &     &30  &   &   &    &    &     \\
55&Cs & $6s^1    $ &$^2S$&&    &    &   &     &    &   &   &    &    &     \\
56&Ba & $6s^2    $ &$^1S$&& 1  &    &   &     &    &   &   &    &    &     \\
72&Hf & $6s^25d^2$ &$^3F$&& 1  &    &   &     &    & 4 & 2 & 1  & 8  & 1   \\
73&Ta & $6s^25d^3$ &$^4F$&& 1  &    &   &     &    & 6 & 3 & 3  & 15 & 8   \\
74&W  & $6s^25d^4$ &$^5D$&& 1  &    &   &     &    & 8 & 4 & 6  & 21 & 21  \\
75&Re & $6s^25d^5$ &$^6S$&& 1  &    &   &     &    &10 & 5 & 10 & 35 & 35  \\
76&Os & $6s^25d^6$ &$^5D$&& 1  &    &   &     &    &12 & 6 & 15 & 35 & 35  \\
77&Ir & $6s^25d^7$ &$^4F$&& 1  &    &   &     &    &14 & 7 & 21 & 43 & 36  \\
78&Pt & $6s^15d^9$ &$^3D$&&    &    &   &     &    & 9 & 5 & 36 & 56 & 56  \\
79&Au & $6s^15d^{10}$&$^2S$&&  &    &   &     &    &10 & 5 & 45 & 70 & 70  \\
80&Hg & $6s^2     $&$^1S$&& 1  &    &   &     &    &   &   &    &    &     \\
81&Tl & $6s^26p^1$ &$^2P$&& 1  & 2  & 1 &     &    &   &   &    &    &     \\
82&Pb & $6s^26p^2$ &$^3P$&& 1  & 4  & 2 &  1  &3   &   &   &    &    &     \\
83&Bi & $6s^26p^3$ &$^4S$&& 1  & 6  & 3 &  3  &9   &   &   &    &    &     \\
84&Po & $6s^26p^4$ &$^3P$&& 1  & 8  & 4 &  1  &13  &   &   &    &    &     \\ 
85&At & $6s^25p^5$ &$^2P$&& 1  & 10 & 5 &     &20  &   &   &    &    &     \\
\hline
\end{tabular}
\end{center}
\end{table}
