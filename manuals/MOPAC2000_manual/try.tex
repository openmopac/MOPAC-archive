
$$ H_{i,j} (\mathrm{millidynes/\AA}) = 10^5\frac{({\mathrm{Kcal\
to\ ergs}})}
 {(\mathrm{\AA\ to\ cm})^2(\mathrm{Mole\ to\
molecule})} H_{i,j}(\mathrm{kalmol/\AA  })
 $$




or
$$
H_{i,j} (\mathrm{millidynes/\AA}) = 10^5\frac{4.184*10^3*10^7}{(10^{-8*2})
(6.022*10^{23})}H_{i,j}(\mathrm{kcal/mol/\AA ^2}).
$$
Diagonalization of this matrix yields the force constants of the system.

In order to calculate the vibrational frequencies, the Hessian matrix is first
mass-weighted:
$$
H^m_{i,j} = \frac{H_{i,j}}{\sqrt{M_i*M_j}}.
$$

Then the Hessian is converted from millidynes per \AA ngstrom to dynes per
centimeter by multiplying by 10$^5$.

Diagonalization of this matrix yields eigenvalues, $\epsilon$, which represent
the quantities $\sqrt{k/\mu}$, from which the vibrational frequencies can be
calculated:
$$
\bar{\nu}_i = \frac{1}{2\pi c}\sqrt{\epsilon_i}.
$$

\subsection{Mechanism of the frame in FORCE calculation}
\index{Frame!description of}
The FORCE calculation uses Cartesian coordinates, and all 3N  modes are
calculated, where N is the number of atoms in the system.  Clearly, there will
be 5 or 6 ``trivial'' vibrations,  which  represent  the  three translations
and two or three rotations.  If the molecule is exactly at a stationary point,
then these ``vibrations'' will have a  force  constant and  frequency  of
precisely  zero.   If the force calculation was done correctly, and the
molecule was not exactly at a stationary point,  then the  three  translations
should be exactly zero, but the rotations would be non-zero.  The extent to
which  the  rotations  are  non-zero  is  a measure of the error in the
geometry.

If  the  distortions  are  non-zero,  the  trivial  vibrations  can interact
with  the  low-lying genuine vibrations or rotations, and with the transition
vibration if present.

To prevent this the analytic form of the rotations  and  vibrations is
calculated,  and arbitrary eigenvalues assigned; these are 500, 600, 700, 800,
900, and 1000 millidynes/\AA ngstrom for Tx, Ty, Tz, Rx,  Ry  and Rz  (if
present),  respectively.  The rotations are about the principal axes of inertia
for the system, taking  into  account  isotopic  masses. The ``force matrix''
for these trivial vibrations is determined, and added on to the calculated
force matrix.  After diagonalization the  arbitrary eigenvalues are subtracted
off the trivial vibrations, and the resulting numbers are the ``true'' values.
Interference with genuine vibrations  is thus avoided.

\subsection{Vibrational Analysis}\index{Vibrational analysis}
Analyzing normal coordinates is very tedious.  Users  are  normally familiar
with the internal coordinates of the system they are studying, but not familiar
with the Cartesian coordinates.  To  help characterize the  normal
coordinates,  a very simple analysis is done automatically, and users are
strongly encouraged to use this analysis first,  and  then to look at the
normal coordinate eigenvectors.

In the analysis, each pair of bonded atoms is examined  to  see  if there  is
a  large  relative  motion  between them.  By bonded is meant \index{Van der
Waals|ff} within the van der Waals' distance.  If there  is  such  a  motion,
the indices  of  the  atoms,  the  relative  distance  in \AA ngstroms, and the
percentage radial motion are printed.   Radial  plus  tangential  motion adds
to  100\%,  but  as there are two orthogonal tangential motions and only one
radial, the radial component is printed.

Vibrations in the range +50 to {\em i}50 cm$^{-1}$ cannot be described
accurately in the vibrational analysis, due to numerical difficulties. However,
the normal coordinates and frequencies are printed in the section above
``DESCRIPTION OF VIBRATIONS''.

\subsection{Reduced masses}\index{Reduced mass|ff}\index{Simple harmonic motion}
\index{Mass!reduced}
A molecular vibration normally involves all the atoms moving simultaneously.
This is clearly very different from the  simple harmonic motion of a mass
attached to a spring that is attached to an immovable object.  Nevertheless, it
is convenient to visualize a molecular vibration as consisting of a single
mass, $M$, on the end of a spring of force constant $k$. For such a  system,
the period of vibration, $T$, is given by:
$$
T=2\pi\sqrt{\frac{M}{k}}.
$$

How, then, do we relate the complicated motion of a molecular vibration to the
mass and spring model?

During a molecular vibration, each atom follows a simple harmonic motion. So
the problem is, to what extent does each atom contribute to the mass, and to
what extent does each atom contribute to the spring?

In order to answer this, first consider some simple systems.  In the system
H--X, where X has a very large mass, compared to that of the H,  the effective
mass is obviously that of H.  In H$_2$, the effective mass is half that of a
single H.   Why is this so?  In H--X, particle X is stationary, and particle H
contributes 100\% of the energy to the vibration.  In H$_2$, each particle
obviously contributes 50\%, but now the center of mass is half way between the
two particles.  If the force constants are the same in H--X and in H--H, then
the period of vibration of H--X will be $\sqrt{2}$ times that of H--H.  This is
the same period as for a system of two particles, each of which having a mass
twice that of a H particle.  For a system of two particles, $A$ and $B$, having
masses $M_A$ and $M_B$,  the vibrational wavefunction, $\psi_v$, is:
$$
\psi_v=\sqrt{\frac{M_B}{M_A+M_B}}\psi_A-\sqrt{\frac{M_A}{M_A+M_B}}\psi_B.
$$
This can be interpreted as particle $A$ moves $(\sqrt{\frac{M_B}{M_A+M_B}})^2$
in the time particle $B$ moves $(\sqrt{\frac{M_A}{M_A+M_B}})^2$.  The center of
mass, $\rho$, stays constant:
$$
\rho=\sum_iM_i\delta x_i = M_A\frac{M_A}{M_A+M_B}- M_B\frac{M_A}{M_A+M_B} = 0.
$$
The square of the coefficients of the wavefunction represent the contribution
to the motion.  The effective mass, $\mu$, for this system is:
$$
\mu = \frac{M_A\times M_B}{M_A+M_B}.
$$
What fraction is due to $A$ and what fraction is due to $B$?  From the
wavefunction, the intensity of $A$ is $\frac{M_B}{M_A+M_B}$, and the relative
rate of motion is also $\frac{M_B}{M_A+M_B}$, so the contribution to the
effective mass due to $A$ is:
$$
(\frac{M_B}{M_A+M_B})M_A;
$$
likewise, for $B$:
$$
(\frac{M_A}{M_A+M_B})M_B.
$$
Consider two particles, $A$ and $B$, of mass 1 and 4, respectively.   The
wavefunction for the vibration is:
$$
\psi_v = \sqrt{\frac{4}{5}}\psi_A-\sqrt{\frac{1}{5}}\psi_B,
$$
where $A$ contributes
$$
\frac{16}{25}\times 1 = 0.64
$$
and $B$ contributes
$$
\frac{1}{25}\times 4 = 0.16
$$
to the effective mass of $\frac{4}{5}$.

In other words, the contribution to the effective mass is equal to the
intensity of the wavefunction on each atom, times the mass of the atom, times
the intensity of the wavefunction.  This is intuitively correct:  the total
vibration is composed of contributions from each particle, and the amount each
particle contributes is proportional to its intensity in the wavefunction.  The
mass of each particle is also proportional to its intensity in the
wavefunction.

Extension to polyatomic molecules is now trivial.  The effective mass is given
by:
$$
\mu = \sum_A <\!\psi_AM_A\psi_A\!>\times <\!\psi_A|\psi_A\!>.
$$
When written in this way, the quantum nature of the expression is obvious.
However, for computational convenience, the effective mass is rewritten as:
$$
\mu = \sum_A(\psi_A^2)^2\times M_A
$$
or
$$
\mu = \sum_A(\sum_{i=1}^3c_{A_i}^2)^2M_A.
$$
This expression is suitable for most systems.  However, it is not a
well-defined quantity.  Under certain circumstances involving degenerate
vibrations, the quantity $\mu$ can become ill-defined.  This phenomenon can be
attributed to the  fact that the reduced mass is not an observable.

\subsection{Effective masses}\index{Mass!effective}\index{Effective mass}
Another way of regarding the effective mass of a mode can be derived from
consideration of the simple harmonic oscillator:
$$
   E =  \sqrt{\frac{k}{\mu}}.
$$
Diagonalization of the mass-weighted Hessian yields the energies, and from the
normal coordinates the force-constants can readily be derived.  From these two
quantities, the effective mass can readily be calculated:
$$
\mu = \frac{k}{E^2}.
$$
For a homonuclear diatomic, the effective mass calculated this way is equal
to the mass of one atom.

\subsection{Travel}\index{Travel}
To continue the idea of representing a normal mode as a simple harmonic
oscillator, the distance the atoms move through can be represented as the
distance the idealized mass moves through.  This can be calculated knowing the
energy of the mode and the force constant:
$$
     E = \frac{1}{2}kx^2.
$$
Here $k$ is the force-constant for the mode, and is given by
$$
k = <\psi|Hessian|\psi>;
$$
$E$ is the energy of the mode.

From this, the distance, $x$, which the system moves through, can be calculated
from
$$
x =\sqrt{\frac{2\times 1.196266\times 10^8 \times 1000 \times 10^8 \nu}{N k}},
$$
where $1.196266\times 10^8$ is the conversion factor from cm$^{-1}$ to ergs,
1000 converts from millidynes to dynes, $10^8$ converts from cm to \AA , and
$N$ converts from moles to molecules.

Note that $x$, which in the output is called TRAVEL, is in mass weighted
space, not simple space.  This quantity can also be calculated using the DRC,
by depositing one quantum of energy into a vibrational mode.  For a system at a
stationary point, the relevant keywords would be \comp{IRC=1 DRC t=1m}.  For
larger systems, the time may need to be increased.  At least one coordinate
must have an optimization flag set to 1.  This is required in order to instruct
the DRC to print the turning points. \index{Normal coordinates!calculation
of|)}
