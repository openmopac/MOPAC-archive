\section{Dynamic and Intrinsic Reaction Coordinates}\index{Coordinates!reaction}
\label{t_irc}\index{Molecular dynamics}\index{Time-dependent phenomena}
The Intrinsic Reaction Coordinate method pioneered and developed by Mark
Gordon, North Dakota State University,\index{gordon@{\bf Gordon, Mark}}   has
been incorporated into MOPAC in a modified form.  As this facility is quite
complicated all the keywords associated with  the IRC have been grouped
together in this section (these can be seen later on  in this section).
\index{DRC|(} \index{DRC!definition}

The Dynamic Reaction Coordinate is the path  followed  by  all  the atoms  in
a  system  assuming  conservation  of  energy;  i.e.,  as the potential energy
changes the kinetic energy of  the  system  changes  in exactly  the  opposite
way  so  that  the  total  energy  (kinetic plus potential) is a constant.  It
is equivalent to the molecular mechanics molecular dynamics calculation, except
that bond-breaking and bond-making are supported, as are all the electronic
phenomena of the semiempirical methods.

If started at a  ground  state  geometry,  no \index{Transition state!use with
DRC} significant  motion should be seen.  Similarly, starting at a transition
state geometry should not produce  any  motion---after  all  it  is  a
stationary point and during the lifetime of a calculation it is unlikely to
accumulate enough momentum to travel far from the starting position.

In order to calculate the DRC path from a transition state,  either an
initial  deflection  is  necessary  or some initial momentum must be supplied.

Because of the time-dependent nature of the DRC  the  time  elapsed since the
start of the reaction is meaningful, and is printed.

\subsection*{Description}
The course of a molecular vibration can be followed by  calculating the
potential  and  kinetic  energy  at  various  times.   Two  extreme conditions
can be identified:  (a) gas phase, in which the total  energy is a constant
through time, there being no damping of the kinetic energy allowed, and (b)
liquid phase, in which kinetic energy is always set  to zero, the motion of the
atoms being infinitely damped.\index{Liquids}

All possible degrees of damping  are  allowed.   In  addition,  the facility
exists  to  dump  energy into the system, appearing as kinetic energy.  As
kinetic energy is a function of velocity, a vector quantity, the  energy
appears  as  energy of motion in the direction in which the molecule would
naturally move.  If the system  is  a  transition  state, then  the  excess
kinetic  energy  is added after the intrinsic kinetic energy has built up to at
least 0.2 kcal/mol.\index{Kinetic energy!damping}

For ground-state systems, the excess energy sometimes  may  not  be added;  if
the  intrinsic kinetic energy never rises above 0.2 kcal/mol then the excess
energy will not be added.


\subsection*{Equations used}
Force acting on any atom:
$$ g(i) + g'(i)t + g''(i)t^2 = \frac{dE}{dx(i)} +
        \frac{d^2E}{dx(i)^2} + \frac{d^3E}{dx(i)^3} $$
Acceleration due to force acting on each atom:
$$ a(i) = \frac{1}{M(i)} (g(i) + g'(i)t + g''(i)t^2) $$
New velocity:
$$ V(o) + \frac{1}{M(i)} \left(\Delta t g(i) + (1/2) \Delta  t^2 g'(i) +
(1/3) \Delta t^3g''(i)\right) $$
or:
$$ V(i) = V(i) + V'(i)t + V''(i)t^2 + V'''(i)t^3 $$
That is, the change in velocity is equal to the integral  over  the
time interval of the acceleration.

New position of atoms:
$$ X(i) = X(o) + V(o)t + (1/2) V't^2 + (1/3) V''t^3 + (1/4) V'''t^4 $$
That is, the change in position is equal to the integral  over  the
time interval of the velocity.

The velocity vector is accurate to the extent that  it  takes  into account
the  previous velocity, the current acceleration, the predicted acceleration,
and the change in predicted  acceleration  over  the  time interval.    Very
little  error  is  introduced  due  to  higher  order contributions to the
velocity; those that do occur  are  absorbed  in  a re-normalization of the
magnitude of the velocity vector after each time interval.

The magnitude of $\Delta t$, the time interval, is determined mainly by the
factor   needed   to   re-normalize  the  velocity  vector.   If  it  is
significantly different from unity, $\Delta t$ will be reduced; if  it  is
very close to unity, $\Delta t$ will be increased.

Even with all this, errors creep in and a system,  started  at  the transition
state,  is  unlikely  to  return precisely to the transition state  unless  an
excess  kinetic  energy  is  supplied,  for   example 0.2 kcal/mol.

The calculation  is  carried  out  in  Cartesian  coordinates,  and converted
into   internal  coordinates  for  display.   All  Cartesian coordinates must
be allowed to vary, in order to  conserve  angular  and translational
momentum.\index{DRC!conservation of momentum}

\subsection*{IRC}
\index{IRC|(}
The Intrinsic Reaction Coordinate is the path followed by  all  the atoms  in
a  system  and assumes that  all kinetic energy is completely lost at every
point; i.e., as the potential energy changes  the  kinetic  energy generated
is  annihilated  so  that  the  total  energy  (kinetic  plus potential) is
always equal to the potential energy only.

The IRC is intended for use in calculations in which the starting geometry is
that of the transition state.  A   normal  coordinate  is  chosen,  usually
the  reaction coordinate, and the system  is  displaced  in  either  the
positive  or negative  direction  along  this  coordinate.   The  internal
modes are obtained by calculating the mass-weighted  Hessian  matrix  in  a
force calculation   and   translating  the  resulting  Cartesian  normal  mode
eigenvectors to conserve  momentum.   That  is,  the  initial  Cartesian
coordinates  are  displaced  by  a  small  amount  proportional  to  the
eigenvector coefficients plus a translational constant; the constant  is
required  to  ensure that the total translational momentum of the system is
conserved as zero.  At the present time there may be  small  residual
rotational  components  which  are not annihilated; these are considered
unimportant, and will not materially affect the calculation.

\subsection*{General description of the DRC and IRC}
As the IRC usually requires a normal coordinate, a  force  constant
calculation  normally  has to be done first.  If IRC is specified on its own, a
normal coordinate is not used and the IRC calculation is performed on the
supplied geometry.

A recommended sequence of operations to start an IRC calculation is
as follows:
\begin{enumerate}
\item Calculate the transition state geometry.  If  the  transition state  is
not first  optimized,  then  the  IRC  calculation  may  give  very misleading
results.  For example, if NH$_3$ inversion  is  defined as  the  planar
system  but  without the N--H bond length being optimized, the first normal
coordinate might be for N--H  stretch rather  than  inversion.   In  that case
the IRC will relax the geometry to the optimized planar structure.

\index{ISOTOPE!use with IRC}
\index{FORCE!use with IRC}
\item Do a normal FORCE calculation, specifying \comp{ISOTOPE} in  order  to
save  the  FORCE  matrices.   (Note: Do  not  attempt  to  run the IRC at this
point directly unless you have confidence that  the FORCE  calculation will
work as expected.  If the IRC calculation is run directly, specify
\comp{ISOTOPE} anyway:  that will save the FORCE matrix and if the
calculation  has  to  be  re-done  then  \comp{RESTART} will work correctly.)

\item Using \comp{IRC=$n$} and \comp{RESTART}, run the IRC calculation.   If
\comp{RESTART} is specified with \comp{IRC=$n$} then the restart is assumed to
be from the FORCE calculation.  If, in an \comp{IRC} calculation,
\comp{RESTART}  is specified, and \comp{IRC=$n$} is {\em not} present,  then
the restart is  assumed to be from an earlier IRC calculation that was  shut
down  before  going  to completion.
\end{enumerate}

A DRC calculation is simpler, in that a force calculation is  not  a
prerequisite;  however,  most  calculations of interest normally involve use of
an internal coordinate.  For this reason IRC=$n$  can  be  combined with  DRC
to  give  a  calculation in which the initial motion (0.3kcal worth of kinetic
energy) is supplied by  the  IRC,  and  all  subsequent motion  obeys
conservation of energy.  The DRC motion can be modified in three ways:
\begin{enumerate}
\item It is possible to calculate the reaction  path  followed  by  a system
in  which  the  generated  kinetic energy decays with a finite half-life.  This
can  be  defined  by  DRC=$n.nnn$,  where $n.nnn$  is  the  half-life in
femtoseconds.  If $n.nn$ is 0.0 this corresponds  to  infinite  damping
simulating  the   IRC.    A limitation  of  the  program is that time only has
meaning when DRC is specified without a half-life.

\item Excess kinetic energy can be added to the calculation by use of
KINETIC=$n.nn$.   After  the  kinetic  energy  has  built  up  to 0.2 kcal/mol
or if IRC=$n$ is used then $n.nn$ kcal/mol of kinetic energy  is  added  to
the  system.   The excess kinetic energy appears as a velocity vector  in  the
same  direction  as  the initial motion.

\index{RESTART!use with IRC/DRC}
\item The RESTART file \verb/<filename>.res/ can be edited to allow the user
to  modify the velocity vector or starting geometry.  This file
is formatted.
\end{enumerate}

Frequently, the  DRC leads to a periodic, repeating orbit.   One  special
type---the  orbit in which the direction of motion is reversed so that the
system retraces its own path---is sensed for  and  if  detected  the
calculation  is  stopped after exactly one cycle.  If the calculation is to be
continued,  the  keyword  \comp{GEO-OK}  will  allow  this  check  to  be
by-passed.

\index{GNORM!use with IRC/DRC}  Sometimes the system will enter a stable state
in which the geometry is always changing, but nothing new is occurring.  One
example would be a system which decomposed into fragments, and the fragments
were moving apart. If all forces acting on the atoms become small, then the
calculation will be stopped.  If the calculation should be continued, then
specify \comp{GNORM=0 LET}.

Due to the potentially very large output files  that  the  DRC  can generate,
extra  keywords  are  provided  to allow selected points to be printed.  Two
types of control are provided:  one controls which points to  print, the other
controls what is printed.

By default, every point calculated is printed.  Often, this is not desirable,
and three keywords are provided to allow printing to be done whenever the
system changes by a preset amount.  These keywords are:

\begin{center}
\begin{tabular}{cll}\hline
          KeyWord &       Default         &   User Specification  \\ \hline
\comp{X-PRIO}  &   0.05 \AA ngstroms   &         \comp{X-PRIORITY=$n.nn$}  \\
\comp{T-PRIO}  &   0.10 Femtoseconds   &      \comp{T-PRIORITY=$n.nn$}  \\
\comp{H-PRIO}  &   0.10 kcal/mol      &      \comp{H-PRIORITY=$n.nn$}\\ \hline
\end{tabular}
\end{center}

By default, only the energies involved are printed (one line per point).  To
allow the geometry to be printed, \hyperref[pageref]{\comp{LARGE} is provided}{, see
p.~}{ for more detail}{large}.  Using \comp{LARGE} a wide range of control
is provided over what is printed.

\subsection*{Option to allow only extrema to be output}
In the geometry specification, if an internal coordinate is  marked for
optimization  then  when that internal coordinate passes through an extremum a
message will be printed and the geometry output.

Difficulties can  arise  from  the  way  internal  coordinates  are
processed.   The  internal  coordinates are generated from the Cartesian
coordinates, so an internal coordinate supplied  may  have  an  entirely
different  meaning  on  output.  In particular the connectivity may have
changed.  For obvious reasons dummy atoms should  not  be  used  in  the
supplied  geometry  specification.   If  there  is  any  doubt about the
internal coordinates or if the starting geometry  contains  dummy  atoms then
run  a  \comp{1SCF} calculation specifying \comp{INT}.  This  will produce an
ARC file with the ``ideal'' numbering---the internal numbering system used  by
MOPAC. \ Use this ARC file to construct a data file suitable for the DRC or
IRC.\index{DRC!dummy atoms in}\index{Dummy atoms!in DRC}

Notes:
\begin{enumerate}
\item Any coordinates marked for optimization  will  result  in  only extrema
being printed.
\item If extrema are being printed then kinetic energy  extrema  will also be
printed.
\end{enumerate}

\subsection*{Keywords for use with the IRC and DRC}
\index{IRC!keywords for}
\label{drckeys}
\begin{enumerate}
\item Setting up the transition state:  \comp{NLLSQ}, \comp{SIGMA}, or \comp{TS}.
\item Constructing the FORCE matrix:  \comp{FORCE} or \comp{IRC=$n$},
\comp{ISOTOPE}, \comp{LET}.
\item Starting an IRC:  \comp{RESTART} and \comp{IRC=$n$},  \comp{X-PRIO}, \comp{H-PRIO}.
\item Starting a DRC:  \comp{DRC} or \comp{DRC=$n.nn$}, \comp{KINETIC=$n.nn$},
\comp{T-PRIO}, etc..
\item Starting a DRC from a transition state:   (\comp{DRC}  or  \comp{DRC=$n$})  and
            \comp{IRC=$n$}, \comp{KINETIC=$n$}.
\item Restarting an IRC:  \comp{RESTART} and \comp{IRC}.
\item Restarting a DRC:  \comp{RESTART} and (\comp{DRC} or \comp{DRC=$n.nn$}).
\item Restarting a DRC starting from a transition state:  \comp{RESTART} and
            (\comp{DRC} or \comp{DRC=$n.nn$}).
\end{enumerate}
Other keywords, such as \comp{T=$nnn$} or \comp{GEO-OK} can be used any time.


\subsection*{Examples of DRC/IRC data}
Use of the IRC/DRC facility is quite complicated.  In the following examples
various `reasonable' options are illustrated for a calculation on water. It is
assumed  that  an  optimized  transition-state  geometry  is available.

Example  1:   Figure~\ref{h2odrc} illustrates a  Dynamic  Reaction   Coordinate
calculation,  starting  at   the transition  state  for  water  inverting, the
initial motion being opposite to the transition normal mode, with 6kcal of
excess kinetic  energy  added  in. Every point calculated is to be printed
(Note all coordinates are marked with a zero, and T-PRIO, H-PRIO and X-PRIO are
all absent).  The results of  an  earlier calculation using the same keywords
is assumed to exist. The earlier calculation would have constructed the force
matrix.   While the  total  cpu  time  is specified, it is in fact redundant in
that the calculation will run to completion in less than 600 seconds.

\index{IRC!example of}
\begin{figure}
\begin{makeimage}
\end{makeimage}
\begin{verbatim}
 KINETIC=6 RESTART  IRC=-1 DRC T=600
 WATER

      H   0.000000 0   0.000000 0   0.000000 0  0 0 0
      O   0.911574 0   0.000000 0   0.000000 0  1 0 0
      H   0.911574 0 180.000000 0   0.000000 0  2 1 0
      0   0.000000 0   0.000000 0   0.000000 0  0 0 0
\end{verbatim}
\caption{\label{h2odrc} Example of DRC calculation}
\end{figure}

Example 2:  Figure~\ref{h2oirc} shows an Intrinsic Reaction Coordinate
calculation.  Here the restart  is from a previous IRC calculation which was
stopped before the minimum was reached.  Recall that RESTART with IRC=$n$
implies  a  restart from  the FORCE calculation.  Since this is a restart from
within an IRC calculation the keyword IRC=$n$ has been replaced by IRC. \  IRC
on its  own (without the ``=$n$'') implies an IRC calculation from the starting
position---here the RESTART position---without initial
displacement.\index{IRC!example of restart}

\begin{figure}
\begin{makeimage}
\end{makeimage}
\begin{verbatim}
 RESTART  IRC  T=600
 WATER

      H   0.000000 0   0.000000 0   0.000000 0  0 0 0
      O   0.911574 0   0.000000 0   0.000000 0  1 0 0
      H   0.911574 0 180.000000 0   0.000000 0  2 1 0
      0   0.000000 0   0.000000 0   0.000000 0  0 0 0
\end{verbatim}
\caption{\label{h2oirc} Example of IRC calculation}
\end{figure}

\subsection*{Output format for IRC and DRC}
The IRC and DRC can produce  several  different  forms  of  output. Because of
the large size of these outputs, users are recommended to use search functions
to extract information.  To facilitate  this,  specific lines  have specific
characters.  Thus, a search for the ``\%'' symbol will summarize the energy
profile while a search  for ``AA'' will  yield  the coordinates of atom 1,
whenever it is printed.  The main flags to use in searches are:

\begin{description}
\item[\comp{\%}] Energies for all points calculated,    excluding extrema
\item[\comp{\%M}] Energies for all turning points
\item[\comp{\%MAX}] Energies for all maxima
\item[\comp{\%MIN}] Energies for all minima
\item[\comp{\%}] Energies for all points calculated
\item[\comp{AA*}] Internal coordinates for atom 1 for every point
\item[\comp{AE*}] Internal coordinates for atom 5 for every point
\item[\comp{123AB*}] Internal coordinates for atom 2 for point 123
\end{description}

As the keywords for the IRC/DRC are interdependent,  the  following list of
keywords illustrates various options.\index{DRC!keyword options}
\index{KINETIC}

\begin{description}
\item[\comp{DRC}] The Dynamic Reaction Coordinate is calculated.
Energy is conserved, and no initial impetus.
\item[\comp{DRC=0.5}] In the DRC kinetic energy is lost with a half-life of
0.5 femtoseconds.
\item[\comp{DRC=1.0}] Energy is put into a DRC with an half-life of
-1.0 femtoseconds, i.e., the system gains   energy.
\item[\comp{IRC}] The Intrinsic Reaction Coordinate is
calculated.  No initial impetus is given.
Energy not conserved.
\item[\comp{IRC=4}] The IRC is run starting with an impetus in the
negative of the 4th normal mode direction. The
impetus is one quantum of vibrational energy.
\item[\comp{IRC1 KINETIC=1}] The first normal mode is used in an IRC, with
the initial impetus being 1.0 kcal/mol.
\item[\comp{DRC KINETIC=5}] In a DRC, after the velocity is defined, 5 kcal
of kinetic energy is added in the direction of
the initial velocity.
\item[\comp{IRC=1 DRC KINETIC=4}] After starting with a 4 kcal impetus in the
direction of the first normal mode, energy is
conserved.
\item[\comp{DRC VELOCITY KINETIC=10}] Follow a DRC trajectory which starts with an
initial velocity read in, normalized to a
kinetic energy of 10 kcal/mol.
\end{description}


Instead of every point being printed, the option  exists  to  print specific
points  determined  by the keywords \comp{T-PRIORITY}, \comp{X-PRIORITY} and
\comp{H-PRIORITY}.  If any one of these words is specified, then the calculated
points  are used to define quadratics in time for all variables normally
printed.  In addition, if the flag for the first atom is set to  ``T''  then
all  kinetic  energy  turning  points  are printed.  If the flag for any other
internal coordinate is set to ``T'' then, when that coordinate  passes through
an extremum, that point will be printed.  As with the PRIORITY's, the point
will be calculated via  a  quadratic  to  minimize  non-linear errors.

N.B.:  Quadratics are unstable in the regions of inflection points; in  these
circumstances linear interpolation will be used.  A result of this is that
points printed in the  region  of  an  inflection  may  not correspond  exactly
to those requested.  This is not an error and should not affect the quality of
the results.

\subsection*{Test of DRC---verification of trajectory path}
Introduction:  Unlike  a  single-geometry  calculation  or  even  a geometry
optimization, verification of a DRC trajectory is not a simple task.  In this
section  a  rigorous  proof  of  the  DRC  trajectory  is presented;  it  can
be used both as a test of the DRC algorithm and as a teaching exercise.  Users
of the DRC are asked to  follow  through  this proof in order to convince
themselves that the DRC works as it should.

\subsection*{The nitrogen molecule}
For the nitrogen molecule (using MNDO) the equilibrium  distance is  $1.103816$
\AA, the heat of formation is 8.25741 kcal/mol and the vibrational frequency is
$2738.8$ cm$^{-1}$.   For  small  displacements, the  energy curve versus
distance is parabolic and the gradient curve is approximately linear, as is
shown in Table~\ref{n2}.         A  nitrogen molecule is thus a good
approximation to a harmonic oscillator.

% 40 lines, including this line
\begin{table}
\caption{\label{n2}Stretching Curve for Nitrogen Molecule}
\begin{center}
\begin{tabular}{rrr}
\multicolumn{1}{c}{N--N DIST} & \multicolumn{1}{c}{$\Delta H_f$}  & \multicolumn{1}{c}{GRADIENT}\\
\multicolumn{1}{c}{(\AA ngstroms)} &\multicolumn{1}{c}{(kcal/mol)} & \multicolumn{1}{c}{(kcal/mol/\AA ngstrom)}\\
\hline
1.11800   &  8.69441  &   60.84599 \\
1.11700   &  8.63563  &   56.70706 \\
1.11600   &  8.58100  &   52.54555 \\
1.11500   &  8.53054  &   48.36138 \\
1.11400   &  8.48428  &   44.15447 \\
1.11300   &  8.44224  &   39.92475 \\
1.11200   &  8.40444  &   35.67214 \\
1.11100   &  8.37091  &   31.39656 \\
1.11000   &  8.34166  &   27.09794 \\
1.10900   &  8.31672  &   22.77620 \\
1.10800   &  8.29611  &   18.43125 \\
1.10700   &  8.27986  &   14.06303 \\
1.10600   &  8.26799  &    9.67146 \\
1.10500   &  8.26053  &    5.25645 \\
1.10400   &  8.25749  &    0.81794 \\
1.10300   &  8.25890  &   -3.64427 \\
1.10200   &  8.26479  &   -8.12993 \\
1.10100   &  8.27517  &  -12.63945 \\
1.10000   &  8.29007  &  -17.17278 \\
1.09900   &  8.30952  &  -21.73002 \\
1.09800   &  8.33354  &  -26.31123 \\
1.09700   &  8.36215  &  -30.91650 \\
1.09600   &  8.39538  &  -35.54591 \\
1.09500   &  8.43325  &  -40.19953 \\
1.09400   &  8.47579  &  -44.87745 \\
1.09300   &  8.52301  &  -49.57974 \\
1.09200   &  8.57496  &  -54.30648 \\
1.09100   &  8.63164  &  -59.05775 \\
1.09000   &  8.69308  &  -63.83363 \\
\end{tabular}
\end{center}
\end{table}

\subsubsection{Period of vibration}
The period of vibration (time taken for the oscillator to undertake one
complete vibration, returning to its original position and velocity) can be
calculated in three ways.  Most direct is  the  calculation  from the  energy
curve; using the gradient constitutes a faster, albeit less direct, method,
while calculating it from the vibrational  frequency  is very  fast  but
assumes  that the vibrational spectrum has already been calculated.

\begin{enumerate}

\item From the energy curve. For a simple harmonic oscillator the period $r$ is
given by: $$ r = 2 \pi \sqrt{\frac{\mu}{k}}  $$ where $k$ is the
force constant.  \index{Reduced mass}\index{Force constant} The
reduced  mass, $\mu$,  (in amu)   of   a   nitrogen  molecule  is
$14.0067/2  =  7.00335$, and  the force-constant, $k$, can be
calculated from: $$E-c = (1/2) k(R-R_o)^2. $$ Given $R_o =
1.1038$, $R = 1.092$, $c = 8.25741$ and $E = 8.57496$~kcal/mol
then:
 \begin{eqnarray*}
k &=& 2*0.31755/(0.0118)^2 \; \mbox{(per mole)}\\
k &=& 4561.2 \mbox{ kcal/mol/A$^2$  (per mole)}\\
k &=& 1.9084*10^{30} \; \mbox{ ergs/cm$^2$ (per mole)}\\
k &=& 31.69*10^5  \; \mbox{ dynes/cm (per molecule)}\\
\end{eqnarray*}

(Experimentally, for N$_2$, k = $23*10^5$ dynes/cm )

Therefore:
$$ r = 2 \times 3.14159 \times \sqrt{\frac{7.0035}{1.9084\times 10^{30}}}
\;{\rm seconds} = 12.037 \times 10^{-15}\;{\rm s} = 12.037\;{\rm fs}. $$
If the frequency is calculated using the other half of the curve ($R=1.118,
E=8.69441$), then $k=12.333$ fs, or $k$, average, = 12.185 fs.

\item From the gradient curve. The force  constant  is  the  derivative  of
the  gradient  wrt distance:
$$ k = \frac{dG}{dx}. $$
Since we are using discrete points,  the  force  constant  is  best
obtained from finite differences:
$$ k = \frac{(G_2-G_1)}{(x_2-x_1)}. $$
For $x_2 = 1.1100$, $G_2 = 27.098$ and for $x_1 =  1.0980$,
$G_1  =  -26.311$,
giving rise to $k = 4450.75$ kcal/mol/\AA$^2$ and a period of $12.185$~fs.

\item From the vibrational frequency. Given a ``frequency'' (wavenumber) of
vibration of N$_2$ of $\bar{\nu}=2738.8$   cm$^{-1}$,  the period of
oscillation, in seconds, is given directly by:
$$ r = \frac{1}{c\bar{\nu}} = \frac{1}{2738.8 \times 2.998 \times 10^{10}} ,$$
or as $12.179$ fs.
\end{enumerate}

Summarizing, by three different methods the period  of  oscillation of N$_2$
is calculated to be $12.1851$, $12.185$ and $12.179$~fs, average $12.183$~fs.

\subsubsection{Initial dynamics of \mbox{N$_{2}$} with N--N distance = 1.094 \AA}
A useful check on the dynamics of N$_2$ is to  calculate  the  initial
acceleration  of  the  two  nitrogen  atoms  after releasing them from a
starting interatomic separation of 1.094 \AA.

At R(N-N) = 1.094 \AA, $G = -44.877$~kcal/mol/\AA\ or $-18.777 \times
10^{19}$~erg/cm. Therefore acceleration, $f = -18.777 \times 10^{19}
/14.0067$~cm/sec/sec, or $-13.405 \times 10^{18}$~cm/s$^2$, which is $ -13.405
\times 10^{15} \times$ Earth surface gravity.

Distance from equilibrium  $= 0.00980$ \AA. After $0.1$ fs, velocity is
$0.1\times  10^{-15} (-13.405 \times  10^{18})$ cm/sec or $1340.5$ cm/s.

In the  DRC  the  time-interval  between  points  calculated  is  a complicated
function of the curvature of the local surface.  By default, the first
time-interval is 0.105fs, so the calculated velocity  at  this time should be
$0.105/0.100 \times 1340.5 = 1407.6$ cm/s, in the DRC calculation the predicted
velocity is $1407.6$ cm/s.

The option is provided to allow sampling of the system at  constant
time-intervals,  the  default being $0.1$~fs.  For the first few points the
calculated velocities are given in Table~\ref{tdrc}.

\begin{table}
\caption{\label{tdrc} Velocities in DRC for N$_2$ Molecule}
\begin{center}
\begin{tabular}{rrrr}\\ \hline
      Time  & Calculated &  Linear &    Diff. in \\
            & Velocity  & Velocity &  Velocity \\ \hline
      0.000 &      0.0  &    0.0   &    0.0  \\
      0.100 &   1340.6  & 1340.5   &   -0.1  \\
      0.200 &   2678.0  & 2681.0   &   -3.0  \\
      0.300 &   4007.0  & 4021.5   &  -14.5  \\
      0.400 &   5325.3  & 5362.0   &  -36.7  \\
      0.500 &   6628.4  & 6702.5   &  -74.1  \\
      0.600 &   7912.7  & 8043.0   & -130.3  \\ \hline
\end{tabular}
\end{center}
\end{table}

As the calculated velocity is  a  fourth-order  polynomial  of  the
acceleration,   and  the  acceleration,  its  first,  second  and  third
derivatives, are all changing, the predicted velocity rapidly becomes  a poor
guide to future velocities.

For simple harmonic motion the velocity at any time is given by:
$$ v = v_0 \sin(2\pi t/r). $$
By fitting the computed velocities to simple harmonic motion, a much better fit
is obtained (Table~\ref{tdrc2}).

\begin{table}
\caption{\label{tdrc2} Modified  Velocities in DRC for N$_2$ Molecule}
\begin{center}
\begin{tabular}{rrrr} \hline
 & Calculated & Simple Harmonic  &    Diff. \\
Time  & Velocity  & 25325.Sin(0.5296t) &\\ \hline
            &           &                   &    \\
     0.000  &     0.0   &       0.0         &    0.0  \\
     0.100  &  1340.6   &    1340.6         &    0.0  \\
     0.200  &  2678.0   &    2677.4         &   +0.6  \\
     0.300  &  4007.0   &    4006.7         &   +0.3  \\
     0.400  &  5325.3   &    5324.8         &   +0.5  \\
     0.500  &  6628.4   &    6628.0         &   +0.4  \\
     0.600  &  7912.7   &    7912.5         &    0.0  \\ \hline
\end{tabular}
\end{center}
\end{table}

The repeat-time required for this  motion  is  $11.86$~fs,  in  good agreement
with  the  three  values calculated using static models.  The repeat time
should not be calculated from the time required to go from a minimum  to  a
maximum and then back to a minimum---only half a cycle. For all real systems
the potential energy is a skewed parabola, so  that the  potential energy
slopes are different for both sides; a compression (as in this case) normally
leads to a higher force-constant, and shorter apparent  repeat  time  (as in
this case).  Only the addition of the two half-cycles is meaningful.

\subsubsection{Conservation of normal coordinate}
So far this analysis has only considered a homonuclear diatomic.  A detailed
analysis  of  a  large  polyatomic  is  impractical,  and  for simplicity a
molecule of formaldehyde will be studied.

In polyatomics, energy can  transfer  between  modes.   This  is  a result  of
the non-parabolic nature of the potential surface.  For small displacements the
surface can be considered as  parabolic.   This  means that  for small
displacements interconversion between modes should occur only very slowly.  Of
the six normal modes, mode 1, at 1209.5~cm$^{-1}$, the in-plane C--H asymmetric
bend, is the most unsymmetric vibration, and is chosen to demonstrate
conservation of vibrational purity.

Mode 1 has a  frequency  corresponding  to  3.46  kcal/mol  and  a predicted
vibrational time of $27.58$~fs.  By direct calculation, using the DRC, the
cycle time is $27.59$~fs.  The rate of decay of this mode  has  an estimated
half-life of a few thousands femtoseconds.

\subsubsection{Rate of decay of starting mode}
For trajectories initiated by an IRC=$n$  calculation,  whenever  the
potential  energy is a minimum the current velocity is compared with the
supplied velocity.  The square of the cosine of the  angle  between  the two
velocity vectors is a measure of the intensity of the original mode in the
current vibration.

\subsubsection{Half-Life for decay of initial mode}
Vibrational purity is assumed to decay according to  zero'th  order kinetics.
The  half-life is thus $-0.6931472t/\log(<\!\psi^2\!>^2)$~fs, where
$<\!\psi^2\!>^2$ is the square of  the overlap integral of the wavefunction for
the  original vibration with that of the current  vibration.   Due to the  very
slow rate of decay of the starting mode, several half-life calculations
should  be  examined.   Only  when successive  half-lives  are  similar  should
any confidence be placed in their value.

\subsubsection{DRC print options}
The amount of output in the DRC is  controlled  by  three  sets  of
options.  These sets are:\index{H--PRIORITY}
\begin{itemize}
\item Equivalent Keywords \comp{H-PRIORITY}, \comp{T-PRIORITY}, and
\comp{X-PRIORITY}.
\item Potential Energy Turning Point option.
\item Geometry Maxima Turning Point options.
\end{itemize}
If \comp{T-PRIORITY} is used then  turning  points  cannot  be  monitored.
%Currently  \comp{H-PRIORITY} and \comp{X-PRIORITY} are not implemented,
%but will be as soon as practical.

\index{``T'' - Optimization flag} To monitor geometry turning points, put  a
``T'' in  place  of  the geometry optimization flag for the relevant geometric
variable. In the example shown in Figure~\ref{t}, the geometry of formaldehyde
would first be optimized, then a \comp{FORCE} calculation run, then a DRC
calculation started, using the first normal mode for the starting velocity.
Whenever the C=O bond length becomes a maximum or a minimum, a message is
printed.

\begin{figure}
\begin{makeimage}
\end{makeimage}
\begin{verbatim}
 IRC=1 DRC T=20
 Formaldehyde
 Monitoring the C=O Bond-length turning points
  O    0.0 0    0.0 0    0.000000 0   0 0 0
  C    1.2 T    0.0 0    0.000000 0   1 0 0
  H    1.0 1  120.0 1    0.000000 0   2 1 0
  H    1.0 1  120.0 1  180.000000 0   2 1 3
  0    0.0 0    0.0 0    0.000000 0   0 0 0
\end{verbatim}
\caption{\label{t} Example of DRC calculation, monitoring a geometric variable}
\end{figure}

To monitor the potential energy turning points, put a ``T'' for  the flag for
atom 1 bond length (Do not forget to put in a bond-length (zero will do)!).

To monitor the geometry, use \comp{LARGE=$n$}.  This will cause the geometry to
be printed once every $n$ steps.

The effect of using these flags together is as follows.
\begin{enumerate}
\item No options:  All calculated points will be printed.  No turning points
will be calculated.

\item Atom 1 bond length flagged with a ``T'': If  \comp{T-PRIO},  etc.\ are
NOT  specified,  then  potential  energy turning points will be printed.

\item Internal coordinate flags set to ``T'':  If \comp{T-PRIO}, etc.  are NOT
specified,  then geometry extrema will be printed.  If only one coordinate is
flagged, then the turning point will be displayed in  chronologic  order; if
several are flagged then all turning points occurring in a given time-interval
will  be  printed  as they  are  detected.   In  other  words,  some  may  be
out of chronologic order.  Note that each coordinate flagged will give rise  to
a different geometry:  minimize flagged coordinates to minimize output.

\item Potential and geometric flags set:  The effect is equivalent to the sum
of the first two options.

\item \comp{T-PRIO} set:  No turning points will be  printed,  but  constant
time-slices  (by  default  $0.1$~fs)  will  be used to control the print.
\end{enumerate}
\index{DRC|)}\index{IRC|)}
