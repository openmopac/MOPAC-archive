\subsection{Outer Valence Green's Function}\label{gf}
This section is based on materials supplied by
\begin{center}Dr David Danovich\\The Fritz Haber Research Center for 
Molecular Dynamics\\ The Hebrew University of Jerusalem\\ 91904 Jerusalem\\
Israel \end{center}

The OVGF technique was used with the self-energy part extended to include
third order perturbation corrections,~\cite{gf1}.  The higher order contributions
were estimated by the renormalization procedure.  The actual expression used to
calculate the self-energy part, $\sum_{pp}(w)$, chosen in the diagonal form,
is given in equation~(\ref{gfeq1}), where $\sum_{pp}^{(2)}(w)$ and $\sum_{pp}^{(3)}(w)$ are
the second- and third-order corrections, and $A$ is the screening factor accounting
for all the contributions of higher orders.
\begin{equation}\label{gfeq1}
\sum_{pp}(w) = \sum_{pp}^{(2)}(w)+(1-A)^{-1}\sum_{pp}^{(3)}(w).
\end{equation}
The particular expression which was used for the second-order corrections is given in
equation~(\ref{gfeq2}).
\begin{equation}\label{gfeq2}
\sum_{pp}^{(2)}(w) = \sum_a\sum_{i,j}\frac{(2V_{paij}-V_{paji})V_{paij}}{w+e_a-e_i-e_j}
+\sum_{a,b}\sum_i\frac{(2V_{piab}-V_{piba})V_{piab}}{w+e_i-e_a-e_b},
\end{equation}
where
$$
V_{pqrs} = \int\int\psi_p^*(1)\psi_q^*(2)(1/r_{12})\psi_r^*(1)\psi_s^*(2){\rm d}\tau_1{\rm d}\tau_2.
$$

In equation~(\ref{gfeq2}), $i$ and $j$ denote occupied orbitals, $a$ and $b$
denote virtual orbitals, $p$ denotes orbitals of unspecified occupancy, and
$e$  denotes an orbital energy. The equations are solved by an iterative
procedure, represented in  equation~(\ref{gfeq3}).
\begin{equation}\label{gfeq3}
w_p^{i+1}=e_p+\sum_{pp}(w^i).
\end{equation}

The SCF energies and the corresponding integrals, which were calculated by one
of the semiempirical methods (MNDO, AM1, or PM3), were taken as the zero'th
approximation and all M.O.s may be included in the active space for the OVGF
calculations.

The expressions used for $\sum_{pp}^{(3)}$ and $A$ are given in \cite{gf2}.

The OVGF method itself, is described in detail in \cite{gf1}.

%\subsubsection{Example of OVGF calculation}
%  The data-set {\bf test\_greenf.dat} will calculate the first 8 I.P.s
%for dimethoxy-$s$-tetrazine.  This calculation is discussed in detail in \cite{gf6}.
%The experimental and calculated I.P.s are shown in Table~\ref{gftab}.
%\begin{table}
%\caption{\label{gftab}OVGF Calculation, Comparison with Experiment}
%\begin{center}
%\begin{tabular}{lccccc}\\
%M.O.    &  Expt*  &   PM3   & Error   &  OVGF(PM3)  &  Error \\
%$n_1$    &  9.05   &   10.15 &  1.10   &   9.46      &   0.41 \\
%$\pi_1$  &  9.6    &   10.01 &  0.41   &   9.65      &   0.05 \\
%$n_2$    &  11.2   &   11.96 &  0.76   &  11.13      &  -0.07 \\
%$\pi_2$  &  11.8   &   12.27 &  0.47   &  11.43      &  -0.37 \\
%\end{tabular}
%
%*: R. Gleiter,  V. Schehlmann, J. Spanget-Larsen, H. Fischer and F. A. Neugebauer,
%{\em J. Org. Chem.}, {\bf 53}, 5756 (1988).
%\end{center}
%\end{table}
%
%
%From this, we see that for PM3 the average error is 0.69eV, but after OVGF 
%correction, the error drops to 0.22eV.  This is typical of nitrogen heterocycle 
%%calculations.
