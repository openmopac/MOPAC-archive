\chapter{Debugging}
There are three potential sources of  difficulty  in  using  MOPAC, each  of 
which  requires special attention.  There can be problems with data, due to
errors in the data, or MOPAC  may  be  called  upon  to  do calculations  for
which it was not designed.  There are intrinsic errors in MOPAC which extensive
testing has  not  yet  revealed,  but  which  a user's novel calculation
uncovers.  Finally there can be bugs introduced by the user modifying MOPAC,
either to make it compatible with the  host computer, or to implement local
features.

For whatever reason, the user may  need  to  have  access  to  more
information  than  the  normal  keywords  can provide, and a second set,
specifically  for  debugging,  is   provided.    These   keywords   give
information  about  the  working  of  individual subroutines, and do not affect
the course of the calculation.

\section{Keywords for use in debugging}
\index{Keywords!debugging}
A full list of keywords for debugging subroutines:

\begin{description}
\item[\comp{1ELEC}] The one-electron matrix. See Note 1.
\item[\comp{CARTAB}] Print all the character tables.
\item[\comp{CHARST}] Calculation of state symmetry characters.
\item[\comp{COMPFG}] Heat of Formation.
\item[\comp{DCART}] Cartesian derivatives.
\item[\comp{DEBUG}] See Note 2.
\item[\comp{DEBUGPULAY}] Pulay matrix, vector, and error-function. See Note 3.
\item[\comp{DENSITY}] Every density matrix. See Note 1.
\item[\comp{DERI1}] Details of DERI1 calculation.
\item[\comp{DERI2}] Details of DERI2 calculation.
\item[\comp{DERITR}] Details of DERITR calculation.
\item[\comp{DERIV}] All gradients, and other data in DERIV.
\item[\comp{DERNVO}] Details of DERNVO calculation.
\item[\comp{DFORCE}] Print Force Matrix. 
\item[\comp{DIAGG2}] Details of the annihilation operation.
\item[\comp{DIIS}] Details of DIIS calculation.
\item[\comp{EIGS}] All eigenvalues.
\item[\comp{FLEPO}] Details of BFGS minimization.
\item[\comp{FMAT}] See Note 1.
\item[\comp{FOCK}] Every Fock matrix.
\item[\comp{GEOCHK}] Details of the check on geometry.
\item[\comp{GROUP}] Determination of symmetry group.
\item[\comp{HCORE}] The one electron matrix, and two electron integrals.
\item[\comp{HYBRID}] Construction of hybrid atomic orbitals.
\item[\comp{ITER}] Values of variables and constants in ITER.
\item[\comp{LARGE}] Increases amount of output generated by other keywords.
\item[\comp{LEWIS}] Print the connectivity and Lewis structure.
\item[\comp{LINMIN}] Function values, step sizes at all points in the line 
minimization (LINMIN or SEARCH).
\item[\comp{MAKVEC}] Construction of the initial LMOs.
\item[\comp{MOLDAT}] Molecular data, number of orbitals, ``U'' values, etc.
\item[\comp{MOLSYM}] Calculation of molecular symmetry.
\item[\comp{MECI}] C.I.\ matrices, M.O.\ indices, etc.
\item[\comp{PL}] Differences between density matrix elements in ITER. See Note 4.
\item[\comp{SIZES}] Print sizes of all arrays created and destroyed.  Note 5.
\item[\comp{SYMOIR}] Symmetry characters for vibrational frequencies.
\item[\comp{SYMTRZ}] Symmetry determination (point group and IR's).
\item[\comp{TIDY}] Tidy up of the LMOS (removing unused space).
\item[\comp{TIMES}] Times of stages within ITER.
\item[\comp{VECTORS}] All eigenvectors on every iteration. See Note 1.
\end{description}

\subsection*{Notes}
\begin{enumerate}
\item These keywords are activated by the  keyword  \comp{DEBUG}.   Thus  if
\comp{DEBUG}  and  \comp{FOCK} are both specified, every Fock matrix on every
iteration will be printed.
\item \comp{DEBUG} is not intended to increase the output,  but  does  allow
other keywords to have a special meaning.
\item \comp{PULAY} is already  a  keyword,  so  \comp{DEBUGPULAY}  was  an 
obvious alternative.
\item \comp{PL} initiates the output of the value of the largest  difference
between  any  two  density  matrix  elements on two consecutive iterations. 
This is very useful when investigating options for increasing the rate of
convergence of the SCF calculation.
\item \comp{SIZES} is very useful in monitoring the sizes and order of creation
and annihilation of arrays.
\end{enumerate}

                    
\subsection*{Suggested procedure for locating bugs}
\index{Bugs!locating}
Users are supplied with the source code for MOPAC, and,  while  the original 
code is fairly bug-free, after it has been modified there is a possibility that
bugs may have been introduced.  In these  circumstances the  author  of  the 
changes  is obviously responsible for removing the offending bug, and the 
following  ideas  might  prove  useful  in  this context.

First of all, and most important, before any modifications are done a  back-up 
copy  of the standard MOPAC should be made.  This will prove invaluable in
pinpointing deviations from the  standard  working.   This point  cannot  be 
over-emphasized---{\em make  a  back-up before modifying MOPAC!}

Clearly, a bug can occur almost  anywhere,  and  a  logical  search sequence is
necessary in order to minimize the time taken to locate it.

If possible, perform the debugging with a small molecule, in  order to  save 
time  (debugging  is,  of  necessity,  time  consuming) and to minimize output.

The two sets of subroutines  in  MOPAC,  those  involved  with  the
electronics  and  those  involved  in  the geometrics, are kept strictly
separate, so the first question to be answered is which set contains the bug.  
If the heats of formation, derivatives, I.P.s, and charges, etc., are correct, 
the  bug  lies  in  the  geometrics;  if  faulty,  in  the electronics.

        
\subsubsection{Bugs discovered in the electronics subroutines}
Use formaldehyde for this test.  Use keywords \comp{1SCF}, \comp{DEBUG}, and 
any others necessary.

The main steps are:
\begin{enumerate}
\item Check  the  starting  one-electron  matrix  and   two-electron integral 
string, using the keyword \comp{HCORE}.  It is normally sufficient to verify
that the two hydrogen atoms  are  equivalent,  and  that  the $\pi$ system 
involves  only $p_z$  on  oxygen  and carbon.  Note that numerical values are
not checked, but only relative values.

If an error is found, use \comp{MOLDAT} to verify the  orbital  character, etc.

If faulty the error lies in \comp{READMO}, \comp{GETGEO} or \comp{MOLDAT}.

Otherwise the error lies in \comp{HCORE}, \comp{H1ELEC} or \comp{ROTATE}.

If the starting matrices are correct, go on to step (2).

\item Check the density or Fock matrix on every iteration,  with  the words
\comp{FOCK} or \comp{DENSITY}.  Check the equivalence of the  two hydrogen
atoms, and the $\pi$ system, as in (1).

If an error is found, check the first Fock matrix.  If faulty,  the bug  lies 
in ITER, probably in the Fock subroutines FOCK1 or FOCK2,  or in the (guessed)
density matrix (MOLDAT).  An exception is  in  the  UHF closed-shell 
calculation,  where  a  small  asymmetry  is introduced to initiate the
separation of the $\alpha$ and $\beta$ UHF wavefunctions.

If no error is found, check the second Fock matrix.  If faulty, the error lies
in the density matrix DENSIT, or the diagonalization RSP.

If the Fock matrix is acceptable, check all the Fock matrices.   If the  error
starts in iterations 2 to 4, the error probably lies in CNVG, if after that, in
PULAY, if used.

If SCF is achieved, and the heat  of  formation  is  faulty,  check HELECT. 
If C.I.\ was used check MECI.

If the derivatives are faulty, use \comp{DCART} to  verify  the  Cartesian
derivatives.   If  these  are  faulty, check DCART and DHC.  If they are
correct,  or  not  calculated,  check  the   DERIV   finite   difference
calculation.   If the wavefunction is non-variationally optimized, check
DERNVO.
\end{enumerate}

\subsubsection{Bugs discovered in the geometric subroutines}
If the geometric calculation is faulty, use \comp{FLEPO} or \comp{PRNT=5}  to 
monitor  the optimization, \comp{DERIV} may also be useful here.

For  the  FORCE  calculation,  \comp{DCART}  or  \comp{DERIV}  are    useful  
for variationally   optimized   functions,   \comp{COMPFG}   for 
non-variationally optimized functions.

For reaction paths, verify that FLEPO is working correctly; if  so, then PATHS
is faulty.

For  saddle-point  calculations,  verify  that  FLEPO  is   working correctly;
if so, then REACT1 is faulty.

Keep in mind the fact that MOPAC is a large calculation, and, while intended 
to  be  versatile,  many combinations of options have not been tested.  If a
bug is found in  the  original  code,  please  communicate details  to    Dr.\
James J.\ P.\ Stewart, Stewart Computational Chemistry, 15210 Paddington
Circle,   Colorado  Springs,  CO 80921-2512; E-mail:
\htmladdnormallink{\comp{jstewart@fujitsu.com}}{mailto:jstewart@fujitsu.com}.
