\chapter {Bibliography }

\subsubsection*{Elements}
\begin{description}
\item[MINDO/3]~\\
The oldest (1975) of the methods used in MOPAC~\cite{mopac} is
MINDO/3~\cite{mindo3}, which was parameterized for H, B, C, N, O, F, Si, P, S,
and Cl~\cite{mindo3}. Some faults were found in the P--O, P--F and P--Cl
parameters~\cite{mindo3-p}, but these were \emph{not} corrected in MOPAC (see
Section~\ref{mindo3-p}).

\item[MNDO]~\\
Later, in 1977 MNDO~\cite{mndo1} appeared.   Initially, parameters were
available for C, H, N, and O only~\cite{mndo1}.  Results were much
better~\cite{mndo2} than  for MINDO/3, and MNDO was soon extended to
B~\cite{mndo-al}, F~\cite{mndo-f},
%Na~\cite{mndo-na},
Al~\cite{mndo-al}, Si~\cite{mndo-si},  P~\cite{mndo-p}, S~\cite{mndo-s},
Cl~\cite{mndo-cl},
% K~\cite{mndo-k},
Zn~\cite{mndo-zn},  Ge~\cite{mndo-ge}, Br~\cite{mndo-br}, Sn~\cite{mndo-sn},
I~\cite{mndo-i}, Hg~\cite{mndo-hg}, and Pb~\cite{mndo-pb}.

\item[AM1]~\\
Many of the faults in MNDO were corrected in AM1~\cite{am1}, which also was
initially available only for H, C, N, and O~\cite{am1}.  AM1 has since been
extended to B~\cite{am1-b}, F~\cite{am1-x}, Al~\cite{am1-al},
Si~\cite{am1-si},  P~\cite{am1-p}, S~\cite{am1-s}, Cl~\cite{am1-x},
Zn~\cite{am1-zn}, Ge~\cite{am1-ge},  Br~\cite{am1-x},  I~\cite{am1-x}, and
Hg~\cite{am1-hg}. Four unpublished elements available under AM1 are As, Se, Sb,
and Te.

\item[PM3]~\\
In 1985, PM3~\cite{pm3-1} was developed.  Initially, 12 elements were
available: H, C, N, O, F, Al, Si, P, S, Cl, Br, and I~\cite{pm3-2}. This set
was expanded in 1991 to include Be, Mg, Zn, Ga, Ge, As, Se, Cd, In, Sn, Sb, Te,
Hg, Tl, Pb, and Bi~\cite{pm3-3}.  In 1993 lithium was
parameterized~\cite{pm3-li}.

\item[MNDO-$d$]~\\
When $d$ orbitals are added to the basis set~\cite{mndod}, the accuracy of the
method rises.  At present (1998), parameters are available for Al, Si, P, S,
Cl, Br, and I~\cite{mndod}.
\end{description}

\subsubsection*{Geometry}
By default, geometries are optimized using Baker's EF routine~\cite{ef-ts}. If
this is not desired, then the
Broyden~\cite{bfgs1}-Fletcher~\cite{bfgs2}-Goldfarb~\cite{bfgs3}-Shanno~\cite{bfgs4}
method can be used. This replaces the older
Davidon~\cite{dfp1}-Fletcher~\cite{dfp2}-Powell~\cite{dfp3} method.   The
Powell line search in both the BFGS and DFP methods has been upgraded with
Thiel's FSTMIN technique~\cite{fstmin}.

The energy minimum for intersystem crossing can be calculated using the method
of Anglada and Bofill~\cite{cross}.

Reaction paths can be followed, but sometimes (unavoidable) numerical
instability causes difficulty~\cite{boyd}.

Various methods can be used to locate transition states. In the region  of a
transition state, TS~\cite{ef-ts} is the best method to use. Other options are
NLLSQ~\cite{nllsq} and SIGMA~\cite{sigma1,sigma2}.

However, getting to this region can often be difficult.  One effective strategy
is to use the reaction path option.  A more costly method is the \comp{SADDLE}
technique~\cite{saddle}.  Once the transition state is located, the geometry
must be refined

All systems can be characterized by determining the number of imaginary
vibrations using FORCE~\cite{pulayf}.  Ground state systems should have none,
and transition states should have exactly one imaginary frequency. The normal
modes can then be used to calculate heat capacities and entropies and other
thermodynamic quantities~\cite{thermo}.

Molecular dynamics can be followed via DRC~\cite{drc}.

\subsubsection*{Electronics}
\begin{description}
\item[SCF]~\\
Semiempirical methods use approximations to the
Roothaan~\cite{roothaan}-Hall~\cite{hall} equations.  The four approximate
methods are MINDO/3~\cite{mindo3}, MNDO~\cite{mndo1}, AM1~\cite{am1}, and
PM3~\cite{pm3-1} For RHF open shell systems, a
Multi-Electron-Configuration-Interaction procedure~\cite{meci} is available.
SCF convergence can be assisted by an inexpensive  SHIFT
technique~\cite{shift},  by a slightly more complicated Direct Inversion of the
Iterative Subspace  method (DIIS)~\cite{diis}, or by an expensive but
sophisticated interpolation  procedure~\cite{king}.

For large systems, an alternative to conventional methods is to use localized
molecular orbitals~\cite{mozyme}.  This is usually more efficient, particularly
for geometry optimization~\cite{crambin}.

\item[Gradients]~\\
For variationally-optimized wavefunctions, two derivative methods are
available.  The default is to use finite difference, but under user request
(\comp{ANALYT}), analytical derivatives can be used~\cite{analyt}. When
analytical derivatives are used, STO-6G~\cite{sto-6g} Gaussian functions are
used instead of Slater orbitals.

When non-variationally optimized wavefunctions are used, the derivatives
are much more complicated~\cite{analci}.
\end{description}
\subsubsection*{General}
Fundamental constants are taken from the CODATA report~\cite{codata}.
A good introduction to MOPAC can be found in Tim Clark's book~\cite{tclark}.
\begin{description}
\item[Results]~\\
The SCF M.O.s, which diagonalize the Fock matrix, can be localized~\cite{local}
to give M.O.s which can be identified with the conventional picture of
two-electron bonds and lone pairs. The localization scheme is faster at the
semiempirical level than the Edmiston-Ruedenberg~\cite{edmrue} or
Boys~\cite{boys} methods.  Associated with each conventional M.O.\ is a
bond-index~\cite{m.o.valency}, which represents the contribution to  the
bond-order matrix due to each M.O.\ Bond orders and valencies can be displayed
by use of \comp{BONDS}~\cite{bonds}. Other phenomena relating to bonding can
also be calculated~\cite{nrm85,mrb86,mb89}.  An alternative to the normal
Coulson density matrix is the Mulliken~\cite{mullik1,mullik2} matrix, invoked
by \comp{MULLIK}.

Dipole moments for ions are coordinate system dependent, so by
definition~\cite{dipoles_for_ions},\index{Ions!dipole moment for} the center of
mass is used. Higher terms, e.g., polarizability, and first and second
hyperpolarizability, can be calculated~\cite{polar} by \comp{POLAR}.

Ionization potentials~\cite{koopmans} can be corrected using Green's
Functions~\cite{gf1,gf2,gf3,gf4,gf5,gf6}

\item[Solvent and Electrostatics]~\\
Solvent phenomena can be studied.  The COSMO technique~\cite{cosmo}, unlike the
self-consistent reaction fields~\cite{scrf}, allows  geometries to be
optimized. Although the Miertus-Scrocco-Tomasi model~\cite{mst,mt} cannot
optimize geometries, is more sophisticated in that it allows cavitation
effects.  This model has been modified~\cite{lno,glo,nol,lbo,obl}  to  allow
NDDO methods to be used. In this, optimized VdW radii~\cite{ojl,blo} are used
to construct~\cite{pastb} a cavity.

The free energy of hydration is computed as the addition of three
contributions:

\begin{enumerate}
\item The electrostatic term, which is computed from the linear free energy
response theory~\cite{mst,mt,lno,glo,nol,lbo,obl}.
\item The cavitation contribution, which is computed from Pierotti's scaled
particle theory~\cite{rap}.
\item The van der Waals terms, which is computed using a linear relation with
the solute accessible surface, and optimized ``hardness''
parameters~\cite{ojl,blo}.
\end{enumerate}

In addition to the free energy of hydration a ``solvent-adapted'' wavefunction
is obtained. Such a wavefunction can be used to determine changes in
solute properties due to the solvent~\cite{lobg,lo,lo2,lobg2}.

Electrostatic potentials can be used with the MST method both by deorthogonalizing
the wavefunction~\cite{lio,alo,lo3} and by keeping the wavefunction
orthogonal~\cite{frr,alo2}

Other ESP methods available are the Merz-Bessler technique~\cite{esp} and
the Ford-Wang procedure~\cite{pmep1,pmep2}.  The Ford-Wang is much faster and
more accurate than the Merz-Bessler method, but is limited to AM1 calculations
on systems containing H, C, N, O, F, and Cl, only.
\end{description}
