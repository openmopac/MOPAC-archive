\subsection{Atom Transition Moments}\index{Transition!dipole moments}\index{Polarizability!atomic transition}
\index{Dipole!transition}\index{Exponents!transition dipole}
The probability, $B_{0\rightarrow *}$, that a photon will be absorbed by a system that has a
ground state $\Psi_0$ and an excited state $\Psi_*$ separated by an energy $\epsilon$ when
irradiated by an energy density $\rho_{\epsilon}$
is given by
$$
B_{0\rightarrow *} = \frac{2\pi}{3\hbar^2}|R_{0*}|^2\rho_{0*},
$$
in which
$$
|R_{0*}|^2 = |X_{0*}|^2 + |Y_{0*}|^2 + |Z_{0*}|^2.
$$

$X_{0*}$ is the matrix element for the $x$ component of the dipole moment:
$$
X_{0*} = \int \Psi_0|e\sum_j \vec{x_j} |\Psi_* d\tau
$$

Evaluation of this integral requires evaluating the effect of the operators  $\vec{x_j}$,
$\vec{y_j}$, and $\vec{z_j}$.
acting on an atomic orbital.  Tables \ref{transx}, \ref{transy}, and \ref{transz} show
the integrals of the type $<\! \phi_{\lambda}
 |\vec{r_j}|\phi_{\sigma}\!>$, where $\phi_{\lambda}$ and $\phi_{\sigma}$ are
 pairs of atomic orbitals.



The integral
 $<\! \phi_{\lambda}|\vec{r}|\phi_{\lambda}\! >$ is simply the appropriate Cartesian coordinate,
that is, the $x$, $y$, or $z$ coordinate of the atom that $\phi_{\lambda}$ is on. $<\! ns|\vec{r}|np\! >$
 and  $<\! np|\vec{r}|nd\! >$
 can be evaluated using the following expressions:

$$
<\! ns|\vec{r}|np\! > = a_0\frac{(2n+1).2^{2n+1}.(\xi_s\xi_p)^{n+1/2}}{\sqrt{3}(\xi_s+\xi_p)^{2n+2}}
$$

$$
<np|\vec{r}|nd>=a_0\frac{(n_p+n_d+1)!.2^{n_p+n_d+1}.\xi_p^{n_p+1/2}.\xi_d^{n_d+1/2}}
{\sqrt{5}(\xi_p+\xi_d)^{n_p+n_d+2}.\sqrt{(2n_p)!(2n_d)!}},
$$
where $ns$, $np$, and $nd$ are $s$, $p$, and $d$ quantum numbers, respectively.
For the $sp$ transition, $n=ns=np$.
The Slater orbital exponents, $\xi_s$, $\xi_p$, and $\xi_d$ are usually
given in atomic units, that is, in
inverse Bohr, therefore they must converted to \AA ngstroms before use, hence the
presence of the $a_0=0.529$ in these expressions.

All integrals of the type used here are in \AA ngstroms, therefore the units of the integral
of the dipole operator on a M.O. is also in \AA ngstroms:
$$
<\! \psi_i|\vec{r}|psi_j\! > =\sum_{\lambda}\sum_{\sigma}c_{\lambda i}c_{\sigma j}
<\! \phi_{\lambda}\vec{r}\phi_{\sigma} \!>,
$$
as is the integral over microstates.  Evaluation of the integrals over microstates is straightforward,
in that all integrals are zero, unless the number of differences between the microstates is exactly two,
in which case the integral is equal to that of the two M.O.s involved, times a phase factor.  That is,
for each pair of microstates that are identical, except for $\psi_i$
in $\Psi_a$ and  $\psi_j$ in $\Psi_b$, the integral is:.

$$
<\! \Psi_a|\vec{r}|\Psi_b\! > =<\! \psi_i|\vec{r}|\psi_j\! >*(-1)^n,
$$
where $n$ is the number of permutations necessary to move $\psi_i$ in microstate $\Psi_a$ to the position
occupied by $\psi_j$ in microstate $\Psi_b$.
\begin{table}
\caption{\label{transx} ``$x$" Transition Integrals}
\begin{center}
\begin{tabular}{l|ccccccccc} \hline
& $s$  &  $p_x$  &  $p_y$  &  $p_z$  &  $d_{x^2-y^2}$  & $d_{xz}$  &
$d_{z^2}$  &  $d_{yz}$  &  $d_{xy}$ \\ \hline
$s$ & X$_A$\\
$p_x$ & sp & X$_A$\\
$p_y$  & 0 & 0 & X$_A$ \\
$p_z$  & 0 & 0 & 0 & X$_A$\\
$d_{x^2-y^2}$ & 0 & pd & 0 & 0 & X$_A$\\
$d_{xz}$      & 0 & 0 & 0 & pd & 0 & X$_A$\\
$d_{z^2}$     & 0 & -$\frac{1}{\sqrt{3}}$pd & 0 & 0 & 0 & 0 & X$_A$\\
$d_{yz}$      & 0 & 0 & 0 & 0 & 0 & 0 & 0 & X$_A$\\
$d_{xy}$      & 0 & 0 & pd & 0 & 0 & 0 & 0 & 0 & X$_A$\\  \hline


\end{tabular}\\
Note: X$_A$ = $<\! \phi_{\lambda}|\vec{r}|\phi_{\lambda}\! >$; sp = $<\! ns|\vec{r}|np\! >$; pd = $<\! np|\vec{r}|nd\! >$.
\end{center}
\end{table}

\begin{table}
\caption{\label{transy} ``$y$" Transition Integrals}
\begin{center}
\begin{tabular}{l|ccccccccc} \hline
& $s$  &  $p_x$  &  $p_y$  &  $p_z$  &  $d_{x^2-y^2}$  & $d_{xz}$  &
$d_{z^2}$  &  $d_{yz}$  &  $d_{xy}$ \\ \hline
$s$ & Y$_A$\\
$p_x$ & 0 & Y$_A$\\
$p_y$  & sp & 0 & Y$_A$ \\
$p_z$  & 0 & 0 & 0 & Y$_A$\\
$d_{x^2-y^2}$ & 0 & 0 & -pd & 0 & Y$_A$\\
$d_{xz}$      & 0 & 0 & 0 & 0 & 0 & Y$_A$\\
$d_{z^2}$     & 0 & 0 & -$\frac{1}{\sqrt{3}}$pd & 0 & 0 & 0 & Y$_A$\\
$d_{yz}$      & 0 & 0 & 0 & pd & 0 & 0 & 0 & Y$_A$\\
$d_{xy}$      & 0 & pd & 0 & 0 & 0 & 0 & 0 & 0 & Y$_A$\\  \hline


\end{tabular}\\
\end{center}
\end{table}

\begin{table}
\caption{\label{transz} ``$z$" Transition Integrals}
\begin{center}

\begin{tabular}{l|ccccccccc} \hline
& $s$  &  $p_x$  &  $p_y$  &  $p_z$  &  $d_{x^2-y^2}$  & $d_{xz}$  &
$d_{z^2}$  &  $d_{yz}$  &  $d_{xy}$ \\ \hline
$s$ & Z$_A$\\
$p_x$ & 0 & Z$_A$\\
$p_y$  & 0 & 0 & Z$_A$ \\
$p_z$  & 0 & 0 & sp & Z$_A$\\
$d_{x^2-y^2}$ & 0 & 0 & 0 & 0 & Z$_A$\\
$d_{xz}$      & 0 & pd & 0 & 0 & 0 & Z$_A$\\
$d_{z^2}$     & 0 & 0 & 0 & $\frac{2}{\sqrt{3}}$pd & 0 & 0 & Z$_A$\\
$d_{yz}$      & 0 & 0 & pd & 0 & 0 & 0 & 0 & Z$_A$\\
$d_{xy}$      & 0 & 0 & 0 & 0 & 0 & 0 & 0 & 0 & Z$_A$\\  \hline
\end{tabular}\\
\end{center}
\end{table}


Finally, the state transition dipole can be calculated from:
$$
<\! \Psi_A|\vec{r}|\Psi_B\! > =\sum_a\sum_bc_{A a}c_{B b}<\! \Psi_a\vec{r}\Psi_b \!>,
$$

Although the transition dipole is normally regarded as involving the ground and an excited state, it is
possible to calculate the transition between two excited states.  The initial state is, by default, the
ground state, however if \comp{ROOT=n} $n\neq 1$, or any other keyword that specifies
a state other than the ground state, then the initial state will be an excited state.

For degenerate states, the transition dipole is the sum over all states involved.
