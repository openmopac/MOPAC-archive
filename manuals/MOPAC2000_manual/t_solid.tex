\section{Solid state capability}\label{solid-state}\index{Solid state!description}
%{\sc Users are warned that the two and three dimensional calculations are
%relatively new, and only limited testing has been done to verify their
%validity.  Extreme caution should be exercised when these calculations are
%being done.  In particular, care should be taken to ensure that the cluster
%is sufficiently large that the translation vector conditions described
%below are met. Many other considerations, such as the significance of using
%only one point in $k$-space, have not been fully explored.  The solid-state
%capability should be regarded as a research tool, and should not be used
%except be researchers familiar with solid-state concepts.}

\subsection{Constructing Data Sets}
Setting up the data set for a solid is much more complicated than
setting up the equivalent data set for a molecule. For example,
the data set for iodine, I$_2$, might take say 20 seconds to type
up. For crystalline I$_2$, the data set might take 20 minutes to
an hour to set up. The reason for this is the presence in solids
of the translation vectors. In MOPAC, these are represented by the
symbol ``Tv", short for Translation Vector. Translation vectors
must be the last entries in the Z-matrix, are defined in terms of
the positions of atoms or dummy atoms.

The recommended procedure for constructing a data set for a solid
is as follows:

Build a primitive unit cell. In this work, the action of simple
translations on a primitive unit cell would make the solid. That
is, other operations, such as rotation, non-primitive translation,
or reflection should not be necessary.

Use MAKPOL to build a cluster. The number of primitive unit cells
in each direction should be sufficiently large that the shortest
distance between two opposite faces in the cluster is at least 10
\AA ngstroms.

Using an editor (vi, Notepad, or WORD, in order of preference),
add symmetry to the system.

Run using MOPAC

Worked examples of this procedure, with explanation of each step
are shown at the end of this section, see P. \pageref{diamond}.


\subsection{The Cluster}
Unlike more conventional methods, MOPAC does not normally use a fundamental
unit cell.  Neither does it sample the Brillouin Zone in order to model the
electronic structure.  Instead, it uses a large unit cell, called a `cluster',
\index{Born-von K\'{a}rm\'{a}n! periodic boundary conditions} \index{Periodic
boundary conditions|ff} and applies the Born-von K\'{a}rm\'{a}n
~\cite{bornvon1,bornvon2} periodic  boundary conditions.  In this discussion,
the term `solid' is intended to include polymers, layer systems, and true
solids, unless otherwise indicated by the context.

If a unit cell of a solid is large enough,  then  a  single  point  in
\index{Gamma@{$\Gamma$ point!in solid state}} $k$-space,  the $\Gamma$ point,
is sufficient to specify the entire Brillouin zone.  The secular determinant
for this  point  can  be  constructed  by adding together the Fock matrix for
the central unit cell plus those for the adjacent unit cells.  The periodic
boundary conditions are satisfied, and diagonalization yields the correct
density matrix for the $\Gamma$ point.

At this point  in  the  calculation,  conventionally,  the  density matrix
for  each  unit  cell  is constructed.  Instead, the $\Gamma$-point density
and  one-electron  density  matrices  are   combined   with  \index{Coulomb
integrals} ``$\Gamma$-point-like'' Coulomb  and  exchange integral strings to
produce a new Fock matrix.  The  calculation  can  be  visualized  as  being
done entirely in reciprocal space, at the $\Gamma$ point.

Most  solid-state  calculations  take  a  very  long  time.   These
calculations,   called ``Cluster'' calculations   after  the  original
publication, require between 1.3 and 2 times  the  equivalent  molecular
calculation.\index{Cluster model}

A minor `fudge'  is  necessary  to  make  this  method  work.   The
contribution  to  the  Fock  matrix  element  arising  from the exchange
integral between an atomic orbital and all atomic orbitals which are more than
half a unit cell away must be ignored.

The unit cell must be large enough that an atomic  orbital  in  the
\index{Translation vector!requirements} center  of  the  unit  cell has an
insignificant overlap with the atomic orbitals at the ends of the  unit
cell.   In  practice,  a  translation vector  of more that about 7 or 8\AA\ is
sufficient.  For one rare group of compounds a larger translation vector is
needed.  Solids with delocalized  $\pi$--systems,  and  solids  with  very
small band-gaps will require a larger translation  vector,  in  order  to
accurately  sample k-space.   For these systems, a translation vector in the
order of  15--20 \AA ngstroms is needed.


\subsection{Derivatives}
Solid-state derivatives with respect to geometry are handled differently from
molecule derivatives.  If the Cartesian coordinate derivatives are printed,
using \comp{DEBUG} and \comp{DCART}, then for a molecule with an  optimized
geometry all the derivatives will be zero.  This is  not the case for an
infinite system.

An infinite system is represented by cell supplied by the user, called the
Central Unit Cell, or the CUC, and the cells surrounding this CUC. When
\comp{DCART}, \comp{LARGE}, and \comp{DEBUG} are used in an infinite system
calculation for which  the geometry has been optimized, the Cartesian
derivatives for all unit cells are output. Many of these will be quite large,
up to about 60 kcal/mol/\AA .  This is not an error, rather it is a peculiarity
of the way solid-state derivatives are stored.

The Cartesian derivatives of the CUC represent the sum of all forces acting on
the atoms of the CUC due to all the atoms in the CUC.  Thus, if the atoms in
the CUC are the set ($a$,$b$,$c$,$d$,$e$,$f$), then the Cartesian  derivatives
for atom $a$ represent the forces on $a$ due to the set
($b$,$c$,$d$,$e$,$f$).  The Cartesian derivatives of atom $a$ do {\em NOT}
include terms from  the surrounding unit cells.  Because of this, those atoms
in the CUC  which are at the cell boundaries are likely to have large
derivatives.

The Cartesian derivatives of the surrounding unit cells represent the  forces
acting on the atoms in those cells arising from the atoms of the CUC.  Again,
this is an unbalanced set of forces, and those atoms near to the cell
boundaries are likely to have large resultant forces.

It is possible to evaluate the total, balanced, forces acting on the atoms of
the CUC.  This is done by simply adding the forces acting on the atoms of the
three unit cells.  When the keywords given above are used, the last part of the
derivative output consists of the forces acting on the CUC itself.

Only by representing the forces in this unusual manner can the information
necessary for calculating the derivative of the translation vector be
generated.


\subsection{Geometry Specification for Band Structure Calculations}\label{polygeo}
\index{Band structure!geometry requirements}
Before electronic band structure calculations can be done, the sequence of
atoms in the polymer must be supplied in a highly specific order.  For a simple
polymer, the coordinates of all the atoms in the first fundamental unit cell
are given.  These atoms can be in any order.  The next set of atoms defined are
those for the next unit cell.  These atoms {\em must} be in the same order as
the atoms in the first unit cell. For band structures at least two unit cells
must be defined.  If more than two unit cells are defined, the atoms in the
other  unit cells must be defined in the same order as those in the first unit
cell.

For all polymer calculations {\em except} band structures, the order of
atoms is not important.  \htmlref{An example of such a data set}{pthf}
is shown
\begin{latexonly}
on p~\pageref{pthf}
\end{latexonly} for
polytetrahydrofuran.  When band structures are to be calculated,
\hyperref[pageref]{the order of atoms is important}{. For an example,
see p.~}{}{polyc2h4}.

Because of the difficulty in generating data sets for band-structure work,
program \comp{BZ} was written.  Given a suitable data-set, BZ will generate a
MOPAC data set which can then be used for the calculation of band structures.

\subsection{Electronic Band Structure}
\index{Band structure!electronic}
In a normal cluster calculation, the Fock matrix is diagonalized to yield
eigenvalues corresponding to various points  in the Brillouin zone.  For
$m$ unit cells, the points generated  are $0, 1/m, 2/m$, \ldots  up to $1/2$.  If
$m$ is odd, the upper bound  becomes $(m-1)/(2m)$.  No other points in the
Brillouin zone can be generated by diagonalization.

In order to represent a general point, $k$, in the Brillouin  zone, a complex
secular determinant, $F_{k}$, of size $n$ must be  constructed.  The elements
of this matrix are
$$
 F_{k}(\lambda,\sigma) = \sum_{r=-\infty}^{r=
\infty}E(\lambda,\sigma+nr)e^{-ikr2\pi}.
$$
Because interactions between atomic orbitals fall off  rapidly with distance,
the limits of $r$ can be  truncated to include all non-vanishing elements of
$E$, for the sake of convenience. However,  these elements are precisely those
which were used in the  construction of the Fock matrix.  Using this, and the
fact that  periodic boundary conditions were employed in the construction of
the Fock matrix, this summation can be simplified to
$$
 F_{k}(\lambda,\sigma) = \sum_{r=0}^{r=m-1}E(\lambda,\sigma+nr)exp(-ikr'2\pi),
$$
where $r'$, the index of the unit cell, equals $r$ while $r$ is less  than
$m/2$, otherwise $r' = m-r$.  Band structures can then readily be constructed
by varying  the wave-vector $k$ over the range 0--0.5.  Units of $k$ are
$2\pi/a$,  where $a$ is the fundamental unit cell repeat distance.  The band
structure is then constructed by simply joining the points in the
\index{Crossing, bands in solid state} order in which they are generated.
Within band structures, bands  of different symmetry are allowed to cross.
Simply joining the  points does not allow for band crossing.  However, when
the  resulting bands are represented graphically, visual inspection  readily
reveals which bands should, in fact, cross.

\subsection{Electronic Density of States}
\index{Density of states!electronic}\index{Electronic density of states}
The density of states, DoS, is the spectrum of the number of  energy levels per
eV versus energy.  While the energy levels  resulting from the calculation of
the band structure could be  used directly for the calculation of the DoS, the
resulting DoS  would be very rough as a result of the relatively coarse mesh
used.  A better procedure is to assume continuity of the bands, and, by using
an interpolation procedure, numerically integrate.   A possible complication
arises from the incorrect representation  of bands which should cross.  In
practice, however, errors due to  such causes are so small as to not show up in
a normal graphical  representation of the DoS.

At present, the DoS is calculated in MOPAC (not in BZ), and only
for one-dimensional systems, i.e., polymers.

\subsection{Brillouin Zone: Generation of Band Structures}
Using a modified cluster technique, band structures of  polymers can readily be
calculated.  When a sufficiently large  repeat unit is used,  errors introduced
due to  the methodology of the cluster procedure become vanishingly small.
\index{Polyacetylene}\index{Delocalized $\pi$ systems} Even for delocalized
$\pi$ systems, such as polyacetylene, accurate band structures can be
generated  using repeat units of about 20\AA .  For less highly conjugated
systems, a shorter cluster length should be sufficient.   \index{Oligomers} In
contrast to earlier oligomer work, no allowance need be  made for end-effects.
In addition, the set of points in the BZ  to be used is determined explicitly
by the step-size.

The technique outlined here is very fast compared to earlier
methods~\cite{mosol}.

Geometry optimization of clusters of the size reported  here (i.e., having
translation vectors of about 25\AA\,) require  only a little more time than
molecules of similar size, the extra time  being used to calculate the
inter-unit cell interactions.  Band  structure calculations are also very
fast.  The time required  depends on the size of the fundamental unit cell.
For  polyacetylene, this amounted to 3\% of the time for a single self
consistent field calculation of the cluster.

Band structures calculated using the program BZ are accurate in the sense  that
any errors are due to the Hamiltonian used.  A more accurate  \index{ab
initio@{{\em ab initio}}!band structures} method, for example a large basis set
ab initio calculation,  should yield highly accurate band structures.  In
addition,  limited use of symmetry in the construction of the cluster  secular
determinant and in the geometry optimization should  increase the speed of such
a calculation considerably.   Electrical conductivity in semiconductors is
caused by holes  in the valence band and electrons in the conduction band.
Conductivity also depends on the hole and electron effective  masses, which are
readily calculable from the second derivative  of the energy of the band with
respect to wave-vector.  Band  structures for linear polymers, calculated using
semiempirical  methods, should be suitable for calculation of effective
masses,  \index{Electron!effective mass} and consequently electrical
conductivity.  Unfortunately, NDDO  type semiempirical methods have not proven
very accurate at  predicting conduction band levels.  As a result, in order to
rapidly calculate electrical phenomena, it is likely that a  combination of ab
initio methods and the cluster technique will  be necessary.

As generated by MOPAC, the Fock matrix is unsuitable for band-structure work.
First, the matrix represents the cluster, not the unit cell, and second, the
Fock matrix will not exhibit the high symmetry of the associated space-group.
The perturbation is small, but fortunately it can readily be eliminated.

% 56 lines, including this line
\begin{table}
\caption{\label{spgoh7}Space-group operations for $O_h^7$ (diamond)}
\compresstable
\begin{center}
\begin{tabular}{cccc} \hline
 $\{E|000\}$
&$\{C_2(1,1,0)|\frac{1}{2},\frac{1}{2},\frac{1}{2}\}$
&$\{I|\frac{1}{2},\frac{1}{2},\frac{1}{2}\}$
&$\{\sigma_d(0,1,-1)|000\}$ \\
 $\{C_3(1,1,1)|000\}$
&$\{C_2(1,0,1)|\frac{1}{2},\frac{1}{2},\frac{1}{2}\}$
&$\{S_4(0,0,1)|000\}$
&$\{S_6(1,1,1)|\frac{1}{2},\frac{1}{2},\frac{1}{2}\}$ \\
 $\{C_3(1,-1,\! -1)|000\}$
&$\{C_2(0,1,1)|\frac{1}{2},\frac{1}{2},\frac{1}{2}\}$
&$\{S_4(0,1,0)|000\}$
&$\{S_6(1,-1,\! -1)|\frac{1}{2},\frac{1}{2},\frac{1}{2}\}$ \\
 $\{C_3(-1,1,\! -1)|000\}$
&$\{C_2(0,1,-1)|\frac{1}{2},\frac{1}{2},\frac{1}{2}\}$
&$\{S_4(1,0,0)|000\}$
&$\{S_6(-1,1,\! -1)|\frac{1}{2},\frac{1}{2},\frac{1}{2}\}$ \\
 $\{C_3(-1,\! -1,1)|000\}$
&$\{C_2(1,0,-1)|\frac{1}{2},\frac{1}{2},\frac{1}{2}\}$
&$\{S_4^2(0,0,1)|000\}$
&$\{S_6(-1,\! -1,1)|\frac{1}{2},\frac{1}{2},\frac{1}{2}\}$ \\
 $\{C_3^2(1,1,1)|000\}$
&$\{C_2(1,-1,0)|\frac{1}{2},\frac{1}{2},\frac{1}{2}\}$
&$\{S_4^2(0,1,0)|000\}$
&$\{S_6^5(1,1,1)|\frac{1}{2},\frac{1}{2},\frac{1}{2}\}$ \\
 $\{C_3^2(1,-1,\! -1)|000\}$
&$\{C_4(0,0,1)|\frac{1}{2},\frac{1}{2},\frac{1}{2}\}$
&$\{S_4^2(1,0,0)|000\}$
&$\{S_6^5(1,-1,\! -1)|\frac{1}{2},\frac{1}{2},\frac{1}{2}\}$ \\
 $\{C_3^2(-1,1,\! -1)|000\}$
&$\{C_4(0,1,0)|\frac{1}{2},\frac{1}{2},\frac{1}{2}\}$
&$\{\sigma_d(1,1,0)|000\}$
&$\{S_6^5(-1,1,\! -1)|\frac{1}{2},\frac{1}{2},\frac{1}{2}\}$ \\
 $\{C_3^2(-1,\! -1,1)|000\}$
&$\{C_4(1,0,0)|\frac{1}{2},\frac{1}{2},\frac{1}{2}\}$
&$\{\sigma_d(1,0,1)|000\}$
&$\{S_6^5(-1,\! -1,1)|\frac{1}{2},\frac{1}{2},\frac{1}{2}\}$ \\
 $\{C_2(0,0,1)|000\}$
&$\{C_4^3(0,0,1)|\frac{1}{2},\frac{1}{2},\frac{1}{2}\}$
&$\{\sigma_d(0,1,1)|000\}$
&$\{\sigma_h(0,0,1)|\frac{1}{2},\frac{1}{2},\frac{1}{2}\}$ \\
 $\{C_2(0,1,0)|000\}$
&$\{C_4^3(0,1,0)|\frac{1}{2},\frac{1}{2},\frac{1}{2}\}$
&$\{\sigma_d(1,-1,0)|000\}$
&$\{\sigma_h(0,1,0)|\frac{1}{2},\frac{1}{2},\frac{1}{2}\}$ \\
 $\{C_2(1,0,0)|000\}$
&$\{C_4^3(1,0,0)|\frac{1}{2},\frac{1}{2},\frac{1}{2}\}$
&$\{\sigma_d(1,0,-1)|000\}$
&$\{\sigma_h(1,0,0)|\frac{1}{2},\frac{1}{2},\frac{1}{2}\}$ \\ \hline
\end{tabular}
\end{center}
\end{table}
The steps involved in converting the MOPAC Fock matrix into one suitable
for band-structure work are as follows:
\begin{description}
\item[Generation of solid-state Fock matrix]~\\ BZ assumes that the unit cells
used in constructing the MOPAC cluster were \index{MAKPOL|ff} supplied in the
order defined in MAKPOL.  Based on this assumption, the first unit cell will
have the index (0,0,0).  If there are N atomic functions in a unit cell, then
the first N rows of the MOPAC Fock matrix will correspond to the central unit
cell (CUC).  Of all the unit cells, this one is the only one for which the
entire Fock matrix is not present; instead only the lower-half triangle is
available.  However, since the CUC is symmetric, the missing data are generated
by forming the transpose, i.e., H$_{i,j}$ = H$_{j,i}$.

The Fock matrix representing the interaction of the CUC with the next
\index{BCC} unit cell, (0,0,1), or (0,0,2) if BCC is specified, is then
extracted, as are all the small Fock matrices.  Each Fock matrix, representing
the CUC interacting with each unit cell, is stored in a large array, of size
N$^2$ times the number of unit cells used.  As phrases of the type ``The Fock
matrix  representing the interaction of the CUC with unit cell (i,j,k)'' are
cumbersome, from here on, the term ``unit cell (i,j,k)'' should be understood
as having the same meaning.

The indices of each unit cell is also generated and stored.  However, the
cluster theory assumes that the interaction matrix relating two unit cells
which  are separated by more than half the distance of the translation vector
does not represent that interaction. Rather, it represents the interaction of
two unit cells which are separated by less that half the translation vector
distance.  In order to conform with this definition, all unit cell indices more
that half of the number \index{MERS} of unit cells specified by the \comp{MERS}
keyword are changed.  For example, if \comp{MERS(4,4,4)} is used, the unit
cells (0,1,1) and (2,2,2) would be unchanged, but unit cells(0,1,3) and (0,0,4)
would become (0,1,-2) and (0,0,-1), respectively.

As a result of this operation, most of the unit cells surrounding the CUC are
generated.  The next step is to symmetrize the Fock matrices so that they have
the symmetry of the space group.  Note that if symmetrization is not done, the
band-structures generated would be almost, but not quite, identical to those
which use symmetrized Fock matrices.

\item[Symmetrization of Fock matrices]~\\
\index{Fock matrix!symmetrization in solid state calcs.}
From group-theory we know that if a matrix is operated on by every operation of
a group exactly once, the resulting matrix will have the symmetry of that
group.  In other words,
$$
F_{sym} = \frac{1}{M}\sum_{i=1}^M<R_i|F_{unsym}|R_i^T>.
$$
The index $i$ covers all operations of the group, including the identity.

\end{description}

\subsubsection{Space-group operations}
\index{Space group!theory} Space-group operations differ from point-group
operations in that in addition \index{Non-primitive translations|ff} to the
point-group operation, a non-primitive translation may be involved. Thus far,
we have been using as our example the diamond lattice which is suitable for
illustrating space-group operations.  For convenience, \index{Space
group!operations|ff} we will specify a space-group operation thus: \{R$|$T\},
where ``R'' is a point-group operation, e.g.\ C$_2$(0,0,1) or S$_6$(1,1,-1),
and ``T'' is a non-primitive translation, e.g.\
($\frac{1}{2}$,$\frac{1}{2}$,$\frac{1}{2}$), or (0,0,0). The term in
parentheses following the point group operation indicates the axis about which
the operation is to be performed.  Finally, to complete the specification of a
space-group operation, the point about which the operation acts must be
defined.  As this is most conveniently done in fractional unit
\index{Fractional!unit cell coordinates} \index{Coordinates!fractional unit
cell} cell coordinates, or crystal coordinates, the Cartesian coordinates are
converted at this time into fractional unit cell coordinates.

\index{oh7@{$O_h^7$}}\index{Fd3m@{$Fd3m$}} Diamond belongs to the
$Fd3m$ or $O_h^7$ space-group, and has octahedral symmetry; its
associated point-group is O$_h$.  The space-group operations are
given in Table~\ref{spgoh7}.
\subsection{Examples of Solid State Data Set Construction}
 \label{diamond}
 \subsubsection{Diamond}\index{Diamond}\index{Data!Diamond!construction of}\index{Solid state!data sets}

The unit cell of diamond consists of two carbon atoms. If symmetry operations were allowed,
 only one atom would be needed, but only translation operations are allowed, so two atoms
 must be used.

Each carbon atom forms four single bonds with other carbon atoms. 
This can be used in defining the translation vectors: the effect 
of a translation vector acting on atom 1 would be to move it to 
one of the atoms attached to atom 2. 
 Using this fact, the data set can easily be made:

First attempt at a data set for MAKPOL. \begin{verbatim}
MERS=(4,4,4)  Diamond
C
C  1.545 1 0 0 0 0 1
Tv 2.523 1 35.26439 10 0 1 2
Tv 2.523 1 35.26439 1 120 1 1 2 3
Tv 2.523 1 35.26439 1 240 1 1 2 3
\end{verbatim}

In this description, the number of unit cells to be used is too
small: 4 by 4 by 4.  Ideally, at least a 6 by 6 by 6 system
 should be use. The smaller number is used here purely for the
 purposes of illustration.



This data set is, however, not ideal, for the following reasons:

The angle between the translation vectors is 60 degrees. This 
means that the unit cell must be very large in order for the 
distance between opposite faces to be at least 10 \AA ngstroms. 
Diamond is a Body-Centered-Cubic lattice. This means that every 
odd cell (e.g. 001,111,012, etc.) is missing. By specifying BCC, 
the angle between the translation vectors can be increased to 90 
degrees. This is the ideal angle to maximize the distance between 
faces. The system has a lot of symmetry. By adding ``SYMMETRY" two 
objectives can be met: First, symmetry can be used in defining the 
Tv. This is not very important in this unit cell, but does make it 
easier to change the bond-length of atom 2, in that any change in 
this distance is automatically made in the Tv. Second, if SYMMETRY 
is present, MAKPOL will automatically add symmetry to the data 
set.

\begin{verbatim}
Data set for MAKPOL. File name: Make_diamond.dat
 SYMMETRY MERS=(4,4,4) BCC
 Diamond, 64 atoms 
C 
C 1.545 1 0 0 0 0 1 
Tv 1.784 0 54.73561 0 0 0 1 2 
Tv 1.784 0 54.73561 0 120 0 1 2 3 
Tv 1.784 0 54.73561 0 240 0 1 2 3 

2 19 1.1547005 3 4 5 
\end{verbatim}

When this data set is run using MAKPOL, the following data set is 
generated: 
\begin{verbatim}
SYMMETRY MERS=(4,4,4) BCC 
Diamond, 64 atoms 

C  0.000000 0   0.000000 0    0.000000 0 
C  1.545000 1   0.000000 0    0.000000 0 1 
C  3.568025 1  54.735610 1    0.000000 0 1 2 
C  1.545000 0 125.264390 1    0.000000 1 3 1 2 
C  2.522974 1  35.264390 1   60.000000 1 1 2 3 
C  1.545000 0 144.735610 1    0.000000 0 5 1 2 
C  2.522974 0 135.000000 1   45.000000 1 3 1 2 
C  1.545000 0 144.735610 0  -90.000000 1 7 3 1 
C  2.522974 0  90.000000 1   90.000000 0 5 1 2 
C  1.545000 0  90.000000 0 -144.735610 1 9 5 1  

(many lines deleted) 

C  1.545000 0  90.000000 0  160.528779 1 61 57 41 
C  2.522974 0  60.000000 0  -35.264390 0 59 43 26
C  1.545000 0  90.000000 0  160.528779 0 63 59 43 
XX 2.522974 0  90.000000 0  180.000000 0 7   3 1 
XX 2.522974 0  90.000000 0 -125.264390 0 29  9 5 
XX 2.522974 0 180.000000 0    0.000000 0 53 41 42 
Tv 7.136049 1   0.000000 0    0.000000 0 1  65 2 
Tv 7.136049 0   0.000000 0    0.000000 0 1  66 2 
Tv 7.136049 0   0.000000 0    0.000000 0 1  67 2 

2 1 4 6 8 10 12 14 16 18 20 22
2 1 24 26 28 30 32 34 36 38 40 42
2 1 44 46 48 50 52 54 56 58 60 62
2 1 64
4 3 6 11 12 15 16 18 33 34 35 36
4 3 39 40 43 44 45 51 52 53 54 55
4 3 56 59 60 67
5 1 7 9 11 13 15 17 19 21 23 25
5 1 27 29 31 33 35 37 39 41 43 45
5 1 47 49 51 53 55 57 59 61 63 65
5 1 66 67
5 2 17
5 14 17

(many lines deleted) 

\end{verbatim}
Before running this data set, more symmetry can be added 
``by hand." Every angle is symmetry defined, and does not need to 
be optimized, therefore every angle and dihedral optimization flag 
can be set to zero. Every angle and dihedral symmetry relation 
defined at the end of the data set can also be deleted. These are 
the lines that start with a number followed by a 2, a 3, or a 14. 
Every distance can be related to the bond-length of atom 2. The 
position of atoms 3 and 5, and the length of the translation 
vector Tv can be defined by distances that are exactly Sqrt(16/3). 
Sqrt(8/3) and Sqrt(64/3) times the C$_2$-C$_1$ 
distance, respectively. These symmetry relations can be defined 
using MOPAC symmetry function 19. This has the form: Defining-atom 
19 multiplier dependent atom(s) When these changes are made to the 
data set, the final data set is produced. This is: 
\begin{verbatim}
 SYMMETRY MERS=(4,4,4) BCC 
 Diamond, 64 atoms

  C    0.000000  0   0.000000  0    0.000000  0
  C    1.545000  1   0.000000  0    0.000000  0     1
  C    3.568025  0  54.735610  0    0.000000  0     1     2
  C    1.545000  0 125.264390  0    0.000000  0     3     1     2
  C    2.522974  0  35.264390  0   60.000000  0     1     2     3
  C    1.545000  0 144.735610  0    0.000000  0     5     1     2
  C    2.522974  0 135.000000  0   45.000000  0     3     1     2
  
  (many lines deleted) 
 
  C    1.545000  0  90.000000  0  160.528779  0    61    57    41
  C    2.522974  0  60.000000  0  -35.264390  0    59    43    26
  C    1.545000  0  90.000000  0  160.528779  0    63    59    43
 XX    2.522974  0  90.000000  0  180.000000  0     7     3     1
 XX    2.522974  0  90.000000  0 -125.264390  0    29     9     5
 XX    2.522974  0 180.000000  0    0.000000  0    53    41    42
 Tv    7.136049  0   0.000000  0    0.000000  0     1    65     2
 Tv    7.136049  0   0.000000  0    0.000000  0     1    66     2
 Tv    7.136049  0   0.000000  0    0.000000  0     1    67     2

   2 19 2.3094  3
   2 19 1.6330  5
   2 19 4.6188 68
   2  1    4    6    8   10   12   14   16   18   20   22
   2  1   24   26   28   30   32   34   36   38   40   42
   2  1   44   46   48   50   52   54   56   58   60   62
   2  1   64
   5  1    7    9   11   13   15   17   19   21   23   25
   5  1   27   29   31   33   35   37   39   41   43   45
   5  1   47   49   51   53   55   57   59   61   63   65
   5  1   66   67
  68  1   69   70
\end{verbatim} 

Is the use of symmetry worth all this effort? Most 
definitely! If symmetry is NOT used, then for this small system of 
64 atoms, 195 parameters would need to be optimized. That is, 
3*64-6 for the 64 atoms plus 9 parameters for the three 
translation vectors. Optimization of a system with 195 unknowns 
would take much longer than for a system with precisely one 
unknown. For a more realistic system, involving 6 by 6 by 6 
primitive unit cells, symmetry would lower the complexity of the 
calculation from 651 unknowns to precisely 1. 

