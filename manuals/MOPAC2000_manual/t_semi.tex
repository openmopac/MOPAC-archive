\section{Semiempirical Theory}\label{semit}
There are five distinct methods available within MOPAC: MINDO/3, MNDO, AM1,
PM3, and MNDO-$d$. All are semiempirical, and have roughly the same structure.
A complete knowledge of these methods is not necessary in order to use MOPAC;
however, a superficial understanding of these methods and their relationship to
{\em ab~initio} methods is important for using MOPAC and particularly for
interpreting the results.

The five methods within MOPAC have many features in common. They are all
self-consistent field (SCF) methods, they take into account electrostatic
repulsion and exchange stabilization, and, in them, all calculated integrals
are evaluated by approximate means. Further, they all use a restricted basis
set of one $s$ orbital and three $p$ orbitals ($p_x$, $p_y$, and $p_z$) per
atom (except MNDO-$d$, which has five $d$ orbitals in addition to the $s-p$
basis set) and ignore overlap  integrals in the secular equation. Thus, instead
of solving
$$
|H-ES| = 0,
$$
the expression
$$
|H-E| = 0,
$$
in which $H$ is the secular determinant, $S$ is the overlap matrix, and $E$ is
the set of eigenvalues, is solved. These approximations considerably simplify
quantum mechanical calculations on systems of chemical interest. As a result,
larger systems can be studied. Computational methods are only models, and there
is no advantage in rigorously solving Schr\"{o}dinger's equation for a large
system if that system has had to be abbreviated in order to make the
calculations tractable. Semiempirical methods are thus seen to be well
balanced: they are accurate enough to have useful predictive powers, yet fast
enough to allow large systems to be studied.

All five semiempirical methods contain sets of parameters. For MINDO/3 atomic
and diatomic parameters exist, while MNDO, AM1, PM3, and MNDO-$d$ use only
single-atom parameters. Not all parameters are optimized for all methods; for
example, in MINDO/3, MNDO and AM1 the two electron one center integrals are
normally taken from atomic spectra. In the list given in Table~\ref{pars},
parameters optimized for a given method are indicated by `$*$'. A `+' indicates
that the value of the parameter was obtained from experiment (not optimized).
Where neither symbol is given, the associated parameter is not used in that
method.

\begin{table}
\caption{\label{pars} Parameters used in Semiempirical Methods}
\begin{center}
\begin{tabular}{|l|l|cccc|}  \hline
Parameter & Description & MINDO/3 &MNDO & AM1 & PM3 \\ \hline
U$_{ss}$ and U$_{pp}$ &  $s$ and $p$ atomic orbital &  +& *& *& * \\
& one-electron one-center integrals&&&&\\
$\beta_s$ and $\beta_p$ &  $s$ and $p$ atomic orbital one-electron && *& * &*\\
& two-center resonance integral terms&&&&\\
 I$_s$ &  $s$ atomic orbital ionization potential& +&&&\\
& for two center resonance integral term&&&& \\
 I$_p$ & $p$  atomic orbital ionization potential& +&&&\\
& for two center resonance integral term&&&&\\
$\beta_{A,B}$ &  Diatomic two center one-electron & *&&&\\
& resonance integral multiplier&&&&\\
$\xi_s$& $s$-type Slater atomic orbital exponent& *& *& *& *  \\
$\xi_p$ & $p$-type Slater atomic orbital exponent& *& *& *& *  \\
$\alpha_A$ & Atom $A$ core-core repulsion term &&  *& *& *  \\
$\alpha_{AB}$&  Atoms $A$ and $B$ core-core repulsion term &* &&& \\
$ Gss$ & $s-s$ atomic orbital one center  & + & + & + & *  \\
 & two electron repulsion integral&&&&  \\
$ Gsp$ & $s-p$ atomic orbital one center  & +  &+ & + & *  \\
 & two electron repulsion integral  &&&&\\
$ Gpp$ & $p-p$ atomic orbital one center  & + & + & + & *  \\
 & two electron repulsion integral &&&& \\
$ Gp2$ & $p-p'$ atomic orbital one center  & + & + & + & *  \\
 & two electron repulsion integral &&&& \\
$ Hsp$ & $s-p$ atomic orbital one-center  & + & + & + & *  \\
& two-electron exchange integral &&&& \\
K$_{n_A}$ or a$_{n_A}$ & A Gaussian multiplier for $n$th &  &&  *  &*  \\
&  Gaussian of atom A &&&& \\
L$_{n_A}$ or b$_{n_A}$ & A Gaussian exponent multiplier&   &&  * & *  \\
&  for $n$th Gaussian of atom A &&&& \\
M$_{n_A}$ or c$_{n_A}$ & A Radius of center of $n$th  && &   * & *  \\
&  Gaussian of atom A. &&&& \\ \hline
\end{tabular}
\end{center}
\end{table}

All five semiempirical methods also use two experimentally determined constants
per atom: the atomic mass of the most abundant isotope and the heat of
atomization.

\subsection{Approximations used in MNDO, MINDO/3, AM1, PM3, and MNDO-$d$}
All five methods are similar enough to describe simultaneously. In the
following discussion, applications specific to each method will be indicated in
the text. This will allow easy comparison among the methods, a comparison which
is not obvious from their names. MINDO/3 stands for Modified Intermediate
Neglect of Differential Overlap, version 3. MNDO stands for Modified Neglect of
Diatomic Overlap, and AM1 is Austin Model 1: these are the first two of  the
MNDO-type methods. PM3 is the Modified Neglect of Diatomic Overlap, Parametric
Method Number 3, and MNDO-$d$, is really just MNDO with $d$ orbitals.

Earlier methods, such as CNDO \index{CNDO} (Complete Neglect of Differential Overlap) ignored all
terms involving two different atomic orbitals on one atom. Because of this, CNDO
was an example of a ZDO \index{ZDO} (Zero Differential Overlap) method. (For a good
introduction to CNDO, see ``Approximate Molecular Orbital Theory", J. A. Pople,
D. L. Beveridge, McGraw-Hill, New York, 1970. CNDO was first described in: \index{Pople}
J. A. Pople, D. P. Santry, and G. A. Segal, ``Approximate Self-Consistent
Molecular Orbital Theory I. Invariant Procedures", J. Chem. Phys., 1965, 43,
S129-S135 (1965); J. A. Pople and G. A. Segal, ``Approximate Self-Consistent
Molecular Orbital Theory II. Calculations with Complete Neglect of Differential
 Overlap", J. Chem. Phys., 1965, 43, S136-S149 (1965); and J. A. Pople and
 G. A. Segal, ``Approximate Self-Consistent Molecular Orbital Theory. III CNDO
 Results for AB2 and AB3 Systems", J. Chem. Phys., 1966, 44, 3289-3296 (1966).),

MNDO, AM1, PM3, and MNDO-d belong to the family of NDDO (Neglect of Diatomic
Differential Overlap) methods.  In these
methods all terms arising from the overlap of two atomic orbitals which are on
different centers or atoms are set to zero. As this is not the forum for
developing the ideas of Hartree-Fock theory, the derivation of the
Roothaan-Hall equations will be assumed, and our description of the methods
will start with the final Roothaan~\cite{roothaan}-Hall~\cite{hall} equations.

\subsubsection{Basic Roothaan-Hall Equations}
Secular Equation
$$
c_i|F-\epsilon_iS|c_i =0
$$
$c_i$: eigenvector; $F$: Fock matrix; $\epsilon_i$: eigenvalue; $S$: overlap matrix.
The total electronic  energy of the system is given by:
$$
E = 1/2P(H+F),
$$
in which $P$: density matrix; $H$: one-electron matrix.
The general Fock Matrix element is
$$
F_{\mu\nu}^{\alpha} = H_{\mu\nu}+\sum_{\lambda}\sum_{\sigma}
[P_{\lambda\sigma}^{\alpha+\beta}<\mu\nu|\lambda\sigma> - P_{\lambda\sigma}^{\alpha}
<\mu\lambda|\nu\sigma>].
$$
Or, spin-free
$$
F_{\mu\nu} = H_{\mu\nu}+\sum_{\lambda}\sum_{\sigma}
[P_{\lambda\sigma}<\mu\nu|\lambda\sigma> - \frac{1}{2}P_{\lambda\sigma}
<\mu\lambda|\nu\sigma>].
$$

The methods all use a minimum basis set consisting of a maximum of one atomic
orbital for each angular quantum number. The normal basis set for any atom
consists of one $s$ and three $p$ orbitals ($p_x$, $p_y$ and $p_z$).

\subsubsection{Neglect of diatomic overlap integral}
All overlap integrals arising from the overlap of two \index{Neglect of
overlap}\index{Overlap! neglect of} different atomic orbitals are neglected.
This reduces the overlap matrix to a unit matrix. The secular equation thus
reduces to:
$$
c_i|F-\epsilon_i|c_i =0.
$$
In semiempirical theory the Coulson \label{ndoi} density matrix is used, e.g.:
$$
P_{\lambda\sigma}^{\alpha} = \sum_i^{occ}c_{\lambda i}^{\alpha}c_{\sigma  i}^{\alpha},
$$
where the sum is over all occupied spin molecular orbitals.
In RHF calculations, only the total density matrix is calculated:
$$
     P_{\lambda\sigma} = 2\sum_i^{occ}c_{\lambda i}c_{\sigma  i},
$$
where the sum is over all occupied  molecular orbitals.

When a system has more than half the available M.O.s, $N$,  filled, it
is computationally\index{Positron equivalent|ff}
faster to calculate the positron electron equivalent:
$$
     P_{\mu\nu}^{\alpha} = 1-\! \sum_{i=occ+1}^Nc_{\lambda i}^{\alpha}c_{\sigma  i}^{\alpha}
$$
and
$$
     P_{\mu\nu} = 2-\! 2\sum_{i=occ+1}^Nc_{\lambda i}c_{\sigma  i}.
$$

An important exception to this rule is the calculation of the one-electron
two-center integral $H_{\mu\nu}$, which is approximated by:
$$
H_{\mu\nu}= S_{\mu\nu}\frac{1}{2}(U_{\mu\mu}+U_{\nu\nu}),
$$
where $S_{\mu\nu}$ is the overlap integral between atomic orbital $\varphi_{\mu}$
on an atom, and  $\varphi_{\nu}$ on another atom, and the $U$ values are
atomic orbital constants, supplied as data.

\subsubsection{Neglect of three and four center integrals}\label{34}
\index{Rotational invariance|ff}
Continuing with the neglect of differential overlap, all two-electron integrals
involving charge clouds arising from the overlap of two atomic orbitals on
different centers are ignored. Since no rotation can convert a two center
two-electron integral into a set of integrals involving three and four center
terms, rotational invariance is not compromised by this approximation.
Rotational invariance is present if the calculated observables [$\Delta H_f$,
Dipole, I.P., etc] are not dependent on the orientation of the system. The
effects of this approximation on the \mi{Roothaan} equations are as follows:

In the Fock matrix, if $\varphi_{\mu}$ and $\varphi_{\nu}$ are on different
centers the NDDO matrix element $F_{\mu\nu}^{\alpha}$ reduces to
$$
 F_{\mu\nu}^{\alpha} = H_{\mu\nu}- \sum_{\lambda}^A\sum_{\sigma}^B
P_{\lambda\sigma}^{\alpha} <\mu\lambda|\nu\sigma>,
$$
while the MINDO/3 matrix element becomes:
$$
 F_{\mu\nu}^{\alpha} = H_{\mu\nu}- P_{\mu\nu}^{\alpha} <\mu\nu|\mu\nu>.
$$
Equivalent expressions exist for  $F_{\mu\nu}^{\beta}$ and
$P_{\mu\nu}^{\beta}$. Thus no Coulombic terms are present in the two-center
Fock matrix elements.

If $\varphi_{\mu}$ and $\varphi_{\nu}$  are different but on the same center,
then, since a minimal basis set is being used, all integrals of the type
$<\!\mu\nu|\lambda\sigma\!>$ are zero by the orthogonality of the atomic
orbitals unless $\mu=\nu$ and $\lambda=\sigma$, or $\mu=\lambda$ and
$\nu=\sigma$.  The off-diagonal one-center NDDO Fock matrix elements become:
$$
 F_{\mu\nu}^{\alpha} = H_{\mu\nu}+ 2P_{\mu\nu}^{\alpha+\beta} <\mu\nu|\mu\nu>
- P_{\mu\nu}^{\alpha}( <\mu\nu|\mu\nu> +<\mu\mu|\nu\nu>),
$$
while the MINDO/3 element becomes:
$$
F_{\mu\nu}^{\alpha} - P_{\mu\nu}^{\alpha} <\mu\nu|\mu\nu>.
$$
 If  $\varphi_{\mu}$ is the same as $\varphi_{\nu}$,  then, because of the
symmetry of the two-electron integrals, the diagonal NDDO
Fock matrix elements reduce to:
$$
F_{\mu\mu}^{\alpha} = H_{\mu\mu}+ \sum_{\nu}^A(P_{\nu\nu}^{\alpha+\beta}
<\mu\mu|\nu\nu> - P_{\nu\nu}^{\alpha}<\mu\nu|\mu\nu>)
+\sum_B\sum_{\lambda}^B\sum_{\sigma}^BP_{\lambda\sigma}^{\alpha+\beta}<\mu\mu|\lambda\sigma>,
$$
and the corresponding MINDO/3 element becomes:
$$
F_{\mu\mu}^{\alpha} = H_{\mu\mu}+ \sum_{\nu}^A(P_{\nu\nu}^{\alpha+\beta}
<\mu\mu|\nu\nu> - P_{\nu\nu}^{\alpha}<\mu\nu|\mu\nu>)
+\sum_B<AA|BB>\sum_{\lambda}^B P_{\lambda\lambda}^{\alpha+\beta}.
$$

\subsubsection{One-center two-electron integrals}
The MNDO, AM1, and MNDO-$d$ one-center two-electron integrals are derived from
experimental data on isolated atoms. Most were taken from
\mi{Oleari}'s~\cite{oleari} work, but a few were obtained by optimization to
fit molecular properties. The values of PM3 one-center two-electron integrals
were optimized to reproduce experimental molecular properties.

For each atom there are a maximum of five one-center two-electron integrals.
These are $<\!ss|ss\!>$, $<\!ss|pp\!>$,  $<\!sp|sp\!>$, $<\!pp|pp\!>$, and
$<\!pp|p'p'\!>$, where $p$ and $p'$ are two different $p$-type atomic orbitals.
In the original formulation~\cite{dt4689} there was a sixth integral,
$<\!pp'|pp'\!>$, but it can be shown that this integral is related to two of
the other integrals by:
$$
  <pp'|pp'> = \frac{1}{2}(<pp|pp> - <pp|p'p'>).
$$
Proof:
If the molecular frame is rotated by 45$^\circ$ about
the z axis the atomic bases mix thus:
\begin{eqnarray}
   R(45^\circ)p_x = \frac{1}{\sqrt{2}}(p_x + p_y) \\ \nonumber
   R(45^\circ)p_y = \frac{1}{\sqrt{2}}(p_y - p_x) \nonumber
\end{eqnarray}
\begin{eqnarray}
 R(45^\circ)<p_xp_y|p_xp_y>& =& \frac{1}{4}<(p_x+p_y)(p_y-p_x)|(p_x+p_y)(p_y-p_x)> \nonumber \\
& =& \frac{1}{4}(<p_xp_x|p_xp_x> + <p_yp_y|p_yp_y> - <p_xp_x|p_yp_y> - <p_yp_y|p_xp_x>)\nonumber
\end{eqnarray}
or
$$
 R(45^\circ)<p_xp_y|p_xp_y> =\frac{1}{2}(<p_xp_x|p_xp_x> - <p_xp_x|p_yp_y>)
$$
\index{gss@{$G_{ss}$}}
\index{gsp@{$G_{sp}$}}
\index{gpp@{$G_{pp}$}}
\index{gp2@{$G_{p2}$}}
\index{hsp@{$H_{sp}$}}
 For convenience these five integrals are given the following names:
\begin{eqnarray}
    <ss|ss> &=& G_{ss} \nonumber  \\
    <pp|pp>&=&  G_{pp} \nonumber  \\
    <sp|sp>&=&  H_{sp} \nonumber  \\
    <pp|pp>&=&  G_{pp} \nonumber  \\
    <pp|p'p'>&=&  G_{p2}\nonumber
\end{eqnarray}
 Using these definitions, the two-electron one-center
contributions to the Fock matrix become:
\begin{eqnarray}
F_{ss}^{\alpha}&:&P_{ss}^{\beta}G_{ss}+(P_{p_x}^{\alpha+\beta}+P_{p_y}^{\alpha+\beta}+P_{p_z}^{\alpha+\beta})G_{sp}
-(P_{p_x}^{\alpha}+P_{p_y}^{\alpha}+P_{p_z}^{\alpha})H_{sp}\nonumber  \\
F_{sp}^{\alpha}&:&2P_{sp}^{\alpha+\beta}H_{sp}-P_{sp}^{\alpha}(H_{sp}+G_{sp})\nonumber  \\
F_{pp}^{\alpha}&:&P_{ss}^{\alpha+\beta}G_{sp}-P_{ss}^{\alpha}H_{sp}+
P_{pp}^{\beta}G_{pp} +(P_{p'p'}^{\alpha+\beta}+P_{p''p''}^{\alpha+\beta})G_{p2}
-\frac{1}{2}(P_{p'p'}^{\alpha}+P_{p''p''}^{\alpha})(G_{pp}-G_{p2})\nonumber  \\
F_{pp'}^{\alpha}&:&P_{pp'}^{\alpha+\beta}(G_{pp}-G_{p2})
-\frac{1}{2}P_{pp'}^{\alpha}(G_{pp}+G_{p2}) \nonumber
\end{eqnarray}
These expressions are common to all methods.

\subsubsection{NDDO two-electron two-center integrals}

\index{Two electron!integrals} In a local diatomic frame there are 22 unique
two-electron two-center integrals for each pair of heavy
(non-hydrogen) atoms. These are shown in Table~\ref{2e2c}.

\begin{table}
\caption{\label{2e2c} Two-Electron Two-Center Integrals (Local Frame)}
\begin{center}
\begin{tabular}{rlcl}
1&$<ss|ss>$&                                   12&$ <sp_{\sigma} |p_{\pi} p_{\pi} >  $\\
2&$<ss|p_{\pi}p_{\pi} >$&                      13&$ <sp_{\sigma} |p_{\sigma} p_{\sigma}>  $\\
3&$<ss|p_{\sigma} p_{\sigma} >$&               14&$ <ss|sp_{\sigma} >  $\\
4&$<p_{\pi} p_{\pi} |ss>$&                     15&$ <p_{\pi} p_{\pi} |sp_{\sigma} >  $\\
5&$<p_{\sigma} p_{\sigma} |ss>$&               16&$ <p_{\sigma} p_{\sigma} |sp_{\sigma} >  $\\
6&$<p_{\pi} p_{\pi} |p_{\pi} p_{\pi} >$&       17&$ <sp_{\pi} |sp_{\pi} >  $\\
7&$<p_{\pi} p_{\pi} |p_{\pi} 'p_{\pi} '>$&     18&$ <sp_{\sigma} |sp_{\sigma} >  $\\
8&$<p_{\pi} p_{\pi} |p_{\sigma} p_{\sigma} >$& 19&$ <sp_{\pi} |p_{\pi} p_{\sigma} >  $\\
9&$<p_{\sigma} p_{\sigma} |p_{\pi} p_{\pi} >$& 20&$ <p_{\pi} p_{\sigma} |sp_{\pi} >  $\\
10&$<p_{\sigma} p_{\sigma} |p_{\sigma} p_{\sigma} >$& 21&$ <p_{\pi} p_{\sigma} |p_{\pi} p_{\sigma} >  $\\
11&$<sp_{\sigma} |ss>$ &                        22&$ <p_{\pi} p_{\pi} '|p_{\pi} p_{\pi} '>  $
\end{tabular}
\end{center}
\end{table}
 Each integral represents the energy of an electron
density distribution (electron 1) arising from the product
of the first two atomic orbitals interacting with the
electron density distribution (electron 2), which in turn arises
from the product of the second two atomic orbitals. Only if
the first two atomic orbitals are the same and the second two
are the same will the interaction energy have to be
positive, in which case the integral represents an
electron-electron repulsion term. In all other cases the
sign of the integral value may be positive or negative.

 With the exception of integral 22, all the integrals can
be calculated using different techniques without loss of
rotational invariance. That is, no integral depends on the
value of another integral, except for number 22. As with
the $H_{pp}$ monocentric integral, it is easy to show that:
$$
 <p_{\pi} p_{\pi} '|p_{\pi} p_{\pi} '> = \frac{1}{2}(<p_{\pi} p_{\pi} |p_{\pi} p_{\pi} >
- <p_{\pi} p_{\pi} |p_{\pi} 'p_{\pi} '>).
$$
 The electron density distributions are approximated by a
series of point charges. There are four possible types of
distribution. These are given in Table~\ref{mp}.

\begin{table}
\caption{\label{mp}Types of Electron Density Distribution}
\index{Multipoles}\index{Monopole}\index{Dipole}\index{Quadrupole}
\index{Atomic!electron density dist.}
\begin{center}
\begin{tabular}{ll}\hline
 Monopole & Unit negative charge centered on the  \\
 (1 charge)&  nucleus  \\
\\
 Dipole & +1/2 charge located at position (x,y,z),  \\
 (2 charges) & -1/2 charge located at position (-x,-y,-z)  \\
\\
 Linear Quadrupole& +1/2 charge located at the nucleus, -1/4 charge  \\
 (3 charges) & at positions (x,y,z) and at (-x,-y,-z)  \\
\\
 Square Quadrupole& Four charges of magnitude +1/4, -1/4, +1/4  \\
 (4 charges) & and -1/4 forming a square centered on the  nucleus.  \\ \hline
\end{tabular}
\end{center}
\end{table}

These are used to represent the four types of atomic orbital
products (Table~\ref{prod}).

\begin{table}
\caption{\label{prod}Density Distributions Arising from Pairs of Atomic Orbitals}
\begin{center}
\begin{tabular}{clc}\hline
Atomic Orbitals &  Multipole Distribution & Number of Charges \rule[-0.2cm]{0cm}{0.6cm} \\ \hline
$ <ss| $  &  Monopole                      &  1  \\
$ <sp| $  &  Dipole                        &  2  \\
$ <pp| $  &  Monopole plus Linear Quadrupole &4  \\
$ <pp'|$  &  Square Quadrupole             &  4  \\ \hline
\end{tabular}
\end{center}
\end{table}

Each two electron interaction integral is the sum of all the interactions
arising from the charge distribution representing one pair of atomic orbitals
with the charge distribution representing the second pair of atomic orbitals.
Thus, in the simplest case, the $<ss|ss>$ interaction is represented by the
repulsion of two monopoles, while a $ <p_{\pi} p_{\pi} |p_{\pi} 'p_{\pi} '>$, a
much more complicated interaction, is represented by 16 separate terms, arising
from the four charges representing the monopole and linear quadrupole on one
center interacting with the equivalent set on the second center.

 While the repulsion of two like charges is proportional
to the inverse distance separating the charges, boundary
conditions preclude using this simple expression to
represent the interelectronic interactions. Instead, the
interaction energy is approximated by:
$$
<ss|ss> =\frac{27.21}{\sqrt{(R+c_A+c_B)^2+\frac{1}{4}
(\frac{1}{G_A}+\frac{1}{G_B})^2}}.
$$

All that remains is to specify functional forms for the
terms $c$ and $A$. $c$, the distance of a multipole charge from
its nucleus, is a simple function of the atomic orbitals; in
the case of a $s-p$ product, this is a vector of length $D_1$
Bohr pointing along the $p$ axis, where
$$
D_1 = \frac{(2n+1)(4\xi_s\xi_p)^{(n+1/2)}}{\sqrt{3}(\xi_s+\xi_p)^{(2n+2)}}.
$$
The principal quantum number is always the
same in these methods for $s$ and $p$ orbitals on any given
atom.

The corresponding distances of the charges from the
nucleus for the linear and square quadrupoles are $2D_2$ and
$\sqrt{2}D_2$ Bohr, respectively, where
$$
D_2=\left(\frac{4n^2+6n+2}{20}\right)^{1/2}\frac{1}{\xi_p}.
$$
 Now that the distances of the charges from the nucleus
have been defined, the upper boundary condition can be set.
In the limit, when $R=0.0$, the value of the two-electron
integral should equal that of the corresponding monocentric
integral. Three cases can be identified:
\begin{enumerate}
\item  A monopole-monopole interaction, in which case the
integral must converge on $G_{ss}$.
\item A dipole-dipole interaction, where the integral must
converge on $H_{sp}$.
\item The quadrupole-quadrupole interaction where the
integral must converge on $H_{pp}$.
\end{enumerate}
 For convenience, the $G_A$ terms are given special names.
These are given in Table~\ref{aa}.
\index{AM}\index{AD}\index{AQ}
\begin{table}
\caption{\label{aa} Additive Terms for Two-Electron Integrals}
\begin{center}
\begin{tabular}{lll} \hline
 Multipole & Monocentric Equivalent & Name  \\ \hline
 Monopole  & $G_{ss}$ &  AM  \\
 Dipole &  $H_{sp}$ &  AD  \\
 Quadrupole &  $H_{pp} =1/2(G_{pp}-G_{p2})$ & AQ \\ \hline
\end{tabular}
\end{center}
\end{table}
 In practice, $\frac{1}{2}(G_{pp}-G_{p2})$ is used instead of $H_{pp}$.
This eliminates any
possibility of loss of rotational invariance due to an incorrect value
of  $H_{pp}$.

 While $AM$ is given simply by $G_{ss}/27.21$, $AD$ and $AQ$ are
complicated functions of one-center terms and the orbital
exponents---recall that, in the limit, the associated
charges are not all coincident. $AD$ and $AQ$ are solved
iteratively. Given an initial estimate of $AD$ of
$$
AD = \left[\frac{H_{sp}}{27.21D^2_1}\right]^{1/3},
$$
then, by iterating, an exact value of $AD$ can be found. On
iteration $n$ the value of $AD$ is given by
$$
AD_n = AD_{n-2}+(AD_{n-1}-AD_{n-2})\frac{(\frac{H_{sp}}{27.21}-a_{n-2})}{a_{n-1}-a_{n-2}},
$$
where
$$
a_n=2AD_n-2(4D^2_1+AD^{-2}_n)^{-1/2}.
$$

About 5 iterations are needed in order to get AD specified
with acceptable accuracy.

 Similarly, for $AQ$ an initial estimate of
$[\frac{H_{pp}}{27.21(3D^4_2)}]^{1/5}$
is made and, again, by iterating using
$$
AQ_n = AQ_{n-2}+(AQ_{n-1}-AQ_{n-2})\frac{(\frac{H_{pp}}{27.21}-a_{n-2})}{a_{n-1}-a_{n-2}},
$$
where, now,
$$
a_n=4AQ_n-8(4D^2_2+AQ^{-2}_n)^{-1/2}+4(8D^2_2+AQ_n^{-2})^{-1/2},
$$
 an exact value of $AQ$ can be obtained. About 5 iterations
are necessary.

Note that these equations are intrinsically unstable on
finite-precision computers. The denominator $(a_{n-1} -a_{n-2} )$
rapidly becomes vanishingly small; this is, however, necessary in
order to accurately define $AD$ and $AQ$.

\subsubsection{Final Assembly of Two-Electron Two-Center Integrals}
 With all the component parts defined, the two-electron
two-center integrals are assembled from the sum of all the
interactions of the charges on one center with those on the
other center. The distance between the two charges must be
determined---this is the vector addition of $R$, the
interatomic distance in Bohr, and the two $c$ terms defining
the location of the charges from the nucleus---as well as
the appropriate additive terms, $AM$, $AD$ or $AQ$ selected. Two
examples will illustrate this assembly:

$ <ss|ss>$: This is represented by a single term. For
monopoles, $c=0$ and $G_A=AM_A$, $G_B=AM_B$ giving:
$$
<ss|ss> = \frac{14.399}
{(R^2_{AB}+1/4[\frac{1}{AM_A} + \frac{1}{AM_B}]^20.529177^2)^{1/2}}.
$$

$<ss|p_{\pi}p_{\pi}>$: $p_{\pi}p_{\pi}$ is expressed as the sum of a monopole
and a linear quadrupole. This gives rise to a total of four
charges, hence four terms. However, since the interaction
of the monopole with each of the two negative charges of the
dipole are the same, only three terms need to be evaluated.
In  general, symmetry considerations lower the total number of
terms that need to be evaluated, so the maximum number in
any integral is 8. The full integral is then represented as:
\begin{eqnarray}
<ss|p_{\pi}p_{\pi}>& =&
\frac{14.399}
{(R^2_{AB}+1/4[\frac{1}{AM_A} + \frac{1}{AM_B}]^20.529177^2)^{1/2}} \nonumber \\
&&+\frac{1}{2}\frac{14.399}
{([R^2_{AB}+(2D_2^B)^2]+1/4[\frac{1}{AM_A} + \frac{1}{AQ_B}]^20.529177^2)^{1/2}} \nonumber \\
&&-\frac{1}{2}\frac{14.399}
{(R^2_{AB}+1/4[\frac{1}{AM_A} + \frac{1}{AQ_B}]^20.529177^2)^{1/2}}.  \nonumber
\end{eqnarray}
\subsubsection{MINDO/3 Two electron two center integrals}
\index{MINDO/3!Coulomb integrals}
\index{MINDO/3!exchange integrals}
 In marked contrast to the other methods, MINDO/3 Coulomb
and exchange integrals are given by the simple Dewar-Sabelli~\cite{rab,rab2} -
Klopman~\cite{rab3} approximation. The integral is a
function of the atom types and the interatomic distance
only, and is of form
$$
<AA|BB>=\frac{14.399}{(R_{AB}^2+\frac{1}{4}[\frac{14.399}{G_A}+
\frac{14.399}{G_B}]^2)^{1/2}},
$$

where $G_A$ and $G_B$, sometimes called the Gamma-values,
are the averages of the appropriate
one-center two-electron integrals. All finite integrals
over atomic orbitals on two centers are set equal. Thus,
$<s_As_A|s_Bs_B> =$  $<s_As_A|p_Bp_B> =<p_Ap_A|p_Bp_B> =<AA|BB> $.
\subsubsection{The one-center one-electron integral $H_{\mu\mu}$}\label{1e1c}
\index{One-electron integrals|ff}
 This represents the energy an electron in atomic orbital
$\varphi_{\mu}$ would have if all electrons were removed from the system.
This is approximated by adding on to the one-electron energy
of the atomic orbital in the fully ionized atom the
potential due to all the other nuclei in the system. The
one-electron energy is obtained parametrically, and is given
the symbol $U_{\mu\mu}$. $H_{\mu\mu}$ is derived from the fundamental
equation
$$
H_{\mu\mu}=U_{\mu\mu}-\sum_{B\neq A}Z_B<\mu\mu|BB>
$$
by equating the core-electron integral $<\mu\mu|BB>$ to the
corresponding two-electron integral, thus:
(a) NDDO:
$$
<\mu\mu|BB>= Z_B<\mu\mu|ss>
$$
(b) MINDO/3:
$$
<\mu\mu|BB>= Z_B<AA|ss>
$$
\subsubsection{The two-center one-electron integral $H_{\mu\nu}$ }
\index{Resonance integrals}

 Sometimes called the resonance integral, $H_{\mu\nu}$ is
approximated using the overlap integral, $S_{\mu\nu}$. Note that
this violates the NDO approximation, but since resonance
integrals are large, this integral is retained. This is the
origin of Modified in the name of the method.
With Slater atomic orbitals of type
$$
\varphi_\mu = Nr^{n-1}e^{-\xi r},
$$
the overlap integral is given by:
$$
S_{\mu\nu} = <\varphi_{\mu}\varphi_{\nu}>.
$$
Within NDDO, $H_{\mu\nu}$ is approximated by:
$$
H_{\mu\nu} = S_{\mu\nu}\frac{1}{2}(\beta_{\mu}+\beta_{\nu}),
$$
while MINDO/3 has a very different form:
$$
H_{\mu\nu} = S_{\mu\nu}\beta_{AB}(I_{\mu}+I_{\nu}).
$$
 This use of a diatomic parameter is the most distinctive
difference between the NDDO philosophy and that of
MINDO/3. Because of the difficulty in parameterizing an
element at the MINDO/3 level---the number of $\beta$-parameters
rises as the square of the number of elements
parameterized---it is unlikely that any further development
of MINDO/3 will be made.
\subsubsection{Core-core repulsion integrals}
\index{Core-core repulsion}
 From simple electrostatics the core-core repulsion
integral in eV is:
$$
E_N(A,B)=\frac{14.399Z_AZ_B}{R_{AB}}.
$$
However, the electron-electron and electron-core integrals
do not collapse to the form $c/R_{AB}$ ($R$ in \AA ) for distances
beyond the van der Waal's radii. If the simple term given
above is used, there would be a net repulsion between two
neutral atoms or molecules. To correct for this the
core-core repulsion is approximated by:
$$
E_N(A,B)=Z_AZ_B<AA|BB>.
$$
\subsubsection{MINDO/3 modification to the core-core term}
 In addition to this term, account must be taken of the
decreasing screening of the nucleus by the electrons as the
interatomic distance becomes very small. At very small
distances the core-core term should approach the classical
form. To account for this, an additional term is added to
the basic core-core repulsion to give:
$$
E_N(A,B)=Z_AZ_B(<AA|BB>+\;(\frac{14.399}{R_{AB}}-<AA|BB>)e^{-\alpha_{AB}R_{AB}}.
$$
 Here, $\alpha_{AB}$ is a diatomic parameter. In the case of N-H
\index{N-H! core-core terms}
\index{O-H! core-core terms}
and O-H interactions only, this expression is replaced by:
$$
E_N(A,B)=Z_AZ_B(<AA|BB>+\;(\frac{14.399}{R_{AB}}-<AA|BB>)\alpha_{AB}e^{-R_{AB} }.
$$
\subsubsection{MNDO modification to the core-core term}
 The MNDO approximation to the screening effect is
similar to that of MINDO/3 in practice, but has a different
functional form:
$$
E_N(A,B)=Z_AZ_B(<s_As_A|s_Bs_B>(1+e^{-\alpha_AR_{AB}} +e^{-\alpha_BR_{AB}}).
$$
 Again, O-H and N-H interactions are treated differently.
For these interactions, use
$$
E_N(A,B)=Z_AZ_B(<s_As_A|s_Bs_B>(1+e^{-\alpha_BR_{AB}}R_{AB}  +e^{-\alpha_HR_{AB}}).
$$
\subsubsection{AM1 and PM3 modifications to the core-core term}
\index{AM1! core-core terms}
\index{AM1! difference from MNDO}
\index{PM3! difference from MNDO}
 These modifications are the same as that for MNDO with
the addition of an extra term to reduce the excessive
core-core repulsions just outside bonding distances. The
additional term may be considered as a van der Waal's
attraction term. The AM1 and PM3 core-core terms are:
$$
E_N(A,B)=E_N^{MNDO}(A,B)+\frac{Z_AZ_B}{R_{AB}}
(\sum_ka_{kA}e^{-b_{kA}(R_{AB}-c_{kA})^2}+\sum_ka_{kB}e^{-b_{kB}(R_{AB}-c_{kB})^2}).
$$
The extra terms define spherical Gaussian functions, the $a$,
$b$, and $c$ are adjustable parameters. PM3 has two Gaussians
per atom, while AM1 has between two and four.

\begin{quote}
When I originally developed the theory and software for AM1, my purpose
for having the Gaussian modification was to correct for the excessive
repulsions in the region outside the normal bonding distance and less than the
van der Waals distance.  My reasoning was  as follows:  The excessive repulsion
depends on distance only.  This must be due to a fault in the approximations.
It is not necessary to identify either the faulty approximation, or to modify
it so as to correct the fault.  All that is necessary is to add a new term,
using the well-known method of operator equivalents. In this, I decided to treat the
symptom of the fault, and not the cause.  As subsequent events have shown, this
has proved to be an effective approach.  However, it is still intellectually
unsatisfactory, in that the final justification is that ``the end justifies
the means.''  Hopefully, a better approximation than the {\em ad hoc} one
I chose will eventually be developed.
\end{quote}

\subsection{Self-Consistent Field Calculation}\index{SCF|ff}
Once all the integrals needed are calculated, the SCF calculation can be
started.  A trial density matrix is constructed; this is a diagonal matrix
with the diagonal terms chosen so that every atom starts off electrically
neutral (except for ions, when each atom is given an equal charge).

Using this trial density matrix, the one-electron matrix, and the
two-electron integrals,  a trial Fock matrix is constructed.

Diagonalization produces a set of eigenvectors, from which a better
density matrix can be made.

This sequence (constructing the density matrix---constructing the Fock
matrix---diagonalizing the Fock to get new eigenvectors) is repeated until the
density matrix has become self-consistent to within a pre-set limit
(\comp{SELCON}).

\subsection{Calculation of $\Delta H_f$}

\index{Heat of Formation! definition}
\index{$\Delta H_f$! definition}
The SCF calculation produces a density, $P$, and Fock
matrix, $F$. These, together with the one-electron matrix, $H$,
allow the total electronic energy to be calculated via
\index{Electronic energy}
$$
 E_{elect} = \frac{1}{2}\sum_{\mu}\sum_{\nu}P_{\mu\nu}(H_{\mu\nu}+F_{\mu\nu}).
$$

The total core-core repulsion energy is given by:
\index{Nuclear energy}
$$
   E_{nuc} = \sum_A\sum_{B<A}E_N(A,B).
$$
 The addition of these two terms represents the energy
released when the ionized atoms and valence electrons
combine to form a molecule.

 A more useful quantity is the
heat of formation of the compound from its elements in their
standard state. This is obtained when the energy required
\index{Electronic energy!of atoms}
to ionize the valence electrons of the atoms involved
(calculated using semiempirical parameters), $E_{isol}(A)$, and
\index{Heat of atomization}
heat of atomization, $E_{atom}(A)$, are added to the electronic
plus nuclear energy. This yields:
$$
\Delta H_f = E_{elect} + E_{nuc} + \sum_AE_{isol}(A) +\sum_AE_{atom}(A).
$$

This is the quantity which MOPAC calls the ``Heat of Formation''.
An alternative but equivalent definition of $\Delta H_f$, more
suited for comparison with experimental $\Delta H_f$s, is:

``$\Delta H_f$ is the calculated gas-phase heat of formation at 298K
 of one mole of a compound from its elements in their standard state.''

Things to note about this definition:  unlike {\em ab initio} methods, which
yield the energy at $0K$, semiempirical methods give $\Delta H_f $
at $298K$.  This follows from the way in which semiempirical methods
are parameterized: the reference $\Delta H_f $ are conventionally given at
$298K$.  This means that  semiempirical methods will reproduce $\Delta H_f$
for $298K$.  Secondly, note that $\Delta H_f $ are for gas-phase systems.
To calculate $\Delta H_f $ in the liquid or solid phases, additional terms are
necessary.
