\section{``Size'' of a molecule}\index{Size of molecule}\index{Molecular size}
A useful measure of a molecule is its size.  There are several possible ways of
defining the size of a molecule.  The definition used in MOPAC is as follows:

The first dimension is the maximum distance between any pair of atoms.  

For systems  of 20 or fewer atoms, this distance, and the atoms involved, is
worked out explicitly.  For systems of 21 or more atoms, an atom is selected;
the atom, $K$,  most distant from it is then identified, then the atom, $L$,
most distant  from $K$ is identified.  In most systems, the distance R($K-L$)
is the first dimension.  To ensure that it is, the point half-way between $K$
and $L$ is  selected, and atom $K$ is then re-defined as the atom most distant
from  that point.  A new $L$ is determined.  This sequence in repeated up to 10
times, or until atoms $K$ and $L$ no longer change. There is no guarantee that
the first dimension is, in fact, the largest distance, but it is likely to be
close to the largest possible value.

The second dimension is the maximum distance in the plane perpendicular to the
first dimension between any pair of atoms. 

The technique that was used in determining the first dimension for systems of
over 21 atoms is used here.

The third dimension is the maximum distance between any two atoms on the line
perpendicular to the plane of the first two atoms.

This quantity is explicitly calculated.

Note that the second and third dimensions do {\em not} define the smallest 
rectangular slot that a molecule would go through; it will normally be slightly
larger than the minimum slot. Nevertheless, the ``dimensions'' of a molecule
can be regarded as a good measure of the size of hole that the molecule could
pass through.  Of course, allowance must be made for the finite size of atoms.

Monatomic systems have no ``dimension'', linear systems have two zero
``dimensions'', and flat systems have one zero ``dimension''.



