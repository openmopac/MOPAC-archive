\chapter{Generation of Band Structures}
\section{Program BZ}
\subsection{Description} \index{BZ!description of}
Unlike molecules, whose eigenvalue spectrum consists of infinitely sharp lines
(Kronecker-$\delta$ type), the eigenvalues of solids form bands.  These bands
are conventionally represented by Brillouin Zones (BZ).  For a polymer, the BZ
consists of a line, for a layer structure, a surface, and for a solid, a
three-dimensional shape.  The dimensions of the BZ are inverse distance: if two
polymers have unit cells of 10 and 8 \AA, then the ratio of their unit cell
translation vectors is 1.25:1, and the ratio of their BZ's is 1:1.25.

Although MOPAC can generate all the information necessary  to generate
band-structures, for ease of use this job is given to a special program called
BZ. \  MOPAC is designed to run in batch mode, but drawing band-structures is
an interactive operation.

Before BZ can be run, a MOPAC calculation must first be run in order to
generate the data needed by BZ. \  Most of this data is invisible to the user,
but the user must supply to MOPAC one or two data which BZ will use.  These
data are:

\begin{description}
\item[\comp{MERS=($n_1$,$n_2$,$n_3$)}]~\\
(See Chapter~\ref{makpol:mers}  and Section~\ref{mopac:mers})  BZ needs to know how
the Fock matrix is organized, and  \comp{MERS} gives the number of unit cells
in each direction.  MOPAC uses the keyword \comp{MERS} as the prompt to write
out a file for use by BZ.  Within MOPAC, \comp{MERS} has no other use.
\index{MERS}
\item[\comp{BCC} \normalfont{(optional)}]~\\
Like \comp{MERS}, \comp{BCC} is not used by MOPAC. \comp{BCC} (see
Chapter~\ref{makpol:bcc})  indicates that the system is Body-Centered-Cubic; in
other words, that every odd unit cell is missing.  If the keyword \comp{BCC} is
present, MOPAC informs BZ that the system is Body-Centered-Cubic.
\end{description}


BZ is an interactive utility program.  It can perform three types of
calculation:
\begin{itemize}
\item  Points in $k$-space are to be studied.
\item  Lines in $k$-space are to be drawn.
\item  Surfaces in $k$-space are to be drawn.
\end{itemize}

\subsection{Data input for BZ}\index{Data! for BZ}
Once the MOPAC calculation has been completed, the band structure can be
generated. The instruction to generate band structures is:

\comp{bz $<$filename$>$}

where \comp{$<$filename$>$} is the name of the file. This must be the same as
that used for the MOPAC calculation.

When BZ starts, it invites the user to specify the type
of calculation to be performed. The allowed types are:
\begin{description}
\item[0] Points in $k$-space are to be studied.
\item[1] Lines in $k$-space are to be drawn.
\item[2] Surfaces in $k$-space are to be drawn.
\end{description}

Instead of supplying data at the prompt, a data set can be supplied, using the
 command:

\comp{bz $<$filename$>$ $<$filename2$>$}

A second file, called \comp{$<$filename2$>$.bz}, must exist.  This file will
contain the data that would otherwise be supplied by the prompt.

\index{Help! with BZ}
In addition, the user is invited to enter the number `3' to request help. Help
describes diagnostic key-words which allow an interested user to follow the
calculation.

Some of the operations in BZ are very sensitive to perturbations in the Fock
matrix, specifically any perturbations which lower the symmetry.  The first
thing BZ does, therefore, is to restore symmetry.  To do this, BZ reads in a
list of symmetry operations and symmetrizes the Fock matrix.

At the present time, the user must supply all operations of the group.
Eventually this will be done automatically by the program, but for now the job
of defining symmetry is given to the user.

\subsubsection{Structure of $<$filename$>$.ops}
\index{Space group!operations}
This file holds the definitions of the symmetry operations for the space-group.
The format is `free format' - one or more spaces are treated as one space. The
first line should contain one of the following two words: ``Cartesian'' or
``Crystallographic''.  If ``Cartesian'' is selected, then the rotation axis is
relative to the Cartesian coordinates of the atoms; if ``Crystallographic'' is
selected, then the rotation axis is relative to the crystal coordinates of the
atoms.  (In practice, ``Cartesian'' is {\em much} easier to use.)

Each operation is defined by one line and is composed of 12 data.
In order, these data are:
\begin{enumerate}
\item A `1' or `0'.  `1' if the operation is composed of the inversion operation times
a rotation, `0' if the operation is a pure rotation.
\item The non-primitive translations.  These are given as their reciprocals, thus
a (0.5,0.5,0.0) translation would be entered as `2~2~0'. The absence of
a non-primitive translation is indicated by a zero. (three numbers)
\item The rotation expressed as a fraction of a circle.
\item The axis about which the operation is performed. Units are
Cartesian coordinates, or, if ``Crystallographic'' is selected, the crystal
coordinates. (three numbers)
\item The center about which the operation is performed. Units are
Cartesian coordinates, in Angstroms. (three numbers)
\item A text string describing the operation.
\end{enumerate}

The operations list is terminated by an operation which has a `2' for the
inversion option.

Examples of these operations are illustrated in Table~\ref{sgo}
by the 10 classes of operation for the O$_h$ group in O$_h^7$.


The Fock matrix will be symmetrized if, and only if, every operation of
the group is represented once.  For O$_h$, this would involve a total of 49
lines, 48 for the operations of O$_h$, and one line to terminate the operations
file.

% 24 lines, including this one
\begin{table}
\caption{\label{sgo}
 Space-Group Operations of O$_h^7$}
\index{oh7@{$O_h^7$}|ff}
\begin{center}
\begin{tabular}{lllllllrrrll}\\ \hline
i& \multicolumn{3}{c}{N.P.T}  & Rotation & \multicolumn{3}{c}{Axis of Rotation} &
 \multicolumn{3}{c}{Center of Operation}& Name\\ \hline
0& 0& 0& 0& 0.0     &  0.0&0.0 &1.0 &  0.0& 0.0& 0.0 &  Identity  \\
0& 0& 0& 0& 0.3333  &  1.0&1.0 &1.0 &  0.0& 0.0& 0.0 &  C$_3$  \\
0& 0& 0& 0& 0.5     &  1.0&0.0 &0.0 &  0.0& 0.0& 0.0 &  C$_2$=C$_4^2$  \\
0& 2& 2& 2& 0.5     &  1.0&0.0 &-1.0 & 0.0& 0.0& 0.0 &  C$_2$  \\
0& 2& 2& 2& 0.25    &  0.0&0.0 &1.0 &  0.0& 0.0& 0.0 &  C$_4$  \\
1& 2& 2& 2& 0.0     &  0.0&0.0 &1.0 &  0.0& 0.0& 0.0 &  Inversion  \\
1& 0& 0& 0& 0.25    &  0.0&0.0 &1.0 &  0.0& 0.0& 0.0 &  S$_4$  \\
1& 0& 0& 0& 0.5     &  1.0&1.0 &0.0 &  0.0& 0.0& 0.0 &  $\sigma_d$  \\
1& 2& 2& 2& 0.333333&  1.0&1.0 &1.0 &  0.0& 0.0& 0.0 &  S$_6$  \\
1& 2& 2& 2& 0.5     &  0.0&0.0 &1.0 &  0.0& 0.0& 0.0 &  $\sigma_h$  \\
2& 0& 0& 0& 0.0     &  0.0&0.0 &1.0 &  0.0& 0.0& 0.0 &  End  \\ \hline
\end{tabular}

N.P.T.: Non-Primitive Translation \end{center}
\end{table}

\subsection{Calculating Points in $k$-space}
\index{ksp@{$k$-space}!calculating points in}

A point in $k$-space is specified by one number for a polymer, two
numbers for a layer system, and three numbers for a solid.   All symmetry
points and lines for diamond are shown in Table~\ref{sod} and Figure~\ref{oh7}.
% 21 lines, including this one
\begin{figure}
\begin{makeimage}
\end{makeimage}
% \setlength{\unitlength}{0.1cm}
\begin{center}
% \begin{picture}(100,100)
% \put(62,47){$\Gamma$}
% \put(106,42){X}
% \put(108,68){W}
% \put(107.5,58){$\longleftarrow$Z}
% \put(96.5,54){S$\longrightarrow$}
% \put(84,44){$\Delta$}
% \put(100.2,59.5){U}
% \put(76,71){L}
% \put(92,71){Q}
% \put(56,80){K}
% \put(61,63){$\Sigma$}
% \put(69,59){$\Lambda$}
% %\put(14,38){\psfig{figure=oh7.ps}}
% \put(14,38){\includegraphics{oh7}}
% \end{picture}
\includegraphics{oh7}
\end{center}
\caption{\label{oh7}Brillouin zone for Diamond}
\end{figure}
A ``random'' point is indicated by the number 0.123.  (Note: 0.377 = 0.500 -
0.123)

These high-symmetry points have associated subgroup symmetries, thus $\Gamma $
has the symmetry O$_h$, $\Lambda $ has the symmetry C$_{3v}$, and X has the
symmetry D$_{4h}$.

\subsubsection{Little Groups}\index{Little groups|(}
The symmetry operations supplied in \comp{ $<$filename$>$.ops} are used to
generate a little group of the space group.  A ``little group'' is the
sub-group of the space-group at a specific point in $k$-space.  There are too
many little groups to allow all the irreducible representations to be labeled.
In addition, there is no generally-accepted convention for describing them.
Because of this, the little group is given as a group-table.  Thus the little
group for $\Gamma$ would have the form shown in Table~\ref{lgd}.

% 22 lines, including this line
\begin{table}
\begin{center}
\caption{\label{sod} Symmetry Points in $k$-space for Diamond}
\begin{tabular}{lllcc} \\ \hline
\multicolumn{3}{c}{Point} & Little Group* & Symbol\\ \hline
  0.000& 0.000& 0.000  & O$_h$ &  $\Gamma$\\
  0.500& 0.000& 0.000  & (D$_{4h}$)&X\\
  0.500& 0.250& 0.000  & (C$_{2v}$)&W\\
  0.500& 0.125& 0.125  & (C$_{2v}$)&U\\
  0.375& 0.375& 0.000  & (C$_{2v}$)&K\\
  0.250& 0.250& 0.250  & D$_{3d}$ &L\\
  0.123& 0.000& 0.000  & C$_{4v}$ &$\Delta $\\
  0.500& 0.123& 0.000  & (C$_{2v}$) &Z\\
  0.500& 0.045& 0.045  & (C$_{2v}$)&S\\
  0.123& 0.123& 0.000  & C$_{2v}$ &$\Sigma $\\
  0.123& 0.250& 0.377  &  C$_{2}$ &Q\\
  0.123& 0.123& 0.123  & C$_{3v}$ &$\Lambda $\\ \hline
\end{tabular}

* Point groups in parentheses only approximate little groups.\end{center}\end{table}
\begin{table}
\begin{center}
\caption{\label{lgd} Little Group for $\Gamma$ for Diamond}
\begin{tabular}{lrrrrrrrrrrrr}\\ \hline
Level & Energy & $\Gamma$ & E & C$_3$ & C$_2$=C$_4^2$ & C$_2$ & C$_4$
 & I & S$_4$ & $\sigma_d$ & S$_6$
&$\sigma_h$ \\ \hline
  1& -58.0408& 1& 1& 1& 1& 1 &1&1&1&1&1&1 \\
  2&  -9.4687& 2& 3& 0&-1& 1&-1&3&-1&1&0&-1 \\
  3&   0.9807& 3& 3& 0&-1&-1&1&-3&-1&1&0&1 \\
  4&   4.1933& 4& 1& 1& 1&-1&-1&-1&1&1&-1&-1\\ \hline
\end{tabular}\end{center} \end{table}

From this we see that there are four different irreducible representations: 1,
2, 3, and 4. These representations can be equated with those of the O$_h$
point-group, so 1 and 4, being non-degenerate, become $a$, while 2 and 3, being
3-fold degenerate, become $t$. So far we have ($a$, $t$, $t$, $a$).  The effect
of inversion on the eigenfunctions allows us to assign {\em gerade} and  {\em
ungerade}, thus ($a_g$, $t_g$, $t_u$, $a_u$), while C$_4$ resolves the
``1''-``2'' nature of the representations: ($a_{1g}$, $t_{2g}$, $t_{1u}$,
$a_{2u}$).

Inspection of the eigenvectors reveals, as expected, that the two
$a$ representations can be identified with the $s$ atomic orbitals, and the two
$t$ representations, with the six $p$ atomic orbitals.

For points other than $\Gamma$, the little groups become complex. Consider a
point on $\Delta$, the line connecting $\Gamma$ and X, or (0,0,0) and
(1/2,0,0). The little group for (0.123,0,0) has the form shown in
Table~\ref{dig}.
\begin{table}
\begin{center}
\caption{\label{dig} Little Group for $\Delta$ for Diamond}
 \index{Diamond!little group|(}
\begin{tabular}{lrlrrrrr} \\ \hline
Level&Energy&$\Gamma$&E &C$_2$=C$_4^2$ &C$_4$&$\sigma_d$
&$\sigma_h$  \\  \hline
  1& -54.2893 &1 & 1& 1& $\theta$    &   1& $\theta$   \\
  2& -10.2758 &2 & 2&-2& 0           &   0& 0              \\
  3& -10.1336 &3 & 1& 1&-$\theta$    &   1& -$\theta$   \\
  4&   0.9087 &1 & 1& 1& $\theta$    &   1& $\theta$   \\
  5&   1.5003 &2 & 2&-2& 0           &   0& 0  \\
  6&   3.7878 &3 & 1& 1&-$\theta$    &   1& -$\theta$  \\ \hline
\end{tabular}\end{center}\end{table}
Here $\theta$ = $e^{-ik/2\pi}$.


As is seen, complications arise for those operations which involve
\index{Characters!complex}\index{Complex characters} non-primitive
translations,\index{Translations!non-primitive} \index{Non-primitive
translations} e.g.\  C$_{4}$ and $\sigma_h$, when points other than $\Gamma$ are
studied. The complex phase-factor can cause interpretation of the group
characters to be difficult.  One option is to correct for the phase-factor by
multiplying the character by $e^{ik/2\pi}$. This would make the little group
for $\Delta$ easier to read (Table~\ref{mdig}).

\begin{table}
\begin{center}
\caption{\label{mdig} Modified Little Group for $\Delta$ for Diamond}
\begin{tabular}{lrlrrrrr} \\ \hline
Level&Energy&$\Gamma$&E &C$_2$=C$_4^2$ &C$_4$&$\sigma_d$
&$\sigma_h$  \\ \hline
  1& -54.2893 &1 & 1& 1& 1     &   1& 1    \\
  2& -10.2758 &2 & 2&-2& 0     &   0& 0              \\
  3& -10.1336 &3 & 1& 1&-1     &   1& -1    \\
  4&   0.9087 &1 & 1& 1& 1     &   1& 1    \\
  5&   1.5003 &2 & 2&-2& 0     &   0& 0  \\
  6&   3.7878 &3 & 1& 1&-1     &   1& -1   \\ \hline
\end{tabular}\end{center}\end{table}

However, for some little groups, this procedure may cause complications. Thus,
at $L$, the inversion operation becomes sensitive to the point chosen, e.g.
(0.25,0.25,0.25) will give a different character for the $I$ operation than
(0.25,0.25,-0.25).

A third option is to assume that the lowest-lying level represents the totally
symmetric representation.  (Note: this is different from transforming as the
totally symmetric representation.  Only one point, the $\Gamma$-point, can ever
contain a totally symmetric representation.) By making that assumption, all
characters can be related to those of the  lowest-lying level.

Users are given key-words which select these alternatives to the basic
character tables when the $k$-point is specified.  The key-words are:

\begin{description}
\index{One! use in BZ}\index{BZ!use of ``one''}
\item[\comp{one}] Assume the lowest energy level is totally symmetric.  All operations
involving non-primitive translations are then multiplied by a (complex) factor
which will make the lowest energy level totally symmetric. Example: \comp{0.123 0 0 one}.

\index{Byk! use in BZ}\index{BZ!use of ``byk''}
\item[\comp{byk}] Multiply all operations which involve non-primitive translations by
a phase factor (``byk''---``multiply by k'')  which depends on the current value
of $k$. Example: \comp{0.123 0 0 byk}.
\end{description}

Using the points option, assignment of the symmetry of lines and points in
$k$-space becomes very easy.

\subsection{Lines in $k$-space}

A line  in $k$-space is specified by two numbers for a polymer, four numbers
for a layer system, and six numbers for a solid.  The number of points
calculated for each line is proportional to the length of the line in
$k$-space. For a line of length 1.0, for example $\Gamma$ to $\Gamma$, 300
points would be calculated. \index{Walks in
$k$-space}\index{ksp@{$k$-space}!walks} The data needed to specify a
band-structure `walk' from $\Gamma $ to X to  W to L to $\Gamma $ is given in
Table~\ref{walk}, and for diamond, the band-structure for this `walk' is given
in Figure~\ref{walk-pic}.

\begin{table}
\begin{center}
\caption{\label{walk}A ``Walk'' in $k$-Space ($\Gamma $ to X to W to L to $\Gamma $)}
\begin{tabular}{ccccccccc} \\ \hline
Start && Stop&\multicolumn{3}{c}{Start} & \multicolumn{3}{c}{Stop} \\ \hline
$\Gamma$&-&X & 0.000& 0.000& 0.000& 0.500& 0.000& 0.000  \\
X       &-&W        & 0.500& 0.000& 0.000& 0.500& 0.250& 0.000  \\
W       &-&L        & 0.500& 0.250& 0.000& 0.250& 0.250& 0.250  \\
L       &-&$\Gamma$ & 0.250& 0.250& 0.250& 0.000& 0.000& 0.000  \\ \hline
\end{tabular}
\end{center}
\end{table}
\begin{figure}
\begin{makeimage}
\end{makeimage}
\begin{center}
\includegraphics{diamondbs}
\end{center}
\caption{\label{walk-pic}Band structure for Diamond}
\label{pbt}
\end{figure}
\index{Little groups|)}
\index{Diamond!little group|)}


\subsection{Surfaces in $k$-space}
\index{ksp@{$k$-space}!surfaces in}\index{Surfaces in $k$-space} A surface is
defined by seven data.  Three numbers are used to specify the center of the
surface, three are used to define a perpendicular, and one number is used to
define the length of an edge.

For example, to generate the hexagonal face of the Brillouin Zone for diamond,
the specification would be (0.25 0.25 0.25 1 1 1 0.7071), the center of the
plot being (0.25 0.25 0.25), and (1 1 1) defining the direction.

Two dimensional surfaces use 80 points in each direction, and an interpolation
procedure is used to double this number to 160 points in each direction.  The
length of a side in a two-dimensional plot is defined by the user at
run-time.

The cost of generating the surface is quite large.  Because of this, the data
are stored on disk, once they are generated.

The user is prompted for input.

\subsection{Files used by BZ}
\index{Files used by BZ}\index{BZ!files used by}
BZ uses the following files:
\begin{description}
\item[\comp{ $<$filename$>$.ops}] a symmetry file.
\item[\comp{ $<$filename$>$.brz}] a Fock matrix file made by MOPAC.
\end{description}

The file \comp{$<$filename$>$.brz} {\em must} exist.  The file
\comp{$<$filename$>$.ops}
{\em should}, but does not have to, exist.    If \comp{$<$filename$>$.ops} is
written,  the first two lines should look like this:
\begin{verbatim}
0 0 0 0 0.0       0.0       0.0       1.0  0.0 0.0 0.0  'Identity'
2 0 0 0 0.0       0.0       0.0       0.0  0.0 0.0 0.0  'Terminus'
\end{verbatim}
More lines can be added, as needed,  to define the other symmetry operations.
