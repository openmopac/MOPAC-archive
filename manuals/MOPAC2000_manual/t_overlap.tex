\subsection{Overlap Integrals}\index{Overlap!integrals|(}
The particular technique used for the evaluation of the overlap integral
depends on the atoms involved and whether analytic derivatives are used. All
five semiempirical \index{Slater orbitals} methods use Slater-type orbitals,
STOs, although when analytic derivatives are involved~\cite{analyt}, a
Gaussian  \index{Gaussian!orbitals} expansion~\cite{analyt} of STOs is normally
used.

Specific expressions for various of the overlap integrals have appeared in the
literature. These are normally used for those overlaps which involve only small
principal quantum numbers, PQN, $n$, and a low angular quantum \index{Quantum
numbers! in overlap} number, $l$. For the general case, however, in which any
PQN may be encountered, the general overlap integral is used. As the final
expression is rather ungainly, a simple derivation of the overlap integral will
be given.

Slater atomic orbitals are of form\label{so}
$$
\varphi = \frac{(2\xi )^{n+1/2}}{(2n)!^{1/2}}r^{n-1}e^{-\xi r}Y_l^m(\theta ,\phi ),
$$
where the $Y_l^m(\theta ,\phi )$ are the normalized complex spherical
\index{Spherical harmonics}
harmonics. Complex spherical harmonics are chosen for convenience;
$Y_l^m(\theta ,\phi )$ real orbitals have a similar behavior, but require more
manipulation. The $\theta$ dependence of
\index{Laguerre polynomials}
spherical harmonics are the Laguerre polynomials, of form
$$
Y_l^m(\theta ,\phi ) = \frac{e^{im\phi }}{(2\pi )^{1/2}}\left [\frac{(2l+1)(l-m)!}{2(l+m)!}\right ]^{1/2}
\frac{sin^m\theta\ d^{l+m}(cos^2\theta-1)^l}{2^l\ l!\ (d\ cos\theta)^{l+m}}.
$$
For convenience the phase factor is set to +1; this varies according to which
source is used and the purpose for which the Laguerre polynomials are used.

Solving the differential gives
$$
\frac{d^{l+m}}{(d\ cos\theta)^{l+m}}(cos^2\theta-1)^l = \sum_j\frac{l!\ (2j)!\ (-1)^{l+j}}{(l-j)!\ j!\ (2j-l-m)!}(cos\theta )^{2j-l-m},
$$
which, on rearranging to have the summation start at zero,
becomes
$$
\frac{d^{l+m}}{(d\ cos\theta)^{l+m}}(cos^2\theta-1)^l = \sum_{j=0}^{1/2(l-m)}\frac{l!\ (2(l-j))!\ (-1)^j}{j!\ (l-j)!\ (l-m-2j)!}(cos\theta )^{l-m-2j}.
$$
Substituting this into the STO yields
\begin{eqnarray}
\varphi &=& \frac{(2\xi )^{n+1/2}}{(2n)!^{1/2}}\left [\frac{(2l+1)(l-m)!}{2(l+m)!}\right ]^{1/2}\left [\frac{sin^m\theta }{2^l}r^{n-1}e^{-\xi r}\frac{e^{im\phi}}{(2\pi )^{1/2}}\right ] \nonumber \\
&& \sum_{j=0}^{1/2(l-m)}\frac{(2(l-j))!\ (-1)^j}{j!\ (l-j)!\ (l-m-2j)!}(cos\theta )^{l-m-2j}. \nonumber
\end{eqnarray}
At this point it is convenient to collect some of the constants together; thus,
$$
C_{lmj}=\left [\frac{(l-m)!}{(l+m)!}\right ]^{1/2}\frac{(2(l-j))!\ (-1)^j}{2^l\ j!\ (l-j)!\ (l-m-2j)!},
$$
which allows us to represent the STO in a considerably simplified form:
$$
\varphi = \frac{(2\xi )^{n+1/2}}{(2n)!^{1/2}}\frac{(2l+1)^{1/2}}{2^{1/2}}sin^m\theta\
\/ r^{n-1}\ e^{-\xi r}\frac{e^{im\phi}}{(2\pi )^{1/2}}\sum_{j=0}^{1/2(l-m)}C_{lmj}(cos\theta)^{l-m-2j}.
$$
The overlap integral of two STOs can then be represented as
\begin{eqnarray}
<\varphi _a\varphi _b>& =& \frac{(2\xi _a)^{na+1/2}(2\xi _b)^{nb+1/2}}{((2n_a)!(2n_b)!)^{1/2}}
\frac{[(2l_a+1)(2l_b+1)]^{1/2}}{2}\nonumber \\ &&
\int _0^{\infty}sin^m\theta _a\/sin^m\theta _b\/r_a^{na-1}\/r_b^{n_b-1}e^{-r_a\xi _a}\/e^{-r_b\xi _b}\/
\frac{e^{im\phi}e^{im\phi *}}{2\pi } \nonumber \\
 & & \sum_{j_a=0}^{1/2(l_a-m)}C_{i_amj_a}(cos\theta_a)^{l_a-m-2j_a}
  \sum_{j_b=0}^{1/2(l_b-m)}C_{i_bmj_b}(cos\theta_b)^{l_b-m-2j_b}{\rm d}v\nonumber
\end{eqnarray}
It is impractical to solve this integral using polar
\index{Coordinates!prolate spheroidal}
coordinates. Instead, a prolate spheroidal coordinate system is used. Using
the identities:
$$
\begin{array}{ccc}
r_a=\frac{R(\mu + \nu)}{2};  & cos\theta _a = \frac{(1+\mu \nu )}{(\mu + \nu )};&sin\theta _a = \frac{((\mu ^2-1)(1-\nu ^2))^{1/2}}{(\mu + \nu )} \\
 r_b=\frac{R(\mu - \nu)}{2}; &cos\theta _b = \frac{(1-\mu \nu )}{(\mu - \nu )};&
sin\theta _b = \frac{((\mu ^2-1)(1-\nu ^2))^{1/2}}{(\mu - \nu )},
\end{array}
$$
this gives
$
dv = \frac{R^3}{8}(\mu + \nu )(\mu - \nu)d\mu d\nu d\phi.
$

Substituting these identities into the previous
expression we get:
\begin{eqnarray}
<\varphi _a\varphi _b>& =& \int _0^{2\pi }\int _{-1}^{1} \int _1^{\infty}
\xi_a ^{n_a+1/2}\xi_b^{n_b+1/2}\left [\frac{(2l_a+1)(2l_b+1)}{(2n_a)!(2n_b)!}\right ]
^{1/2} \nonumber \\
&&\frac{((\mu ^2-1)(1-\nu ^2))^m}{(\mu+\nu)^m(\mu-\nu)^m}
\frac{R^{n_a-1}}{2^{n_a-1}}(\mu+\nu)^{n_a-1}\frac{R^{n_b-1}}{2^{n_b-1}}(\mu-\nu)^{n_b-1}\nonumber \\
&&\frac{e^{-R\xi_a(\mu+\nu)/2}e^{-R\xi_b(\mu-\nu)/2}}{2\pi}
 \nonumber \\
&&\sum_{j_a=0}^{(l_a-m)/2}\sum_{j_b=0}^{(l_b-m)/2}
C_{l_amj_a}C_{l_bmj_b}\frac{(1+\mu\nu)^{l_a-m-2j_a}(1-\mu\nu)^{l_b-m-2j_b}}
{(\mu+\nu)^{l_a-m-2j_a}(\mu-\nu)^{l_b-m-2j_b}} \nonumber \\&&
\frac{R^3}{8}(\mu+\nu)(\mu-\nu)d\mu d\nu d\phi , \nonumber
\end{eqnarray}
which, on integrating over and rearranging, gives:
\begin{eqnarray}
<\varphi _a\varphi _b>& =& \int _{-1}^{1} \int _1^{\infty}
\frac{\xi_a ^{n_a+1/2}\ \xi_b^{n_b+1/2}}{2}\left [\frac{(2l_a+1)(2l_b+1)}{(2n_a)!(2n_b)!}\right ]
^{1/2}\ R^{n_a+n_b+1} \nonumber \\
&&\sum_{j_a=0}^{(l_a-m)/2}\sum_{j_b=0}^{(l_b-m)/2}
C_{l_amj_a}\ C_{l_bmj_b}(\mu^2-1)^m(1-\nu^2)^m(\mu+\nu)^{n_a-l_a+2j_a}\nonumber \\&&
(\mu-\nu)^{n_b-l_b+2j_b}(1+\mu\nu)^{l_a-m-2j_a}(1-\mu\nu)^{l_b-m-2j_b}
\ e^{-R\xi_a(\mu+\nu)/2}\ e^{-R\xi_b(\mu-\nu)/2}d\mu d\nu . \nonumber
\end{eqnarray}
This is a product of six simple expressions of type $(a+b)^n$. Expanding each
term as a binomial generates six summations:
\begin{eqnarray}
<\varphi _a\varphi _b>& =& \int _{-1}^{1} \int _1^{\infty}
\frac{\xi ^{n_a+1/2}\ \xi^{n_b+1/2}}{2}\left [\frac{(2l_a+1)(2l_b+1)}{(2n_a)!(2n_b)!}\right ]
^{1/2}\ R^{n_a+n_b+1} \nonumber \\
&&\sum_{j_a=0}^{(l_a-m)/2}\sum_{j_b=0}^{(l_b-m)/2}
C_{l_amj_a}\ C_{l_bmj_b}\sum_{k_a=0}^m\sum_{k_b=0}^m
\sum_{P_a}^{n_a-l_a+2j_a}\sum_{P_b}^{n_b-l_b+2j_b}
\sum_{q_a}^{l_a-m-2j_a}\sum_{q_b}^{l_b-m-2j_b} \nonumber \\ &&
\frac{(l_b-m-2j_b)!}{(l_b-m-2j_b-q_b)!q_b!}
\frac{(l_a-m-2j_a)!}{(l_a-m-2j_a-q_a)!q_a!}
\frac{(n_b-l_b+2j_b)!}{(n_b-l_b+2j_b-P_b)!P_b!} \nonumber \\ &&
\frac{(n_a-l_a+2j_a)!}{(n_a-l_a+2j_a-P_a)!P_a!}
\frac{m!^2}{(m-k_b)!k_b!(m-k_a)!k_a!}  \nonumber \\ &&
(-1)^{k_a+k_b+m+P_b+q_b}
\nu^{2k_a+P_a+P_b+q_a+q_b}
\mu^{2k_b+n_a-l_a+2j_a+n_b-l_b+2j_b-P_a-P_b+q_a+q_b}{\rm d}\mu\ {\rm d}\nu.
  \nonumber
\end{eqnarray}

Using integration by parts, and making use of the following
integrals (The ``A'' and ``B'' overlap integrals): \label{ab}
$$
\int_1^{\infty}x^ne^{-ax}dx=e^{-a}\sum_{\mu=1}^{n+1}\frac{n!}{a^{\mu}(n-\mu+1)}=A_n(a)
$$
$$
\int_{-1}^{1}x^ne^{-ax}dx=-e^{-a}\sum_{\mu=1}^{n+1}\frac{n!}{a^{\mu}(n-\mu+1)}
-e^a\sum_{\mu=1}^{n+1}\frac{n(-1)^{n-\mu}}{a^{\mu}(n-\mu+1)!} =B_n(a),
$$
the overlap integral becomes
\begin{eqnarray}
<\varphi _a\varphi _b>& =&
\frac{\xi ^{n_a+1/2}\xi^{n_b+1/2}}{2}\left [\frac{(2l_a+1)(2l_b+1)}{(2n_a)!(2n_b)!}\right ]
^{1/2}R^{n_a+n_b+1} \nonumber \\ &&
\sum_{j_a=0}^{(l_a-m)/2}C_{l_amj_a}\sum_{j_b=0}^{(l_b-m)/2}
C_{l_bmj_b}\sum_{k_a=0}^m\sum_{k_b=0}^m
\sum_{P_a=0}^{n_a-l_a+2j_a}\sum_{P_b=0}^{n_b-l_b+2j_b}
\sum_{q_a=0}^{l_a-m-2j_a}\sum_{q_b=0}^{l_b-m-2j_b} \nonumber \\ &&
\frac{(l_b-m-2j_b)!}{(l_b-m-2j_b-q_b)!q_b!}
\frac{(l_a-m-2j_a)!}{(l_a-m-2j_a-q_a)!q_a!}
\frac{(n_b-l_b+2j_b)!}{(n_b-l_b+2j_b-P_b)!P_b!} \nonumber \\ &&
\frac{(n_a-l_a+2j_a)!}{(n_a-l_a+2j_a-P_a)!P_a!}
\frac{m!^2}{(m-k_b)!k_b!(m-k_a)!k_a!}
(-1)^{k_a+k_b+m+P_b+q_b} \nonumber \\ &&
B_{2k_a+P_a+P_b+q_a+q_b}\left (\frac{R(\xi_a-\xi_b)}{2}\right )
A_{2k_b+n_a-l_a+2j_a+n_b-l_b+2j_b-P_a-P_b+q_a+q_b}\nonumber \\ &&
\left (\frac{R(\xi_a+\xi_b)}{2}\right )
{\rm d}\mu\ {\rm d}\nu , \nonumber
\end{eqnarray}


in which the coefficients $C_{lmj}$ have the numerical values given
in Table~\ref{clmj}.
\begin{table}
\caption{\label{clmj} Values of $C_{lmj}$}
\begin{center}
\begin{tabular}{llllllll} \hline
l & m & j & C$_{lmj}$    & l & m & j & C$_{lmj}$ \\ \hline
0 & 0 & 0 & 1.0        & 3 & 0 & 0 & 5/2 \\
1 & 0 & 0 & 1.0        & 3 & 1 & 0 & (225/48)$^{1/2}$ \\
1 & 1 & 0 & (1/2)$^{1/2}$ & 3 & 2 & 0 & (15/8)$^{1/2}$ \\
2 & 0 & 0 & 3/2        & 3 & 3 & 0 & (5/16)$^{1/2}$ \\
2 & 1 & 0 & (3/2)$^{1/2}$ & 3 & 0 & 1 & -3/2 \\
2 & 2 & 0 & (3/8)$^{1/2}$ & 3 & 1 & 1 & -(3/16)$^{1/2}$ \\
2 & 0 & 1 & -1/2       & \\ \hline
\end{tabular}\\
Note: In subroutine SS the array AFF(l,m,2j) corresponds to C$_{lmj}$ here.
\end{center}
\end{table}
which is the most convenient form for algorithmic use.

In this form, the overlap integral can be found in function SS.
\index{Overlap!integrals|)}

\subsubsection{Rotation of Atomic Orbitals}
As calculated, the overlap integrals represent the overlap of atomic
orbitals that are aligned along the $z$-axis.  In general, this will not
be the case, and the diatomic overlap integral matrix must be rotated in
order to represent the actual orientation used.

The rotation matrices are well known, but the method by which they are
constructed is by no means simple.  Consider the general case of one atom at
the origin, and a second atom at some point $p(x,y,z)$.
Let the angle from the
second atom to the first atom to $z$ axis be $b$, and the angle of the projection
of the second atom onto the $xy$ plane to the $y$ axis be $a$.  Then, the $p$-orbital
transform is as shown in Figure~\ref{p-rot}.
\begin{table}

\begin{tabular}{|c|ccc|} \hline
 &
$p(x)$ &
$p(y)$ &
p(z)
 \\ \hline
%
$\pi +$ &
$\cos a.\cos b$ &
$-\sin a$ &
$\cos a.\sin b$ \\
%
$\pi -$ &
$\sin a.\cos b$ &
$\cos a$ &
$\sin a.\sin b$ \\
%
$\sigma$ &
$-\sin b$ &
0 &
$\cos b$ \\ \hline
\end{tabular}
\caption{\label{p-rot} Angular Dependence of the $p$-orbitals}
\end{table}

The rotation matrices for the higher harmonics are fairly difficult to construct.
In this work, several mathematical tools will be used.  These are presented in
Figure~\ref{maths}.

\begin{figure}
\begin{makeimage}
\end{makeimage}
\begin{enumerate}
\item All odd integrals, i.e., integrals of the type $<\!x^n\!>$, where $n$
is an odd integer, are zero.
\item $<\!x^2\!>=<\!y^2\!>=<\!z^2\!> \neq 0$
\item $<\!x^2y^2\!>=<\!y^2x^2\!>=<\!y^2z^2\!>=<\!z^2x^2\!>\neq 0$
\item $<\!x^4\!>=<\!y^4\!>=<\!z^4\!>\neq 0$
\item All integrals of the type $<\!a^nb^mc^l\!>$ where $a$, $b$, and $c$
are different members of the set ($x$,$y$,$z$) are equal.
\item The normalization condition for the spherical harmonics is:
$$
\int_0^{2\pi}\int_0^{\pi}Y_l^m(\theta\phi)^2\sin\theta {\rm d}\theta{\rm d}\phi =1
$$
\item $\cos^2\theta+\sin^2\theta=1$
\item $\cos 2\theta = 2\cos^2\theta -1$
\item $\sin 2\theta = 2\sin\theta .\cos\theta$
\item All atomic orbitals are assumed to be normalized:
$$
<\!\psi^2\!>=1
$$
\item $x=r\sin\theta.\cos\phi,\ \ y=r\sin\theta.\sin\phi,\ \ z=r\cos\theta.$
\end{enumerate}

When working with the angular components, it is normally easier to use the
Cartesian symbols $x$, $y$, and $z$ instead of the trigonometric forms.  This
avoids the potentially confusing expressions that involve the angles $a$, $b$,
$\theta$, and $\phi$.

\caption{\label{maths} Mathematical Tools for use with Spherical Harmonics}
\end{figure}

Consider the rotation of $d(x^2-y^2)$. The rotation components involved are:
$$
R(x) = (\cos a.\cos b)x +(\sin a.\cos b)y +(-\sin b)z,
$$
and
$$
R(y) = (-\sin a)x +(\cos a)y.
$$
The rotation matrix element $<d(x^2-y^2)|R|d(x^2-y^2)>$ can be derived
using the relationship:
\begin{eqnarray}
R|d(x^2-y^2)> &=& (\cos^2 a.\cos^2 b)x^2+(2\cos a.\cos b.\sin a.\cos b)xy+
                (-2\cos a.\cos b.\sin b)xz+\nonumber \\
              & & (\sin^2a.\cos^2b)y^2+(-2\sin a.\cos b.\sin b)yz+
(\sin^2b)z^2+\nonumber \\
              & &(-\sin^2a)x^2+(2\sin a.\cos a)xy+(-\cos^2a)y^2.\nonumber
\end{eqnarray}

In this expression, the integral over all odd terms vanish, therefore:
\begin{eqnarray}
<d(x^2-y^2)|R|d(x^2-y^2)>&=&
<x^2(\cos^2 a.\cos^2 b-\sin^2a)x^2>+\nonumber \\
& &<x^2(\sin^2a.\cos^2b-\cos^2a)y^2>+\nonumber \\
& &<x^2(\sin^2b)z^2>+\nonumber \\
& &<y^2(-\cos^2 a.\cos^2 b+\sin^2a)x^2>+\nonumber \\
& &<y^2(-\sin^2a.\cos^2b+\cos^2a)y^2>+ \nonumber \\
& &<y^2(-\sin^2b)z^2>.\nonumber
\end{eqnarray}
The integral $<\!x^4\!>$ has the same value as $<\!y^4\!>$, and
$<\!x^2y^2\!>$ has the same
value as $<\!x^2z^2\!>$, etc., therefore this expression can be simplified.
Collecting together terms of the type $<\!x^4\!>$ gives
$$
\cos^2 a.\cos^2 b-\sin^2a -\sin^2a.\cos^2b+\cos^2a,
$$
or
$$
(\cos^2b+1).(2\cos^2a-1).
$$
Collecting terms of the type $<\!y^2x^2\!>$ gives
$$
\sin^2a.\cos^2b-\cos^2a+\sin^2b-\cos^2 a.\cos^2 b+\sin^2a-\sin^2b,
$$
or
$$
-(\cos^2b+1).(2\cos^2a-1).
$$
The rotation matrix element is thus:
$$
<d(x^2-y^2)|R|d(x^2-y^2)>=<\!x^4-2x^2y^2+y^4\!>\frac{1}{2}(\cos^2b+1).(2\cos^2a-1)
$$
As a result of the fact that the angular components of the atomic orbitals
are normalized, i.e., $<\!x^4-2x^2y^2+y^4\!>$=1:
$$
<d(x^2-y^2)|R|d(x^2-y^2)>=(2\cos^2a-1)\cos^2b+\frac{1}{2}(2\cos^2a-1)\sin^2b,
$$
which is the form used in the calculation of the $d$-rotation matrix.
The full set of $d$-orbital rotation matrix elements is presented in
Table~\ref{d_rot}.
\begin{table}
\caption{\label{d_rot} Angular Dependence of the $d$-orbitals}
\begin{tabular}{|l|l|}\hline
 & $\delta + \ \equiv \ d(x^2-y^2)$ \\ \hline
$d(x^2-y^2)$ & $(2\cos^2a-1) \cos^2b+\frac{1}{2}(2\cos^2a-1) \sin^2b $ \\
$d(xz)$ & $(2\cos^2a-1) \sin b.\cos b$ \\
$d(2z^2-x^2-y^2) $ & $\sqrt{\frac{3}{4}}(2\cos^2a-1) \sin^2b$ \\
$d(yz)$ & $-2\sin a.\cos a. \sin b $ \\
$d(xy)$ & $-2\sin a.\cos a. \cos b $ \\ \hline
%
  & $\pi + \ \equiv \ d(xz) $ \\\hline
$d(x^2-y^2) $ &-$\cos a. \sin b.\cos b$ \\
$d(xz) $ & $\cos a. (2\cos^2b-1)$ \\
$d(2z^2-x^2-y^2) $ & $\sqrt{3}\cos a. \sin b.\cos b$ \\
$d(yz)$ & $-\sin a. \cos b$ \\
$d(xy)$ & $\sin a . \sin b$ \\  \hline
%\hline
  & $\sigma  \ \equiv \ d(2z^2-x^2-y^2) $ \\\hline
$d(x^2-y^2) $ & $\sqrt{\frac{3}{4}}\sin^2b$ \\
$d(xz) $ & $-\sqrt{3}\sin b.\cos b$ \\
$d(2z^2-x^2-y^2) $ & $\cos^2b-\frac{1}{2}\sin^2b$ \\
$d(yz)$ & $ 0 $ \\
$d(xy)$ & $ 0 $ \\  \hline
%\hline
  & $\pi -  \ \equiv \ d(yz)$ \\     \hline
$d(x^2-y^2)  $ & $-\sin a. \sin b.\cos b$ \\
$d(xz)  $ & $\sin a. (2\cos^2b-1) $ \\
$d(2z^2-x^2-y^2) $ & $\sqrt{3}\sin a.\sin b.\cos b $ \\
$d(yz)  $ & $\cos a. \cos b $ \\
$d(xy) $ & $-\cos a. \sin b $ \\  \hline
%
  & $\delta -  \ \equiv \ d(xy) $ \\\hline
$d(x^2-y^2) $ &2$\sin a.\cos a. \cos^2b+\sin a.\cos a. \sin^2b$ \\
$d(xz)$ & $2\sin a.\cos a. \sin b.\cos b $ \\
$d(2z^2-x^2-y^2) $ & $\sqrt{3}\sin a.\cos a. \sin^2b $ \\
$d(yz)$ & $(2\cos^2a-1) \sin b $ \\
$d(xy)$ & $(2\cos^2a-1) \cos b $ \\ \hline
\end{tabular}
\end{table}


The $f$-orbitals are even more complicated.  Unlike the $d$-orbitals, there are
two complete and equivalent sets of $f$-orbitals.  The set selected is that set
which transforms as $\sigma$, $\pi$, $\delta$, and $\phi$ when viewed along the
$z$ axis.  The $f$-orbital transform is presented in Table~\ref{f_rot1} and
\ref{f_rot2}.
\begin{table}
\caption{\label{f_rot1} Angular Dependence of the $f$-orbitals}
\begin{tabular}{|l|l|} \hline
 & $\sigma \ \equiv \ f(z^3) \ \equiv \ z(5z^2-3r^2)$ \\ \hline
$f(z^3)$ & $\frac{1}{2}\cos b . (5\cos^2b -3)$ \\
$f(xz^2)$ & $-\frac{\sqrt{6}}{4}\sin b.(5\cos^2b-1)$ \\
$f(yz^2)$ & $0$ \\
$f(xyz)$ & $0$ \\
$f(z(x^2-y^2))$ & $\frac{\sqrt{15}}{2}\cos b.\sin^2b$ \\
$ f(x(x^2-3y^2))$ & $-\frac{\sqrt{10}}{4}\sin^3b$ \\
$f(y(3x^2-y^2))$ & $0$ \\ \hline
\end{tabular}

\begin{tabular}{|l|l|} \hline
 & $\pi + \ \equiv \  f(xz^2) \ \equiv \ x(5z^2-r^2)$ \\ \hline
$f(z^3)$ & $\frac{\sqrt{6}}{4}\cos a.\sin b.(5\cos^2b-1)$ \\
$f(xz^2)$ & $\frac{1}{4}\cos a.\cos b.(15\cos^2b-11)$ \\
$f(yz^2)$ & $-\frac{1}{4}\sin a.(5\cos^2b-1)$ \\
$f(xyz)$ & $\frac{\sqrt{10}}{2}\sin a.\cos b.\sin b$ \\
$f(z(x^2-y^2))$ & $\frac{\sqrt{10}}{4}\cos a.\sin b.(1-3\cos^2b)$ \\
$f(x(x^2-3y^2))$ & $\frac{\sqrt{15}}{4}\cos a.\cos b.\sin^2b$ \\
$f(y(3x^2-y^2))$ & $-\frac{\sqrt{15}}{4}\sin a.\sin^2b$ \\ \hline
 & $\pi - \ \equiv \ f(yz^2)  \ \equiv \ y(5z^2-r^2$)\\ \hline
$f(z^3)$ & $\frac{\sqrt{6}}{4}\sin a.\sin b.(5\cos^2b-1)$ \\
$f(xz^2)$ & $\frac{1}{4}\sin a.\cos b.(15\cos^2b-11)$ \\
$f(yz^2)$ & $\frac{1}{4}\cos a.(5\cos^2b-1)$ \\
$f(xyz)$ & $-\frac{\sqrt{10}}{2}\cos a.\cos b.\sin b$ \\
$f(z(x^2-y^2))$ & $\frac{\sqrt{10}}{4}\sin a.\sin b.(1-3\cos^2b)$ \\
$f(x(x^2-3y^2))$ & $\frac{\sqrt{15}}{4}\sin a.\cos b.\sin^2b$ \\
$f(y(3x^2-y^2))$ & $\frac{\sqrt{15}}{4}\cos a.\sin^2b$ \\ \hline
\end{tabular}
\end{table}

\begin{table}
\caption{\label{f_rot2} Angular Dependence of the $f$-orbitals (cont.)}
\begin{tabular}{|l|l|} \hline
 & $\delta + \ \equiv \ f(xyz) \ \equiv \ xy$z\\ \hline
$f(z^3)$ & $\sqrt{15}\sin a.\cos a.\cos b.\sin^2b$ \\
$f(xz^2)$ & $\frac{\sqrt{10}}{2}\sin a.\cos a.\sin b.(3\cos^2b-1)$ \\
$f(yz^2)$ & $\frac{\sqrt{10}}{2}(\cos^2a-\sin^2a)\sin b.\cos b$ \\
$f(xyz)$ & $(\cos^2a-\sin^2a)(\cos^2b-\sin^2b)$ \\
$f(z(x^2-y^2))$ & $\sin a.\cos a.\cos b.(3\cos^2b-1)$ \\
$f(x(x^2-3y^2))$ & $-\frac{\sqrt{6}}{2}\sin a.\cos a.\sin b.(1+\cos^2b)$ \\
$f(y(3x^2-y^2))$ & $-\frac{\sqrt{6}}{2}(\cos^2a-\sin^2a)\sin b.\cos b$ \\ \hline
 & $\delta -\ \equiv \ f(z(x^2-y^2)) \ \equiv \ z(x^2-y^2$)\\ \hline
$f(z^3)$ & $\frac{\sqrt{15}}{2}(\cos^2a-\sin^2a)\cos b.\sin^2b$ \\
$f(xz^2)$ & $\frac{\sqrt{10}}{4}(\cos^2a-\sin^2a)\sin b.(3\cos^2b-1)$ \\
$f(yz^2)$ & $-\sqrt{10}\sin a.\cos a.\sin b.\cos b$ \\
$f(xyz)$ & $-2\sin a.\cos a.(\cos^2b-\sin^2b)$ \\
$f(z(x^2-y^2))$ & $\frac{1}{2}(\cos^2a-\sin^2a)\cos b.(3\cos^2b-1)$ \\
$f(x(x^2-3y^2))$ & $-\frac{\sqrt{6}}{4}(\cos^2a-\sin^2a)\sin b.(1+\cos^2b)$ \\
$f(y(3x^2-y^2))$ & $\sqrt{6}\sin a.\cos a.\sin b.\cos b$ \\ \hline
\end{tabular}

\begin{tabular}{|l|l|} \hline
 & $\phi + \ \equiv \ f(x(x^2-3y^2)) \ \equiv \ x(x^2-3y^2$)\\ \hline
$f(z^3)$ & $\frac{\sqrt{10}}{4}((\cos^2a-\sin^2a)\cos a - 2\sin a.\cos a.\sin a)\sin^3b$ \\
$f(xz^2)$ & $\frac{\sqrt{15}}{4}((\cos^2a-\sin^2a)\cos a - 2\sin a.\cos a.\sin a)\cos b.\sin^2b$ \\
$f(yz^2)$ & $-\frac{\sqrt{15}}{4}((\cos^2a-\sin^2a)\sin a +2\sin a.\cos a.\cos a)\sin^2b$ \\
$f(xyz)$ & $-\frac{\sqrt{6}}{2}((\cos^2a-\sin^2a)\sin a +2\sin a.\cos a.\cos a)\sin b.\cos b$ \\
$f(z(x^2-y^2))$ & $\frac{\sqrt{6}}{4}((\cos^2a-\sin^2a)\cos a - 2\sin a.\cos a.\sin a)\sin b.(1+\cos^2b)$ \\
$f(x(x^2-3y^2))$ & $\frac{1}{4}((\cos^2a-\sin^2a)\cos a - 2\sin a.\cos a.\sin a)\cos b.(3+\cos^2b)$ \\
$f(y(3x^2-y^2))$ & $-\frac{1}{4}((\cos^2a-\sin^2a)\sin a +2\sin a.\cos a.\cos a)(1+3\cos^2b)$ \\ \hline
 & $\phi - \ \equiv \ f(y(3x^2-y^2)) \ \equiv \ y(3x^2-y^2)$\\  \hline
$f(z^3)$ & $\frac{\sqrt{10}}{4}((\cos^2a-\sin^2a)\sin a +2\sin a.\cos a.\cos a)\sin^3b$ \\ 
$f(xz^2)$ & $\frac{\sqrt{15}}{4}((\cos^2a-\sin^2a)\sin a +2\sin a.\cos a.\cos a)\cos b.\sin^2b$ \\ 
$f(yz^2)$ & $\frac{\sqrt{15}}{4}((\cos^2a-\sin^2a)\cos a - 2\sin a.\cos a.\sin a)\sin^2b$ \\ 
$f(xyz)$ & $\frac{\sqrt{6}}{4}((\cos^2a-\sin^2a)\cos a - 2\sin a.\cos a.\sin a)2\sin b.\cos b$ \\ 
$f(z(x^2-y^2))$ & $\frac{\sqrt{6}}{4}((\cos^2a-\sin^2a)\sin a +2\sin a.\cos a.\cos a)\sin b.(1+\cos^2b)$ \\
$f(x(x^2-3y^2))$ & $\frac{1}{4}((\cos^2a-\sin^2a)\sin a +2\sin a.\cos a.\cos a)\cos b.(3+\cos^2b)$ \\
$f(y(3x^2-y^2)$ & $\frac{1}{4}((\cos^2a-\sin^2a)\cos a - 2\sin a.\cos a.\sin a)(1+3\cos^2b)$ \\ \hline
\end{tabular}
\end{table}

