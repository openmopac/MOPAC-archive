\subsection{EigenFollowing}\index{Jensen@{\bf Jensen, Frank}}\label{t_ef}
\begin{center}
Description of the EF and TS function \\
by\\
Dr Frank Jensen  \\
Department of Chemistry  \\
Odense University  \\
5230 Odense  \\
Denmark
\end{center}

The current version of the EF optimization routine is a combination of the 
original EF algorithm of Simons et al.\ (J.\ Phys.\ Chem.\ 89, 52) as
implemented by  Baker (J.\ Comp.\ Chem.\ 7, 385) and the QA algorithm of Culot
et al.\  (Theo.\ Chim.\ Acta 82, 189), with some added features for improving
stability.

\index{Taylor expansion}\index{Hessian}
The geometry optimization is based on a second order Taylor expansion of  the
energy around the current point. At this point the energy, the gradient  and
some estimate of the Hessian are available. There are three fundamental  steps
in determining the next geometry based on this information:
\begin{itemize}
\item  finding the ``best'' step within or on the hypersphere with 
    the current trust radius.
\item possibly reject this step based on various criteria.
\item update the trust radius.
\end{itemize}

\begin{enumerate}\index{Eigenvector!following}\index{OMIN}\index{NONR}\index{RSCAL}
\item For a minimum search the correct Hessian has only positive eigenvalues. 
For a Transition State (TS) search the correct Hessian should  have exactly one
negative  eigenvalue, and the corresponding eigenvector should be in the
direction  of the desired reaction coordinate.  The geometry step is
parameterized  as $g/(s-H)$, where $s$ is a shift factor which ensure that the
step-length is  within or on the hypersphere. If the Hessian has the correct
structure, a pure  Newton-Raphson step is attempted. This corresponds to
setting the shift factor  to zero. If this step is longer than the trust
radius, a P-RFO step is attempted.  If this is also too long, then the best
step on the hypersphere is made via  the QA formula.  This three step procedure
is the default. The pure NR step can  be skipped by giving the keyword
\comp{NONR}. An alternative to the QA step is to  simply scale the P-RFO step
down to the trust radius by a multiplicative  constant, this can be
accomplished by specifying \comp{RSCAL}.

\index{RMIN}\index{RMAX}\index{CYCLE}\index{MODE}
\item Using the step determined from 1), the new energy and gradient are 
evaluated. If it is a TS search, two criteria are used in determining whether 
the step is ``appropriate''. The ratio between the actual and predicted energy 
change should ideally be 1. If it deviates substantially from this value, the 
second order Taylor expansion is no longer accurate. \comp{RMIN} and \comp{RMAX}
(default  values 0 and 4) determine the limits on how far from 1 the ratio can
be before the step is rejected. If the ratio is outside the \comp{RMIN} and
\comp{RMAX} limits, the  step is rejected, the trust   radius reduced by a
factor of two and a new step  is determined. The second criteria is that the
eigenvector along which the  energy is being maximized should not change
substantially between iterations.  The minimum overlap of the TS eigenvector
with that of the previous iteration  should be larger than \comp{OMIN},
otherwise the step is rejected. Such a step  rejection can be recognized in the
output by the presence of (possibly more)  lines with the   same CYCLE number.
The default \comp{OMIN} value is 0.8, which allows  fairly large changes to
occur, and should be suitable for most uncomplicated  systems. See below for a
discussion of how to use \comp{RMIN}, \comp{RMAX} and \comp{OMIN} for 
difficult cases. The selection of which eigenvector to follow towards the TS
is  given by \comp{MODE=$n$}, where $n$ is the number of the Hessian
eigenvector to follow.  The default is \comp{MODE=1}. These features can be
turned off by giving suitable  values as keywords, e.g.\ \comp{RMIN=-100
RMAX=100} effectively inhibits step  rejection. Similarly setting \comp{OMIN=0}
disables step rejection based on large  changes in the structure of the TS
mode. The default is to use mode following  even if the TS mode is the lowest
eigenvector. This means that the TS mode may  change to   some higher mode
during the optimization. To turn of mode  following, and thus always follow the
mode with lowest eigenvalue, set \comp{MODE=0}. If  it is a minimum search the
new energy should be lower than the previous. 

The  acceptance criteria used is that the actual/predicted ratio should be
larger  than \comp{RMIN}, which for the default   value of \comp{RMIN=0} is
equivalent to a lower  energy. If the ratio is below \comp{RMIN}, the step is
rejected, the trust radius  reduced by a factor of two and a new step is
predicted. The \comp{RMIN}, \comp{RMAX} and \comp{OMIN}  features has been
introduced in the current version of EF to improve the  stability of TS
optimizations. Setting \comp{RMIN} and \comp{RMAX} close to one will give a 
very stable, but also very slow, optimization. Wide limits on \comp{RMIN} and
\comp{RMAX} may  in some cases give a  faster convergence, but there is always
the risk that  very poor steps are accepted, causing the optimization to
diverge. The default  values of 0 and 4 rarely rejects steps which would lead
to faster convergence,  but may occasionally accept poor steps. If TS searches
are found to cause problems, the first try should be to lower the limits to 0.5
and 2. Tighter  limits like 0.8 and 1.2, or   even 0.9 and 1.1, will almost
always slow the  optimization down significantly but may be necessary in some
cases. 

In minimum  searches it is usually desirable that the energy decreases in each
iteration.  In certain very rigid systems, however, the initial diagonal
Hessian may be so  poor that the   algorithm cannot find an acceptable step
larger than DDMIN, and  the optimization terminates after only a few cycles
with the ``TRUST RADIUS  BELOW DDMIN'' warning long before the stationary point
is reached. In such cases  the user can add \comp{DDMIN=0.0} and  \comp{RMIN} 
set to some negative value, say -10, thereby allowing  steps which allow the
energy to increase. An alternative is to use \comp{LET DDMIN=0.0}. 

The algorithm has the capability of following  Hessian eigenvectors other than
the one with the lowest eigenvalue toward a TS.  Such higher mode following are
always much more difficult to make converge.  Ideally, as the optimization
progresses, the TS mode should at some point  become the lowest eigenvector.
Care must be taken during the optimization,  however, that the   nature of the
mode does not change all of a sudden, leading to  optimization to a different
TS than the one desired. \comp{OMIN} has been designed for  ensuring that the
nature of the TS mode only changes gradually, specifically  the overlap between
to successive TS modes should be higher than OMIN. While  this concept at first
appears very promising, it is not without problems when  the Hessian is
updated. 

As the updated Hessian in each step is only  approximately correct,   there is
a upper limit on how large the TS mode  overlap between steps can be. To
understand this, consider a series of steps  made from the same geometry (e.g.\
at some point in the optimization), but with  steadily smaller step-sizes. The
update adds corrections to the Hessian to  make it a better approximation to
the exact Hessian. As the step-size become  small, the updated Hessian
converges toward the exact Hessian, at least in the  direction of the step.  
The old Hessian is constant, thus the overlap between  TS modes thus does not
converge toward 1, but rather to a constant value which  indicate how well the
old approximate Hessian resembles the exact Hessian. Test  calculations suggest
a typical upper limit around 0.9, although cases have been  seen where the
limit is more like 0.7. It appears that an updated Hessian in  general is not
of sufficient accuracy for reliably rejecting steps with TS  overlaps much  
greater than 0.80. The default \comp{OMIN} of 0.80 reflects the  typical use of
an updated Hessian. If the Hessian is recalculated in each step,  however, the
TS mode overlap does converge toward 1 as the step-size goes toward  zero, and
in this cases there is no problems following high lying modes. 


Unfortunately setting \comp{RECALC=1} is very expensive in terms of computer
time, but  used in conjecture with \comp{OMIN=0.90} (or possibly higher), and
maybe also tighter  limits on \comp{RMIN}   and \comp{RMAX}, it represents an
option of locating transitions  structures that otherwise might not be
possible. If problems are encountered with  many step rejections due to small
TS mode overlaps, try reducing \comp{OMIN}, maybe  all the way down to 0. This
most likely will work if the TS mode is the lowest  Hessian eigenvector, but it
is doubtful that it will produce any useful results  if a high lying   mode is
followed. Finally, following modes other than the  lowest toward a TS indicates
that the starting geometry is not ``close'' to the  desired TS. In most cases it
is thus much better to further refined the  starting geometry, than to try
following high lying modes. There are cases,  however, where it is very  
difficult to locate a starting geometry which has  the correct Hessian, and
mode following may be of some use here.
\end{enumerate}
