\section{Parametric Molecular Electrostatic Potential (PMEP)}\index{PMEP|ff}
\index{Ford@{\bf Ford, George}}
\index{Wang@{\bf Wang, Bingze}}
\index{Electrostatic Potential!PMEP model}
The PMEP procedure~\cite{pmep1,pmep2} is a technique for rapidly calculating 
the electrostatic properties of a molecule.  Written by Prof.\ George Ford and
Dr.\ Bingze  Wang\footnote{Current address: IRBM, Via Pontina Km.30.600, 00040
Pomezia (Roma), Italy} at Southern Methodist University, Dallas, Texas, the
procedure is ideally suited for large systems.  

The PMEP procedure has two main functionalities: first, to generate a 2-D grid
of points giving the Electrostatic Potential (ESP) in a cross-section through a
system, and second, to generate atomic charges based on the calculated ESP. At
present, the method is limited to AM1 systems containing H, C, N, O, F, Cl,
only. 

\subsection{2-D Electrostatic Potential Plots}
ESP plots are generated in two steps.  First, a MOPAC calculation generates a
2-D grid of points.  This grid is then converted into a picture by the utility
program \comp{ESPLOT}.  \comp{ESPLOT} is very simple to use: the command is
\comp{esplot $<$filename$>$}, where \comp{$<$filename$>$} is the name of the
data-set.  \comp{ESPLOT} generates an on-line picture of the PMEP, and a HPGL
file suitable for use in generating hard-copy.  Because \comp{ESPLOT} is so
simple, it will not be discussed further.  Instead, the rest of this discussion
applies to the MOPAC calculation.

The grid generated by MOPAC consists of a 2-D array of points representing a
cross-section through the system.  The distance between points is a constant
0.1 \AA ngstroms.  The size represented by the grid is roughly 4 \AA ngstroms 
plus twice the size of the system.  For example, N$_2$ has a N--N distance of
1.1 \AA , and the default associated grid represents a rectangular area of 5.8
by 4.6 \AA .    Each grid point represents the potential in kcal/mol which a
unit positive charge would experience due to the electrostatic field of the
system.

ESP grids are generated by specifying 
\hyperref[pageref]{\comp{PMEP}}{(see p.~}{)}{pmep} and \comp{PRTMEP}.
An example of a data-set for the PMEP procedure is shown in
Figure~\ref{pmepdata}. The PMEP plot for this data set is shown in
Figure~\ref{pmepplot}. This plot can be compared with the
\hyperref[pageref]{\comp{MEP} plot}{ on page~}{}{mep1plot}.

\begin{figure}
\begin{makeimage}
\end{makeimage}
\begin{verbatim}
 1scf  AM1 PMEP MINMEP PRTMEP
 Formaldehyde (Cross-section in plane of molecule)
 Generate a 2-D grid of PMEP potentials for `esplot' to use
  O    0.00000000 0     0.0000000 0    0.0000000 0    0 0 0     -0.2759
  C    1.22732374 1     0.0000000 0    0.0000000 0    1 0 0      0.1384
  H    1.11047287 1   122.2253516 1    0.0000000 0    2 1 0      0.0688
  H    1.11048351 1   122.2158646 1  179.9998136 1    2 1 3      0.0687
\end{verbatim}
\caption{\label{pmepdata}Data Set for PMEP Calculation of Formaldehyde}
\end{figure}

\begin{figure}
\begin{makeimage}
\end{makeimage}
\begin{center}
\includegraphics{ch2o-ford}
\end{center}
\caption{\label{pmepplot}Parametric Molecular Electrostatic Potential around
Formaldehyde}
\end{figure}

\subsubsection{Choice of Plane to be Calculated}
By default, the grid is centered on the center of the molecule, and the X-Y
plane at Z=0 is selected.  Other grids can be chosen using \comp{PMEPR}.
\comp{PMEPR} uses three atoms and an optional offset to define the plane to be
used.  It has enough options to allow any plane to be easily specified.

\subsection{Atomic Charges}
By use of \comp{QPMEP}, a set of atomic charges can be calculated.  This set of
charges is the best least squares fit to the charges which reproduce the ESP of
the Connolly or Williams surfaces.
