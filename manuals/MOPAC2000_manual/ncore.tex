\subsection*{\comp{NCORE(100)}}
Almost all of the arrays that depend on the size of the system  are assigned
dynamically.  At the start of the calculation, two large blocks of memory are
reserved. One block, \comp{MCORE}, will be used to hold all the integer arrays,
and the other, \comp{CCORE},  is used to hold all the real arrays. The starting
addresses of the arrays are given in \comp{NCORE}. Thus, for example, the
first dynamic real array assigned is \comp{GEO}, and the corresponding starting
address can be found in \comp{NCORE(3)}.  The order in which the arrays are
stored in \comp{NCORE} is not important, however, each array is associated with
a unique address in \comp{NCORE}.  A full list of the 100 addresses in 
\comp{NCORE} is as follows:

\begin{table}
\caption{\label{asl} Array Storage Locations in MOPAC}
\begin{center}
\begin{tabular}{lllllllll} \hline
MOZYME&MOPAC&No.&MOZYME&MOPAC&No.&&MOPAC&No.\\\hline
Special &  Special &  1 & \comp{TXTATM} & \comp{TXTATM} & 34 & & \comp{NAR} & 67\\
Special &  Special &  2 & \comp{ISORT} & \comp{TOM} & 35 & & \comp{DIRVEC} & 68\\
\comp{GEO} &  \comp{GEO} &  3 & & & 36 & & \comp{BH} & 69\\
\comp{NA,NB,NC} & \comp{NA,NB,NC} &  4 & & & 37 & & \comp{IATSP} & 70\\
\comp{XPARAM} &  \comp{XPARAM} & 5 & \comp{EIGF/EIGV} & \comp{EIGS} & 38 & & \comp{NN} & 71\\ 
\comp{LOC} & \comp{LOC} & 6 & & \comp{EIGB} & 39 & & \comp{QDEN} & 72\\
\comp{COORD} & \comp{COORD} & 7 & \comp{IOPT} & & 40 & & \comp{CH} & 73\\
\comp{LABELS} & \comp{LABELS} & 8 & \comp{JOPT} & \comp{JOPT} & 41 & & \comp{NSETF} & 74\\
\comp{NAT} &  \comp{NAT} &  9 & \comp{OLDXYZ} & & 42 & & \comp{TOM} & 75\\
\comp{NFIRST/NLAST} &  \comp{NFIRST/NLAST} & 10 & \comp{DXYZ} & \comp{DXYZ} & 43 & & & 76\\
\comp{USPD} &  \comp{USPD} & 11 & \comp{PARTP} & PA & 44 & & & 77\\
\comp{PSPD} &  \comp{PSPD} & 12 & \comp{PARTH} & \comp{PB} & 45 & & & 78\\
\comp{IORBS} &  \comp{IORBS} & 13 & \comp{PARTF} & \comp{FB} & 46 & & & 79\\
\comp{IJBO} &  \comp{VECTCI} & 14 & \comp{KOPT} & \comp{KOPT} & 47 & & & 80\\
\comp{H} & \comp{H} & 15 & & & 48 & & \comp{AMAT} & 81\\
\comp{W} & \comp{W} & 16 & & \comp{HESINV} & 49 & & \comp{BMAT} & 82\\
\comp{P} & \comp{P} & 17 & & \comp{JELEM} & 50 & & \comp{CMAT} & 83\\
\comp{F} & \comp{F} & 18 & & \comp{ATMASS} & 51 & & \comp{COSURF} & 84\\
\comp{NCE} & \comp{CONF} & 19 & & \comp{REACT} & 52 & & \comp{ISUDE} & 85\\
\comp{NCF} & & 20 & & \comp{C0} & 53 & & \comp{SUDE} & 86\\
\comp{NCOCC} & \comp{WK} & 21 & & \comp{NC} & 54 & & \comp{PHINET} & 87\\
\comp{NCVIR} & \comp{HQ} & 22 & & \comp{INTERP} & 55 & & \comp{QSCNET} & 88\\
\comp{GRAD} & \comp{GRAD} & 23 & & \comp{ERRFN} & 56 & & \comp{QDENET} & 89\\
\comp{NNCF} & \comp{NSET} & 24 & & \comp{AIDER} & 57 & & \comp{IPIDEN} & 90\\
\comp{NNCE} & & 25 & & \comp{ALPARM} & 58 & & \comp{GDEN} & 91\\
\comp{ICOCC} & \comp{XY} & 26 & & & 59 & & \comp{QSCAT} & 92\\
\comp{ICVIR} & \comp{CIMAT} & 27 & & \comp{NAMO} & 60 & & \comp{IDENET}  & 93\\
\comp{COCC} & \comp{C} & 28 & & \comp{JNDEX}  & 61 & & \comp{ARAT} & 94\\
\comp{CVIR} & \comp{CB} & 29 & & \comp{IPO} & 62 & & \comp{WK} & 95\\
\comp{NBOND} & \comp{DIJKL} & 30 & & \comp{DXYZR} & 63 & & & 96\\
\comp{IFACT/I1FACT} & \comp{IFACT/I1FACT} & 31 & & \comp{XPAREF} & 64 & & Special & 97\\
\comp{IBONDS} & \comp{NPERMA} & 32 & & \comp{PROFIL} & 65 & & Special& 98\\
\comp{NFMO} & \comp{NPERMB} & 33 & & \comp{SRAD} & 66 & & Special& 99\\
 & & & & & & & Special& 100\\
\hline
\end{tabular}
\end{center}
\end{table}

\begin{enumerate}
\item {\bf Start of Real:}  Not an array address, but the starting address of
the first unused element of \comp{CCORE}.  Any new permanent real array would
be given the starting address \comp{NCORE(1)}. At the start of the calculation,
\comp{NCORE(1)=1}.
\item {\bf Start of Integer:}  Not an array address, but the starting address
of the first unused element of \comp{MCORE}.  Any new permanent integer array
would be given the starting address \comp{NCORE(2)}. At the start of the
calculation, \comp{NCORE(2)=1}.
\item \comp{GEO:} The internal coordinates. (3$*$NATOMS)
\item \comp{NA,NB,NC:} The connectivity indices.  Set in \comp{COPY1}. If an
atom position is defined in Cartesian coordinates, then the corresponding
NA($i$), NB($i$), and NC($i$) would all be zero. (3$*$NATOMS)
\item \comp{XPARAM:} The parameters to be optimized. (3$*$NATOMS)
\item \comp{LOC:} An array of size 2 by 3$\times$\comp{NATOMS}, \comp{LOC} is
the list of atom coordinates which are to be optimized.  Each pair of elements
in \comp{LOC} represents the atom  and the coordinate (bond length, angle or
dihedral, or $x$, $y$, or $z$). If all three coordinates of an atom are to be
optimized, then six array elements in \comp{LOC} are used.   Set in
\comp{COPY2}. In  \comp{XPARAM($n$)}, the associated atom would be
\comp{LOC(1,n)}, and the associated coordinate would be \comp{LOC(2,n}).
(6$*$NATOMS)
\item \comp{COORD:} A two dimensional array holding the Cartesian coordinates
of all real atoms, stored as $x$, $y$, $z$ for each atom, in turn.  The order
of the atoms is the same as that in the supplied data set. Units: \AA ngstroms.
Although there are only 3$*$NUMAT of these, during the conversion from
\comp{GEO}, extra storage is  needed to hold the Cartesian coordinates of the
dummy atoms and translation vectors. (3$*$NATOMS)
\item \comp{LABELS:} A one-dimensional array of labels of all atoms, real  and
dummy.  If there are no dummy atoms, this array is the same as the \comp{NAT} 
array.  Set in \comp{COPY1}, \comp{FMAT}, \comp{FORCE}, \comp{POLAR},
\comp{REACT1} and \comp{GEOCHK}. These are integers in the range 1 to 107.
(3$*$NATOMS)
\item \comp{NAT:} The labels of all real atoms. NAT($i$) holds the atomic
number of the $i$'th real atom. Set in \comp{MOLDAT} and \comp{RESEQ}. (NUMAT)
\item \comp{NFIRST, NLAST:} The starting and stopping atomic orbital indices
for all real atoms.  If the first three atoms were C, H, and +, then the values
of (\comp{NFIRST, NLAST}) would be (1,4), (5,5), and (6,5).  Note that the
number of atomic orbitals on an atom $i$ is given by  NLAST($i$)-NFIRST($i$)+1.
(NUMAT)
\item \comp{USPD:} The initial values of the one-electron one-center energies
for all atoms.  (NORBS)
\item \comp{PSPD:} The initial atomic orbital occupancies.  (NORBS)
\item \comp{IORBS:} The number of atomic orbitals on each atom. IORBS($i$) is
the same as NLAST($i$)-NFIRST($i$)+1. (NUMAT)
\item 
\begin{description}
\item[(MOZYME) \comp{IJBO:}] A two-dimensional array giving the starting 
addresses of every calculated atom-pair in \comp{H}, \comp{F}, \comp{P},
\comp{PARTH}, \comp{PARTF}, and  \comp{PARTP}. If atom-pair $i,j$ is
calculated, the starting address is defined as  IJBO($i,j$)+1.  For the first
atom, atom 1, the starting address is IJBO(1,1)+1 = 1. If an atom-pair is not
calculated, the associated array element in \comp{IJBO} is set to \comp{--2} if
electrostatic polarization terms are to be used (see \hyperref{\comp{CUTOF2} and
\comp{CUTOF1}}{, p.~}{}{cutoff}), or to \comp{--1} is only simple
electrostatic terms are to be considered.

\comp{IJBO} is set in subroutine \comp{GETLIM} only.  Once set, \comp{IJBO} is
never changed.
\item[\comp{VECTCI:}] The state vectors (eigenvectors of the C.I. matrix) for
the root requested. (5*LAB)
\end{description}

\item \comp{H:} The one-electron integrals.  \index{H Matrix!structure} The way
in which the array elements are stored in conventional and LMO methods is
different.  In conventional work, the density (\comp{P}), one-electron
(\comp{H}), and Fock (\comp{F}) matrices are stored in packed, lower-half,
triangular form, as shown in Fig~\ref{plhtm}.
\begin{figure}
\begin{makeimage}
\end{makeimage}
\begin{verbatim}
       
            Conventional Method                   LMO Method
      
           s px py pz    s px py pz               s px py pz    s px py pz

       s   1                                   s  1
      px   2  3                               px  2  3
      py   4  5  6                            py  4  5  6
      pz   7  8  9 10                         pz  7  8  9 10

       s  11 12 13 14   15                     s 11 12 13 14   27
      px  16 17 18 19   20 21                 px 15 16 17 18   28 29
      py  22 23 24 25   26 27 28              py 19 20 21 22   30 31 32
      pz  29 30 31 32   33 34 35 36           pz 23 24 25 26   33 34 35 36
\end{verbatim}
\begin{center}
Order of occurrence of array elements for two atoms, each with a $s-p^3$ basis
set.
\end{center}
\caption{\label{plhtm} Order of Storage of Matrix Elements}
\end{figure}

In conventional work, the address of an array element involving atomic orbitals
$\lambda$ and $\sigma$ can be determined from the order in which they occur in
the basis set for the system.  If $\lambda$ is the $i$'th atomic orbital, and
$\sigma$ is the $j$'th, with $i < j$, then the address of the array element is
given by:
$$
\lambda,\sigma  = P((j(j+1))/2+i).
$$

In LMO work, the order in which the array elements in \comp{P},  \comp{H}, and
\comp{F} are stored is the same, but is different from that used in
conventional work. The starting address of the array elements involving atoms
$A$ and $B$ is stored in the two-dimensional array \comp{IJBO}. If  $\lambda$
is the $i$'th atomic orbital on atom $A$ and  $\sigma$ is the $j$'th atomic
orbital on atom $B$, and $A$ occurs before $B$, then the address, $\lambda
\sigma$, of the array element is:
$$
\lambda,\sigma = P(IJBO(A,B)+(i-1)\times N_a+j),
$$
in which $N_a$ is the number of atomic orbitals on atom $A$.

If both $\lambda$ and $\sigma$ are on the same atom, then the address is given
by:
$$
\lambda,\sigma = P(IJBO(A,B)+(i(i-1))/2+j).
$$
Note that the first element of any diatomic or monatomic interaction is given
by \comp{IJBO(A,B)}+1.  As a result of this, the first entry in the 
\comp{IJBO} array, \comp{IJBO(1,1)} is zero.

The reason that elements in arrays in LMO methods are stored differently arises
from the fact that not all diatomic interactions are considered. If two atoms
are sufficiently far apart, then the corresponding array elements are not
considered.  This is indicated in the \comp{IJBO} array by the presence of a
negative number.  By ignoring such terms, a large saving can be made in the
amount of memory needed. (MPACK)

\item \comp{W:} The two-electron integrals. 
\begin{description}
\item[(MOPAC) \comp{W:}]  A one-dimensional array of the two-electron
integrals. The number of these integrals (the size of \comp{W}) is $LM61^2$.
\item[(MOZYME) \comp{W:}]\label{n16} A one-dimensional array of the
two-electron integrals. The number of these integrals (the size of \comp{W}) is
given as the sum of the following quantities:

\begin{itemize}
\item 2025 for each pair of atoms with $d$-orbitals calculated in \comp{H}.
\item 450 for each $d$-orbital - heavy atom pair  calculated in \comp{H}.
\item 45 for each $d$-orbital - light atom pair  calculated in \comp{H}.
\item 100 for each pair of heavy atoms calculated in \comp{H}.
\item 10 for each heavy-light pair of atoms calculated in \comp{H}.
\item 1 for each pair of light atoms calculated in \comp{H}.
\item 7 for each pair of heavy or $d$-orbital atoms which are {\em not} 
calculated in \comp{H}, and which do have polarization functions.
\item 4 for each pair of atoms, one heavy or $d$-orbital and one light,  which 
are {\em not} calculated in \comp{H}, and which do have polarization functions.
\item 1 for each pair of atoms which are {\em not} calculated in \comp{H}, and
which do not have polarization functions.
\end{itemize}

%what's missing here???
eV. Set in \comp{OUTER2} and \comp{ROTATE} only, and not modified during the
SCF calculation. (N2ELEC)
\end{description}
\item \comp{P:} The density matrix. (MPACK)
\item \comp{F:} The Fock matrix. (MPACK)
\item 
\begin{description}
\item[(MOPAC) \comp{CONF:}] The state eigenvectors.  This is the
orthonormal set of linear combination of configurations that form
the states. They are constructed in \comp{MECI}. 
\item[(MOZYME) \comp{NCE:}]
The number of atoms involved in the virtual LMOs.
(NVIR)
\end{description}
\item {\bf (MOZYME) \comp{NCF:}}  The number of atoms involved in the occupied
LMOs. (NOCC)
\item 
\begin{description}
\item[(MOPAC) \comp{WK:}] The two-electron two-center exchange integrals used
in polymer and other extended solids work.  This array is not used in
calculations on molecules.  (N2ELEC) 
\item[(MOZYME) \comp{NCOCC:}] The starting addresses of the atomic orbital
coefficients in the occupied LMOs.  The starting address for LMO $i$ is
NCOCC($i$)+1. (NOCC)
\end{description}
\item 
\begin{description}
\item[(MOPAC) \comp{HQ:}] Used by the Tomasi solvation method. (MPACK)
\item[(MOZYME) \comp{NCVIR:}] The starting addresses of the atomic orbital
coefficients in the virtual LMOs The starting address for LMO $i$ is
NCVIR($i$)+1. (NVIR)
\end{description}
\item \comp{GRAD:} The gradients of \comp{XPARAM}, that is, the derivative
of the heat of formation, in kcal.mol$^{-1}$, with respect to the coordinate
XPARAM($i$) in \AA ngstroms or radians. (3$*$NATOMS)
\item 
\begin{description}
\item[(MOPAC) \comp{NSET:}] Used by the COSMO method, \comp{NSET} is the set
of points on a surface associated with each atom.
\item[(MOZYME) \comp{NNCF:}]  The starting addresses of the atom-lists in 
\comp{ICOCC}.  For the first atom in occupied LMO $i$, the  address in
\comp{ICOCC} is NNCF($i$)+1. (NOCC)
\end{description}
\item {\bf (MOZYME) \comp{NNCE:}}  The starting addresses of the atom-lists in
\comp{ICVIR}.  For the first atom in virtual LMO $i$, the  address in
\comp{ICVIR} is NNCE($i$)+1. (NVIR)
\item 
\begin{description}
\item[(MOPAC) \comp{XY:}] The two-electron integrals over molecular orbitals.
Used in the C.I., $XY(i,j,k,l)$ is the integral involving electron 1 in M.O.s
$\psi_i$ and $\psi_j$ interacting with electron 2 in M.O.s $\psi_k$ and
$\psi_l$. (NMOS$^4$)
\item[(MOZYME) \comp{ICOCC:}] The set of atoms in each occupied M.O.  All the
occupied LMOs are represented in \comp{ICOCC}. All the atom numbers of any
given LMO are contiguous.  Between each LMO is an unused space which can be
used if the LMO expands in \comp{DIAGG2}. Set in \comp{MLMO} or
\comp{PINOUT}, and modified in \comp{TIDY} and \comp{DIAGG2}.

The first atom in LMO $i$ is at address ICOCC($j$), $j$=NCF($i$)+1, and there
are NNCF($i$) atoms in that LMO.
\end{description}
\item 
\begin{description}
\item[(MOPAC) \comp{CIMAT:}] The configuration interaction matrix used 
in \comp{MECI}. (At least (LAB*(LAB+1))/2)
\item[(MOZYME) \comp{ICVIR:}] The set of atoms in each virtual M.O.  All the
virtual  LMOs are represented in \comp{ICVIR}. All the atom numbers of any
given LMO are contiguous.  Between each LMO is an unused space which can be
used if the LMO expands in \comp{DIAGG2}. Set in \comp{MLMO} or \comp{PINOUT},
and modified in \comp{TIDY} and \comp{DIAGG2}.

The first atom in LMO $i$ is at address ICVIR($j$), $j$=NCE($i$)+1, and there
are NNCE($i$) atoms in that LMO.
\end{description}
\item 
\begin{description}
\item[(MOPAC) \comp{C:}] The set of eigenvectors or M.O.s.  In  UHF
calculations, this would be the $\alpha$ set. (NORBS$^2$)
\item[\bf (MOZYME) \comp{COCC:}] The atomic orbital coefficients for the  set
of occupied LMOs.  All the coefficients of any given LMO are contiguous. 
Between each LMO is an unused space which can be used if the LMO expands in
\comp{DIAGG2}.

The contents of \comp{COCC} are modified in \comp{DIAGG2}, \comp{MAKVEC},
\comp{MLMO}, \comp{PINOUT}, and \comp{TIDY}, only.

For LMO $i$, the first coefficient is in COCC($j$), $j$=NCOCC($i$)+1.  The
number of coefficients depends on the atoms in the LMO, see \comp{ICOCC}.
\end{description}
\item 
\begin{description}
\item[(MOPAC) \comp{CB:}] In UHF work, the set of $\beta$ eigenvectors.
(NORBS$^2$)
\item[(MOZYME) \comp{CVIR:}] The atomic orbital coefficients for the  set of
virtual LMOs.  All the virtual  LMOs are represented in \comp{CVIR}.  All the
coefficients of any given LMO are contiguous.  Between each LMO is an unused
space which can be used if the LMO expands in \comp{DIAGG2}.

The contents of \comp{CVIR} are modified in \comp{DIAGG2}, \comp{MAKVEC},
\comp{MLMO}, \comp{PINOUT}, and \comp{TIDY}, only. For LMO $i$, the first
coefficient is in CVIR($j$), $j$=NVIR($i$)+1.  The number of coefficients
depends on the atoms in the LMO, see \comp{ICVIR}.
\end{description}
\item 
\begin{description}
\item[(MOPAC) \comp{DIJKL:}] Derivatives of two-electron integrals 
over M.O.s with respect to geometry.  Used by \comp{DERNVO}.
\item[(MOZYME) \comp{NBOND:}] The number of atoms bonded to each atom. (NUMAT)
\end{description}
\item \comp{IFACT/I1FACT:} FACT($i$)=$(i*(i-1))/2$, I1FACT($i$)=$(i*(i+1))/2$.
The array \comp{I1FACT} shares the same storage as \comp{IFACT}, but has a
starting address in \comp{MCORE} one more than \comp{IFACT}.  That is, the
starting address of \comp{IFACT} is $NCORE(31)$ and the starting address of
\comp{I1FACT} is  $NCORE(31)+1$.
\item 
\begin{description}
\item[(MOPAC) \comp{NPERMA:}] The $\alpha$ molecular orbital occupancy for the 
microstates in a C.I.\ calculation. (At least LAB) 
\item[(MOZYME) \comp{IBONDS:}] The atom numbers of the atoms attached to any
given atom.  For atom $i$, the number of attached atoms is NBONDS($i$), and
the  attached atoms are IBONDS(1:NBONDS($i$),$i$).
\end{description}
\item 
\begin{description}
\item[(MOPAC) \comp{NPERMB:}] The $\beta$ molecular orbital occupancy for the 
microstates in a C.I.\ calculation. (At least LAB) 
\item[(MOZYME) \comp{NFMO:}] This array is used in the annihilation routine 
\comp{DIAGG1}, only. \comp{NFMO} holds the number of filled M.O.s that each
virtual M.O.\ interacts with significantly. (NORBS)
\end{description}
\item \comp{TXTATM:} A short description of each atom.  This description can be
made in two ways.  In the data-set, a short description can be given in
parenthesis after each atom symbol.  For example ``H(on a COO)''.  The
description can be up to 13 characters long.

Alternatively, a useful description can be generated by use of \comp{RESIDUES}
or \comp{RESEQ}, and consists of 13 letters: `$nnnnn$~$RES$f$mmm$', where
`$nnnnn$' is the number of the atom, `$RES$' is the three-letter abbreviation
for the residue, `f' is a space or an asterisk, and `$mmm$' is the residue
number.  Explicit atom charges can also be specified in an atom label.  Thus a
cation would be specified by the presence of a `+' symbol in an atom label, and
an anion by the presence of a `--' symbol. Because both \comp{RESIDUES} and
\comp{RESEQ} create new labels, these keywords should not be used when explicit
charges are present.  Instead, the labels should be generated in one run, then
the explicit charges assigned.  Thereafter, \comp{RESIDUES} should not be used.

When printed, the description is enclosed in parentheses, for example
`(~3674~GLY~231)'.  Set in \comp{GETGEO} and \comp{NAMES}. See also
\hyperref[pageref]{``\comp{RESIDUES}''}{ on p.~}{}{res}. \comp{TXTATM}
is a character string of length 16 bytes, and is  contained in
\comp{CCORE}. (NATOMS)
\item 
\begin{description}
\item[(MOPAC) \comp{TOM:}] An array used by the Tomasi method.
\item[(MOZYME) \comp{ISORT:}] The order in which LMOs increase in energy. In
\comp{VALUES}, the energies of the LMOs are calculated.  The LMO number of the
lowest energy M.O.\ is then put in ISORT(1), then the number of the next higher
M.O.\ is put in ISORT(2), etc. (NORBS)
\end{description}
\addtocounter{enumi}{2} 
\item 
\begin{description}
\item[(MOPAC) \comp{EIGS:}] The eigenvalues of the M.O.s.  If UHF, then the
$\alpha$ eigenvalues. (NORBS)
\item[(MOZYME) \comp{EIGF/EIGV:}] The energy levels of the occupied and virtual
sets of LMOs. (NORBS)
\end{description}
\item \comp{EIGB:} In UHF work, the eigenvalues of the $\beta$ M.O.s (NORBS)
\item \comp{IOPT:} When \comp{RESIDUES} is specified, IOPT($i$) contains the
residue number to which atom $i$ belongs. If atom $i$ is a backbone atom, then
IOPT($i$) holds the {\em negative} of the residue number.  (In a residue, the
sequence of atoms is -NH-CH(R)-CO-, the backbone atoms are -NH-CH-CO- and the
side-chain atoms are those in the `R' group. The hydrogen atom at the -NH$_2$
end is considered part of backbone of residue 1, and the -OH group at the -COOH
end is considered part of the  backbone of the last residue.) Set in
\comp{PICOPT} only. (NUMAT)
\item \comp{JOPT:} The atoms used in a partial geometry optimization.  By
default, JOPT($i$)=$i$.  If \comp{RAPID} is used, then only those atoms that
move are in \comp{JOPT}.   This set is defined as the set of atoms to be
optimized plus all the atoms attached to the atoms that are to be optimized. 
Thus if CH$_3$-NH$_2$ were to be calculated, and the coordinate of the nitrogen
were to be  optimized, then \comp{JOPT} would contain all the atoms except the
hydrogens on the CH$_3$. If the atoms are in the order C, H, H, H, N, H, and H,
the \comp{JOPT} would be: 1, 5, 6, 7. The number of atoms defined in
\comp{JOPT} is \comp{NUMRED}. In the first SCF, all atoms are used. 
\comp{JOPT} only affects the subsequent SCFs.  \comp{JOPT} interacts with the
SCF in \comp{TIDY}. Set in \comp{PICOPT}.
\item \comp{OLDXYZ:} In \comp{RAPID} calculations, \comp{OLDXYZ} holds the
reference Cartesian coordinate gradients. The name should be read as `Old
DXYZ', not `Old XYZ'.   These are the gradients due to all  atoms {\em except}
those involved in the geometry optimization. ($3*$NUMAT)
\item \comp{DXYZ:} A two-dimensional array of derivatives of the energy with
respect to displacement in the $x$, $y$, and $z$ directions for each real
atom.  Units: kcal/mol/\AA ngstrom. \comp{DXYZ} is set in \comp{DCART}.
Except for systems with translational vectors, the size of \comp{DXYZ} is
$3\times $NUMAT.
\item 
\begin{description}
\item[(MOPAC) \comp{PA:}] The $\alpha$ density matrix. (MPACK)
\item[(MOZYME) \comp{PARTP:}] The partial density matrix due to all atoms that
do not move in a \comp{RAPID} calculation. (MPACK)
\end{description}
\item
\begin{description}
\item[(MOPAC) \comp{PB:}] In a UHF calculation, the $\beta$ density  matrix.
(MPACK)
\item[(MOZYME) \comp{PARTH:}] The partial one-electron matrix due to all atoms
that do not move in a \comp{RAPID} calculation.  (MPACK)
\end{description}
\item
\begin{description}
\item[(MOPAC) \comp{FB:}] In a UHF calculation, the Fock matrix. (MPACK)
\item[(MOZYME) \comp{PARTF:}] The partial Fock matrix due to all atoms that do
not move in a \comp{RAPID} calculation.  (MPACK)
\end{description}
\item \comp{KOPT:} In a \comp{RAPID} calculation,  \comp{KOPT} is the set of
atoms which are found in the LMOs used in an SCF. If the LMOs for all the atoms
are used in the SCF, then \comp{KOPT} is the full set.  If only the LMOs for a
few atoms are used (the number of atoms in \comp{JOPT} is small), then
\comp{KOPT} is the set of atoms in that sub-set of LMOs.  Note that the set in
\comp{KOPT} is larger than that in \comp{JOPT}, unless the set in \comp{JOPT}
is full.   Set in \comp{SETUPK}. (NUMAT)
\addtocounter{enumi}{1}
\item \comp{HESINV:} The inverse Hessian matrix in the BFGS method, and the
Hessian matrix in Baker's EigenFollowing \comp{EF} method. ((NVAR$*$(NVAR+1))/2)
\item \comp{JELEM:} In calculating symmetry properties, $JELEM(i,j)$ holds the
atom-number of the atom that $j$ would be moved to under operation 
$i$. ($20*$NUMAT)
\item \comp{ATMASS:} The atomic masses of the atoms.  Set by default, and can
be reset by the input data. (NUMAT)
\item \comp{REACT:} The values of the reaction coordinates in a reaction path
calculation.
\item \comp{C0:} An array used by the Tomasi method.
\item \comp{NC:} An array used by the Tomasi method.
\item \comp{INTERP:} A set of arrays used by the Camp-King SCF converger. 
These arrays can be eliminated by use of \comp{UNSAFE}. ($5*$NORBS$^2$)
\item \comp{ERRFN:} In RHF open-shell calculations, \comp{ERRFN} 
holds the difference
between the ``exact" (\comp{DERNVO}) derivatives and those calculated using 
\comp{DCART} only.  These differences are used in the geometry optimization, by
allowing the derivatives to be calculated using \comp{DCART}, and then applying
a correction.
\item \comp{AIDER:} {\em Ab initio} derivatives.  These are read in from the
data set in an attempt to improve the geometry.
\item \comp{ALPARM:}  In reaction path calculations, \comp{ALPARM} holds the
last three geometries.  These are used to extrapolate to the next geometry. 
($3*$NVAR)
\addtocounter{enumi}{1}
\item \comp{NAMO:} The symmetry labels of the eigenvectors (M.O.s, 
vibrations, or states). 
\item \comp{JNDEX:} The quantum numbers for the symmetry labels (The first
occurrence of symmetry label $X$ would have quantum number 1, the second: 2,
and so on.)
\item  \comp{IPO:} In a FORCE calculation, $IPO(i,j)$  holds the number of the
atom that atom $i$ would be moved to under operation $j$. ($120*$NUMAT)
\item \comp{DXYZR:} In RHF open-shell gradient calculations, \comp{DXYZR} holds
the gradient components due to the frozen core.
\item \comp{XPAREF:} Similar to \comp{XPARAM}.  \comp{XPAREF} is used to
temporarily store the geometric variables.
\item \comp{PROFIL:} The values of all heats of formation in a reaction
path calculation, in which the path coordinates are defined using
\comp{STEP=n.nn} and \comp{POINTS=m}.
\item \comp{SRAD:}  Used by the COSMO method. (NUMAT)
\addtocounter{enumi}{1}
\item \comp{DIRVEC:} Used by the COSMO method. (3246=3*1082)
\item \comp{BH:} Used by the COSMO method.  (LENABC)
\item \comp{IATSP:} Used by the COSMO method.  (LENABC+1)
\item \comp{NN:} Used by the COSMO method. ($3\times $NUMAT)
\item \comp{QDEN:} Used by the COSMO method. (LN61)
\item \comp{CH:} In electrostatic calculations, \comp{CH} hold atomic
charges. (NUMAT)
\item \comp{VECTCI:} In C.I. calculations, \comp{VECTCI} hold the state
vectors that would be used in calculating derivatives. (5 State Vectors)
\item \comp{TOM:} Used in the Tomasi method.
\addtocounter{enumi}{5}
\item \comp{AMAT:} Used by the COSMO method. (LENABC$^2$/2)
\item \comp{BMAT:} Used by the COSMO method. (LENABC$\times$ LM61)
\item \comp{CMAT:} Used by the COSMO method. ((LM61$\times$ (LM61+1))/2)
\item \comp{COSURF:} Used by the COSMO method. (LENABC+1)
\addtocounter{enumi}{12}
\item {\bf Reserved:} The starting address of a new temporary array is
placed in $NCORE(97)$.  This address must be used immediately, otherwise
it might be overwritten when a new temporary array is created.
\item {\bf Start of Integer:} The highest address in \comp{MCORE} available
for use in making temporary integer arrays. Any new temporary integer arrays
would be placed at the end of \comp{MCORE}.  At the start of the calculation,
$NCORE(98)$ is equal to the size of \comp{MCORE}; as temporary arrays are
created, $NCORE(98)$ decreases.
\item {\bf Start of Real:} The highest address in \comp{CCORE} available
for use in making temporary real arrays. Any new temporary real arrays
would be placed at the end of \comp{CCORE}.  At the start of the calculation,
$NCORE(99)$ is equal to the size of \comp{CCORE}; as temporary arrays are
created, $NCORE(99)$ decreases.
\item {\bf Not used}
\end{enumerate}

\subsection*{NASIZE(100)}
An array of size 100, \htmlref{\comp{NASIZE}}{nasize} holds the sizes of all the permanent arrays
that depend on the system being calculated. 
\begin{latexonly}
See p.~\pageref{nasize} for more detail.
\end{latexonly}
