\chapter{Description of Subroutines}
\index{Subroutines!description of|(}
\begin{description}
\item[\comp{AABABC}] Calculates the configuration interaction matrix element
between two configurations differing by exactly one alpha M.O.   Called by
\comp{MECIH} only.

\item[\comp{AABACD}] Calculates the configuration interaction matrix element
between two configurations differing by exactly two alpha M.O.s. Called by
\comp{MECIH} only.

\item[\comp{AABBCD}] Calculates the configuration interaction matrix element
between two configurations differing by exactly two M.O.s; one configuration
has alpha M.O.\ ``A'' and beta M.O.\ ``C'' while the other configuration has
alpha M.O.\ ``B'' and beta M.O.\ ``D''.  Called by \comp{MECIH} only.

\item[\comp{ADDFCK/ADDFCN}] Part of COSMO.

\item[\comp{ADDHB}] Hydrogen bonds and other weak bonds that  are not generated
automatically during the first few iterations of the first SCF are explicitly
constructed by \comp{ADDHB}.  A weak bond is presumed to exist if the scalar of
the Fock matrix elements connecting two atoms is greater than a preset limit
(1.0eV on iteration 3, 0.1 on iteration 6, 0.001 on iteration 9, and 0.0001 on
iteration 12).  If the scalar of the density matrix elements connecting the two
atoms is zero, then \comp{ADDHB} will create a weak bond between the two atoms.

\item[\comp{ADDHCR}] Part of COSMO, \comp{ADDHCR} adds in the effect of the
solvent  to the one-electron or core matrix, \comp{H}.

\item[\comp{ADDNUC}] Part of COSMO, \comp{ADDNUC} adds in the effect of the
solvent  to the core-core, or nuclear, energy, \comp{ENUCLR}.

\item[\comp{ADJVEC}] Two almost orthogonal LMOs \htmlref{are made
orthogonal}{reorth} by mixing them together.  
\begin{latexonly}
See p.~\pageref{reorth} for details.
\end{latexonly}

\item[\comp{AIJM}] Part of MNDO-$d$, \comp{AIJM} calculates the A$_{ij}$ values
used in the evaluation of two-center two-electron integrals.

\item[\comp{AINTGS}] Within the overlap integrals, \htmlref{calculates
the A-integrals}{ab}.  Dedicated to function \comp{SS} within
\comp{DIAT}
\begin{latexonly}
, see \pageref{ab} for more
details
\end{latexonly}.

\item[\comp{ALPHAF}] Part of \comp{POLAR}, \comp{ALPHAF} calculates the
frequency dependent response matrices $U_A$ and densities $D_A$.  It also is
used in computing the frequency-dependent polarizability and in solving the
second-order problem.

\item[\comp{ANALYT}] Calculates the analytical derivatives of the energy with
respect to Cartesian coordinates for all atoms. Use only if the mantissa is
short (less than 52 bits) or out of interest.  Should not be used for routine
work.

\item[\comp{ANAVIB}] Gives a brief interpretation of the modes of vibration of
the molecule. The principal pairs of atoms  involved in each vibration are
identified, and the mode of motion (tangential or radial) is output.

\item[\comp{ANSUDE}] Part of COSMO.  Calculates the area of two intersecting 
spheres.

\item[\comp{AROM}] \comp{AROM} determines if the $\pi$ bond between two atoms
is part of an aromatic ring.  It does this by checking to see if two other
$\pi$ bonds already exist in the ring.  If they do, then the two atoms form the
third and final $\pi$ bond. Used in generating Lewis structures for aromatic
and graphitic structures.

\item[\comp{AROM2}] Determines if a $\pi$ bond between two atoms is part of an
aromatic ring.  If a $\pi$ bond already exists in a ring, then a check is done
to see if a third $\pi$ bond could be made later on.  If it can, then the two
atoms are part of an aromatic ring. Used in generating Lewis structures for
aromatic and graphitic structures.

\item[\comp{ASUM}] Part of the Greens Function code.

\item[\comp{ATOMRS}] All atoms in a given residue, residue $N$, in a protein
are identified.  In the array \comp{LUSED}, backbone atoms are set to -$N$, and
side-chain atoms are set to $N$.

\item[\comp{AVAMEM}] If a run requires more memory than is available, then the
job cannot continue.  \comp{AVAMEM} will print out the maximum amount of memory
that the computer can provide.

\item[\comp{AXIS}] Works out the three principal moments of inertia of a
molecule. If the system is linear, one moment of inertia is zero. Prints
moments in units of cm$^{-1}$ and $10^{-40}$ g cm$^2$.

\item[\comp{BABBBC}] Calculates the configuration interaction matrix element
between two configurations differing by exactly one beta M.O.  Called by
\comp{MECI} only.

\item[\comp{BABBCD}] Calculates the configuration interaction matrix element
between two configurations differing by exactly two beta M.O.s. Called by
\comp{MECI} only.

\item[\comp{BANGLE}] Given a set of coordinates, \comp{BANGLE} will calculate
the angle between any three atoms.

\item[\comp{BDENIN}] Part of \comp{POLAR}. 

\item[\comp{BDENUP}] Part of \comp{POLAR}. 

\item[\comp{BEOPOR}] Part of \comp{POLAR}, it iteratively evaluates the 
$\beta$ value used in the electrooptic Pockel's effect and optical
rectification.

\item[\comp{BETAF}] Part of \comp{POLAR}, it iteratively evaluates  the $\beta$
value used in second harmonic generation.

\item[\comp{BETAL1}]  Part of \comp{POLAR}. 

\item[\comp{BETALL}] Part of \comp{POLAR}. 

\item[\comp{BETCOM}] Part of \comp{POLAR}.

\item[\comp{BFN}] Calculates the $B$-functions in the Slater overlap.

\item[\comp{BINTGS}] \htmlref{Calculates the $B$-integrals}{ab} in the Slater overlap
\begin{latexonly}
, see
p.~\pageref{ab} for details
\end{latexonly}.

\item[\comp{BLDSYM}] Part of the point-group symmetry package.  \comp{BLDSYM}  
constructs the point-group symmetry operations as 3 by 3 matrices.
\item[\comp{BMAKUF}] Part of \comp{POLAR}.

\item[\comp{BONDN/BONDS}] Evaluates and prints the valencies of atoms and 
bond-orders between atoms. Main argument: density matrix. No results are passed
to the calculation, and no data  are changed. Called by \comp{WRITMN} only.

\item[\comp{BONDN/BONDSZ}] Evaluates and prints the valencies of atoms and 
bond-orders between atoms calculated by \comp{MOZYME}. \ Main  argument:
density matrix.  No results are passed to the calculation, and no data  are
changed. Called by \comp{WRITMN} only.

\item[\comp{BPRINT}] Part of the Tomasi Solvation method.

\item[\comp{BRLZON/BRLZNN}] \comp{BRLZON} generates a band structure,  or
phonon structure, for high polymers.  Called by \comp{WRITMO} and
\comp{FREQCY}.

\item[\comp{BUILDF/BUILDN}] Constructs a new Fock matrix from the one-electron
matrix, the density matrix, and the two-electron integrals.

\item[\comp{CADIMA}] Part of the Tomasi Solvation method.

\item[\comp{CALPAR}] When external parameters are read in via \comp{EXTERNAL=},
the derived parameters are worked out using \comp{CALPAR}.  Note that all
derived parameters are calculated for all parameterized elements at the same
time.

\item[\comp{CANON}] In a MOZYME calculation, if \comp{EIGEN} is {\em not} 
present, then \comp{CANON} converts the way LMOs are stored into the way M.O.s
are stored in MOPAC: that is, it forms a square array of orthonormal vectors.
It also creates a MOPAC-type lower half triangle of the one-electron integrals.

\item[\comp{CAPCOR}] Capping atoms, of type Cb, should not contribute to  the
energy of a system.  \comp{CAPCOR} calculates the energy  contribution due to
the Cb and subtracts it from the  electronic energy.

\item[\comp{CARTAB}] Constructs the point-group character tables. Called by
\comp{MOLSYM} only.

\item[\comp{CCPROD}] Part of Greens Function.

\item[\comp{CCREP}] Calculates the core-core repulsion in MNDO-$d$.

\item[\comp{CDIAG}] Complex diagonalization.  Used in generating eigenvalues of
complex Hermitian secular determinant for band structures. Called by
\comp{BRLZON} only.

\item[\comp{CHARMO}] Calculates the character of an operation acting on a
molecular orbital, i.e., $<\psi|R_i|\psi>$.  Called by \comp{SYMOIR} only.

\item[\comp{CHARST}] Calculates the character of an operation acting on a state
function, i.e., $<\Psi|R_i|\Psi>$.  Called by \comp{SYMOIR} only.

\item[\comp{CHARVI}] Calculates the character of an operation acting on a
normal coordinate, i.e.\ an eigenvector of a vibration. 

\item[\comp{CHECK}] Re-normalized the LMOs after \comp{DIAGG2} has annihilated
energy matrix terms.

\item[\comp{CHI}] Determines the existence of specific point-group operators.
CHI returns a `1' in IELEM($IOPER$) if the operation $IOER$ is present,
otherwise IELEM($IOPER$)=0.

\item[\comp{CHKION}] Identifies all remaining ionized atoms in a system.
Examples: metal ions, halide ions, carbanions and carbocations.

\item[\comp{CHKLEW}] Identifies most of the elements of the Lewis structure,
such as $\sigma$ bonds, lone pairs, $\pi$ bonds, and some ions, such as
-NH$_3^+$.

\item[\comp{CHRGE/CHRGN}] Calculates the total number of valence electrons on
each atom. Main arguments: density matrix, array of atom charges (empty on
input). Called by \comp{ITER} only.

\item[\comp{CHRGE/CHRGEZ}] Calculates the total number of valence electrons on
each atom. Main arguments: density matrix, array of atom charges (empty on
input). Called by \comp{ITER} only.

\item[\comp{CIINT}] Part of COSMO.

\item[\comp{CIOSCI}] Calculates the electronic transition oscillator strength
in  \comp{MECI}.

\item[\comp{CNVG/CNVGZ}] Used in SCF cycle. \comp{CNVG} does a three-point
interpolation of the last three density matrices.  Arguments: Last three
density matrices, Number of iterations, measure of self-consistency (empty on
input). Called by \comp{ITER} only.

\item[\comp{COE}] Within the general overlap routine \comp{COE} calculates  the
angular coefficients for the $s$, $p$ and $d$ real atomic orbitals given the
axis and returns the rotation matrix.

\item[\comp{COLLID}] Part of the ESP package.

\item[\comp{COLLIS}] Part of the PMEP package.

\item[\comp{COLLIT}] Part of the MST solvation package.

\item[\comp{COMMOP}] The conventional matrix algebra SCF and gradient
calculation.

\item[\comp{COMMOZ}] The localized molecular orbital SCF and gradient
calculation.

\item[\comp{COMPCT}]] (COMPACT) removes unused space from between LMOs. During
the selection of LMOs to be used in the SCF, the `active' LMOs are put at the
start of the appropriate arrays.  If there is insufficient storage to hold all
the LMOs selected, then extra storage is created by compacting together the
other LMOs, thus removing the gaps caused when the active LMOs were moved.

\item[\comp{COMPFG}] Evaluates the total heat of formation of the supplied
geometry and, if requested,  the derivatives. This is the nodal point
connecting the electronic and geometric parts of the program. Main arguments:
on input: geometry, on output: heat of formation, gradients.

\item[\comp{COPY1}] At the start of a calculation, \comp{COPY1} copies  INTEGER
data from the COMMON arrays to \comp{MCORE}.

\item[\comp{COPY2}] At the start of a calculation, \comp{COPY1}  copies REAL
data from the COMMON arrays to \comp{CCORE}.

\item[\comp{COPYM}] Part of \comp{POLAR}.

\item[\comp{COSCAV/COSCAN}] Part of \comp{COSMO}.

\item[\comp{COSCL1/COSCL2}] Part of \comp{COSMO}.

\item[\comp{COSINI}] Part of \comp{COSMO}.

\item[\comp{COUL}] Part of the Tomasi Solvation method.

\item[\comp{CSUM}] Part of Greens Function.

\item[\comp{DANG}] Called by \comp{XYZINT}, \comp{DANG} computes the  angle
between a point, the origin, and a second point.

\item[\comp{DAREA1}] Part of \comp{POLAR}.

\item[\comp{DAREAD}] Part of \comp{POLAR}.

\item[\comp{DASUM}] A BLAS routine.

\item[\comp{DATIN}] Reads in external parameters for use within MOPAC. 
Originally used for the testing of new parameters, \comp{DATIN} is now a
general purpose reader for parameters. Invoked by the keyword \comp{EXTERNAL}.

\item[\comp{DAWRIT}] Part of \comp{POLAR}.

\item[\comp{DAWRT1}] Part of \comp{POLAR}.

\item[\comp{DAXPY}] A BLAS routine.

\item[\comp{DCART/DCARN}] Called by \comp{DERIV}, \comp{DCART} sets up a  list
of Cartesian derivatives of the energy with respect to coordinates which 
\comp{DERIV} can then use to construct the internal coordinate derivatives.

\item[\comp{DCART/DCARNZ}] Called by \comp{DERIV}, \comp{DCART}  sets up a list
of Cartesian derivatives of the energy with respect to coordinates  which
\comp{DERIV} can then use to construct the internal coordinate derivatives.

\item[\comp{DCOPY}] A BLAS routine. 

\item[\comp{DDOT}] A BLAS routine.

\item[\comp{DDPO}] Part of MNDO-$d$.

\item[\comp{DELMOL}] Part of analytical derivatives. Two-electron.

\item[\comp{DELNEW}] Part of intersystem crossing method.

\item[\comp{DELRI}] Part of analytical derivatives. Two-electron.

\item[\comp{DELSTA}] Calculates the derivative of the energy with respect to
coordinates, for a pair of atoms that have electrostatic interactions only.

\item[\comp{DENROT/DENROZ}] Converts the ordinary density matrix into  a
condensed density matrix over basis functions $s-\sigma$,  $p-\sigma$ and
$p-\pi$, i.e., three basis functions. Useful  in hybridization studies. Has
capability to handling $d$  functions, if present.

\item[\comp{DENSF}] Part of \comp{POLAR}.

\item[\comp{DENSIT}] Constructs the Coulson electron density matrix from the
eigenvectors. Main arguments: Eigenvectors, number of singly and doubly
occupied levels, density matrix (empty on input) Called by \comp{ITER}.

\item[\comp{DENSIZ}] Constructs the Coulson electron density matrix from the
LMOs. Main arguments: LMOs, number of doubly occupied levels,  density matrix
(Partially filled or empty on input) Called by \comp{ITENZ}.

\item[\comp{DERI0}] Part of the analytical C.I.\ derivative package. 
Calculates the diagonal dominant part of the super-matrix.

\item[\comp{DERI1/DERN1}] Part of the analytical C.I.\ derivative package. 
Calculates the frozen density contribution to the derivative of  the energy
with respect to Cartesian coordinates, and the derivatives of  the frozen Fock
matrix in M.O.\ basis. Its partner is \comp{DERI2}.

\item[\comp{DERI2/DERN2}] Part of the analytical C.I.\ derivative package. 
Calculates the relaxing density contribution to the derivative of the energy
with respect to Cartesian coordinates. Uses the results of \comp{DERI1}.

\item[\comp{DERI21}] Part of the analytical C.I.\ derivative package.  Called
by \comp{DERI2} only.

\item[\comp{DERI22}] Part of the analytical C.I.\ derivative package.  Called
by \comp{DERI2} only.

\item[\comp{DERI23}] Part of the analytical C.I.\ derivative package.  Called
by \comp{DERI2} only.

\item[\comp{DERITR}] Calculates derivatives of the energy with respect to
internal coordinates using full SCFs.  Used as a foolproof way of calculating
derivatives.  Not recommended for normal use.

\item[\comp{DERIV/DERIN}] Calculates the derivatives of the energy with respect
to the geometric variables.  This is done either by  using initially Cartesian
derivatives (normal mode), by  analytical C.I.\ RHF derivatives, or by full SCF
calculations  (\comp{NOANCI} in half-electron and C.I.\ mode). Arguments: on 
input: geometry, on output: derivatives. Called by \comp{COMMOP/COMMOZ}.

\item[\comp{DERNVO/DERNVN}] Analytical C.I.\ Derivative main subroutine. 
Calculates the derivative of the energy with respect to geometry for a
non-variationally optimized wavefunction (a SCF-CI wavefunction).

\item[\comp{DERP}] Analytical derivative of diatomic energy with respect to
distance in a polymer or other infinite system, in the point-charge region.

\item[\comp{DERS}] Called by \comp{ANALYT}, \comp{DERS} calculates the
analytical derivatives of the overlap matrix within the molecular frame.

\item[\comp{DEX2}] A function called by \comp{ESP}.

\item[\comp{DFIELD/DFIENZ/DFIELZ}] Calculates the derivative  of the energy
with respect to an applied external electric field.

\item[\comp{DFOCK2}] Part of the analytical C.I.\ derivative package. Called by
\comp{DERI1}, \comp{DFOCK2} calculates the frozen density contribution to  the
derivative of the energy with respect to Cartesian coordinates.

\item[\comp{DFPSAV/DFPSAN}] Saves and restores data used by the  BFGS geometry
optimization. Main arguments: parameters being optimized, gradients of
parameters, last heat  of formation, integer and real control data. Called by
\comp{FLEPO}.

\item[\comp{DGEDI}] Part of EF.

\item[\comp{DGEFA}] Part of EF.

\item[\comp{DGEMM}] Part of COSMO.

\item[\comp{DGER}] Part of COSMO.

\item[\comp{DGESV}] Part of COSMO.

\item[\comp{DGETF2}] Part of COSMO.

\item[\comp{DGETRF}] Part of COSMO.

\item[\comp{DGETRS}] Part of COSMO.

\item[\comp{DHC}] Called by \comp{DCART} and calculates the energy of a pair of
atoms using the SCF density matrix.  Used in the finite difference derivative
calculation.

\item[\comp{DHCORE}] Part of the analytical C.I.\ derivative package. Called by
\comp{DERI1}, \comp{DHCORE} calculates the derivatives of the 1 and 2  electron
integrals with respect to Cartesian coordinates.

\item[\comp{DIAG}] Rapid pseudo-diagonalization. Given a set of vectors which
almost block-diagonalize a secular determinant, \comp{DIAG} modifies the
vectors so that the block-dia\-gon\-al\-iz\-at\-ion is more exact. Main
arguments: Old vectors, Secular Determinant,  New vectors (on output).  Called
by \comp{ITEN}.

\item[\comp{DIAGG}] The occupied - virtual LMO energy matrix diagonalizer. By
doing a 2 $\times$ 2 rotation of the LMOs, the associated energy term is
annihilated.

\item[\comp{DIAGG1}] Constructs the occupied - virtual LMO energy term.

\item[\comp{DIAGG2}] Annihilates the  occupied - virtual LMO energy term.

\item[\comp{DIAGI}] Calculates the electronic energy arising from a given
configuration. Called by MECI.

\item[\comp{DIAT}] Calculates overlap integrals between two atoms in  general
Cartesian space. Principal quantum numbers up to 6, and angular quantum numbers
up to 2 are allowed. Main arguments: Atomic numbers and Cartesian coordinates
in \AA ngstroms of the  two atoms, Diatomic overlaps (on exit). Called by
H1ELEC.

\item[\comp{DIAT2}] Calculates reduced overlap integrals between atoms  of
principal quantum numbers 1, 2, and 3, for $s$ and $p$ orbitals. Faster than
the SS in DIAT. This is a dedicated subroutine, and is unable to stand alone
without considerable backup. Called by DIAT.

\item[\comp{DIEGRD}] Part of COSMO.

\item[\comp{DIGIT}] Part of READA.  DIGIT assembles numbers given a  character
string.

\item[\comp{DIHED}] Calculates the dihedral angle between four atoms. Used in
converting from Cartesian to internal coordinates.

\item[\comp{DIJKL1}] Part of the analytical C.I.\ derivative package. Called by
DERI1, DIJKL1 calculates the two-electron integrals over M.O.\ bases, e.g.\
$<$i,j(1/r)k,l$>$.

\item[\comp{DIJKL2}] Part of the analytical C.I.\ derivative package. Called by
DERI2, DIJKL2 calculates the derivatives of the two-electron  integrals over
M.O.\ bases, e.g.\ $<$i,j(1/r)k,l$>$, wrt Cartesian coordinates.

\item[\comp{DIMENS}] Calculates and prints the size of the molecule. ``Size''
is defined as follows: The first dimension is the greatest distance across the
system, the second dimension is the greatest distance across the molecule in
the plane perpendicular to the first dimension. The third dimension is the
greatest distance in the line perpendicular to the plane of the first two
dimensions.

\item[\comp{DIPIND}] Similar to DIPOLE, but used by the POLAR calculation 
only.

\item[\comp{DIPOLE/DIPOLN/DIPOLZ}] Evaluates and, if requested, prints  dipole
components and dipole for the molecule or ion. Arguments: Density matrix, 
Charges on every atom, coordinates, dipoles (on exit). 

The dipole is calculated in two steps:
\begin{enumerate}
\item Using the charge on each atom and the geometry, the dipole arising from
point  charges is evaluated.

\item The contribution to the dipole arising from lone pairs is then added in. 
This is  given by the $s-p$ contributions to the density matrix.  All other
terms are ignored. Thus, the $s-d$ terms do not contribute to the dipole, and
very few, if any,  compounds have non-zero $p-d$ terms in the density matrix.
Called by WRITMO and FMAT.
\end{enumerate}

\item[\comp{DLASWP}] Part of COSMO.

\item[\comp{DMECIP}] Used in analytical C.I.\ derivatives, \comp{DMECIP}
calculates the change in density matrix due to MECI.

\item[\comp{DNRM2}] A BLAS routine.

\item[\comp{DOCK}] Rotates and translates one of the geometries in a 
\comp{SADDLE} calculation so that the root mean square difference between  the
two geometries is a minimum.

\item[\comp{DOFS}] Calculates the density of states within a  Brillouin zone. 
Used in polymer work only.

\item[\comp{DOPEN}] Modifies the density matrix in BONDS to account for ROHF
open shell populations.

\item[\comp{DOPRO}] Part of intersystem crossing method.

\item[\comp{DOT}] Given two vectors, X and Y, of length N, function DOT returns
with the dot product X.Y. I.e., if X=Y, then DOT = X$^2$ Called by FLEPO.

\item[\comp{DPRO}] Part of intersystem crossing method.

\item[\comp{DRC/DRN}] The dynamic and intrinsic reaction coordinates  are
calculated by following the mass-weighted trajectories.

\item[\comp{DRCOUT}] Sets up DRC and IRC data in quadratic form  preparatory to
being printed.

\item[\comp{DREPP2}] Part of the PMEP.

\item[\comp{DROTAT}] Part of the PMEP.

\item[\comp{DSCAL}] A BLAS routine.

\item[\comp{DSUM}] Part of Greens Function.

\item[\comp{DSWAP}] Part of Greens Function.

\item[\comp{DTRANS/DTRAN2}] Performs a point-group operation on a $d$-shell.
  Called by CDIAG.

\item[\comp{DTRSM}] Part of COSMO.

\item[\comp{DUMMY}] Does nothing.  Called at the start of a run, \comp{DUMMY}
is used in a \comp{FTCHECK} of the program, to reduce the number of apparent
errors.  Specifically, when several arrays in COMMON blocks are used in the
program as a single array, \comp{FTCHECK} will flag that as an error.
\comp{DUMMY} will deceive \comp{FTCHECK} into not flagging the error.

\item[\comp{DVFILL}] Part of COSMO.

\item[\comp{EA08C}] Part of CDIAG

\item[\comp{EA09C}] Part of CDIAG

\item[\comp{EC08C}] Part of CDIAG

\item[\comp{EF/EN}] EF is the Eigenvector Following routine.   EF implements
the keywords EF and TS.

\item[\comp{EFSAV}] Saves and restores data used by the Eigenvector Following
subroutine.  Called by EF only.

\item[\comp{EFSTR}] Within EF, EFSTR reads in all data needed by EF.

\item[\comp{EIGEN/EIGENN}] Calculates the canonical M.O.s or eigenvectors.  It 
does this by diagonalizing the SCF Fock matrix.

\item[\comp{EIMP}] Calculates the scalar of the Fock matrix terms between
atoms. The result is put into the $s$-$s$ elements of \comp{P}.

\item[\comp{EINVIT}] Part of the diagonalizer.

\item[\comp{EISCOR}] Part of MNDO-$d$.

\item[\comp{ELAU}] Part of the diagonalizer.

\item[\comp{ELENUC}] Part of MNDO-$d$.

\item[\comp{ELESP/ELESN}] Within the ESP, ELESP calculates the electronic 
contribution to the electrostatic potential.

\item[\comp{EMPIRI}] Calculates and prints the chemical empirical formula.

\item[\comp{ENPART}] Partitions the energy of a molecule into its monatomic and
diatomic components. Called by WRITMO when the keyword  ENPART is specified. No
data are changed by this call.

\item[\comp{EPSAB}] Part of POLAR.

\item[\comp{EPSETA}] Calculates the machine precision and dynamic range for use
by the two diagonalizers.

\item[\comp{EPSLON}] Part of diagonalizer.

\item[\comp{EQLRAT}] Part of diagonalizer.

\item[\comp{ERRION}] If there are a large number of ions in the system -
probably as a result of errors in the data set - then \comp{ERRION} will print
out the ionized atoms.   Data sets with a large number of ions should be
regarded with suspicion!

\item[\comp{ERROR}] Part of the Tomasi Solvation method.

\item[\comp{ESP/ESN/}] ESP calculates the atomic charges which would reproduce
the electrostatic potential of the nuclei and electronic wavefunction.

\item[\comp{ESPFIT}] Part of the ESP.  ESP fits the quantum mechanical
potential to a classical point charge model

\item[\comp{ETRBK3}] Part of diagonalizer.

\item[\comp{ETRED3}] Part of diagonalizer.

\item[\comp{EVVRSP}] Part of diagonalizer.

\item[\comp{EXCHNG}] Dedicated procedure for storing 3 parameters and one array
in a store. Used by SEARCH.

\item[\comp{FBX}] Part of MNDO-$d$.

\item[\comp{FCNPP}] Part of Greens Function.

\item[\comp{FFHPOL}] Part of the POLAR calculation.  Evaluates the  effect of
an electric field on a molecule.

\item[\comp{FFREQ1}] Part of POLAR.

\item[\comp{FFREQ2}] Part of POLAR.

\item[\comp{FHPATN}] Part of POLAR.

\item[\comp{FILLC}] Part of the Tomasi Solvation method.

\item[\comp{FILLIJ/FILLIN}] Works out the various cutoffs, and fills the 
MOZYME matrix \comp{IJBO}.  Works out the sizes of the \comp{H} and \comp{W}
matrices.

\item[\comp{FILMAT}] Part of the Tomasi model.

\item[\comp{FINDN1}] In a protein, \comp{FINDN1} finds the atom number of the
first NH$_2$.    

\item[\comp{FINISH}] Causes a safe and orderly shutdowm of MOPAC. \ Deletes 
temporary or scratch files.

\item[\comp{FLEPO/FLEPN}] Optimizes a geometry by minimizing the energy. Makes
use of the first and estimated second derivatives to achieve this end. 
Arguments: Parameters to be optimized,  (overwritten on exit with the optimized
parameters), Number of  parameters, final optimized heat of formation. Called
by RMOPAC,  REACT1, and FORCE.

\item[\comp{FLUSHM}] Flushes the buffers.  By default, the output of a
calculation is not immediately available while the job is running. 
\comp{FLUSHM} will temporarily close and re-open the output file, so that the
results can be seen while the job is running.

\item[\comp{FMAT}] Calculates the exact Hessian matrix for a system This is
done by either using differences of first derivatives  (normal mode) or by four
full SCF calculations (half electron  or C.I.\ mode). Called by FORCE.

\item[\comp{FOCD2Z}] Adds on to Fock matrix the two-center two electron terms
arising from MNDO-$d$. Called by \comp{FOCK2Z} only.

\item[\comp{FOCK1}] Adds on to Fock matrix the one-center two electron  terms.
Called by \comp{ITEN} only.

\item[\comp{FOCK1Z}] Adds on to Fock matrix the one-center two electron  terms.
Called by ITENZ only.

\item[\comp{FOCK2/FOCK2N/FOCK2Z}] Adds on to Fock matrix the two-center  two
electron  terms. Called by \comp{ITEN} or \comp{ITENZ} and \comp{DERIV}.  In
\comp{ITEN} and \comp{ITENZ} the  entire Fock matrix is filled; in
\comp{DERIV}, only diatomic Fock matrices are  constructed.

\item[\comp{FOCKD2}] Adds on to Fock matrix the two-center two electron terms
arising from MNDO-$d$. Called by \comp{FOCK2} only.

\item[\comp{FORCE/FORCN}] Performs a force-constant and vibrational  frequency
calculation on a given system. If the starting  gradients are large, the
geometry is optimized to reduce the  gradient norm, unless \comp{LET} is
specified in the keywords.  Isotopic substitution is allowed. Thermochemical
quantities  are calculated. Called by \comp{RMOPAC}.

\item[\comp{FORDD}] Part of MNDO-$d$

\item[\comp{FORMD}] Called by \comp{EF}. \ \comp{FORMD} constructs the next
step in the geometry optimization or transition state location.

\item[\comp{FORMXY}] Part of \comp{DIJKL1}. \comp{FORMXY} constructs part of 
the two-electron integral over M.O.s.

\item[\comp{FORSAV}] Saves and restores data used in \comp{FMAT} in
\comp{FORCE}  calculation. Called by \comp{FMAT}.

\item[\comp{FRAME}] Applies a very rigid constraint on the translations and
rotations of the system. Used to separate the trivial vibrations in a
\comp{FORCE} calculation.

\item[\comp{FREDA}] Part of diagonalizer.

\item[\comp{FREQCY}] Final stage of a \comp{FORCE} calculation. Evaluates and
prints the vibrational frequencies and modes.

\item[\comp{FSUB}] Part of \comp{ESP}.

\item[\comp{GATHER\_*}] Subroutines used in multiprocessor calculations.

\item[\comp{GENUN}] Part of \comp{ESP}. Generates unit vectors over a sphere.
called by \comp{SURFAC} only.

\item[\comp{GEOUT}] Prints out the current geometry. Can be called at  any
time. Does not change any data.

\item[\comp{GENVEC}] Creates the Connolly surface unit vectors on a sphere,
part of \comp{PMEP}.

\item[\comp{GEOCHK/GEOCHN}] Checks the supplied geometry to see of it conforms
to the rules for the type of system which can be run.

\item[\comp{GEOUT/GEOUN}] Prints out the current geometry. \comp{GEOUT} can be
called at any time. Does not change any data.

\item[\comp{GEOUTG/GEOUTN}] Prints out the current geometry in Gaussian
Z-matrix format.

\item[\comp{GET2C}] Part of the Tomasi Solvation method.

\item[\comp{GET3C}] Part of the Tomasi Solvation method.

\item[\comp{GETA1}] Part of the Tomasi Solvation method.

\item[\comp{GETCC1}] Part of the Tomasi Solvation method.

\item[\comp{GETDAT}] Reads in all the data, and puts it in a scratch file on
channel 5.

\item[\comp{GETGEG}] Reads in Gaussian Z-matrix geometry. Equivalent to
\comp{GETGEO} and \comp{GETSYM} combined.

\item[\comp{GETGEO}] Reads in geometry in character mode from specified
channel, and stores parameters in arrays. Some error-checking  is done. Called
by \comp{READMO} and \comp{REACT1}.

\item[\comp{GETHES}] Constructs the Hessian matrix in \comp{EF}.

\item[\comp{GETPDB}] Reads in the Z-matrix using Protein Data Base format.

\item[\comp{GETSYM}] Reads in symmetry data. Used by \comp{READMO}.

\item[\comp{GETTXT}] Reads in \comp{KEYWRD}, \comp{KOMENT} and \comp{TITLE}.

\item[\comp{GETVAL}] Called by \comp{GETGEG}, \comp{GETVAL} either gets an
internal coordinate or a logical name for that coordinate.

\item[\comp{GMETRY/GMETRN}] Fills the Cartesian coordinates array. Data are 
supplied from the array \comp{GEO}, \comp{GEO} can be (a) in internal 
coordinates, or (b) in Cartesian coordinates. If \comp{STEP} is  non-zero, then
the coordinates are modified in light of the  other geometry and \comp{STEP}.
Called by \comp{HCORE},  \comp{DERIV}, \comp{READMO}, \comp{WRITMO}, 
\comp{MOLDAT}, etc.

\item[\comp{GOVER}] Calculates the overlap of two Slater orbitals which have
been expanded into six gaussians.  Calculates the STP-6G overlap integrals.

\item[\comp{GREEK}] Creates the greek subscripts for atoms, e.g., C$_{\alpha}$,
for use in PDB format.

\item[\comp{GREENF}] Corrects ionization potentials using Greens' Function
method.

\item[\comp{GRID}] Calculates a grid of points for a 2-D search in coordinate
space.  Useful when more information is needed about a reaction surface.

\item[\comp{GRIDS}] Constructs the Williams surface in \comp{PMEP}.

\item[\comp{GSTORE}] Part of Greens Function.

\item[\comp{H1ELEC/H1ELEZ}] Given any two atoms in Cartesian space,
\comp{H1ELEC}  calculates the one-electron energies of the off-diagonal 
elements of the atomic orbital matrix.  $$ H_{\lambda\sigma} =
-S_{\lambda\sigma}(\beta_{\lambda}+\beta_{\sigma})/2 $$ Called by \comp{HCORE}
or \comp{HCORZ} and \comp{DERIV}.

\item[\comp{HADDON}] The symmetry operation subroutine, \comp{HADDON} relates
two geometric variables by making one a dependent function of the  other.
Called by \comp{SYMTRY} only.

\item[\comp{HBONDS}] Identifies the hydrogen bonds (long-range weak covalent
interactions) which have not been created during the first few iterations of
the first SCF.

\item[\comp{HCORE/HCORN/HCORZ}] Sets up the energy terms used in calculating
the SCF heat of formation. Calculates the one and two electron matrices, and
the nuclear energy. Called by \comp{COMMOP/COMMOZ}.

\item[\comp{HELECT/HELECZ}] Given the density matrix, and the one electron and
Fock matrices, calculates the electronic energy. No data are  changed by a call
of \comp{HELECT}. Called by \comp{ITEN} or  \comp{ITENZ} and \comp{DERIV}.

\item[\comp{HESINI}] Part of intersystem crossing method, calculates the
inverse Hessian matrix.

\item[\comp{HESINI}] Part of intersystem crossing method.

\item[\comp{HMUF}] Part of \comp{POLAR}.

\item[\comp{HPLUSF}] Part of \comp{POLAR}.

\item[\comp{HXVEC}] Part of intersystem crossing method.

\item[\comp{HYBRID}] Constructs the starting localized monatomic hybrid atomic
orbitals.

\item[\comp{IJKL}] Fills the large two-electron array over a M.O.\ basis set.
Called by \comp{MECI}.

\item[\comp{INID}] Part of MNDO-$d$, it initializes various arrays used by the
integral package.

\item[\comp{INIGHD}] Part of MNDO-$d$, it evaluates the one-center 
two-electron integrals.

\item[\comp{INSYMC}] Part of Greens Function.

\item[\comp{INTERP/INTERN}] Runs the Camp-King converger. q.v.

\item[\comp{INTFC}] In a \comp{FORCE} calculation, \comp{INTFC} calculates the
force constants for the supplied coordinate system.  If these are internal,
then internal force constants will be calculated.  The force constants will be
printed.

\item[\comp{IONOUT}] Prints all ions found by \comp{GEOCHK}.

\item[\comp{ISITSC}] (IS IT Self Consistent) Determines if the system has
satisfied the self-consistency criteria.

\item[\comp{ITER/ITEN/ITERZ/ITENZ}] Given the one and two electron matrices,
ITER calculates the Fock and density matrices, and the electronic energy.
Called by \comp{COMMOP/COMMOZ}.

\item[\comp{JAB}] Calculates the coulomb contribution to the Fock matrix in
NDDO formalism. Called by \comp{FOCK2}.

\item[\comp{JCARIN}] Calculates the difference vector in Cartesian coordinates
corresponding to a small change in internal coordinates.

\item[\comp{KAB}] Calculates the exchange contribution to the Fock matrix in
NDDO formalism. Called by FOCK2.

\item[\comp{LDIMA/LDIMN}] Part of the Tomasi Solvation method.

\item[\comp{LEWIS}] Works out the connectivity, that is, which atoms are bonded
together.  It does this using the interatomic distance and the van der Waals
radii.

\item[\comp{LIGAND}] Makes labels for atoms which are not part of the protein
itself; mainly small molecules or ions such as phosphate.

\item[\comp{LINMIN}] Called by the BFGS geometry optimized \comp{FLEPO},
\comp{LINMIN} takes a step in the search-direction and if the energy drops,
returns.  Otherwise it takes more steps until if finds one which causes the
energy to drop.

\item[\comp{LOCAL}] Given a set of occupied eigenvectors, produces a  canonical
set of localized bonding orbitals, by a series of  $2\times 2$ rotations which
maximize $\langle \psi^4 \rangle$.  Called by \comp{WRITMN}.

\item[\comp{LOCAL2}] Converts hybridized, but non-localized atomic orbitals
into localized hybrid atomic orbitals, by maximizing  $\langle \psi^4 \rangle$.

\item[\comp{LOCALZ/LOCANZ}] Re-localizes LMOs.  The MOZYME LMOs are usually not
fully localized.  \comp{LOCALZ} completes the localization, so that the results
can be compared with standard MOPAC localized M.O.s.

\item[\comp{LOCMIN}] In a gradient minimization, \comp{LOCMIN} does a 
line-search to find the gradient norm minimum. Main arguments: current
geometry, search direction, step, current gradient  norm; on exit: optimized
geometry, gradient norm.

\item[\comp{LYSE}] Breaks all covalent bonds between different parts of a
protein.  The set of bonds that are broken is the set X-Y, where X$\neq$C and
Y$\neq$(C, N, or O), e.g.\ S-O, S-N, and S-S.  Used by \comp{NAMES} and
\comp{RESEQ} to work out the residues in a protein.

\item[\comp{MAKEUF}] Part of \comp{POLAR}.

\item[\comp{MAKOPR}] Part of the point-group package, \comp{MAKOPR} builds the 
 operations based on the point group of the system.  A check is made to verify
 that the operations are valid. Called by \comp{SYMTRZ} only.

\item[\comp{MAKSYM}] The code to make \comp{AUTOSYM} work.

\item[\comp{MACVEC/MAKVEN}] Constructs the starting localized mono- and
di-atomic molecular orbitals.

\item[\comp{MAMULT}] Matrix multiplication. Two matrices, stored as lower half
triangular packed arrays, are multiplied together, and the result stored in a
third array as the lower half triangular array. Called from \comp{PULAY}.

\item[\comp{MAT33}] Part of the point-group package,  \comp{MULT33} performs a 
eulerian rotation on a symmetry operation.

\item[\comp{MATOU1/MATON1}] Similar to \comp{MATOUT}, but also prints the
symmetry labels.

\item[\comp{MATOUT/MATOUN}] Matrix printer. Prints a square matrix, and a 
row-vector, usually eigenvectors and eigenvalues. The indices  printed depend
on the size of the matrix: they can be either  over orbitals, atoms, or simply
numbers, thus M.O.s are over  orbitals, vibrational modes are over numbers.

\item[\comp{MBONDS}] Using the localized hybrid atomic orbitals, \comp{MBONDS}
constructs di-atomic bonds and antibonds.

\item[\comp{ME08A, ME08B}] Utilities: Part of the complex diagonalizer, and
called by  \comp{CDIAG}.

\item[\comp{MECI/MECN}] Main function for Configuration Interaction,
\comp{MECI} constructs the appropriate C.I.\ matrix, and evaluates the roots,
which correspond to the electronic energy of the states of the system. The
appropriate root is then returned. Called by \comp{ITEN}  only.

\item[\comp{MECID}] Constructs the differential C.I.\ secular determinant.

\item[\comp{MECIH}] Constructs the normal C.I.\ secular determinant.

\item[\comp{MECIP}] Reforms the density matrix after a MECI calculation.

\item[\comp{MEPCHG}] Part of the PMEP.

\item[\comp{MEPMAP}] Part of the PMEP.

\item[\comp{MEPROT}] Part of the PMEP.

\item[\comp{MFINEL}] Part of intersystem crossing method.

\item[\comp{MINGEO}] Very simple routine to print the input geometry.

\item[\comp{MINLOC}] If a heavy atom has less than 3 atoms attached to it, then
when the hybrid orbitals are generated, two or three of them are ill-defined.
This ill-definition can be resolved by additional constraints, specifically by
requiring certain coefficients to be zero.

\item[\comp{MINV}] Matrix inverter.  This is inefficient, and should be coded
out.

\item[\comp{MKBMAT}] In COSMO, fills the ``B'' matrix.

\item[\comp{MLIG}] Part of MNDO-$d$.

\item[\comp{MLMO}] The starting localized M.O.s are put into their appropriate
arrays by \comp{MLMO}.

\item[\comp{MO}] Part of Greens Function.

\item[\comp{MODCHG}] Prints the charge due to (a) backbone residue atoms, and
(b) all side-chain atoms in each residue, and (c) the net charge ($a$ plus $b$)
on each residue, in a protein.

\item[\comp{MODGRA}] Calculates and prints the gradient norms for all residues
(side-chains and backbones).

\item[\comp{MOIETY}] In protein systems, \comp{MOIETY} identifies atoms in
small molecules that are associated with the protein.

\item[\comp{MOINT}] Part of Greens Function.

\item[\comp{MOLDAT/MOLDAN}] Sets up all the invariant parameters used during
the calculation, e.g.\ number of electrons, initial atomic  orbital
populations, number of open shells, etc. 

\item[\comp{MOLSYM}] Part of the point-group package, \comp{MOLSYM} calculates
the molecular symmetry as a point-group symbol.

\item[\comp{MOLVAL}] Calculates the contribution from each M.O.\ to the total
valency in the molecule.  Empty M.O.s normally have a negative molecular
valency.

\item[\comp{MOPAC}] MAIN program. \comp{MOPAC} first reads in data using 
\comp{READMO}, then calls \comp{RMOPAC} to do the actual calculation. Starts
the timer.

\item[\comp{MOPEND}] Closes all open files and stops the job.  The last
subroutine to be called.

\item[\comp{ MPCBDS, MPCPOP, MPCSYB}] Used in generating data for the TRIPOS
package SYBYL.

\item[\comp{MTXM}] Part of the matrix package. Multiplies together two
rectangular packed arrays, i.e., C = A.B.

\item[\comp{MTXMC}] Part of the matrix package.  Similar to \comp{MTXM}.

\item[\comp{MULLIK/MULLIN/MULLIZ}] Constructs and prints the Mulliken
Population  Analysis. Available only for RHF calculations. Called by
\comp{WRITMN}.

\item[\comp{MULT}] Used by \comp{MULLIK} only, \comp{MULT} multiplies two
square  matrices together. 

\item[\comp{MULT33}] Part of the point-group package,  \comp{MULT33} performs
 a  eulerian rotation on a symmetry operation.

\item[\comp{MXM}] Part of the matrix package. Similar to \comp{MTXM}.

\item[\comp{MXMT}] Part of the matrix package. Similar to \comp{MTXM}.

\item[\comp{MYWORD}] Called in \comp{WRTKEY}, \comp{MYWORD} checks for  the
existence of a specific string. If it is found, \comp{MYWORD} is set true, and
the all occurrences of string are deleted. Any words not recognized will be
flagged and the job stopped.

\item[\comp{NAICAP/NAICAN}] Called by \comp{ESP}.

\item[\comp{NAICAS}] Called by \comp{ESP}.

\item[\comp{NAMES}] Determines the names of all residues in a protein.

\item[\comp{NEWDEL}] Part of intersystem crossing method.

\item[\comp{NEWFLG}] Used when \comp{NEWGEO} is requested, all the coordinates
are converted to internal coordinates, then \comp{NEWFLG} converts the backbone
atoms of a protein into Cartesian coordinates.

\item[\comp{NEWHES}] Part of intersystem crossing method.

\item[\comp{NEWMAT}] Creates temporary and permanent integer and real arrays,
and destroys temporary arrays.

\item[\comp{NGAMTG}] Part of \comp{POLAR}.

\item[\comp{NGEFIS}] Part of \comp{POLAR}.

\item[\comp{NGIDRI}] Part of \comp{POLAR}.

\item[\comp{NGOKE}] Part of \comp{POLAR}.

\item[\comp{NLLSQ/NLLSN}] Used in the gradient norm minimization.

\item[\comp{NONBET}] Part of \comp{POLAR}.

\item[\comp{NONOPE}] Part of \comp{POLAR}.

\item[\comp{NONOR}] Part of \comp{POLAR}.

\item[\comp{NOTLFT}] A check in the static polarizability calculation, to find
out if there is enough time left for another step.

\item[\comp{NUCHAR}] Takes a character string and reads all the numbers in it
and stores these in an array.

\item[\comp{NXTMER}] Identifies the backbone atoms of the next residue (mer) in
a protein.

\item[\comp{OPENDA}] Part of \comp{POLAR}.

\item[\comp{OPTBR}] Sets geometry optimization flags in a protein.  Either
backbone atoms or side chain atoms or both can be set.

\item[\comp{ORIENT}] Part of the point-group package, \comp{ORIENT} orients
cubic systems.

\item[\comp{OSINV}] Inverts a square matrix. Called by \comp{PULAY} only.

\item[\comp{OUTER1}] Calculates the point charge electrostatic energy term 
arising from the charge on atoms which are further away than \comp{CUTOF1}.

\item[\comp{OUTER2}] Calculates the point charge and dipole electrostatic
 energy terms arising from the charge on atoms which are further away than
 \comp{CUTOF2}.

\item[\comp{OVERLP}] Part of \comp{EF}. \comp{OVERLP} decides which normal mode
to  follow.

\item[\comp{OVLP}] Called by \comp{ESP} only.  \comp{OVLP} calculates the
overlap over Gaussian STOs.

\item[\comp{PACKP}] Part of \comp{PMEP}, \comp{PACKP} copies the monatomic
density matrix terms from a normal lower half triangular matrix into a packed
matrix. 

\item[\comp{PARSAV/PARSAV}] Stores and restores data used in the gradient-norm
minimization calculation.

\item[\comp{PARTXY}] Called by \comp{IJKL} only, \comp{PARTXY} calculates the
partial product $<$i,j(1/r) in $<$i,j(1/r)k,l$>$.

\item[\comp{PATHK}] Calculates a reaction coordinate which uses a constant
step-size.  Invoked by keywords \comp{STEP} and \comp{POINTS}. 

\item[\comp{PATHS}] Given a reaction coordinate as a row-vector,  \comp{PATHS}
performs a \comp{EF} or \comp{FLEPO} geometry optimization for each point, the
later geometries being initially guessed from a knowledge of the already
optimized geometries, and the current step. Called by \comp{RMOPAC} only.

\item[\comp{PDBOUN}] Prints the geometry in PDB format. (``N'' is used, because
this is part of \comp{GEOUN}.)

\item[\comp{PDGRID}] Part of \comp{ESP}. Calculates the Williams surface.

\item[\comp{PEDRA}] Part of the Tomasi Solvation method.

\item[\comp{PERM}] Permutes $n1$ electrons of alpha or beta spin among $n2$
   M.O.s. 

\item[\comp{PICOPT}] Determines which atoms are to be included in the SCF
calculation.

\item[\comp{PINOUT/PINOUN}] Writes out the localized molecular orbitals to a 
`.den' file, and reads them in when \comp{OLDENS} is specified.

\item[\comp{PLATO}] Part of the point-group package, \comp{PLATO} orients a
 cubic system.

\item[\comp{PMEP/PMEP1/PMEPCO}] Generates the Parametric Molecular
Electrostatic Potential.

\item[\comp{POLAR/POLAN}] Calculates the static and frequency dependent
polarizability volumes using the ``Sum over States'' (SOS) method.

\item[\comp{POLARZ/POLANZ}] Calculates the static polarizability volumes  for a
molecule or ion, using 19 SCF calculations.

\item[\comp{POTCAL}] Part of \comp{ESP}.

\item[\comp{POWSAV/POWSAN}] Calculation store and restart for \comp{SIGMA} 
calculation. Called by \comp{POWSQ}.

\item[\comp{POWSQ/POWSN}] The McIver - Komornicki gradient  minimization
routine. Constructs a full Hessian matrix  and proceeds by line-searches Called
from \comp{RMOPAC} when  \comp{SIGMA} is specified.

\item[\comp{PRINTP}] Part of MNDO-$d$, \comp{PRINTP} prints the value of a
single parameter used in a MNDO-$d$ calculation. 

\item[\comp{PRJFC}] Part of \comp{EF}.

\item[\comp{PROJE}] Part of intersystem crossing method.

\item[\comp{PRTDRC}] Prints \comp{DRC} and \comp{IRC} results  according to
instructions. Output can be (a) every point calculated (default), (b) in
constant steps in time, space or energy.

\item[\comp{PRTGRA}] Prints all the gradients in internal coordinates. If
\comp{XYZ} is used, the gradients will be in Cartesian coordinates.

\item[\comp{PRTHCO}] Part of MNDO-$d$, \comp{PRTHCO} prints the \comp{H}
matrix.

\item[\comp{PRTHES}] Prints the Hessian in \comp{EF}.

\item[\comp{PRTLMO/PRTLMN}] Prints the localized molecular orbitals.

\item[\comp{PRTPAR}] Part of MNDO-$d$, \comp{PRTPAR} prints the values of all
parameters used in a MNDO-$d$ calculation.  Very useful in debugging.

\item[\comp{PRTTIM}] Prints the time with an elegant layout.

\item[\comp{PULAY}] A new converger. Uses a powerful  mathematical
non-iterative method for obtaining the SCF Fock  matrix. Principle is that at
SCF the eigenvectors of the Fock  and density matrices are identical, so
[\comp{F.P}] is a measure of the non-self consistency. While very powerful,
\comp{PULAY} is not  universally applicable. Used by \comp{ITEN}.

\item[\comp{PURDF1}] Part of the Tomasi Solvation method.

\item[\comp{QNALG/QNALN}] Part of intersystem crossing method.

\item[\comp{QNSAVE}] Part of intersystem crossing method.

\item[\comp{QNSTEP}] Part of intersystem crossing method.

\item[\comp{QUADR}] Used in printing the \comp{IRC} - \comp{DRC} results.  Sets
up a quadratic in time of calculated quantities so that \comp{PRTDRC}  can
select specific reaction times for printing.

\item[\comp{REACT1/REACT2}] Uses reactants and products to find the  transition
state. A hypersphere of $N$ dimensions is centered on  each moiety, and the
radius steadily reduced. The entity of  lower energy is moved, and when the
radius vanishes, the  transition state is reached. Called by \comp{RMOPAC}
only.

\item[\comp{READA}] General purpose character number reader. Used to enter
numerical data in the control line as $<$variable$>$=$n.nnn$ where
$<$variable$>$ is a mnemonic such as \comp{SCFCRT} or \comp{CHARGE}.  Called by
\comp{READ}, \comp{FLEPO}, \comp{ITER}, \comp{FORCE}, and  many other
subroutines.

\item[\comp{READMO}] Almost all the data are read in through \comp{READMO}.  \
There is a lot of data-checking in \comp{READMO}, but very little  calculation.
Called by \comp{MOPAC}.

\item[\comp{REFER}] Prints the original references for atomic data. If an atom
does not have a reference, i.e., it has not been parameterized, then a warning
message will be printed and the calculation stopped.

\item[\comp{REORTH}] Re-orthogonalizes the LMOs.  During a calculation, the
LMOs become increasingly non-orthogonal.  Although the total error is usually
very small, whatever error there is due to non-orthogonality, that error is
removed by \comp{REORTH}.

\item[\comp{REPP}] Calculates the 22 two-electron reduced repulsion  integrals,
and the 8 electron-nuclear attraction integrals. These are in a local
coordinate system. Arguments: atomic numbers of the two atoms, interatomic
distance, and arrays to  hold the calculated integrals. Called by \comp{ROTATE}
and \comp{OUTER2} only.

\item[\comp{REPPD}] More-or-less like \comp{REPP}, this routine should be
deleted.

\item[\comp{REPPD2}] More-or-less like \comp{REPP}, but extended to include 
``$d$'' orbitals.

\item[\comp{RESEQ}] A unique routine, \comp{RESEQ} will re-sequence the atoms
so that the atoms are in the order of the residues; i.e., the atoms of residue
1 first, then those of residue 2, etc.  This routine will automatically stop
the job.

\item[\comp{RESET/RESEN}] In \comp{RESEN}, geometric information is  copied
from the dynamic memory back into the static COMMON arrays at the end of a
calculation. This is necessary so that it can be available to the next
calculation.

\item[\comp{RESOLV}] If two or more LMOs have the same atomic contributions,
then any linear combination of these LMOs is an acceptable solution. This is
undesirable, in that the energies of these LMOs is therefore not defined. 
\comp{RESOLV} will identify such sets of LMOs, and resolve them so that they
have a zero energy interaction, that is, the integral $<\!$psi(1)|F|psi(2)$\!>$
is zero.

\item[\comp{RFIELD/RFIELN}] Part of the Tomasi Solvation method.

\item[\comp{RING5}] Detects the presence of a five membered conjugated ring of
the type found in the fullerenes.

\item[\comp{RMOPAC}] The main subroutine in MOPAC, \comp{RMOPAC} calls all the
geometric functions \comp{EF}, \comp{FLEPO}, \comp{DRC},  \comp{FORCE}, etc. 
These subroutines are not called from \comp{MOPAC} because they do not have
access to the elements of \comp{CCORE} or \comp{MCORE}.

\item[\comp{ROTAT}] Rotates analytical two-electron derivatives from atomic to
molecular frame.

\item[\comp{ROTATD}] Rotates analytical two-electron derivatives from atomic to
molecular frame in MNDO-$d$.

\item[\comp{ROTATE}] All the two-electron repulsion integrals, the electron-
nuclear attraction integrals, and the nuclear-nuclear repulsion term between
two atoms are calculated here. Typically 100 two- electron integrals are
evaluated.

\item[\comp{ROTLMO/ROTLMN}] In \comp{POLAR} calculations, the molecule is
rotated so that it is orientated along the Principle Moments of Inertia.  A
consequence of this rotation is that the LMOs no longer block diagonalize the
Fock matrix. \comp{ROTLMN} then rotates the LMOs so that they are now correct
for the new orientation of the molecule.

\item[\comp{ROTMAT}] Calculates the unitary matrices for rotation of atomic
orbitals from a diatomic into a molecular frame in MNDO-$d$.

\item[\comp{ROTMOL}] Part of the point-group package, \comp{ROTMOL} rotates the
Cartesian coordinates of a molecule. \item[\comp{RPOL1}] Part of the Tomasi
Solvation method.

\item[\comp{RSP}] Rapid diagonalization routine. Accepts a secular determinant,
and produces a set of eigenvectors and  eigenvalues. The secular determinant is
destroyed. 

\item[\comp{SCFCRI}] Determines the SCF criteria.

\item[\comp{SCHMIB}] Part of Camp-King converger.

\item[\comp{SCHMIT}] Part of Camp-King converger.

\item[\comp{SCPRM}] Calculates the Slater-Condon parameters in MNDO-$d$.

\item[\comp{SEARCH}] Part of the \comp{SIGMA} and \comp{NLLSQ}  gradient
minimizations.  The line-search subroutine, \comp{SEARCH} locates the gradient 
minimum and calculates the second derivative of the energy  in the search
direction. Called by \comp{POWSQ} and \comp{NLLSQ}.

\item[\comp{SECOND}] Contains machine specific code. Function \comp{SECOND}
returns the number of CPU seconds elapsed since an arbitrary  starting time. If
the \comp{shut} command has been issued,  the CPU time is in error by exactly
10,000,000 seconds, and  the job usually terminates with the message ``time
exceeded''.

\item[\comp{SELMOS}] Selects the LMOs to be used in the SCF.  If only a part of
the system is of interest, for example the active site, then the SCF can be
limited to a subset of the system.  The LMOs involving the active atoms are
identified, and moved to the start of the LMO set.  Then, when \comp{DIAGG} is
called, only the active LMOs will be used.  This runs much faster than when the
entire system is involved.

\item[\comp{SET}] Called by \comp{DIAT2}, evaluates some terms used in overlap
calculation.

\item[\comp{SETUP3}] Sets up the Gaussian expansion of Slater orbitals using a
STO-3G basis set. Used in \comp{ESP} only.

\item[\comp{SETUPG}] Sets up the Gaussian expansion of Slater orbitals using a
STO-6G basis set.

\item[\comp{SETUPI}] At the start of a calculation, \comp{SETUPI} works out the
sizes of all permanent INTEGER arrays.  A second call to \comp{SETUPI} creates
these arrays.

\item[\comp{SETUPK}] Determines the set of atoms to be used in re-constructing
the Fock matrix.  The set of atoms is the set of all LMOs used in \comp{DIAGG}.

\item[\comp{SETUPR}] At the start of a calculation, \comp{SETUPR} works out the
sizes of all permanent REAL arrays.  A second call to \comp{SETUPR}  creates
these arrays. 

\item[\comp{SNAPTH}] ``Tidies up'' input angles, for example, converts 109.471
into the exact, 16 figure, tetrahedral angle.

\item[\comp{SOLBOX/SOLBON}] Part of the Tomasi Solvation method.


\item[\comp{SOLROT}] For Cluster systems, adds all the two-electron integrals
of the same type, between different unit cells, and stores them in a single
array. Has no effect on molecules.

\item[\comp{SORT}] Part of \comp{CDIAG}, the complex diagonalizer.

\item[\comp{SPCORE}] In MNDO-$d$, \comp{SPCORE} calculates the 10 $s-p$ - core
integrals.

\item[\comp{SPLINE}] Part of Camp-King converger.

\item[\comp{SS}] An almost general Slater orbital overlap calculation. Called
by DIAT.

\item[\comp{STATE}] At the start of a calculation \comp{STATE} determines the
state of the system (number of electrons, orbitals, and the number of $\alpha$
and $\beta$ electrons (if UHF), or number of filled and partly filled levels,
etc. (if RHF).

\item[\comp{SUMA2}] Part of Greens Function.

\item[\comp{SUPDOT}] Matrix multiplication A=B.C

\item[\comp{SUPERD}] Calculates and prints the superdelocalizabilities.

\item[\comp{SURCLO}] In COSMO, calculates the Solvent Accessible Surface (SAS)
in the waist or neck between two atoms.

\item[\comp{SURFA}] Part of the PMEP.

\item[\comp{SURFAC}] Part of the ESP.

\item[\comp{SURFAT}] Part of the Tomasi Solvation method.

\item[\comp{SWAP}] Used with \comp{FILL=n}, SWAP ensures that a specified 
M.O.\ is filled. Called by ITER only.

\item[\comp{SWITCH}] At the start of a calculation, \comp{SWITCH} copies
parameters appropriate to the method into reference arrays.

\item[\comp{SYMDEC}] DECend in SYMmetry)  For a given point group,
\comp{SYMDEC} checks to see of the point group operations present in the
molecule match the operations of the group.  If they do, then \comp{SYMDEC} is
set to ``true''.

\item[\comp{SYMH/SYMN}] Rotates a dipole moment vector.  Used in the 
\comp{FORCE} calculation.

\item[\comp{SYMOIR}] Part of the point-group package, \comp{SYMOIR} calculates
 the  Irreducible Representation of an eigenvector, given the character table
 and the characters of the representation.

\item[\comp{SYMOPR}] Part of the point-group package,  \comp{SYMOPR} performs a
 symmetry operation on the Cartesian coordinates.

\item[\comp{SYMP}] Part of the point-group package,  \comp{SYMP} generates all
 the operations of a group, starting with the operations stored in
 \comp{CARTAB}.

\item[\comp{SYMPOP}] Rotates the dipole moment vector. Used in the \comp{FORCE}
 calculation.

\item[\comp{SYMR}] Used in the rapid construction of the \comp{FORCE} Hessian
matrix, \comp{ SYMH} will add information about an atom to the Hessian matrix,
using point-group information.

\item[\comp{SYMT}] In a \comp{FORCE} calculation, \comp{SYMT} will symmetrize
 the Hessian matrix so that the resulting eigenvectors will transform properly.

\item[\comp{SYMTRY/SYMTNN}] Calculates values for geometric parameters from
known geometric parameters and symmetry data. Called whenever \comp{GMETRY} is
called.

\item[\comp{SYMTRZ/SYMTRN}] The point-group package.  Identifies molecular
point-groups, and characterizes M.O.s, normal coordinates (vibrations), and
state functions.

\item[\comp{TF}] Part of \comp{POLAR}.

\item[\comp{THERMO}] After the vibrational frequencies have been  calculated,
\comp{THERMO} calculates thermodynamic quantities such as internal energy, heat
capacity, entropy, etc, for translational, vibrational, and rotational, degrees
of freedom.

\item[\comp{TIDY/TIDN}]\label{tidy}  Tidies up all the LMOs.   Atoms with
negligible intensity in any LMO are deleted.  The resulting unused storage is
eliminated, and all the gaps between LMOs are made equal.

\item[\comp{TIMER}] Prints times of various steps.

\item[\comp{TIMOUT}] Prints total CPU time in elegant format.

\item[\comp{TMPI}] Determines the amount of temporary INTEGER storage needed by
the calculation.

\item[\comp{TMPMR}] Determines the amount of temporary REAL storage needed by
the calculation, when conventional SCFs are done.. 

\item[\comp{TMPZR}] Determines the amount of temporary REAL storage needed by
the calculation, when the SCF is solved using localized M.O.s. 

\item[\comp{TRANSF}] Part of POLAR.

\item[\comp{TX}] Part of MNDO-$d$, involved in generating two-electron
two-center integrals in a molecular frame.

\item[\comp{TXTYPE}] Constructs the PDB label for each atom in a residue.

\item[\comp{UPCASE}] Puts a character string into upper case.

\item[\comp{UPDATE}] Given a set of new parameters, stores these  in their
appropriate arrays.  Invoked by \comp{EXTERNAL}.

\item[\comp{UPDHES}] Called by \comp{EF}, \comp{UPDHES} updates the Hessian
matrix.

\item[\comp{UPDHIN}] Updates the inverse Hessian matrix, using the BFGS recipe.

\item[\comp{VALUES/VALUEN}] Calculates the energy levels of the LMOs, and sorts
the LMOs into increasing energy order.  Used whenever the LMOs are to be
printed or output for graphical use.

\item[\comp{VECPRT/VECPRN/VECPRZ/VECPZZ}] Prints out a packed, lower-half
triangular matrix.  The labeling of the sides of the matrix depend on the
matrix's size: if it is equal to the number of orbitals, atoms, or other.
Arguments: The matrix to be printed, size of matrix. No data are changed by a
call of \comp{VECPRT}.

\item[\comp{VOLUME}] Calculates the volume of a unit cell.

\item[\comp{W2MAN}] In MNDO-$d$, copies the two-electron integrals, \comp{WW},
into the appropriate part of the two-electron integral array \comp{W}.

\item[\comp{WALLC}] Prints the wall-clock time (as opposed to CPU time).

\item[\comp{WORDER}] Part of Greens Function.

\item[\comp{WRDKEY}] Within a compound word, such as \comp{POLAR(E=(0.5))},
\comp{WRDKEY} will read the `simple' keywords, \comp{E=(0.5)}.

\item[\comp{WRITMO/WRITMN}] Most of the results are printed here. All  relevant
arrays are assumed to be filled. A call of \comp{WRITMO} only  changes the
number of SCF calls made, this is reset to zero.  No other data are changed. 

\item[\comp{WRTKEY}] Prints all keywords and checks for compatibility and to
see if any are not recognized.   \comp{WRTKEY}] can stop the job if any errors
are found.

\item[\comp{WRTMOZ}] In \comp{WRTKEY}, \comp{WRTMOZ} prints the keywords that
are specific to the \comp{MOZYME} function.

\item[\comp{WRTTXT}] Writes out \comp{KEYWRD}, \comp{KOMENT} and 
\comp{TITLE}.  The inverse of \comp{GETTXT}.

\item[\comp{WSTORN}] Adds the one-center two-electron integrals to \comp{W} and
symmetrizes the matrix.

\item[\comp{WWSTEP}] Part of Greens Function.

\item[\comp{XERBLA}] Part of COSMO.

\item[\comp{XXX}] Forms a unique logical name for a Gaussian Z-matrix logical. 
Called  by \comp{GEOUTG} only.

\item[\comp{XYZCRY}] Prints the derivatives of the energy with respect to
fractional crystal coordinates.

\item[\comp{XYZGEO}] Converts from Cartesian coordinates into internal, given a
connectivity table.

\item[\comp{XYZINT}] Converts from Cartesian coordinates into internal.
\comp{XYZINT} sets up its own numbering system, so no connectivity is needed. 
However, a connectivity can be supplied, if desired.
\end{description}
\index{Subroutines!description of|)}
