\subsection{Koopmans' Theorem.}\index{Koopmans' theorem}
\index{Ionization potential}
Koopmans' theorem~\cite{koopmans} can be understood as follows:  for
closed-shell systems, the negative of the HOMO energy is the ionization
potential. That is, the energy required to form the cation {\em provided that
the ionization process is adequately represented by the removal of an electron
from an orbital without change in the wave-functions of the other electrons.}

MOZYME does not use  eigenvectors (LMOs are not eigenvectors), therefore,
Koopmans' theorem cannot be used unless eigenvectors are generated.  To use
Koopmans' theorem, add \comp{VECTORS} and \comp{EIGEN} to the keyword line.
This will cause the eigenvectors to be generated, and from the eigenvalues the
ionization potential can be calculated.

The only alternative way to calculate the I.P.\ is to calculate the $\Delta
H_f$ of the parent species, then, without allowing the geometry to relax,
calculate the $\Delta H_f$ of the ionized system.  The difference in  $\Delta
H_f$, in kcal.mol$^{-1}$, divided by 23.06, is the predicted I.P., in eV.

\subsection{Dipole moments.}\index{Dipole moment!definition}
\index{Dipole moment!for ions}
For neutral systems, the dipole moment is calculated from the atomic charges
and the lone-pairs as
\begin{equation}
\mu_x  = cC\sum_AQ_Ax_A + cCa_o2\sum_A P(s-p_x)_AD_1(A)
\end  {equation}
\begin{equation}
\mu_y  = cC\sum_AQ_Ay_A + cCa_o2\sum_A P(s-p_y)_AD_1(A)
\end  {equation}
\begin{equation}
\mu_z  = cC\sum_AQ_Az_A + cCa_o2\sum_A P(s-p_z)_AD_1(A)
\end  {equation}
\begin{equation}
\mu = \mu_x+\mu_y+\mu_z
\end  {equation}
Where $c$ = speed of light, $C$ = charge on the electron, and $a_o$ = Bohr
radius, or $cC = 2.99792458 \times 1.60217733 = 4.8032066$, and $cCa_o2 = 
2.99792458 \times 1.60217733 \times 0.529177249 \times 2.0  = 5.0834948$.  $D_1(A)$ is defined  in
Table~\ref{aa}
\begin{latexonly}
(see p.~\pageref{aa})
\end{latexonly}.

Formally, the dipole moment for an ion is undefined; however, it is convenient
to set up a `working definition.'  Consider a heteronuclear diatomic ion in a
uniform electric field.  The ion will accelerate.  To compensate for this, it
is convenient to consider the ion in an accelerating frame of reference.  The
ion will  experience a torque which acts about the center of mass, in a manner
similar to that of a polar molecule.  This allows us to define the dipole of an
ion as the dipole the system would exhibit while accelerating in a uniform
electric field.  To formalize this definition:
\begin{equation}
\mu_x  = cC\sum_AQ_A(x_A-x_{cog}) + cCa_o2\sum_A P(s-p_x)_AD(A)
\end  {equation}
\begin{equation}
\mu_y  = cC\sum_AQ_A(y_A-y_{cog}) + cCa_o2\sum_A P(s-p_y)_AD(A)
\end  {equation}
\begin{equation}
\mu_z  = cC\sum_AQ_A(z_A-z_{cog}) + cCa_o2\sum_A P(s-p_z)_AD(A)
\end  {equation}
\begin{equation}
\mu = \mu_x+\mu_y+\mu_z,
\end  {equation}
where $x_{cog}$ is the $x$-coordinate of the center of gravity of the system
\begin{equation}
x_{cog} = \sum_AM_Ax_A,
\end  {equation}
and $y_{cog}$ and $z_{cog}$ have similar definitions.  This general expression
will work for all discrete species, charged and uncharged, and is rotation and
position invariant.


