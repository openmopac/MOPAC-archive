\section{Level of Precision within MOPAC}
\index{Precision} \index{Criteria}
Several users have criticized the  tolerances  within  MOPAC.   The point 
made  is  that significantly different results have been obtained when
different starting conditions have been used, even  when  the  same conformer 
should  have  resulted.  Of course, different results must be expected---there
will always be small differences---nonetheless  any differences   should   be 
small,  for example,  heats  of  formation   ($\Delta H_f$) \index{$\Delta
H_f$!precision} \index{Heat of Formation!precision} differences should be less
than about 0.1  kcal/mol.   MOPAC  has  the flexibility  to  allow  users  to 
specify a much higher precision than the default when circumstances warrant it.

\subsection{Fundamental Physical Constants}
\index{Fundamental Physical Constants}
The fundamental physical constants used in MOPAC were updated in 1993 to
conform with the 1986 CODATA recommendations~\cite{codata}.  The constants used
in MOPAC are given in Table~\ref{codata}.  As a result of this update, all
calculated quantities in MOPAC, except molecular weight, will change slightly
when compared to earlier MOPACs (MOPAC 6 and earlier).  Most of the time,
changes in $\Delta H_f$ are less than 0.1 kcal/mol.  It is not anticipated that
the physical constants will change again.  If they do, however, the effect on
calculated properties should be very small.

The derived quantities, AM, AD, AQ, EISOL, DD, and QQ are functions of the
fundamental constants.  Rather than change all of these, starting with MOPAC
93, they are evaluated  at the start of each calculation.  This is a quick
operation, taking only about 0.1s, and prevents any mistakes being introduced
due to human error. 

\begin{table}
\index{Boltzmann constant}
\index{Constants!physical}
\index{Definitions!Boltzmann constant}
\index{Definitions!velocity of light}
\index{Gas constant, R}
\caption{\label{codata} Fundamental Physical Constants}
\begin{center}
\begin{tabular}{|ccll|}\hline
Physical Constant & Symbol &Value &Units \\ \hline 
Speed of Light    &     c  &  299 792 458 &m sec$^{-1}$ (Definition)\\
Planck constant   &   $h$  & 6.626 075 5(40)  $\times$ 10$^{-34}$ &J sec \\
&&  6.626 075 5(40) $\times$ 10$^{-27}$ & erg s\\
Avogadro constant & $N$    & 6.022 136 7(36)  $\times$  10$^{23}$&mol$^{-1}$ \\
Molar gas constant & $R$   & 1.987 215 6  & cal/mol/degree\\
            & & 8.314 510(70) & J/mol/K \\
Volume of 1 mol of gas& $V_0$ & 22.414 10(19) & l/mol (at 1 atm, 25 C) \\
Electron volt     & eV     & 1.602 177 33(49) $\times$ 10$^{-19}$&J \\
Electron charge   & e      & 1.602 177 33(49) $\times$ 10$^{-19}$&C \\
Hartree           & $E_h$  & 27.211 396 1(81) & eV \\
Electrostatic energy & $E_ha_{\circ}$ & 14.399 651 782 565 &eV \\
Bohr radius       & $a_{\circ}$ & 0.529 177 249(24) $\times$  10$^{-10}$ & m \\
Boltzmann constant& $k$ = $R/N$ & 1.380 658(12) $\times 10^{-23}$& J/K \\
                  &              & 1.380 658(12) $\times 10^{-16}$& erg/K\\
pi & $\pi$ & 3.141 592 653 589 79 & \\
Joule             & J/cal  & 4.184        & J/cal (Definition)\\
 cm$^{-1}$ &  $h$c/eV   & 1.239 842 4  $\times$  10$^{-4}$ & eV\\
 cm$^{-1}$ & $h$c$N$/(1000J/cal)& 2.859 144  $\times$  10$^{-3}$ & kcal/mol\\
 cm$^{-1}$&$h$c$N$/(J/cal)& 2.859 144 &cal/mol \\
cm$^{-1}$   & & 1.196 266 $\times$  10$^{8}$ & erg = dyne \AA $^{-1}$\\
Atomic unit (a.u.)     &        & 8.657 10 $\times$  10$^{-33}$ &e.s.u. \\
a.u.      &  &  2.541 747 8 $\times$  10$^{-40}$ & Debye \\
a.u.        & & 51.422 082     & V m$^{-1}$\\
kcal/mol    & & 6.947 700 $\times$  10$^{-3}$ & erg\\
1 J &&1.$ \times 10^7$ & erg\\
1 eV &&23.060 542 301 389  &kcal/mol\\
&& 627.509 6& kcal/mol\\
1 atm && 1.013 25 $\times 10^5$ &Pa \\
&& 1.013 25 $\times 10^6$ &dyn/cm$^2$ (Definition)\\ \hline
\end{tabular}~\\
Note: The precision of derived constants should {\em not} be used as an
indication of their accuracy. The uncertainty in the fundamental constants is
given in parenthesis after the value. 
\end{center}
\end{table}


\subsection{Various precision levels}
In normal (non-publication quality) work the default  precision  of MOPAC  is
recommended.  This will allow reasonably precise results to be obtained  in  a 
reasonable  time.    Unless   this   precision   proves unsatisfactory, use
this default for all routine work.

The  best  way  of  controlling  the  precision  of  the   geometry
optimization  and gradient minimization is by specifying a gradient norm
\index{GNORM|ff} which must be satisfied.  The gradient norm is the scalar of
the vector of derivatives of the energy with respect to the geometric variables
flagged for optimization. I.e.,
$$
{\rm  GNORM}=\sqrt{\sum_i\left(\frac{d(\Delta H_f)}{dx_i}\right)^2}
$$
where $x_i$ refers to coordinates flagged for optimization.  Note that the
calculated GNORM may be very small and at the same time the geometry might not
be at a stationary point.  This can easily happen when (a) less than 3$N$-6
coordinates are flagged for optimization, (b) \comp{SYMMETRY} has been used
incorrectly, or (c) (less common) only 3$N$-6 coordinates are flagged for
optimization and dummy atoms are used.  If any one of these three conditions
occurs, then the warning message,  ``WARNING -- GEOMETRY IS NOT AT A STATIONARY
POINT''  \index{Error message!WARNING -- GEOM\ldots } will be printed.

A less common, but not unknown, situation arises when internal
coordinates are used.  In this strange situation, the internal derivatives might
all be zero, but the Cartesian derivatives are large.  An example of such a
system is shown in Figure~\ref{xch2o}.
\begin{figure}
\begin{makeimage}
\end{makeimage}
\begin{verbatim}
Line 1:  GRADIENTS 1SCF DEBUG DCART
Line 2:  EXAMPLE OF INTERNAL COORDINATE DERIVATIVES ZERO
Line 3:  AND CARTESIAN DERIVATIVES LARGE
Line 4: O
Line 5: C   1.2075664 1
Line 6: H   1.2114325 1   103.347931 1    0 0   2 1
Line 7: H   1.0904182 1   180.000000 1   90 1   2 1 3
\end{verbatim}
\caption{\label{xch2o} Example of Spurious Stationary Point}
\end{figure}

In this strange system, the bond-angle of the second hydrogen is 180$^\circ$,
and the dihedral is 0$^\circ$.  Obviously, the derivative of the dihedral will
be zero. The derivative of the angle is not so obvious.  If the angle changes,
then the fourth atom will move out of the plane of the other three atoms. The
energy will change in the same way regardless of whether the angle increases or
decreases; therefore, the derivative of the angle must be zero.

Because of the unusual nature of this type of system, users may be unaware of
the danger.  If such a system is detected, a warning will be given and the job
stopped.  For all other cases, the ``WARNING -- GEOMETRY IS NOT AT A 
STATIONARY POINT'' message will be printed on completion of the calculation. 
In the unlikely event that the calculation should not be stopped when the
strange system is detected, calculation can be continued by specifying
\comp{GEO-OK}.

Modification of  GNORM is done via the keyword \comp{GNORM=$n.nn$}.  Altering
the   GNORM  automatically disables the other termination tests resulting in
the gradient norm dominating the calculation.  This works both  ways:
\comp{GNORM=20}  will give a very crude optimization while \comp{GNORM=0.01} 
will give a very precise optimization.  The default is \comp{GNORM=1.0}.

When the highest precision is needed, such as in exacting  geometry work,  or 
when  you want results whose precision cannot be improved, then use the
combination keywords \comp{GNORM=0.0} and either \comp{RELSCF=0.01} or
\comp{SCFCRT=1.D-NN};  \index{SCFCRT} (\comp{NN} should  be  in  the range 
5--15).   By default, EigenFollowing is used in geometry optimization.  One
reason is that EigenFollowing is nearer to a gradient minimizer than it is to
an  energy minimizer.  Because of this, if there is any difference between the
gradient minimum and the energy minimum, it  will give better reproducibility
of the optimized geometry than the alternative \comp{BFGS} method.

In practice, optimized geometries for ``well behaved'' systems can be obtained
with \comp{GNORM}s of less than 0.0001.

Increasing the SCF criterion (the default is \comp{SCFCRT=1.D-4}) improves the
precision of the gradients; however, it can lead to excessive run times, so
take care.  Also, there is an increased chance of  not  achieving  an SCF when
the SCF criterion is excessively increased.

Superficially, requesting \comp{GNORM=0}   might  seem  excessively stringent, 
but  as soon as the run starts, it will be cut back to 0.01. Even that might
seem too  stringent.   The  geometry  optimization  will continue to lower the
energy, and hopefully the GNORM, but frequently it will not prove possible to
lower  the   GNORM to  0.01.   If,  after  10 cycles,  the energy does not drop
then the job will be stopped.  At this point you have the best geometry that
MOPAC, in its  current  form,  can give.

If a slightly less than highest precision is needed,  such  as  for normal
publication quality work, set the \comp{GNORM} to the limit wanted.  For
example, for a flexible system, a \comp{GNORM} of 0.1 to 0.5 will  normally  be
good enough for all but the most demanding work.

If higher than the default, but still not very  high  precision  is wanted, 
then  use  the  keyword  \comp{PRECISE}.  This will tighten up various criteria
so that higher-than-routine precision will be given.

If high precision is used, so that the printed  GNORM is 0.000,  and the  
resulting   geometry   resubmitted   for  one  SCF  and  gradients calculation,
then normally a  GNORM higher than 0.000 will result.   This is  {\em not}  an
error in MOPAC:  the geometry printed is only precise to eight figures after
the decimal point.  Geometries may need  to  be  specified  to more than eight
decimals in order to drive the  GNORM to less than 0.000.

If you want to test MOPAC, or use it  for  teaching  purposes,  the
\comp{GNORM}  lower limit of 0.01 can be overridden by specifying \comp{LET},
in which case you can specify any limit for \comp{GNORM}.  However, if it is
too low the job  may  finish  due to an irreducible minimum in the heat of
formation being encountered.  If this happens, the ``STATIONARY POINT'' message
will be printed.

Examples of highly optimized geometries can be found in the \comp{port.dat}
file.  When this job is run, most gradients will be less than 0.001
kcal/mol/\AA .  A few will be larger. These exceptions fall into two classes:
diatomics, for which a simple line-search is sufficient to locate the optimum
geometry, in which case the  GNORM criterion is {\em not} used; and
non-variationally optimized systems, where the analytical C.I.\ derivatives 
are used.  These derivatives are of lower precision than the variational
derivatives, but are still much better than finite difference derivatives using
full SCFs. \index{Analytical derivatives}\index{Derivatives!analytical|ff}
Finally there is  a  full  analytical  derivative  function~\cite{analyt}
within \index{STO-6G}\index{Gaussian!wavefunctions} MOPAC.  These use STO-6G
Gaussian wavefunctions because the derivatives of the overlap integral are
easier to calculate  in  Gaussians  than  in STOs.  Consequently, there will be
a small difference in the calculated $\Delta H_f$ when analytical derivatives
are used.  If  there  is  any  doubt \index{Analytical derivatives} about  the 
accuracy of the finite derivatives, try using the analytical derivatives.  They
are a bit slower than finite derivatives but are more precise  (a  rough 
estimate is 12 figures for finite difference, 14 for analytical).

Some calculations, mainly open shell RHF or closed shell  RHF  with C.I., have
untracked errors which prevent very high precision.  For these systems
\comp{GNORM} should be in the range 1.0 to 0.1.

\subsection{Reasons for low precision}
There are several reasons for obtaining low quality  results,   the most 
obvious  of which is that for general work the default criteria will result in 
a  difference  in  $\Delta H_f$   of  less  than  0.1 kcal/mol.    This   is  
only  true  for  fairly  rigid  systems,  e.g.\ formaldehyde and benzene.  For
systems with low barriers to rotation  or \index{Flat potential surfaces}
\index{Aniline, dimer}\index{Water, dimer} flat  potential  surfaces,  such
as   aniline  or  water dimer, quite large $\Delta H_f$  errors can result.

\subsection{How large can a gradient be and still be acceptable?}
A common source of confusion is the limit to which the \comp{GNORM} should be 
reduced  in  order  to  obtain acceptable results.  There is no easy answer.
However, a few guidelines can be given.

\index{LET}\index{ANALYT} First of all, setting  the \comp{GNORM} to an
arbitrarily  small  number  is not  sensible.   If \comp{GNORM=0.000001} and
\comp{LET} are used, a geometry can be obtained which is precise to about 
0.000001 \AA. \ If \comp{ANALYT} is also used, the results obtained will be
slightly different. Chemically, this change is meaningless, and no 
significance  should  be attached  to  such  numbers.   In  addition,  any 
minor  change  to the algorithm, such as porting it to a new machine, will give
rise to  small changes  in  the optimized geometry.  Even the small changes
involved in going from one version of MOPAC to another causes  small  changes 
in  the optimized geometry of test molecules.

As a guide, a \comp{GNORM} of 0.1 is sufficient for all  heat-of-formation
work,  and  a  \comp{GNORM}  of  0.01 for most geometry work.  If the system is
large, you may need to settle for a \comp{GNORM} of 1.0--0.5.

This whole topic was raised by Dr.\ Donald B. Boyd while he was at Lilly
Research  Laboratories,  who provided unequivocal evidence for a failure of
MOPAC and convinced me of the importance of increasing  precision  in certain
circumstances.\index{boyd@{\bf Boyd, Donald B.}}

\subsection{Convergence tests in subroutine ITER}
\index{Convergence tests!SCF}
\subsubsection{Self-consistency test}
\index{SCF!test}
The SCF iterations are stopped when two tests are satisfied.  These are (1)
when the difference in electronic energy, in eV, between any two consecutive
iterations drops below the adjustable parameter, \comp{SELCON}, and the 
difference between any three consecutive iterations drops below ten times
\comp{SELCON}, and (2) the difference in density matrix elements  on  two
successive iterations falls below a preset limit, which is a multiple of
\comp{SELCON}. \index{SELCON}

\comp{SELCON} is set initially to 0.0001 kcal/mol; this can be  made  100
times  smaller by specifying \comp{PRECISE} or \comp{FORCE}.   It can be
over-ridden by explicitly defining the SCF criterion {\em via}
\comp{SCFCRT=1.D-12}, or by use of \comp{RELSCF=0.1}.

\comp{SELCON} is further modified by the value of the  gradient  norm,  if
known.   If  \comp{GNORM} is large, then a more lax SCF criterion is
acceptable, and \comp{SCFCRT} can be relaxed up to 50 times its  default  value
(using \comp{RELSCF=50}).   As  the gradient norm drops, the SCF criterion
returns to its default value.

The SCF test is performed using the energy calculated from the Fock matrix 
which  arises  from  a  density matrix, and not from the density matrix which
arises from a Fock.  In the limit, the two  energies  would be  identical, 
but  the first converges faster than the second, without loss of precision.
