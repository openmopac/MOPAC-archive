\subsection{Mulliken populations}
\index{Mulliken populations}
By default, the density matrix printed is the Coulson matrix, which
assumes that the atomic orbitals are orthogonalized.

If the assumption of orthogonality is not made, then the Mulliken density
matrix can be constructed. To construct the Mulliken density matrix (also known
as the Mulliken population analysis), the M.O.s must first be re-normalized,
using the overlap matrix, $S$:
$$
\psi_i^{'} = \psi_i\times S^{-\frac{1}{2}}. 
$$
From these M.O.s, a Coulson population is carried out. The off diagonal terms
are simply the Coulson terms multiplied by the overlap:
$$
P_{\lambda\sigma\neq\lambda}'=S_{\lambda\sigma}2\sum_{i=1}^{occ}c_{\lambda i}
c_{\sigma i},
$$
while the on-diagonal terms are given by the Coulson terms, plus half the sum
of the off-diagonal elements:
$$
P_{\lambda \lambda}' =S_{\lambda\sigma}2\sum_{i=1}^{occ}c_{\lambda i}c_{\lambda i}
 + \frac{1}{2}\sum_{\sigma\neq\lambda}P_{\lambda \sigma}'.
$$
A check of the correctness of the Mulliken populations is to add the diagonal
terms: these should equal the number of electrons in the system.

\subsubsection*{Theory of Mulliken Populations}
The NDDO methods (MNDO, AM1, PM3, and MNDO-$d$) all use Slater orbitals,
but an implication of one of the approximations made, that $\sum(F_{\mu \nu}-E_i\delta_{\mu\nu})
C_{\nu i} =0$, is that the conventional molecular orbitals are normalized
to unity:
$$
\psi_i=\sum_{\lambda}c_{\lambda i}\phi_{\lambda}
$$
with
$$
<\psi_i^2> = 1 = \sum_{\lambda}c_{\lambda i}^2
$$
For example, for H$_2$, the occupied M.O.\ is:
$$
\psi_1 = \sqrt{\frac{1}{2}}(\phi_{H_1}+\phi_{H_2}),
$$
and the unoccupied M.O.\ is:
$$ 
\psi_2 = \sqrt{\frac{1}{2}}(\phi_{H_1}-\phi_{H_2}). 
$$ 
The diagonal of the density matrix is then constructed using the Coulson
formula:
$$
P_{1,1}=P_{2,2}=2.0\times\left(\sqrt{\frac{1}{2}}\right)^2 =1.0.
$$
The off-diagonal terms are constructed in the same way:
$$
P_{1,2}=P_{1,2}=2.0\times\left(\sqrt{\frac{1}{2}}\right)^2 =1.0.
$$

If, instead of using $\sum(F_{\mu \nu}-E_i\delta_{\mu\nu}) C_{\nu i} =0$,
$\sum(F_{\mu \nu}-E_i) C_{\nu i} =0$ is used, then the occupied and unoccupied
M.O.s become:
$$ 
\psi_1 = \sqrt{\frac{1}{2(1+S)}}(\phi_{H_1}+\phi_{H_2}),
$$
and the unoccupied M.O.\ is:
$$
\psi_1 = \sqrt{\frac{1}{2(1-S)}}(\phi_{H_1}-\phi_{H_2}).
$$       
where $S$ is the overlap integral: $\int\phi_{H_1}\phi_{H_2}{\rm d}v$.

In this case, the Coulson population would give 
$$
\begin{array}{cc|cc|}
   &   & \frac{1}{1+S} & \frac{1}{1+S} \\
P  & = &               &               \\
   &   & \frac{1}{1+S} & \frac{1}{1+S} \\
\end{array}
$$
From this we see that the Coulson representation is unsuitable for  two
reasons: first, the number of electrons in the system, represented by the
diagonal terms, does not add to 2.0. Second, the off-diagonal terms, which
should represent the  number of electrons resulting from the overlap of the two
atomic orbitals, becomes unity as the overlap {\em decreases}.  

To correct for this, it is physically meaningful to multiply the matrix 
elements by the overlap.  This gives:
$$
\begin{array}{cc|cc|}
   &   & \frac{1}{1+S} & \frac{S}{1+S} \\
P  & = &               &               \\
   &   & \frac{S}{1+S} & \frac{1}{1+S} \\
\end{array}
$$
Now the off-diagonal terms accurately represent the number of electrons which are
associated with the overlap electron density.  The total number of electrons
in the system is now correct: $ \frac{1}{1+S} $ on atom 1, $ \frac{1}{1+S} $
on atom 2, and $ \frac{2S}{1+S} $ in the overlap region, giving a total of 2.0.

Although this representation is correct, it is potentially misleading, in that
the diagonal terms do not add to the number of electrons.  Mulliken reasoned
that the electron density resulting from the overlaps should be divided into
two equal parts and added to the diagonal terms.  When that is done, we get:
$$
\begin{array}{cc|ll|}
   &   & \frac{1}{1+S}+\frac{S}{1+S} & \frac{S}{1+S} \\ 
P  & = &               &               \\ 
   &   & \frac{S}{1+S} & \frac{1}{1+S} +\frac{S}{1+S}\\
\end{array} 
$$
or
$$
\begin{array}{cc|ll|}
   &   & 1.0  & \frac{S}{1+S} \\ 
P  & = &               &               \\ 
   &   &  \frac{S}{1+S} & 1.0\\ 
\end{array} 
$$
This simple example can be extended to systems involving heteroatoms and to
polyatomics, and is fully general.

The Mulliken analysis can be applied to semiempirical methods.  To do this, it
is necessary to first convert the M.O.s from solutions of  $\sum(F_{\mu
\nu}-E_i\delta_{\mu\nu}) C_{\nu i} =0$ to solutions of $\sum(F_{\mu \nu}-E_i)
C_{\nu i} =0$.  The simplest way to do this is to take the conventional M.O.s
and multiply them by $S^{-\frac{1}{2}}$.  In the case of H$_2$, the resulting
M.O.s are exactly correct; in general, a small error is introduced.  This error
arises from the incomplete annihilation of the secular matrix elements, and is
quite unimportant.

