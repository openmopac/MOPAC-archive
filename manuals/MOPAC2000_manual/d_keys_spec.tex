\section{Keywords}
\subsection{Specification of Keywords}
All control data are entered in the form of keywords, which form the first 
line  of  a data-file.  A description of what each keyword does is given in
Section~\ref{defkey}. The order in which keywords  appear is not important,
although they must be separated by a space.  Some keywords can be abbreviated;
allowed abbreviations are noted in Section~\ref{defkey}  (for example,
\comp{1ELECTRON} can be entered as \comp{1ELECT}).  However, the full keyword
is preferred in order to  more  clearly  document  the  calculation  and  to
prevent  the  possibility  that  an  abbreviated  keyword  might  not  be
recognized.  If there is insufficient space in the first line for all the
keywords  needed,  then consider abbreviating the longer words.  One type of
keyword, those with an equal sign, such as \comp{BAR=0.05},  may  not  be
abbreviated, and the full word needs to be supplied.

Keywords which involve an equal sign, such as \comp{SCFCRT=1.D-12}, cannot be
written with spaces before or after the  equal  sign.   Thus  \comp{SCFCRT=
1.D-12}, \comp{SCFCRT  =  1.D-12}, etc., are not allowed.  

If two keywords which are  incompatible,  like  \comp{UHF}  and \comp{C.I.}, 
are supplied,  or  a keyword which is incompatible with the species supplied,
for instance \comp{TRIPLET} and a  methyl  radical,  then  error  trapping 
will normally occur, and an error message will be printed.  This usually takes
an insignificant amount of time, so data are quickly checked for obvious
errors.

Keywords are normally supplied as the first line of a data-file.  Options
exist, however, which allow users to:
\begin{itemize}
\item Supply more than one line of keywords.
%\item Change other data, namely the title of the job and (optionally)
%the description of the job into keywords.
\item To have frequently-used sets of keywords stored in an auxiliary file.
\end{itemize}
