\subsection*{\comp{IKST(100)}}
Because MOPAC can calculate many quantities, such as the number of atoms, heat
of formation, etc., a method has to exist to allow these quantities to be
transmitted throughout the program. This is accomplished by two arrays, \comp{IKST}
and \comp{RKST}.  Each element of these
arrays is reserved for use as a storage site for a scalar.  \comp{IKST} holds
integer scalars and \comp{RKST} holds real scalars.  Table~\ref{cuim}
summarizes these quantities.   At the start of a calculation, every element of
\comp{IKST} is set to -99999, and every element of \comp{RKST} is set to
-99999.99. As a result of this, if an element of these arrays is examined, it
is immediately obvious whether or not it has been set.  Thus, until the number
of atoms is known, IKST(2) (NUMAT), is -99999.

\begin{table}
\caption{\label{cuim} Constants used in MOPAC}
\begin{center}
\begin{tabular}{lllllllll} \hline
\comp{RKST} & \comp{IKST} & No. & \comp{RKST} & \comp{IKST} & No. & \comp{RKST} & \comp{IKST} & No.\\
\hline
\comp{ESCF} & \comp{NATOMS} & 1 & \comp{EOUTER} & \comp{ICOMPF} & 34 & & & 67\\
\comp{ENUCLR} & \comp{NUMAT} &  2 & \comp{THRESH} & \comp{ISAFE} & 35 & & & 68\\
\comp{EE} & \comp{NELECS} &  3 & \comp{RSOLV} & \comp{LIMSCF} & 36 & & & 69\\
\comp{ATHEAT} & \comp{NALPHA} &  4 & & \comp{NA1} & 37 & & & 70\\
\comp{FRACT} & \comp{NBETA} &  5 & \comp{CORHYB} & \comp{IGROUP} & 38 & & & 71\\
\comp{EMIN} & \comp{NCLOSE} &  6 & \comp{ESCFL} & \comp{NCLASS} & 39 & & & 72\\
\comp{COSINE} & \comp{NOPEN} &  7 & \comp{VOLUME} & \comp{NIRRED} & 40 & & & 73\\
\comp{GNORM} & \comp{NORBS} &  8 & \comp{FEPSI} & \comp{NSPA} & 41 & & & 74\\
\comp{TIME0} & \comp{ISCF} &  9 & \comp{RDS} & \comp{NPS} & 42 & & & 75\\
\comp{EZQ} & \comp{NSCF} & 10 & \comp{DISEX2} & \comp{NPS2} & 43 & & & 76\\
\comp{EEQ} & \comp{ID} & 11 & \comp{CUTOF1} & \comp{IOLDCV} & 44 & & & 77\\
\comp{AREA} & \comp{NVAR} & 12 & \comp{CUTOF2} & \comp{NSTATE} & 45 & & & 78\\
\comp{EDIEL} & \comp{MOLNUM} & 13 & \comp{CUTOFS} & \comp{NUMPTS} & 46 & & & 79\\
\comp{GVW} & \comp{LATOM} & 14 & \comp{CUTOFP} & & 47 & & & 80\\
\comp{GVWS} & \comp{LPARAM} & 15 & \comp{FNSQ} & \comp{LENABC} & 48 & & & 81\\
\comp{ELC1} & \comp{ITRY} & 16 & \comp{CLOWER} & \comp{MESP} & 49 & & & 82\\
\comp{TLEFT} & \comp{ITYPE} & 17 & \comp{CUPPER} & \comp{NUMRED} & 50 & & & 83\\
\comp{TDUMP} & \comp{MFLAG} & 18 & \comp{DLM} & \comp{NORRED} & 51 & & & 84\\
\comp{ELECT} & \comp{ITERQ} & 19 & \comp{XNORM} & \comp{NELRED} & 52 & & & 85\\
\comp{STEP} & \comp{IFLEPO} & 20 & \comp{DIFE} & \comp{NRES} & 53 & & & 86\\
\comp{TVEC(1,1)} & \comp{MPACK} & 21 & \comp{PRESSURE} & \comp{NUMCAL} & 54 & & & 87\\
\comp{TVEC(2,1)} & \comp{N2ELEC} & 22 & & \comp{NIP} & 55 & & & 88\\
\comp{TVEC(3,1)} & \comp{NMOS} & 23 & & \comp{IDIAGG} & 56 & & & 89\\
\comp{TVEC(1,2)} & \comp{LAB} & 24 & & \comp{MDRCRS} & 57 & & & 90\\
\comp{TVEC(2,2)} & \comp{NELEC} & 25 & & & 58 & & & 91\\
\comp{TVEC(3,2)} & & 26 & & & 59 & & & 92\\
\comp{TVEC(1,3)} & & 27 & & \comp{IPAD2} & 60 & & & 93\\
\comp{TVEC(2,3)} & & 28 & & \comp{IPAD4} & 61 & & & 94\\
\comp{TVEC(3,3)} & \comp{MAXTXT} & 29 & & \comp{NL1} & 62 & & & 95\\
\comp{RJKAB1} & \comp{LAST} & 30 & & \comp{NL2} & 63 & & \comp{ICROS} & 96\\
\comp{WTMOL} & \comp{NDEP} & 31 & & \comp{NL3} & 64 & & \comp{MORB} & 97\\
\comp{VERSON} & \comp{LABSIZ} & 32 & & & 65 & & \comp{MOP/MOZ} & 98\\
\comp{TOTIME} & \comp{LM61} & 33 & & & 66 & & & 99\\
 & & & & & & & \comp{DERIV} & 100\\
\hline
\end{tabular}
\end{center}
\end{table}

\begin{enumerate}
\item \comp{NATOMS} The number of atoms supplied by the data set,
including dummy atoms and translational vectors.  \comp{NATOMS} is set in the
subroutine that
reads in the geometry (\comp{GETGEO}, \comp{GETGEG}, or \comp{GETPDB}).

With one exception, once \comp{NATOMS} is determined, it is not changed.
In \comp{FORCE}, only real atoms are used, therefore, for convenience,
the geometry is re-defined so that any dummy atoms are deleted.
\item \comp{NUMAT} The number of real atoms.  Here, ``real'' atoms are
the sparkles (100\% ionized atoms) and normal atoms.  The difference
between \comp{NATOMS} and \comp{NUMAT} is the number of dummy atoms plus
any translational vectors.
\item\comp{NELECS} The total number of valence electrons in the system.  This
is calculated by adding up the number of valence electrons on each atom, and
subtracting the net charge on the system.
\item\comp{NALPHA} In a \comp{UHF} calculation, \comp{NALPHA} is the number
of $\alpha$ electrons in the system.  By default, the number of $\alpha$
electrons is the same as the number of $\beta$ electrons (for closed-shell
systems), or one more than the number of $\beta$ electrons (for odd-electron
systems). If a spin is defined (\comp{TRIPLET}, \comp{QUARTET}, etc.), then the
number of $\alpha$ electrons is two to four more than the number of $\beta$
electrons. For example, if \comp{QUINTET} is specified, then \comp{NALPHA}
would be four more than \comp{NBETA}. The value of \comp{NALPHA} is set in
\comp{STATE}.

In an RHF calculation, \comp{NALPHA}=0.
\item\comp{NBETA} In a \comp{UHF} calculation, \comp{NBETA} is the number
of $\beta$ electrons in the system. In an RHF calculation, \comp{NBETA}=0.
The value of \comp{NBETA} is set in
\comp{STATE}.


\item\comp{NCLOSE} In an RHF calculation, \comp{NCLOSE} is the number of
doubly-occupied M.O.s. In a UHF calculation, \comp{NCLOSE}=0.
The value of \comp{NCLOSE} is set in  \comp{STATE}.

\item\comp{NOPEN} In an RHF calculation, \comp{NOPEN} is the number of
the highest M.O. that would be fractionally occupied in the SCF calculation.
The ``fraction'' can range from 0.0 to 2.0.  For an isolated carbon atom, the
fractional occupancy (two $p$ electrons in three atomic orbitals) would be
0.6667, and \comp{NCLOSE} would be 1, and \comp{NOPEN} would be 4.  By default,
the fractional occupancy of the SOMO in odd-electron
systems is 1.0. In a UHF calculation, \comp{NOPEN}=0.
The value of \comp{NCLOSE} is set in  \comp{STATE}.
\item\comp{NORBS} The number of atomic orbitals in a system.  Each hydrogen atom
contributes one atomic orbital, every other atom contributes four, except atoms
with $d$-orbitals, which contribute nine, and sparkles, which do not
contribute any.  \comp{NORBS} is set in \comp{STATE}.
\item\comp{ISCF} At the end of a calculation, \comp{ISCF=1} if an SCF exists,
otherwise \comp{ISCF=2}.  Set in \comp{ISITSC}.
\item \comp{NSCF} The number of SCF calculations done.  Each time an SCF
calculation is done, \comp{NSCF} is incremented by one.
\item \comp{ID} The number of dimensions for polymeric systems.  For molecules,
\comp{ID=0}; for polymers, \comp{ID=1}; for layer systems, \comp{ID=2}; and for
solids, \comp{ID=3}.  Set in \comp{GMETRY}.
\item \comp{NVAR} The number of geometric variables.  Normally, this is the
number of coordinates flagged by a ``1'' in the input data set.  In a \comp{FORCE}
calculation, or one that implies a \comp{FORCE} calculation, such as an IRC,
then \comp{NVAR=3*NUMAT}.
\item \comp{MOLNUM}  The number of the system.  Each new molecule run is given
a new number, with the first molecule having \comp{MOLNUM=1}.  If several
geometric operations are done on one system, they all share the
same \comp{MOLNUM}, however, if \comp{OLDGEO} is used, then the system
is considered to be ``new''. The only exception is when an IRC is run, in which
case the value of \comp{MOLNUM} is increased between the \comp{FORCE}
and the \comp{IRC} calculations.  This is needed, because the system is
re-oriented before the \comp{IRC} is run.
\item \comp{LATOM} In a reaction coordinate, \comp{LATOM} is the atom number
of the atom that moves.  This coordinate is indicated in the input data set
by the ``-1'' flag.  The atom can be real or dummy. Set in \comp{READMO}.
See also \comp{LPARAM}.
\item \comp{LPARAM} In a reaction coordinate, \comp{LPARAM} is the coordinate
(first, second, or third) of the atom that moves.  Set in \comp{READMO}. See
also \comp{LATOM}.
\item \comp{ITRY}  The number of iterations allowed in attempting to generate
an SCF.  By default, \comp{ITRY=200}, but this can be changed by use of
\comp{ITRY=n}.
\item \comp{ITYPE} The type of calculation  to be run.
If an MNDO calculation, \comp{ITYPE=4}; AM1, \comp{ITYPE=2}; PM3,
\comp{ITYPE=3}, MINDO/3, \comp{ITYPE=4}; MNDO-$d$, \comp{ITYPE=5}.
\item \comp{MFLAG} Used by the Tomasi method.
\item \comp{ITERQ} Used by the Tomasi method
\item \comp{IFLEPO} A flag to indicate how the geometry optimization
terminated.  Initialized in \comp{RMOPAC} and normally set in
\comp{EF}, \comp{FLEPO}, \comp{NLLSQ}, and \comp{POWSQ}.  Set in \comp{ITER}
when the SCF fails.  The various meanings of IFLEPO are defined in \comp{DATA}
statements in \comp{WRITMN}.
\item \comp{MPACK} The number of array elements in the density, one-electron,
and Fock matrices.  In conventional work, this is  \comp{(NORBS*(NORBS+1))/2}
and is set in \comp{STATE},
in MOZYME work, \comp{MPACK} is determined from the number of atoms that
interact with each other.  For large systems, \comp{MPACK} is normally
much smaller than the value in conventional work.
\item \comp{N2ELEC} The number of two-electron integrals.  The number
of integrals (0,1,16, or 81) contributed by an atom depends on the
number of orbitals (0, 1, 4, or 9) on that atom.  The NDDO
approximation indicates that for atom pairs, the number of integrals is
$(n(n+1))*(m(m+1))/4)$, where $n$ and $m$ are the number of orbitals on
each atom.  In MOZYME, \htmlref{the number is reduced when two atoms
are far apart}{m2el}.
\begin{latexonly}
See p.~\pageref{m2el} for details.
\end{latexonly}
\item \comp{NMOS} The number of M.O.s involved in the active space in a C.I.
calculation.  Set in \comp{STATE} only.
\item \comp{LAB} The number of microstates or configurations included in the
C.I.\ calculation.  Set in \comp{MECI} only.
\item \comp{NELEC} The number of electrons involved in the C.I. calculation.
Set in \comp{MECI} only.
\addtocounter{enumi}{3}
\item \comp{MAXTXT} Maximum number of characters in the description of an
atom.  (An atom label can include a short description of up to 13 letters,
for example ``\verb:N(  634 ASN  46):)''
\item \comp{LAST} \comp{LAST=0} if the current SCF is {\em not} the last SCF. \
Only on the last SCF, when \comp{LAST=1}, are the symmetries of the M.O.s
determined.  If the system is RHF open shell, then in \comp{WRITMN}, \comp{LAST}
is set to 3, and \comp{MECI} is called.  This prints the symmetries of the
states.
\item \comp{NDEP} The number of symmetry-dependent coordinates.  Set in
\comp{GETGEG}, \comp{GETSYM}, and \comp{MAKSYM}.
\item \comp{LABSIZ} The number of states to be calculated in a C.I. (usually
less than \comp{LAB}, until the last C.I., when all states are calculated.)
\item\comp{LM61} The number of matrix elements in the one-center part of the
density, one-electron, or Fock matrices.  Each hydrogen contributes 1 element,
each $s-p$ atom contributes 10 ($(4*(4+1))/2$), and each $s-p-d$ atom
contributes 45 elements ($(9*(9+1))/2$).
\item\comp{ICOMPF} Used by the Tomasi method.
\item\comp{ISAFE} A flag to indicate whether some large arrays can be
ignored.  If \comp{ISAFE=1} then these arrays are used.  If \comp{ISAFE=0}, then
the amount of memory used is reduced, but some functionalities, such as
the Camp-King converger, will not work.  Normally, this is not important.
\comp{ISAFE} is set at compile time, and can be re-set at run time by
\comp{SAFE} or \comp{UNSAFE}.
\item\comp{LIMSCF} In the SCF, if \comp{LIMSCF=1}, then the SCF can be stopped
before the usual criteria are satisfied, if the energy is obviously lower
or obviously higher.  This saves time, in that the geometry is still not
near to the stationary point.  \comp{LIMSCF=0} if the usual SCF criteria must
be met. This is necessary in \comp{FORCE} and transition state calculations.
\item \comp{NA1}  A flag to indicate the nature of the coordinates.
If \comp{NA1=1} then all coordinates are Cartesian.  Default: \comp{NA1=0}.
\item\comp{IGROUP} The number of the point-group of the system.
\item \comp{NCLASS} The number of classes necessary to define the point-group.
For $I_h$, for example, four operations (four classes) are necessary: $E$,
$C_5$, $C_3$, and the inversion operation.
\item \comp{NIRRED} The number of irreducible representations in the point-group.
\item \comp{NSPA} Used by the COSMO method.
\item \comp{NPS} Used by the COSMO method.
\item \comp{NPS2} Used by the COSMO method.
\addtocounter{enumi}{1}
\item \comp{NSTATE} The number of states used in describing the electronic
state of a system.  Normally 1, the number can increase when degenerate
states are involved.  Thus in CH$_4^+$, without Jahn-Teller distortion,
the value of \comp{NSTATE} would be 3 (representing the $^2T_u$ state).
Note: For degenerate open shell systems, \comp{OPEN(n,m)} will normally be
needed.
\item \comp{NUMPTS} Used by the COSMO method.
\addtocounter{enumi}{1}
\item \comp{LENABC} Estimated maximum number of surface points in the
COSMO method.
\item \comp{MESP} Size of the ESP array in the electrostatic potential
calculation, set in  ``sizes.h''.
\item \comp{NUMRED} The number of atoms involved in the SCF when the \comp{RAPID}
option is used.  The number of atoms that move plus the number of atoms attached
to them.
\item \comp{NORRED} The number of LMOs involved in the SCF when the \comp{RAPID}
option is used.
\item \comp{NELRED} The number of electrons involved in the SCF when the
\comp{RAPID} option is used.
\item \comp{NRES} The number of residues identified in a protein when
\comp{RESIDUES} or \comp{RESEQ} is specified.
\item \comp{NUMCAL} For a given calculation (input geometry), the number
of the type of calculation.  For example, an initial SCF (\comp{NUMCAL=1}),
followed by a \comp{FORCE} calculation (\comp{NUMCAL=2}).
\addtocounter{enumi}{1}
\item \comp{IDIAGG} Used in the annihilation operation in MOZYME, \comp{IDIAGG}
increments every time \comp{DIAGG} is called.
\item \comp{MDRCRS} Controls whether the restart file in a DRC or IRC run should
be formatted or unformatted.  Set in \comp{MOPAC} from \comp{sizes.h}.
If \comp{MDRCRS=1}, then a formatted restart file is used.  This can be read
easily, but uses more space and is less precise than unformatted (\comp{MDRCRS=0}.
\addtocounter{enumi}{2}
\item \comp{IPAD2} The estimated maximum average number of atoms in a LMO.
The default value is usually sufficient, but it can be reset by \comp{NLMO=nnn}.
\item \comp{IPAD4} The estimated maximum average number of atomic orbitals
in a LMO.  Set as a function of \comp{IPAD2}.
\addtocounter{enumi}{34}
\item \comp{ICROS} The type of calculation: 1 if an intersystem crossing,
0 otherwise.
\item \comp{MORB} The maximum number of orbitals on any atom: 1, 4, or 9.
\item \comp{unnamed} The type of method used in solving the SCF equations.
0 indicates conventional (MOPAC type), 1 indicates LMO (MOZYME type).
\addtocounter{enumi}{1}
\item \comp{unnamed} The type of derivative calculation: 1 = MOZYME -1 =
remove contributions to derivative arising from moving atoms; 2 = MOZYME 0 =
calculate MOZYME derivatives {\em de novo}; 3 = MOZYME +1 = add in contributions
to derivatives arising from moving atoms; 4 = MOPAC RHF; 5 = MOPAC UHF.
\end{enumerate}
