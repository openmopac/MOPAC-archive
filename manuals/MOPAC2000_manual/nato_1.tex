\section{Overlap Integrals}\index{Overlap!integrals|(}
 The particular technique used for the evaluation of the
overlap integral depends on the atoms involved and whether
analytic derivatives are used. All four semiempirical
\index{Slater orbitals}
methods use Slater-type orbitals, STOs, although when
analytic derivatives are involved~\cite{analyt}, a Gaussian 
\index{Gaussian!orbitals}
expansion~\cite{analyt}
of STOs is normally used. 

 Specific expressions for various of the overlap
integrals have appeared in the literature. These are
normally used for those overlaps which involve only small
principal quantum numbers, PQN, $n$, and a low angular quantum
\index{Quantum numbers! in overlap}
number, $l$. For the general case, however, in which any PQN
may be encountered, the general overlap integral is used.
As the final expression is rather ungainly, a simple
derivation of the overlap integral will be given.

 Slater atomic orbitals are of form\label{so}
$$
\varphi = \frac{(2\xi )^{n+1/2}}{(2n)!^{1/2}}r^{n-1}e^{-\xi r}Y_l^m(\theta ,\phi ),
$$
where the $Y_l^m(\theta ,\phi )$ are the normalized complex spherical
\index{Spherical harmonics}
harmonics. Complex spherical harmonics are chosen for
convenience; $Y_l^m(\theta ,\phi )$ real orbitals have a similar behavior,
but require more manipulation. The $\theta$ dependence of
\index{Laguerre polynomials}
spherical harmonics are the Laguerre polynomials, of form
$$
Y_l^m(\theta ,\phi ) = \frac{e^{im\phi }}{(2\pi )^{1/2}}\left [\frac{(2l+1)(l-m)!}{2(l+m)!}\right ]^{1/2}
\frac{sin^m\theta\ d^{l+m}(cos^2\theta-1)^l}{2^l\ l!\ (d\ cos\theta)^{l+m}}.
$$
For convenience the phase factor is set to +1; this varies
according to which source is used and the purpose for which
the Laguerre polynomials are used.

 Solving the differential gives
$$
\frac{d^{l+m}}{(d\ cos\theta)^{l+m}}(cos^2\theta-1)^l = \sum_j\frac{l!\ (2j)!\ (-1)^{l+j}}{(l-j)!\ j!\ (2j-l-m)!}(cos\theta )^{2j-l-m},
$$
which, on rearranging to have the summation start at zero,
becomes
$$
\frac{d^{l+m}}{(d\ cos\theta)^{l+m}}(cos^2\theta-1)^l = \sum_{j=0}^{1/2(l-m)}\frac{l!\ (2(l-j))!\ (-1)^j}{j!\ (l-j)!\ (l-m-2j)!}(cos\theta )^{l-m-2j}.
$$
Substituting this into the STO yields
\begin{eqnarray}
\varphi &=& \frac{(2\xi )^{n+1/2}}{(2n)!^{1/2}}\left [\frac{(2l+1)(l-m)!}{2(l+m)!}\right ]^{1/2}\left [\frac{sin^m\theta }{2^l}r^{n-1}e^{-\xi r}\frac{e^{im\phi}}{(2\pi )^{1/2}}\right ] \nonumber \\
&& \sum_{j=0}^{1/2(l-m)}\frac{(2(l-j))!\ (-1)^j}{j!\ (l-j)!\ (l-m-2j)!}(cos\theta )^{l-m-2j}. \nonumber
\end{eqnarray}  
At this point it is convenient to collect some of the
constants together; thus,
$$
C_{lmj}=\left [\frac{(l-m)!}{(l+m)!}\right ]^{1/2}\frac{(2(l-j))!\ (-1)^j}{2^l\ j!\ (l-j)!\ (l-m-2j)!},
$$
which allows us to represent the STO in a considerably
simplified form:
$$
\varphi = \frac{(2\xi )^{n+1/2}}{(2n)!^{1/2}}\frac{(2l+1)^{1/2}}{2^{1/2}}sin^m\theta\ 
\/ r^{n-1}\ e^{-\xi r}\frac{e^{im\phi}}{(2\pi )^{1/2}}\sum_{j=0}^{1/2(l-m)}C_{lmj}(cos\theta)^{l-m-2j}.
$$
The overlap integral of two STOs can then be represented as
\begin{eqnarray}
<\varphi _a\varphi _b>& =& \frac{(2\xi _a)^{na+1/2}(2\xi _b)^{nb+1/2}}{((2n_a)!(2n_b)!)^{1/2}}
\frac{[(2l_a+1)(2l_b+1)]^{1/2}}{2}\nonumber \\ &&
\int _0^{\infty}sin^m\theta _a\/sin^m\theta _b\/r_a^{na-1}\/r_b^{n_b-1}e^{-r_a\xi _a}\/e^{-r_b\xi _b}\/
\frac{e^{im\phi}e^{im\phi *}}{2\pi } \nonumber \\
 & & \sum_{j_a=0}^{1/2(l_a-m)}C_{i_amj_a}(cos\theta_a)^{l_a-m-2j_a}
  \sum_{j_b=0}^{1/2(l_b-m)}C_{i_bmj_b}(cos\theta_b)^{l_b-m-2j_b}{\rm d}v\nonumber
\end{eqnarray}
It is impractical to solve this integral using polar
\index{Coordinates!prolate spheroidal}
coordinates. Instead, a prolate spheroidal coordinate
system is used. Using the identities:
$$
\begin{array}{ccc}
r_a=\frac{R(\mu + \nu)}{2};  & cos\theta _a = \frac{(1+\mu \nu )}{(\mu + \nu )};&sin\theta _a = \frac{((\mu ^2-1)(1-\nu ^2))^{1/2}}{(\mu + \nu )} \\
 r_b=\frac{R(\mu - \nu)}{2}; &cos\theta _b = \frac{(1-\mu \nu )}{(\mu - \nu )};&
sin\theta _b = \frac{((\mu ^2-1)(1-\nu ^2))^{1/2}}{(\mu - \nu )},
\end{array}
$$
this gives
$
dv = \frac{R^3}{8}(\mu + \nu )(\mu - \nu)d\mu d\nu d\phi.
$

 Substituting these identities into the previous
expression we get:
\begin{eqnarray}
<\varphi _a\varphi _b>& =& \int _0^{2\pi }\int _{-1}^{1} \int _1^{\infty}
\xi_a ^{n_a+1/2}\xi_b^{n_b+1/2}\left [\frac{(2l_a+1)(2l_b+1)}{(2n_a)!(2n_b)!}\right ]
^{1/2} \nonumber \\
&&\frac{((\mu ^2-1)(1-\nu ^2))^m}{(\mu+\nu)^m(\mu-\nu)^m}
\frac{R^{n_a-1}}{2^{n_a-1}}(\mu+\nu)^{n_a-1}\frac{R^{n_b-1}}{2^{n_b-1}}(\mu-\nu)^{n_b-1}\nonumber \\ 
&&\frac{e^{-R\xi_a(\mu+\nu)/2}e^{-R\xi_b(\mu-\nu)/2}}{2\pi}
 \nonumber \\
&&\sum_{j_a=0}^{(l_a-m)/2}\sum_{j_b=0}^{(l_b-m)/2} 
C_{l_amj_a}C_{l_bmj_b}\frac{(1+\mu\nu)^{l_a-m-2j_a}(1-\mu\nu)^{l_b-m-2j_b}}
{(\mu+\nu)^{l_a-m-2j_a}(\mu-\nu)^{l_b-m-2j_b}} \nonumber \\&& 
\frac{R^3}{8}(\mu+\nu)(\mu-\nu)d\mu d\nu d\phi , \nonumber
\end{eqnarray}
which, on integrating over and rearranging, gives:
\begin{eqnarray}
<\varphi _a\varphi _b>& =& \int _{-1}^{1} \int _1^{\infty}
\frac{\xi_a ^{n_a+1/2}\ \xi_b^{n_b+1/2}}{2}\left [\frac{(2l_a+1)(2l_b+1)}{(2n_a)!(2n_b)!}\right ]
^{1/2}\ R^{n_a+n_b+1} \nonumber \\
&&\sum_{j_a=0}^{(l_a-m)/2}\sum_{j_b=0}^{(l_b-m)/2}
C_{l_amj_a}\ C_{l_bmj_b}(\mu^2-1)^m(1-\nu^2)^m(\mu+\nu)^{n_a-l_a+2j_a}\nonumber \\&&
(\mu-\nu)^{n_b-l_b+2j_b}(1+\mu\nu)^{l_a-m-2j_a}(1-\mu\nu)^{l_b-m-2j_b}
\ e^{-R\xi_a(\mu+\nu)/2}\ e^{-R\xi_b(\mu-\nu)/2}d\mu d\nu . \nonumber
\end{eqnarray}
This is a product of six simple expressions of type $(a+b)^n$. Expanding each term as a binomial generates six summations:
\begin{eqnarray}
<\varphi _a\varphi _b>& =& \int _{-1}^{1} \int _1^{\infty}
\frac{\xi ^{n_a+1/2}\ \xi^{n_b+1/2}}{2}\left [\frac{(2l_a+1)(2l_b+1)}{(2n_a)!(2n_b)!}\right ]
^{1/2}\ R^{n_a+n_b+1} \nonumber \\
&&\sum_{j_a=0}^{(l_a-m)/2}\sum_{j_b=0}^{(l_b-m)/2}
C_{l_amj_a}\ C_{l_bmj_b}\sum_{k_a=0}^m\sum_{k_b=0}^m
\sum_{P_a}^{n_a-l_a+2j_a}\sum_{P_b}^{n_b-l_b+2j_b}
\sum_{q_a}^{l_a-m-2j_a}\sum_{q_b}^{l_b-m-2j_b} \nonumber \\ &&
\frac{(l_b-m-2j_b)!}{(l_b-m-2j_b-q_b)!q_b!}
\frac{(l_a-m-2j_a)!}{(l_a-m-2j_a-q_a)!q_a!}
\frac{(n_b-l_b+2j_b)!}{(n_b-l_b+2j_b-P_b)!P_b!} \nonumber \\ &&
\frac{(n_a-l_a+2j_a)!}{(n_a-l_a+2j_a-P_a)!P_a!} 
\frac{m!^2}{(m-k_b)!k_b!(m-k_a)!k_a!}  \nonumber \\ &&
(-1)^{k_a+k_b+m+P_b+q_b}
\nu^{2k_a+P_a+P_b+q_a+q_b} 
\mu^{2k_b+n_a-l_a+2j_a+n_b-l_b+2j_b-P_a-P_b+q_a+q_b}{\rm d}\mu\ {\rm d}\nu
  \nonumber 
\end{eqnarray}

Using integration by parts, and making use of the following
integrals
$$                
\int_1^{\infty}x^ne^{-ax}dx=e^{-a}\sum_{\mu=1}^{n+1}\frac{n!}{a^{\mu}(n-\mu+1)}=A_n(a) 
$$
$$
\int_{-1}^{1}x^ne^{-ax}dx=-e^{-a}\sum_{\mu=1}^{n+1}\frac{n!}{a^{\mu}(n-\mu+1)}
-e^a\sum_{\mu=1}^{n+1}\frac{n(-1)^{n-\mu}}{a^{\mu}(n-\mu+1)!} =B_n(a),
$$                
the overlap integral becomes
\begin{eqnarray}
<\varphi _a\varphi _b>& =&
\frac{\xi ^{n_a+1/2}\xi^{n_b+1/2}}{2}\left [\frac{(2l_a+1)(2l_b+1)}{(2n_a)!(2n_b)!}\right ]
^{1/2}R^{n_a+n_b+1} \nonumber \\ &&
\sum_{j_a=0}^{(l_a-m)/2}C_{l_amj_a}\sum_{j_b=0}^{(l_b-m)/2}
C_{l_bmj_b}\sum_{k_a=0}^m\sum_{k_b=0}^m
\sum_{P_a=0}^{n_a-l_a+2j_a}\sum_{P_b=0}^{n_b-l_b+2j_b}
\sum_{q_a=0}^{l_a-m-2j_a}\sum_{q_b=0}^{l_b-m-2j_b} \nonumber \\ &&
\frac{(l_b-m-2j_b)!}{(l_b-m-2j_b-q_b)!q_b!}
\frac{(l_a-m-2j_a)!}{(l_a-m-2j_a-q_a)!q_a!}
\frac{(n_b-l_b+2j_b)!}{(n_b-l_b+2j_b-P_b)!P_b!} \nonumber \\ &&
\frac{(n_a-l_a+2j_a)!}{(n_a-l_a+2j_a-P_a)!P_a!}
\frac{m!^2}{(m-k_b)!k_b!(m-k_a)!k_a!} 
(-1)^{k_a+k_b+m+P_b+q_b} \nonumber \\ &&
B_{2k_a+P_a+P_b+q_a+q_b}\left (\frac{R(\xi_a-\xi_b)}{2}\right ) 
A_{2k_b+n_a-l_a+2j_a+n_b-l_b+2j_b-P_a-P_b+q_a+q_b}\nonumber \\ &&
\left (\frac{R(\xi_a+\xi_b)}{2}\right ) 
{\rm d}\mu\ {\rm d}\nu , \nonumber
\end{eqnarray}


in which the coefficients $C_{lmj}$ have the numerical values given
in Table~\ref{clmj}.
\begin{table}
\caption{\label{clmj} Values of $C_{lmj}$}
$$
\begin{array}{lllllllll} \\ \hline
l & m & j & C_{lmj} & &    l & m\rule[-0.2cm]{0cm}{0.6cm} & j & C_{lmj} \\ \hline
0 & 0 & 0 & 1.0        & & 3 & 0 & 0 & 5/2 \\
1 & 0 & 0 & 1.0        & & 3 & 1 & 0 & (225/48)^{1/2} \\
1 & 1 & 0 & (1/2)^{1/2}& & 3 & 2 & 0 & (15/8)^{1/2} \\
2 & 0 & 0 & 3/2        & & 3 & 3 & 0 & (5/16)^{1/2} \\
2 & 1 & 0 & (3/2)^{1/2}& & 3 & 0 & 1 & -3/2 \\
2 & 2 & 0 & (3/8)^{1/2}& & 3 & 1 & 1 & -(3/16)^{1/2} \\ 
2 & 0 & 1 & -1/2       & & \\ \hline
\end{array}
$$
\begin{center}
 Note: In subroutine SS the array AFF(l,m,2j) corresponds to C$_{lmj}$ here.
\end{center}
\end{table}
which is the most convenient form for algorithmic use.

In this form, the overlap integral can be found in function SS.
\index{Overlap!integrals|)}

