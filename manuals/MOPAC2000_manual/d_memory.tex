\section{Memory Considerations}\index{Memory!how to minimize}
Of their nature, large systems use a large amount of memory.  This will
sometimes prevent a calculation from being run.  There are, however, several
ways to reduce the amount of memory needed by a specific calculation.  These
can be summarized as follows:
\begin{itemize}
\item Run using the MOZYME function.

For large systems, the MOZYME function uses much less memory than 
the default MOPAC.  The MOZYME function only works for closed shell RHF 
ground state systems.  Also, it cannot be used when high precision is
needed, such as in \comp{FORCE}, hyperpolarizability, or \comp{IRC}
calculations.

\item Run the job with \comp{SIZES}.

Run the job that fails using \comp{SIZES}.  This will print out the sizes of
all arrays created dynamically.  By inspection of these arrays, ways can
usually be found to reduce the memory demand.  For example, the default 
geometry optimizer using the \comp{MOZYME} function, \comp{EF}, uses a large 
amount of memory.  Switching to the BFGS optimizer (\comp{BFGS}) will save a
lot of memory.

\item Run in \comp{UNSAFE} mode.  

Normally, by default, MOPAC runs in \comp{SAFE} mode.  This means that when
things start to go wrong in the SCF, special procedures can be used in an
attempt to generate a SCF.  These procedures use a large amount of memory, and
are normally not used.  Most likely, if memory is limited, the job can be run
in \comp{UNSAFE} mode.  This will save a lot of memory, but at the same time,
if things start to go wrong, the job has an increased chance of failing. If the
job fails, the results will usually indicate the failure in an obvious way.  

\item Reduce the cutoffs.

In a \comp{MOZYME} calculation, useful results can often be obtained even when
the values of \comp{CUTOF1} and \comp{CUTOF2} are reduced. When these are
reduced, the amount of memory needed drops quite rapidly.

If the objective of the calculation is to optimize the geometry, then reducing
the cutoffs will both increase the speed and reduce the memory demand without
significant loss of accuracy.
\end{itemize}


