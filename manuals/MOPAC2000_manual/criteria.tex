\chapter{Criteria}\label{criteria}
\index{MOPAC!criteria}
MOPAC uses various criteria which  control  the  precision  of  its stages.  
These criteria are chosen as the best compromise between speed and acceptable
errors in the results.  The user can override the default settings  by  use 
of  keywords;  however,  care  should be exercised as increasing a criterion
can introduce the potential for  infinite  loops, and decreasing a criterion
can result in unacceptably imprecise results. These are usually characterized
by `noise' in a reaction path, or  large values for the trivial vibrations in a
force calculation.

\section{SCF criterion}\index{SCF!criterion|ff}\index{Criteria!SCF}
\index{SCFCRT}
\subsubsection*{SCFCRT}
\begin{description}
\item[Defined in:] \comp{ITER}
\item[Default value:] 0.0001 kcal/mol
\item[Basic test:] Change in energy in kcal/mol on successive iterations is 
less than \comp{SCFCRT}.
\item[Exceptions:]~
\begin{itemize}
\item If \comp{PRECISE} is specified, \comp{SCFCRT}=0.000001
\item If a polarization calculation, \comp{SCFCRT}=1.D-11
\item If a \comp{FORCE} calculation, \comp{SCFCRT}=0.0000001
\item If \comp{SCFCRT=$n.nnn$} is specified \comp{SCFCRT}=$n.nnn$
\item If a BFGS optimization, \comp{SCFCRT} becomes a function of the 
difference between the current energy and the lowest energy of previous SCFs.
\end{itemize}
\item[Secondary tests:]~
\begin{enumerate}
\item Change in density matrix elements on two successive iterations must be 
less than 0.001
\item Change in energy in eV on three successive iterations must be less than 
10 $\times$ \comp{SCFCRT}.
\end{enumerate}
\end{description}

\section{Geometric optimization criteria}
\index{TOLERX}
\index{DELHOF}
\index{TOLERG}
\index{TOLERX}
\index{TOL2}
\index{TOLS1}
\index{EF!criterion}
\subsubsection*{GNORM ``RMS GRADIENT =  $n.nn$  IS LESS THAN CUTOFF =  $m.mm$''}
\begin{description}
\item[Defined in:] \comp{EF}
\item[Default value:] 1.0
\item[Basic test:] The calculated GNORM is less than that defined by \comp{GNORM}.    \\
\end{description}

\subsubsection*{TOLERX ``Test on X Satisfied''}
\index{Heat of Formation!criteria}
\index{$\Delta H_f$!criteria}
\begin{description}
\item[Defined in:] \comp{FLEPO}
\item[Default value:] 0.0001 \AA ngstroms
\item[Basic test:] The projected change in geometry is less than \comp{TOLERX} \AA ngstroms.
\item[Exception:] If \comp{GNORM} is specified, the TOLERX test is not used.
\end{description}

\subsubsection*{DELHOF ``Herbert's Test Satisfied''}
\begin{description}
\item[Defined in:] \comp{FLEPO}
\item[Default value:] 0.001
\item[Basic test:] The projected change in geometry is less than \comp{DELHOF} kcals/mol.
\item[Exception:] If \comp{GNORM} is specified, the DELHOF test is not used.
\end{description}

\subsubsection*{TOLERG ``Test on Gradient Satisfied''}
\begin{description}
\item[Defined in:] \comp{FLEPO}
\item[Default value:] 1.0
\item[Basic test:] The gradient norm in kcals/mol/\AA ngstrom is less than
TOLERG multiplied by the square root of the number of coordinates to be
optimized.
\item[Exceptions:]~
\begin{itemize}
\item If \comp{GNORM=$n.nnn$} is specified, TOLERG=$n.nnn$ divided by the square
root of the number of coordinates to be optimized, and the secondary tests are
not done.  If \comp{LET} is not specified, $n.nnn$ is reset to 0.01, if it was
smaller than 0.01.
\item If \comp{PRECISE} is specified, TOLERG=0.2 \index{PRECISE}
\item If a \comp{SADDLE} calculation, TOLERG is made a function of the last
gradient  norm.
\end{itemize}
\end{description}

\subsubsection*{TOLERF ``Heat of Formation Test Satisfied''}
\begin{description}
\item[Defined in:] \comp{FLEPO}
\item[Default value:] 0.002 kcal/mol
\item[Basic test:] The calculated heats of formation on two successive cycles 
differ by less than TOLERF.
\item[Exception:] If \comp{GNORM} is specified, the TOLERF test is not used.
\item[Secondary tests:] For the TOLERG, TOLERF, and TOLERX tests, a second 
test in which no individual component of the gradient should be larger than 
TOLERG must be satisfied.
\item[Other tests:] If, after the TOLERG, TOLERF, or TOLERX test has been
satisfied three consecutive times the heat of formation has dropped by less
than 0.3kcal/mol, then the optimization is stopped.
\end{description}

\subsubsection*{TOL2}
\begin{description}
\item[Defined in:] \comp{POWSQ}
\item[Default value:] 0.4
\item[Basic test:] The absolute value of the largest component of the gradient 
is less than \comp{TOL2}
\item[Exceptions:]~
\begin{itemize}
\item If {bf PRECISE} is specified, \comp{TOL2}=0.01 \index{PRECISE}
\item If \comp{GNORM=$n.nn$} is specified, \comp{TOL2}=$n.nn$
\item If \comp{LET} is not specified, \comp{TOL2} is reset to 0.01, if $n.nn$ 
was smaller than 0.01.
\end{itemize}
\end{description}

\subsubsection*{TOLS1}
\begin{description}
\item[Defined in:] \comp{NLLSQ}
\item[Default value:] 0.000 000 000 001
\item[Basic test:] The square of the ratio of the projected change in the
geometry to the actual geometry is less than \comp{TOLS1}.
\end{description}

\begin{description}
\item[Name: ] $<$none$>$
\item[Defined in:] \comp{NLLSQ}
\item[Default value:] 0.2
\item[Basic test:] Every component of the gradient is less than 0.2.
\end{description}
