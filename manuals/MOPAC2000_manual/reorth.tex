\subsubsection{Re-Orthogonalization of LMOs}\label{reorth}
During the course of a long calculation, the LMOs will become increasingly
non-orthogonal.  Subroutine \comp{CHECK} ensures that the LMOs remain
normalized; this is a very rapid calculation.  Ensuring that the LMOs are all
orthogonal is not so simple.  Re-orthogonalizing the LMOs is a lengthy
calculation, and is not routinely performed.  However, the option exists to
re-orthogonalize the LMOs, and this can be done either during a \comp{1SCF}
calculation (the preferred way), or periodically during a geometry
optimization.

Re-orthogonalizing consists of taking pairs of LMOs, $\psi_i$ and $\psi_j$, and
forming linear combinations such that the overlap ($<\!\psi_i|\psi_j\!>$) is
zero.  The LMOs involved form the full set, that is, both occupied  plus
virtual sets are used.

The two LMOs can be regarded as unit vectors that are almost at 90$^{\circ}$ to
each other.  Let the difference from  90$^{\circ}$ be $\alpha$.  If the vectors
are each rotated by -$\frac{1}{2}\alpha$, then they will become perfectly
orthogonal. This operation is most conveniently performed using perturbation
theory. Let:
$$
<\!\psi_i|\psi_i\!> \simeq 1 \simeq <\!\psi_j|\psi_j\!>
$$
and 
$$
|<\!\psi_i|\psi_j\!>| = S_{ij} \ll 1,
$$
then
$$
\psi_i' = \psi_i - \frac{1}{2}S_{ij}\psi_j
$$
and 
$$
\psi_j' = \psi_j - \frac{1}{2}S_{ij}\psi_i.
$$

That the new LMOs are orthogonal can readily be demonstrated:
\begin{eqnarray}
<\!\psi_i'|\psi_j'\!> &=&<(\psi_i - \frac{1}{2}S_{ij}\psi_j)|
(\psi_j - \frac{1}{2}S_{ij}\psi_i)\!> \nonumber \\
 &=&<\psi_i|\psi_j>-\frac{1}{2}S_{ij}<\!\psi_i|\psi_i>-
\frac{1}{2}S_{ij}<\!\psi_j|\psi_j\!>+\frac{1}{4}S_{ij}^2<\!\psi_j|\psi_i\!> 
\nonumber \\
&=& S_{ij}-\frac{1}{2}S_{ij}-\frac{1}{2}S_{ij} +\frac{1}{4}S_{ij}^3 \nonumber \\
&=& 0 \nonumber
\end{eqnarray}

The calculation of the overlaps, $S_{ij}$, is most conveniently done for one
LMO, $\phi_i$, with all other LMOs.  Because of this, $\phi_i$ should not be
modified while the re-orthogonalization is done.  In order to avoid modifying
$\phi_i$, the rotation is changed so that $\phi_i$ remains stationary and all
the rotation is incurred by $\phi_j$, thus:
$$
\psi_i' = \psi_i 
$$
and
$$
\psi_j' = \psi_j - S_{ij}\psi_i.
$$

Before the re-orthogonalization, the LMOs are almost orthogonal, and the use of
perturbation theory here is fully justified.
