% Titlepage/ foreword/ etc of MOPAC manual --- set with Roman numbering
%\documentstyle[headerfooter]{book}
\documentstyle{book}
% Set up page parameters---basically A4
\topmargin 0pt
\headheight 12pt % Height of page header
\headsep 25pt    % Distance between header and text start
\footskip 30pt   % Distance between text bottom and footer bottom
\footheight 12pt % Height of page footer
\oddsidemargin  0.25in  % extra left margin for odd pages
\evensidemargin 0.125in % extra left margin for even pages
\textwidth 5.875in
\marginparwidth 0.8in % width of marginal notes
\marginparsep   11pt % Distance between text and marginal notes
\textheight 53\baselineskip
\advance\textheight by \topskip
\begin{document}
\begin{titlepage}
\begin{center}
{\huge\bf MOPAC manual (Sixth Edition)}\\
\vfill
{\LARGE\bf James J. P. Stewart}\\
\ \\
{\Large\bf Frank J. Seiler Research Laboratory\\
\ \\
United States Air Force Academy\\
\ \\
CO 80840}

\vfill

{\Large\sf October 1990}
\end{center}
\end{titlepage}

\pagenumbering{Roman}
%\pageheader{}{}{}
%\pagefooter{}{\thepage}{}

\begin{center}{\Large\bf Preface to sixth edition, version 6.00}\end{center}

      As indicated at the time of release of MOPAC  5.00,  there  has
 been  a  gap of two years' duration.  It is likely that a second gap
 of two years' duration will follow this release.

      The main change from a user's point of view (who else matters?)
 in  MOPAC  6.00 has been that MOPAC now runs faster than before.  In
 addition, the range of PM3 and AM1 elements is increased.   Finally,
 some  bells  and whistles have been added, such as Gaussian Z-matrix
 input and output, polymer electronic and phonon band structures  and
 densities  of states.  Experienced users should refer to the `update
 release notes' for a concise description of all modifications.



\begin{center}{\Large\bf Updates from version 5.00}\end{center}

      Except for MOPAC 5.00, MOPAC has  been  updated  once  a  year.
 This is the best compromise between staying current and asking users
 to continuously change their software.  Updates may be obtained from
 QCPE  at  the  same  cost as the original, or from sites that have a
 current copy.  All VAX versions of MOPAC have the same QCPE number -
 455;   they   are  distinguished  by  version  numbers.   Users  are
 recommended to update their programs at least once every two  years,
 and preferably every year.

\begin{center}{\Large\bf New Features of Version 6.0}\end{center}
\begin{enumerate}
\item PM3 has been extended to include Be, Mg, Zn, Ga, Ge, As, Se, Cd,
 In, Sn, Sb, Te, Hg, Tl, Pb, and Bi.
\item Changes to the IRC/DRC
 \begin{enumerate}
 \item The amount of output has been reduced.  The `missing' output
      can be printed by using appropriate keywords.
 \item Half-lives are now accurately generated.   Earlier  versions
      had  a  small  error  due  to  calculation  start-up.   Both
      positive and negative half-lives are now accurate.
 \end{enumerate}
\item The energy partition output has been rewritten so that all terms
 having  to  do  with  each  diatomic pair are now printed on one
 line.
\item A LOG file will normally be generated.  This is intended  to  be
 read  while  the  calculation  is  running.  The LOG file can be
 suppressed by the user.
\item Elements can be labeled with up to six alphanumeric characters.
\item Gaussian Z-matrices can be input and printed (in the ARC file).
\item Multiple data-sets can be run in one job.
\item Up to three lines of keywords can be specified.
\item The DEBUG, 1SCF, and C.I. keywords have been re-defined.
\item An Eigenvector Following option has been added.
\item Polymer electronic band structure and density of states added.
\item Polymer phonon band structure and density of states added.
\item The GRID option has been rewritten.
\item The PATH option has been extended.
\end{enumerate}

\pagebreak

\begin{center}{\Large\bf Keywords added in version 6.00}\end{center}

      In going to Version 6.00, many keywords were added.  These  are
 defined  fully  later  on.   The  complete  set  of  new or modified
 keywords follows:
\begin{verbatim}
 &            +          AIDER     AIGIN      AIGOUT     
 DIPOLE       DIPX       DIPZ      DIPZ       DMAX=n.n      
 EF           EIGINV     ESP       ESPRST     HESS=n       
 IUPD=n       K=(n.nn,n) MODE=n    MS=n       NOANCI      
 NODIIS       NOLOG      NONR      NOTHIEL    NSURF   
 OLDGEO       ORIDE      POINT     POINT1=n   POINT2=n      
 POTWRT       RECALC=n   SCALE     SCFCRT=    SCINCR  
 SETUP        SETUP=name SLOPE     STEP       SYMAVG      
 STO3G        TS         WILLIAMS
\end{verbatim}

                  
\begin{center}{\Large\bf Keywords dropped from version 5.00}\end{center}

 FULSCF  Reason:   Line  searches  now  always   involve   full   SCF
 calculations.   The  frozen  density  matrix  option  is  no  longer
 supported.

 CYCLES=n Reason:  The maximum number of cycles is now  not  defined.
 Users should control jobs via `` t=n.nn''.


                   
\begin{center}{\Large\bf Errors corrected in version 5.0}\end{center}

\begin{enumerate}
\item Force constants and frequencies  calculated  using  non-variationally
 optimized wavefunctions were faulty.
\item A full keyword line (no extra spaces) would be corrupted if the first
 character was not a space.
\item PRECISE FORCE calculations on triatomics had spuriously large trivial
 vibrations.
\item FORCE calculations with many more hydrogen atoms  than  MAXLIT  would
 fail to generate force constants or normal coordinates.
\item The EXTERNAL option was limited to AM1.
\item Vibrational transition dipoles were in error by about 30%.
\item The reformation  of  the  density  matrix  when  a  non-variationally
 optimized wavefunction was used was incomplete.
\end{enumerate} 

\begin{center}{\Large\bf Help with MOPAC}\end{center}

  Telephone and mail support is given by the   
  Frank J. Seiler Research Laboratory on a time
  permitting basis.  If you need help, call    
  the Seiler MOPAC Consultant at (719) 472--2655.
  Similarly, mail should be addressed to:       
                                               
{\noindent MOPAC Consultant\\                      
          FJSRL/NC\\                              
          U.S. Air Force Academy, CO 80840-6528\\}
                                               
\newpage

\begin{center}{\Large\bf Acknowledgements}\end{center}

      For her unflagging patience in checking the manual for  clarity
 of  expression, and for drawing to my attention innumerable spelling
 and grammatical errors, I thank my wife, Anna.

      Over the years a large amount of advice,  ideas  and  code  has
 been  contributed  by various people in order to improve MOPAC.  The
 following incomplete list recognises various contributors:
\begin{itemize}
\item Prof. Santiago Olivella: Critical analysis of Versions 1 to 3.
\item Prof. Tsuneo Hirano: Rewrite of the Energy Partition.
\item  Prof. Peter Pulay: Designing the rapid pseudodiagonalization.
\item  Prof. Mark Gordon: Critical comments on the IRC.
\item  Prof. Henry Kurtz: Writing the polarizability and hyperpolarizability.
\item  Prof. Henry Rzepa: Providing the code for the BFGS optimizer.
\item  Dr. Yoshihisa Inoue: Many suggestions for improving readability.
\item  Major Donn Storch and Lt. Col. Skip Dieter: Critical review of versions 
 3-5.
\item  Lt. Cols. Larry Davis and Larry Burggraf: Designed the form
 of the DRC and IRC.
\item Dr. John McKelvey: Numerous suggestions for improving output.
\item Dr. Erich Wimmer: Suggestions for imcreasing the speed of calculation.
\item Dr. James Friedheim: Testing of Versions 1 and 2.
\item Dr. Eamonn Healy: Critical evaluation of Versions 1-4.
\end{itemize}
             
This list does not include  the  large  number  of  people  who
developed  methods  which  are  used  in  MOPAC.  The more important
contributions are given in the References at the end of this Manual.

        I  wish  to  thank  Prof.\ Michael J. S. Dewar  for   providing   the
   facilities and funds during the initial development of the MOPAC program,
   the staff of the Frank J. Seiler Research Laboratory  and  the  Chemistry
   Department at the Air Force Academy for their support.
\end{document}

